\begin{example}
$f(x)= 1+x+x^4~$(23)\newline
$f(x)$ is a primitive polynomial, which means $\epsilon_0=2^M-1$ and the order of $\varphi_0,~\epsilon_0=15$. This means that for a weight-2 PC of the form $h(x)=1+x^{\alpha},~\alpha$ should be a multiple of 15.

The PCs have a general for
$$h(x)=1+x^{15\ell}$$
while the $a(x)$ can be expressed as 
$$ \sum_{i=0}^{\ell} x^{15i} \left(1 +x +x^2 +x^3+x^5+x^7+x^8+x^{11} \right)$$


For the weight-3 PCs, we refer to the table for $\GF(2^4)$ (Table \ref{gf-16-table}), and we observe that there are $7~(m,n)$ pairs which satisfy $x^m+x^n \equiv 1$. We define $\cM_i$ and $\cN_i,~i=0,1,2^{M-1}-2$ as follows
\begin{align}
	\cM_0 &:= \{15\ell + 1\}_{\ell \geq 0},~\cN_0 := \{15\ell + 4\}_{\ell \geq 0}\cr
	\cM_1 &:= \{15\ell + 2\}_{\ell \geq 0},~\cN_1 := \{15\ell + 8\}_{\ell \geq 0}\cr
	\cM_2 &:= \{15\ell + 3\}_{\ell \geq 0},~\cN_2 := \{15\ell + 14\}_{\ell \geq 0}\cr
	\cM_3 &:= \{15\ell + 5\}_{\ell \geq 0},~\cN_3 := \{15\ell + 10\}_{\ell \geq 0}\cr
	\cM_4 &:= \{15\ell + 6\}_{\ell \geq 0},~\cN_4 := \{15\ell + 13\}_{\ell \geq 0}\cr
	\cM_5 &:= \{15\ell + 7\}_{\ell \geq 0},~\cN_5 := \{15\ell + 9\}_{\ell \geq 0}\cr
	\cM_6 &:= \{15\ell + 11\}_{\ell \geq 0},~\cN_6 := \{15\ell + 12\}_{\ell \geq 0}
\end{align}
Then 
\begin{equation*}
\begin{split}
(\alpha,~\beta) \in &\bigcup_{i=0}^6 \cM_i^1\otimes\cN_i^1 
\end{split}
\end{equation*}

\begin{table}[htbp]
 \caption{Non-zero Elements of $\GF (2^4)$ generated by $f(x)=1+x+x^4$}
\centering
 \begin{tabular}{c c} 
 \toprule
 power representation & polynomial representation \\ [0.5ex] 
\midrule
$x^0~=x^{15}$ & $1$\\
\hline
$x$ & $x$\\
\hline
$x^2$ &  $x^2$\\
\hline
$x^3$ &  $x^3$\\
\hline
$x^4$ &  $1+x$\\
\hline
$x^5$ &  $x+x^2$\\
\hline
$x^6$ &  $x^2+x^3$\\
\hline
$x^7$ &  $1+x+x^3$\\
\hline
$x^8$ &  $1+x^2$\\
\hline
$x^9$ &  $x+x^3$\\
\hline
$x^{10}$ &  $1+x+x^2$\\
\hline
$x^{11}$ &  $x+x^2+x^3$\\
\hline
$x^{12}$ &  $1+x+x^2+x^3$\\
\hline
$x^{13}$ &  $1+x^2+x^3$\\
\hline
$x^{14}$ &  $1+x^3$\\
 \bottomrule
 \end{tabular}
 \label{gf-16-table}
\end{table}
\end{example}

\begin{example}
$f(x)=1+x^2+x^3+x^4$ (35)\newline

We may write $f(x)$ as 
$$f(x) =(1+x)(1+x+x^3)$$
and since $1+x$ is a factor, there are no weight-3 PCs associated with $f(x)$.

With respect to weight-2PCs, $\epsilon_0=1$ and $\epsilon_1=7$ and $\alpha$ for $h(x)=1+x^{\alpha}$ should be a multiple of $\lcm(\epsilon_0,\epsilon_1)=7$. $h(x)$ can then be written as 
$$h(x) =1+x^{7\ell}$$ and 
$$a(x) = \sum_{}^{}x^{7i}(1+x^2+x^3)$$
The detailed patterns of $a(x)$ and $h(x)$ for $1 \leq \ell \leq 2$ are listed in Table \ref{novelTab-extra2}.
\begin{table}[htbp]
 \caption{Weight-2 PCs for $f(x)=1+x^2+x^3+x^4$}
\centering
 \begin{tabular}{c c c} 
 \toprule
 $a(x)$ & $h(x)$ \\ [0.5ex] 
 \midrule
$1+x+x^2$
 & $1+x^{7}$ \\
\hline
$1+x^2+x^3+x^7+x^9+x^{10}$
 & $1+x^{14}$ \\
 \bottomrule
 \end{tabular}
 \label{novelTab-extra2}
\end{table}

\end{example}