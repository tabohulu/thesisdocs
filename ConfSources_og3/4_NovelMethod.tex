\section{The patterns of the low-weight PCs}
\label{sec3}
We assume $f(x)$ can be factorized into $K$ prime polynomials as 
\begin{align}
f(x)&=\prod_{k=0}^{K-1}f_k^{\gamma_k}(x)
\end{align}
where $\gamma_0,\gamma_1,\cdots,\gamma_{K-1}$ are positive integers and let $\varphi_k$ be a root of $f_{k}(x)$ of order $\epsilon_k$. After that, refering to \eqref{eq:low-weight-parity}, we consider the solution of
\begin{align}
	h(x) \mod f(x) \equiv 0
	\label{Eq:condition}
\end{align}

We start from the simplest case $K=1$, \ie, $f(x) = f_0^{\gamma_0}(x)$. Then, \eqref{eq:low-weight-parity} indicates that each root of $f(x)$ is also a root of $h(x)$ and we distinguish the cases $\gamma_0 = 1$ and $\gamma_0 > 1$. For the former case, since all $\varphi_0^i$, $0 \leq i < \epsilon_0$, are distinct from each other, the equation
\begin{align}
	h(\varphi_0^i)=0,~~~ 0 \leq i < \epsilon_0
	\label{Eq:rootcondition}
\end{align}
is a necessary and sufficient condition of \eqref{Eq:condition} while it is necessary but not sufficient for the latter case. Thus, for the case $\gamma_0 >1$, we obtain extra differential equations as
\begin{align}
\left.\frac{d^{(j)}h(x)}{d x^j}\right|_{x=\varphi_0^i}=0,~~~0 \leq i < \epsilon_0,~1 \leq j < \gamma_0
\label{Eq:differential}
\end{align}
where the derivation is calculated using the {\it Hasse derivative} defined as
\begin{align}
\frac{d^{j}x^k}{dx^j}=
	\begin{cases}
	\left({}_kC_j\mod 2\right)x^{k-j}, & k \geq j\cr
	0, & \mathotherwise
	\end{cases}
\end{align}
for the bionomial coefficient ${}_kC_j$.

For the case where $K>1$, we may repeat the above discussion for the roots $\varphi_k$, $0 < k < K$, and take the intersection of the results to determine the low-weight PCs.

\subsection{The weight-2 PCs}
\label{sec:PC2}
Each weight-2 PC can be written as 
\begin{equation}
h(x)=1+x^{\alpha}
\label{eq:wt2-gen-form}
\end{equation}
without loss of generality. Thus from \eqref{Eq:rootcondition}, we have 
\begin{equation}
(\varphi_0^i)^{\alpha} =1,~~~ 0 \leq i < \epsilon_0
\label{novelEq5b}
\end{equation}
On the other hand, the order of $\varphi_0,~\epsilon_0$ is the least integer satisfying $\varphi_0^{\epsilon_0} = 1$, thus, $\alpha$ should satisfy the condition
\begin{equation}
\alpha \equiv 0 \mod \epsilon_0~~~\mathor~~~\epsilon_0|\alpha
\label{eq:wt2-alpha}
\end{equation}

\subsection{The weight-3 PCs}

Without loss of generality, theh weight-3 PCs can be written as 
\begin{equation}
h(x)=1+x^{\alpha}+x^{\beta},~\alpha < \beta
\label{novelEqwt3}
\end{equation}
and hence, $(\alpha,\beta)$ should satisfy the condition
\begin{equation}
\varphi_0^{\alpha}+\varphi_0^{\beta}= 1
\label{Eq:novelEq5b}
\end{equation}
The pairs satisfying \eqref{Eq:novelEq5b} can be found by referring to the table of the extended field for $\GF\left(2^M\right)$. 
Let $(m,n)$ be such a pair, and we let $\cM=\{\epsilon_0 \ell+m\}$ and $\cN=\{\epsilon_0 \ell+n\}$, $\ell \geq 0$. Then it is obvious that each pair $(\alpha,~\beta) \in \cM \otimes \cN$ satisfies
\eqref{Eq:novelEq5b}. For a fixed $\alpha$, on the other hand, since $\alpha+i$, $0 \leq i < \epsilon_0$, are distinct from each other, any integer $\beta$ that satisfies \eqref{Eq:novelEq5b} must be such that $n\equiv \beta \mod \epsilon_0$.

%Furthermore, let  We denote the set of all pairs as $\cM \bar{\otimes} \cN$, where the elements of this set are formed using a tensor product approach. Then, $(\alpha,~\beta) \in \cM \bar{\otimes} \cN$
\subsection{Examples}

\begin{example}$f(x)=1+x+x^2$

Since $x^1=x$, $x^2 \equiv 1+x \mod f(x)$, and $x^3 \equiv 1 \bmod f(x)$, $f(x)$ is a primitive polynomial with root $\varphi_0$ of order $\epsilon_0=3$. Thus, $\alpha$ in the weight-2 PCs $1+x^\alpha$ should be a multiple of $3$ and we may express such PCs as $h(x)=1+x^{3\ell}$, $\ell>1$, while the corresponding $a(x)$ is given by 
\begin{equation*}
	a(x)=\sum_{i=0}^{\ell-1} x^{3i}(1+x)
\end{equation*}
The detailed patterns of $a(x)$ and $h(x)$ for $1 \leq \ell \leq 4$ are listed in Table \ref{novelTab2}.
To determine the weight-3 PCs, we can see from Table \ref{novelTab7} that there is a pair $(1,2)$ satisfying $x^1+x^2 \equiv 1 \mod f(x)$. Thus, if we let $\cM = \{3\ell+1\}_{\ell\geq 0}$ and $\cN = \{3\ell+2\}_{\ell\geq 0}$, 
$x^\alpha+x^\beta \equiv 1 \mod f(x)$ for each pair $(\alpha,\beta) \in \cM\otimes\cN$. The PCs $h(x)=1+x^{3\ell_0+1}+x^{3\ell_1+2}$ for $0 \leq \ell_0,\ell_1 \leq 1$ and the corresponding $a(x)$ are listed in Table \ref{novelTab8}.


\label{ex-1}
\end{example}




\begin{example}
$f(x)=1+x+x^2+x^3+x^4$

Since the order of $\varphi_0$ is $\epsilon_0=5< 15$, $f(x)$ is a prime but not primitive polynomial. The weight-2 PCs $h(x)=1+x^{5\ell}$, $1 \leq \ell \leq 4$, and the corresponding $a(x)$ are shown in Table \ref{novelTab3}. For weight-3 PCs, on the other hand, we can see from Table \ref{novelTabWt3-2} that there is no pair $(m,~n)$ satisfying $x^m+x^n \equiv 1$, and hence, the given $f(x)$ does not yield weight-3 PCs.




\label{ex-2}
\end{example}

\begin{example}
	$f(x)=1+x^2$ and $f(x)=1+x^4$\newline
	
	If we rewrite the polynomials as $f(x)=(1+x)^2$ and $f(x)=(1+x)^4$,	the order of the root $\varphi_0$ is $\epsilon_0=1$. Since $\GF(2)$ has single non-zero element, it does not provide a pair $(m,~n)$ satisfying $x^m+x^n \equiv 1$ and there are no weight-3 PCs associated with $f(x)$.
	
	Regarding the weight-2 PCs, since $\epsilon_0 = 1$, each $\alpha \in \bbZ^+$ satisfies
	\begin{align}
		h(x) = 1+x^\alpha = 0
	\end{align}
	However, the following second order differential equation
	\begin{align}
	\frac{dh(x)}{dx} = (\alpha\mod 2) x^{\alpha-1} = 0
	\end{align}
	implies $\alpha$ should be even numbers. Thus, for the case $f(x) = 1+x^2$, we write the PCs as  $h(x)=1+x^{2\ell}$ and listed corresponding $a(x)$s in Table \ref{novelTab1}.
	

	For the case $f(x) = 1+x^4$, from \eqref{Eq:differential}, we have
	\begin{align}
		\begin{cases}
		\frac{d^2h(x)}{dx^2} = \left[\frac{\alpha(\alpha-1)}{2} \mod 2\right]x^{\alpha-2} = 0 \\
		\frac{d^3h(x)}{dx^3} = \left[\frac{\alpha(\alpha-1)(\alpha-2)}{6} \mod 2\right]x^{\alpha-3} = 0
		\end{cases}
		\label{Eq:Ex3Diff}
	\end{align}
\label{ex-3}
\end{example}
and $\alpha \in 4\bbZ^+$ satisfies \eqref{Eq:Ex3Diff} simultaneously. The $a(x)$s corresponding to $h(x)=1+x^{4\ell}$, $1 \leq \ell \leq 4$ are listed in Table \ref{Tb:novelTab2}.
%=====================Deleted Examples  =======================%
%\begin{example}
%$f(x)=1+x^2+x^3+x^4$\newline
%$f(x)$ can be written as 
%$$f(x)=\prod_{k=0}^{1}f_k(x)$$
%where 
%$$f_0(x)=1+x,~f_1(x)=1+x+x^3$$ 
%For $f_0(x), x \equiv 1$, and $x^1 \equiv 1 \bmod f_0(x)$, which means the order of the root $\varphi_0$ is $\epsilon_0=1$ and $\alpha_0$ should be a multiple of $1$. Again, for  $f_1(x), x^3 \equiv 1+x$, and $x^7 \equiv 1 \bmod f_1(x)$, which means the order of the root $\varphi_1$ is $\epsilon_1=7$ and $\alpha_1$ should be a multiple of $7$.
%Finally, valid values of $\alpha$ should be a multiple of the least common multiples of $\alpha_0$ and $\alpha_1$, which means $\alpha$ should be a multiple of $7$.
%The corresponding values for $a(x)$ and $h(x)$ are shown in Table \ref{novelTab1-a} with general forms for $\ell>1$.
%\begin{table}[htbp]
%\renewcommand{\arraystretch}{1.3}
%\parbox{.3\linewidth}{
 %\caption{$f(x)=1+x^2+x^3+x^4$}
 %\centering
%\begin{tabular}{c c } 
%\hline
 %$\alpha(x)=\sum_{\ell=0}^{L-1} x^{7\ell}(1+x^2+x^3)$ & $h(x)=1+x^{7\ell}$ \\ [0.5ex] 
%\hline\hline
%$1+x^2+x^3$ & $1+x^7$\\ 
%$1+x^2+x^3+x^7+x^9+x^{10}$ & $1+x^{14}$ \\
%$1+x^2+x^3+x^7+x^9+x^{10}+x^{14}+x^{16}+x^{17}$ & $1+x^{21}$
%\end{tabular}
% \label{novelTab1-a}
%\end{table}
%\end{example}

%\begin{example}
%$f(x)=1+x+x^2+x^3+x^4+x^5+x^6$\newline
%$f(x)$ can be written as 
%$$f(x)=\prod_{k=0}^{1}f_k(x)$$
%where 
%$$f_0(x)=1+x^2+x^3,~f_1(x)=1+x+x^3$$ 
%For $f_0(x), x^3 \equiv x^2+1$, and $x^7 \equiv 1 \bmod f_0(x)$, which means the order of the root $\varphi_0$ is $\epsilon_0=7$ and $\alpha_0$ should be a multiple of $7$. Again, for  $f_1(x), x^3 \equiv 1+x$, and $x^7 \equiv 1 \bmod f_1(x)$, which means the order of the root $\varphi_1$ is $\epsilon_1=7$ and $\alpha_1$ should be a multiple of $7$.
%Finally, valid values of $\alpha$ should be a multiple of the least common multiples of $\alpha_0$ and $\alpha_1$, which means $\alpha$ should be a multiple of $7$.
%The corresponding values for $a(x)$ and $h(x)$ are shown in Table \ref{novelTab1-b} with general forms for $\ell>1$.
%\begin{table}[htbp]
%\renewcommand{\arraystretch}{1.3}
%\parbox{.3\linewidth}{
% \caption{$f(x)=1+x+x^2+x^3+x^4+x^5+x^6$}
 %\centering
%\begin{tabular}{c c } 
%\hline
 %$\alpha(x)=\sum_{\ell=0}^{L-1} x^{7\ell}(1+x)$ & $h(x)=1+x^{7\ell}$ \\ [0.5ex] 
%\hline\hline
%$1+x$ & $1+x^7$\\ 
%$1+x+x^7+x^8$ & $1+x^{14}$ \\
%$1+x+x^7+x^8+x^{14}+x^{15}$ & $1+x^{21}$
%\end{tabular}
% \label{novelTab1-b}
%\end{table}
%\end{example}
%=====================================End of deleted examples========================%
\begin{example}
$f(x)=1+x^2+x^3+x^4+x^6$

$f(x)$ can be decomposed into the multiplication of two prime polynomials $f_0(x)=1+x+x^2$ and $f_1(x)=1+x+x^2+x^3+x^4$, and we know from Example \ref{ex-1} and Example \ref{ex-2} that $\epsilon_0=3$ and $\epsilon_1=5$, respectively. Thus, $\alpha$ for the weight-2 PCs $h(x) = 1+x^\alpha$ should be a multiple of $\epsilon_0$ and $\epsilon_1$, hence, a multiple of $\lcm(\epsilon_0,\epsilon_1)=15$, and there are no weight-3 PCs associated with $f(x)$ since $f_1(x)$ does not yield any weight-3 PC as shown in Example \ref{ex-2}. Two weight-2 PCs can be found from Table \ref{novelTab1-c} with corresponding $a(x)$s.



%We may write the weight-2 PCs in general form as $h(x)=1+x^{7\ell},~\ell>1$ and the corresponding $a(x)$ is given by 
%\begin{equation*}
%a(x)=\sum_{\ell=0}^{L-1} x^{7\ell}(1+x)
%\end{equation*}
\label{ex-4}
\end{example}

\begin{example}
$f(x)=1+x+x^5$

For this case, the polynomial can be decomposed into the multiplication of $f_0(x)=1+x+x^2$ and $f_1(x)=1+x^2+x^3$.  We know from Example \ref{ex-1} that $\epsilon_0=3$ and $\epsilon_1=7$. Similar to Example \ref{ex-4}, the weight-2 PCs have the general  form of $h(x)=1+x^{21\ell}$ while the corresponding $a(x)$ can be expressed as $\sum_{i=0}^{\ell-1} x^{21i}(1+x^2+x^3+x^4+x^6+x^8+x^{4}+x^{6}+x^{8}+x^{11}+x^{12}+x^{16})$.


In order to determine weight-3 PCs, we rewrite $\cM$ and $\cN$ in Example \ref{ex-1} as $\cM^0$ and $\cN^0$, respectively,
and referring to Table \ref{novelTabWt3-5}, let
\begin{align}
	\cM_0^1 &:= \{7\ell + 1\}_{\ell \geq 0},~\cN_0^1 := \{7\ell + 5\}_{\ell \geq 0}\cr
	\cM_1^1 &:= \{7\ell + 2\}_{\ell \geq 0},~\cN_1^1 := \{7\ell + 3\}_{\ell \geq 0}\cr
	\cM_2^1 &:= \{7\ell + 4\}_{\ell \geq 0},~\cN_2^1 := \{7\ell + 6\}_{\ell \geq 0}
\end{align}



Then, we have
\[
(\alpha_0,~\beta_0) \in \cM^0\otimes\cN^0 
\]
and
\begin{equation*}
\begin{split}
(\alpha_1,~\beta_1) \in &\bigcup_{i=0}^2 \cM_i^1\otimes\cN_i^1 
\end{split}
\end{equation*}
Therefore, by taking the intersection, we can identify $(\alpha,\beta) \in \left(\cM^0\otimes\cN^0\right)\bigcap\left(\bigcup_{i=0}^2 \cM_i^1\otimes\cN_i^1 \right) $. As an example, we listed three valid $h(x)=1+x^\alpha+x^\beta$ in Table \ref{novelTab8-b} with corresponding $a(x)$s.
\label{ex-5}
\end{example}

%=====Merged Example Tables =====
\begin{table}[htbp]
\caption{$a(x)$ and $h(x)$ for various $f(x)$}
\centering
\begin{tabularx}{0.7\textwidth}{l|cXl} 
\toprule
$f(x)$ & weight & $a(x)$ & $h(x)$\\
\midrule
 &  & $1+x$ & $1+x^{3}$ \\
&  &$1+x+x^3+x^4$ & $1+x^{6}$  \\
& 2 &$1+x+x^3+x^4+x^6+x^{7}$ &  $1+x^{9}$ \\
$1+x+x^2$&& $1+x+x^3+x^4+x^6+x^{7}+x^9+x^{10}$&  $1+x^{12}$ \\
\cline{2-4}
(Example 1)& &$1$ & $1+x+x^2$\\ 
& 3&$1+x+x^2$ &  $1+x^2+x^4$\\
& &$1+x+x^3$ & $1+x^4+x^5$\\
& &$1+x^2+x^3$ & $1+x+x^5$ \\
\cline{1-4}
&&$1+x$ &$1+x^5$\\ 
$1+x+x^2+x^3+x^4$&2&$1+x+x^5+x^6$ &$1+x^{10}$  \\
(Example 2)&&$1+x+x^5+x^6+x^{10}+x^{11}$ & $1+x^{15}$ \\
&&$1+x+x^5+x^6+x^{10}+x^{11}+x^{15}+x^{16}$ &$1+x^{20}$  \\
\cline{1-4}
&&$1$ & $1+x^2$\\ 
$1+x^2$&2&$1+x^2$ & $1+x^4$ \\
(Example 3)&&$1+x^2+x^4$ & $1+x^6$\\
&&$1+x^2+x^4+x^6$ & $1+x^8$\\
\cline{1-4}
&&$1$ & $1+x^4$\\ 
$1+x^4$&2&$1+x^4$ & $1+x^8$ \\
(Example 3)&&$1+x^4+x^8$ & $1+x^{12}$\\
&&$1+x^4+x^8+x^12$ & $1+x^{16}$\\
\cline{1-4}
$1+x^2+x^3+x^4+x^6$&2&$1+x^2+x^3+x^6+x^7+x^9$ & $1+x^{15}$\\ 
(Example 4)&&$1+x^2+x^3+x^6+x^7+x^9+x^{15}+x^{17}+x^{18}+x^{21}+x^{22}+x^{24}$ & $1+x^{30}$ \\
\cline{1-4}
&2&$1+x^2+x^3+x^4+x^6+x^8+x^{4}+x^{6}+x^{8}+x^{11}+x^{12}+x^{16}$ &  $1+x^{21}$\\
\cline{2-4}
$1+x+x^5$&&$1$ & $1+x+x^{5}$\\ 
(Example 5)&3&$1+x+x^5$ &  $1+x^2+x^{10}$\\
&&$1+x+x^2+x^3+x^4+x^{6}+x^{8}$ & $1+x^{11}+x^{13}$\\
\cline{1-4}
$1+x^2+x^3+x^4$&&$1+x+x^2$& $1+x^{7}$ \\
(Example 7)&2&$1+x^2+x^3+x^7+x^9+x^{10}$	& $1+x^{14}$ \\
\bottomrule
\end{tabularx}
\label{examples-table}
\end{table}
%=====End Of Merged Example Tables =====






%===== Merged GF(2^M) tables=======
\begin{table}[htbp]
\caption{Galois Field Elements for various prime polynomials $f(x)$}
\centering
\begin{tabularx}{0.7\textwidth}{X| c|c|c|c} 
\toprule
\multicolumn{1}{c|}{Power Representation}&\multicolumn{4}{c}{polynomial representation}\\
\cmidrule(lr){1-5}
&$1+x+x^2$&$1+x^2+x^3$&$1+x+x^2+x^3+x^4$&$1+x+x^4$\\
\cmidrule(lr){1-5}
$x^0$			&1			&1				&1                                                                   &1\\
$x$			&$x$		&$x$			&$x$                                                              &$x$\\
$x^{2}$		&$1+x$	&$x^2$		&$x^2$                                                  	  &$x^2$\\
$x^{3}$		&			&$1+x^2$		&$x^3$                                                        &$x^3$\\
$x^{4}$ 		& 			&$1+x+x^2$   &$1+x+x^2+x^3$ 					 	& $1+x$\\
$x^{5}$		&			&$1+x$		&                                                                   &$x+x^2$\\
$x^{6}$		&			&$x+x^2$		&                                                                   &$x^2+x^3$\\
$x^{7}$		&			&				&                								&$1+x+x^3$\\
$x^{8}$		&			&				&										&$1+x^2$\\
$x^{9}$		&			&				&										&$x+x^3$\\
$x^{10}$		&			&				&										&$1+x+x^2$\\
$x^{11}$		&			&				&										&$x+x^2+x^3$\\
$x^{12}$		&			&				&										&$1+x+x^2+x^3$\\
$x^{13}$		&			&				&										&$1+x+x^3$\\
$x^{14}$		&			&				&										&$1+x^3$\\
\bottomrule
\end{tabularx}

\label{GF-tables}
\end{table}

%=====End of Merged GF(2^M) tables=======
 







