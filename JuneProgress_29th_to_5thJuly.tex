\documentclass[11pt, oneside, dvipdfmx]{book}
\newcommand{\folder}{/usr/local/share/texmf}
%\newcommand{\folder}{/home/hanchenggao/Documents/texmf}
\input{\folder/hfiles/ebook}
\usepackage {graphicx}
\usepackage {graphics}
%\setCJKmainfont{SimSun}
\title{``
Progress So Far'' }
\author{Kwame Ackah Bohulu}
\date{\today}
\begin{document}

\maketitle
\section{Notation}
\begin{enumerate}
\item RTZ (Return-To-Zero) input :- A RTZ input is a binary input which causes a RSC encoder's final state to be return to zero after it has exited the zero state.

\item $\tau$ :- cycle length of the RSC encoder. For the $5/7$ RSC encoder $\tau = 3$

\item $N$ :- Interleaver length. 

\item $\cN$:- Integer set of $\{0,1,\cdots,N-1\}$

\item $\bbN$: Indexed set  of $\{0,1,\cdots,N-1\}$ in the natural order.

\item We assume that $N/\tau=C$

\item $\cC$ and $\bbC$ are definded in a similar manner.

\item $\cC^{t}:=\left\{c+t\right\}_{c \in \cC}$ and $\bbC^t$ is the indexed set with the elements of $\cC^t$ where  $t=(0,1,...,\tau-1)$. Where it becomes necessary to distinguish between the elements of $\cC^{t}$ and $\bbC^{t}$, we will write the elements of $\bbC^{t}$ as $c_{x'}^{t'}$ and the elements of $\cC^{t}$ as $c_x^{t}$

\item Permutation matrix 
\begin{equation*}
\bPi = \begin{bmatrix}
\bpi^0\cr
\bpi^1\cr
\vdots\cr
\bpi^{K-1}
\end{bmatrix}
= \begin{bmatrix}
\bpi_0 , \bpi_1,\cdots,\bpi_{\tau-1}
\end{bmatrix}
= \begin{bmatrix}
\pi_{t}^{(i)}
\end{bmatrix}_{i=0,~t=0}^{K-1,~\tau -1}
\end{equation*}
where $\pi_{t}^{(i)} \in \{0,1,\tau-1\}$. 

\item For the row vector $\bpi^{(i)}$, let $\mathscr{S}^e[\bpi^{(i)}]$ be the left-hand cycle shift of $\bpi^{(i)}$ and $\mathscr{S}^e[\bpi_t]$ be the up cycle shift of $\bpi_t$
\item We assume that the operation outputs the elements in $\bbC^t$ in order while $t$ is appeared in $\bpi^k$. For example, $\bpi^0 = (0,0,1)$ outputs $(c_0^0,c_1^0,c_0^1)$. From this example, we can see that the column index of $i$ in $\pi^{(i)}$ represents the coset it belongs to before interleaving and the value $\pi_{j}^{(i)}$ specifies the coset after interleaving
\item Our goal is to find a prefer $\bPi$ and $\bbC^t$, $t = 0,1,\cdots,\tau-1$.
\end{enumerate}


\section{Cosets and RTZ inputs}

\begin{enumerate}
\item a weight $2$ input sequence
\begin{itemize}
	\item polynomial: $P(x)=x^{h\tau+t}(1+x^{\alpha \tau}) = x^t(x^{h\tau}+x^{(h+\alpha)\tau})$
	\item coset: the $h$th and $(h+\alpha)$th elements in $\bbC^t$
\end{itemize}
\item a weight $3$ input sequence
\begin{itemize}
	\item polynomial: $Q(x) =x^{h\tau+t}(1+x^{\beta \tau +1}+x^{\gamma \tau +2})=x^{h\tau+t}+x^{(h+\beta) \tau +t+1}+x^{(h+\gamma) \tau +t+2}$. 
	Notice that $h \leq \beta$ is not a necessary condition.
	\item coset: the $h$th element in $\bbC^{t}$, $(h+\beta)$th element in $\bbC^{[t+1]_\tau}$, and $(h+\gamma)$th element in $\bbC^{[t+2]_\tau}$.
\end{itemize}
\end{enumerate}

\section{Representation of interleaver}
If the mapping relationship between elements in $\bx$ and $\by$ are read column wise as shown below

$$  
 \begin{bmatrix}
0 & 1 & 2 & 3 & 4 & 5 & 6 & 7 & 8 \\
0 & 5 & 1 & 6 & 2 & 7 & 3 & 8 & 4 \\
\end{bmatrix}
$$
the interleaver is represented by $\bbN=\{0,5,1,6,2,7,3,8,4\}$.

Let $\bbC^0=\{0,6,3\}$, $\bbC^1=\{1,7,4\}$, and $\bbC^2=\{5,2,8\}$. Then, the permutation matrix of $\bbN$ is
$\bPi = (0,2,1)$. Notice the row of $\bPi$ takes cyclicly.


\section{Coset Interleaver Design For Weight-$2$ RTZ inputs}
From the definition of Weight-$2$ RTZ inputs in the previous section, we know that the index of the ``1'' bits are in the same coset. Our aim is to make sure that the interleaver that we design is either able to break such weight-$2$ RTZ inputs or convert it into a large separation weight-$2$ RTZ. 
The condition to break weight-$2$ RTZs is given as

\begin{equation}
\pi_{j}^{(i)} \neq \pi_{j}^{(i')},~|i-i'| \leq N_c
\label{eq1}
\end{equation}

Since $\bPi$ consisting of $\tau$ elements, the maximum length of column elements consisting of values different each other is $\tau$. Thus, the cut-off interleaver length for which (\ref{eq1}) is satisfied is $N_c=\tau=3$.
For this interleaver length, we investigate 3 different compositions of permutation matrices that can be used to achieve this condition in in \ref{eq1}

\begin{enumerate}
	\item One cycle permutation: Each row is permutation of the sequence $(0,1,2)$. Setting the element at the first row and first column to $0$, there are exactly 4 permutation matrices that exist for cut-off length $N_c$.
	Let
	\begin{equation*}
	\bpsi=\begin{bmatrix} 0\cr 1\cr 2\cr \end{bmatrix},~
	\bpsi'=\begin{bmatrix} 0\cr 2\cr 1\cr \end{bmatrix}
	\end{equation*}
We then have 
	\begin{equation}
	\begin{split}
	[\bpsi,\mathscr{S}^1[\bpsi],\mathscr{S}^2[\bpsi]]=
	&
	\begin{bmatrix}
	0 & 1 & 2\cr
	1 & 2 & 0\cr
	2 & 0 & 1
	\end{bmatrix}:=\bpsi(\bpsi) \\
	[\bpsi',\mathscr{S}^1[\bpsi'],\mathscr{S}^2[\bpsi']]=
	&
	\begin{bmatrix}
	0 & 1 & 2\cr
	2 & 0 & 1\cr
	1 & 2 & 0
	\end{bmatrix}:=\bpsi(\bpsi')\\
	[\bpsi,\mathscr{S}^2[\bpsi],\mathscr{S}^1[\bpsi]]=
	&
	\begin{bmatrix}
	0 & 2 & 1\cr
	2 & 1 & 0\cr
	1 & 0 & 2
	\end{bmatrix}:=\bpsi'(\bpsi)\\
	[\bpsi',\mathscr{S}^2[\bpsi'],\mathscr{S}^1[\bpsi']]=
	&
	\begin{bmatrix}
	0 & 2 & 1\cr
	1 & 0 & 2\cr
	2 & 1 & 0
	\end{bmatrix}:=\bpsi'(\bpsi')\\
	\end{split}
	\end{equation}
	
	%{\bf find all such matrices.}
	
	\item Two cycle permutation: Two rows are permutation of the sequence $(0,0,1,1,2,2)$.
	
	There are no permutation matrices that satisfying cut-off length $N_c $. This is because the sequence length is not divisible by $N_c$, there will always be 2 elements of the same value in each row of $\bPi$
	
	
	\item Three cycle permutation: Three rows are permutation of the sequence$(0,0,0,1,1,1,2,2,2)$. 
	
	Example of the permutation matrices satisfying cut-off length $N_c = 9$ are shown in \ref{tb1}
	
	
\end{enumerate}





Table \ref{tb1} shows all unique coset interleaving arrays of length $N_c$ that convert weight-$2$ RTZ inputs to non-RTZ inputs. They are labeled from $A$ to $X$. A coset interleaving array is unique if a shift of the elements in the array does not produce another another coset interleaving array.

\begin{table}[h!]
\centering
\begin{tabular}{|c || c | c|| c|c || c | c|| c|} 
 \hline
 $A$ & $ \begin{bmatrix}0 & 0 & 0\cr1 & 1 & 1\cr2 & 2 & 2\end{bmatrix}$ 
 &
  $B$ & $\begin{bmatrix} 0 & 0 & 0 \cr 1 & 1 & 2 \cr 2 & 2 & 1\end{bmatrix}$ 
  &
  $C$ &$\begin{bmatrix} 0 & 0 & 0 \cr 1 & 2 & 1 \cr 2 & 1 & 2\end{bmatrix}$
  &
  $D$ & $\begin{bmatrix}0 & 0 & 0 \cr 1 & 2 & 2 \cr 2 & 1 & 1\end{bmatrix}$\\
 \hline
  $E$ & $\begin{bmatrix}0 & 0 & 0 \cr 2 & 1 & 1 \cr 1 & 2 & 2\end{bmatrix}$ 
 &
 $F$ & $\begin{bmatrix}0 & 0 & 0 \cr 2 & 1 & 2 \cr 1 & 2 & 1\end{bmatrix}$ 
 &
  $G$ & $\begin{bmatrix}0 & 0 & 0 \cr 2 & 2 & 1 \cr 1 & 1 & 2\end{bmatrix}$ 
 &
  $H$ & $\begin{bmatrix} 0 & 0 & 0\cr 2 & 2 & 2\cr 1 & 1 & 1\end{bmatrix}$\\ 
 \hline
   $I$ & $\begin{bmatrix} 0 & 0 & 1 \cr 1 & 1 & 0 \cr 2 & 2 & 2\end{bmatrix}$
 &
  $J$ & $\begin{bmatrix}0 & 0 & 1 \cr 1 & 2 & 0 \cr 2 & 1 & 2\end{bmatrix}$ 
 &
  $K$ & $\begin{bmatrix}0 & 0 & 1 \cr 2 & 1 & 0 \cr 1 & 2 & 2 \end{bmatrix}$
 &
 $L$ & $\begin{bmatrix}0 & 0 & 1 \cr 2 & 2 & 0 \cr 1 & 1 & 2\end{bmatrix}$\\ 
 \hline
 $M$ & $\begin{bmatrix}0 & 0 & 2 \cr 1 & 1 & 0 \cr 2 & 2 & 1 \end{bmatrix}$
 &
  $N$ & $\begin{bmatrix}0 & 0 & 2 \cr 1 & 2 & 0 \cr 2 & 1 & 1 \end{bmatrix}$ 
 &
 $O$ & $\begin{bmatrix}0 & 0 & 2 \cr 2 & 1 & 0 \cr 1 & 2 & 1\end{bmatrix}$ 
&
 $P$ & $\begin{bmatrix}0 & 0 & 2 \cr 2 & 2 & 0 \cr 1 & 1 & 1\end{bmatrix}$\\ 
 \hline
 $Q$ & $\begin{bmatrix}0 & 1 & 0 \cr 1 & 0 & 1 \cr 2 & 2 & 2 \end{bmatrix}$
&
  $R$ & $\begin{bmatrix}0 & 1 & 0 \cr 1 & 0 & 2 \cr 2 & 2 & 1\end{bmatrix}$ 
&
 $S$ & $\begin{bmatrix}0 & 1 & 0 \cr 1 & 2 & 1 \cr 2 & 0 & 2 \end{bmatrix}$
 &
 $T$ & $\begin{bmatrix}0 & 1 & 0 \cr 2 & 0 & 1 \cr 1 & 2 & 2\end{bmatrix}$\\ 
 \hline
 $U$ & $\begin{bmatrix}0 & 1 & 0 \cr 2 & 0 & 2 \cr 1 & 2 & 1\end{bmatrix}$ 
 &
 $V$ & $\begin{bmatrix}0 & 1 & 0 \cr 2 & 2 & 1 \cr 1 & 0 & 2\end{bmatrix}$ 
 &
 $W$ & $\begin{bmatrix}0 & 1 & 1 \cr 1 & 2 & 0 \cr 2 & 0 & 2 \end{bmatrix}$
 &
 $X$ & $\begin{bmatrix}0 & 2 & 0 \cr 2 & 0 & 2 \cr 1 & 1 & 1\end{bmatrix}$\\ 
   \hline

  \end{tabular}
\caption{All unique coset interleaving arrays of length $N_c =9$ for weight-$2$ RTZ inputs}
\label{tb1}
\end{table}

The interleaver length used in turbo coding are way greater than $N_c$ and it is not possible to transform weight-$2$ RTZ inputs into non-RTZ inputs for all values of $i$. All is not lost however, since not all weight-$2$ RTZ inputs produce low-weight codewords. 
The formula for calculating the Hamming weight ($w_H$) of the Turbo codeword produced by a weight-$2$ RTZ input occuring in both component codes is given by[SunTakeshita] 
\begin{equation}
\begin{split}
w_H=&2+(2 + \frac{\Delta_\pi}{\tau} )w_0+ (2 + \frac{\Delta_{\pi'}}{\tau})w_0\\
=&6+\Big(\frac{\Delta_\pi+\Delta_{\pi'}}{\tau}\Big)w_0,~w_0=2
\end{split}
\label{eq3}
\end{equation}

For all the $\Pi$ in Table \ref{tb1}, since $\Delta \pi = 9=3\tau$ and $\Delta \pi':=(\pi^{(h'+\alpha')}_{t}-\pi^{(h')}_{t})$  we have
\begin{equation}
\begin{split}
w_H=&6+\Big(3+\frac{\Delta \pi'}{3}\Big)w_0,~w_0=2
\end{split}
\end{equation}


%This means that by altering the value of $t$ and $s$, we can increase the weight of the codeword produced. With this knowledge,
%all we need to do is to make sure that the if the input to the interleaver is a weight-$2$ RTZ inputs with a small value of $t$, it is converted to a weight-$2$ RTZ with a large value of $s$

%In summary, an interleaver designed to deal with weight-$2$ RTZ inputs when $N>N_c$ should
%\begin{enumerate}
%\item Convert  weight-$2$ RTZ inputs to non-RTZ inputs

%\item Convert  weight-$2$ RTZ inputs to a weight-$2$ RTZ inputs with a large value of $s$ when condition 1 isnt possible.

%\end{enumerate}

%An interleaver design method which makes use of the above points is as follows.
%Given an interlever with length $N$, we break it up into $\frac{N}{N_c}$ blocks each of length $N_c$. At the beginning of the $n$th block, the coset interleaving pattern is repeated until the last block. By applying this method, we make sure that when the weight-$2$ RTZ occurs within a  block, condition 1 is met and condition 2 is met when the weight-$2$ RTZ occurs in 2 consecutive blocks i.e. when  $t=N_c$.

%This method only works when $N_c | N$ and we will delay the details of what to do when $N_c \not| N$.

\section{Coset Interleaver Design For Weight-$3$ RTZ inputs}
As mentioned earlier, a weight-$3$ RTZ input is formed when the indices of the ``1'' bits each occur in different cosets.  It goes without saying that the simplest way to convert a weight-$3$ RTZ input into a non-RTZ input is to make sure that at least two of indices of the ``1'' bits occur within the same coset after interleaving.
\begin{equation}
w_H=
7+2(l+l') 
\label{eq6}
\end{equation}



Unique permutation matrices which meet this criteria are shown in Table \ref{tb2} and they are labeled from $A$ to $L$

\begin{table}[h!]
\centering
\begin{tabular}{|c || c  |c  ||c  |} 
 \hline
 $A$ & $\begin{bmatrix} 0 & 0 & 0 \cr 1 & 1 & 1 \cr 2 & 2 & 2\end{bmatrix}$ 
  &
 $B$ & $\begin{bmatrix}0 & 0 & 0 \cr 1 & 1 & 2 \cr 1 & 2 & 2\end{bmatrix}$\\ 
 \hline
$C$ & $\begin{bmatrix}0 & 0 & 0 \cr 1 & 1 & 2 \cr 2 & 1 & 2\end{bmatrix}$ 
 &
$D$ & $\begin{bmatrix}0 & 0 & 0 \cr 1 & 1 & 2 \cr 2 & 2 & 1\end{bmatrix}$\\ 
 \hline
 $E$ & $\begin{bmatrix}0 & 0 & 0 \cr 2 & 2 & 1 \cr 1 & 1 & 2\end{bmatrix}$ 
 &
 $F$ & $\begin{bmatrix}0 & 0 & 0 \cr 2 & 2 & 1 \cr 1 & 2 & 1\end{bmatrix}$\\ 
 \hline
 $G$ & $\begin{bmatrix}0 & 0 & 0 \cr 2 & 2 & 1 \cr 2 & 1 & 1\end{bmatrix}$ 
 &
  $H$ & $\begin{bmatrix}0 & 0 & 1 \cr 0 & 1 & 1 \cr 2 & 2 & 2\end{bmatrix}$\\ 
 \hline
  $I$ & $\begin{bmatrix}0 & 0 & 1 \cr 1 & 1 & 2 \cr 2 & 0 & 2\end{bmatrix}$ 
 &
 $J$ & $\begin{bmatrix}0 & 0 & 2 \cr 0 & 2 & 2 \cr 1 & 1 & 1\end{bmatrix}$\\ 
 \hline
  $K$ & $\begin{bmatrix}0 & 0 & 2 \cr 2 & 2 & 1 \cr 1 & 0 & 1\end{bmatrix}$
 &
  $L$ & $\begin{bmatrix}0 & 1 & 0 \cr 1 & 1 & 2 \cr 2 & 0 & 2\end{bmatrix}$\\ 
 \hline
\end{tabular}
\caption{All unique permutation matrices of length $N_c =9$ for weight-$3$ RTZ inputs}
\label{tb2}
\end{table}

Depending on which permutation matrix is chosen from Table \ref{tb2}, Equation \ref{eq6} can be simplified. 

In general $w_H$ for turbo codewords as a result of weight-$3$ RTZ inputs can be written as $$w_H=3 + w_p+ w'_p$$, where $w_p,w'_p$ refer to the pre-interleaving parity weight and the post-interleaving parity weight respectively. The value of $w_p$ for the pre-interleaving weight-$3$ is dependent on the elements in $\cC^t$

Let $(\pi^{(h)}_{t},~c^{(h+\beta)}_{t+1},~c^{(h+\gamma)}_{t+2})$ be the vector representing a weight-$3$ RTZ input
%\begin{equation}
%\begin{split}
%c^{h}_{t}&=\alpha\tau+ t\\
%c^{h+\beta}_{t+1}&=(\alpha+\beta)\tau+ t+1\\
%c^{h+\gamma}_{t+2}&=(\alpha+\gamma)\tau+ t+2\\
%\end{split}
%\label{eq7}
%\end{equation}
Without loss of generality, we can assume that $h =t =0$. We then have 
\begin{equation}
l=\max{(\beta,\gamma)}
\label{eq8}
\end{equation}
And 
\begin{equation}
w_p=
2(\max{(\beta,\gamma)})+2
\label{eq9}
\end{equation}
By deciding on the $\Pi$ we can easily calculate all values of $l$ and $w_p$
$w'_p,\beta',\gamma' $ and $l'$ are similarly defined and are dependent on the elements in $\bbC^{t},~t=0,1,..,\tau-1$

As an example, Table \ref{tb3} shows all the weight-$3$ RTZ inputs and the corresponding equations for calculating $w_H$

\begin{table}
\centering
\begin{tabular}{||c |c  |c  |c |} 
 \hline
 RTZ index  & $ l$ & $w_p$& $w_H$\\
 \hline
 $(0~4~8)$ & $2$ & $6$ & $11+2(\max{(\beta',\gamma')})$\\
 \hline
 $(0~5~7)$ &  $2$ & $6$ &$11+2(\max{(\beta',\gamma')})$\\
 \hline
 $(1~3~8) \equiv (0~2~7)$ &  $2$ & $6$ & $11+2(\max{(\beta',\gamma')})$\\
 \hline
 $(1~5~6) \equiv(0~4~5)$ &  $1$ & $4$ & $9+2(\max{(\beta',\gamma')})$\\
 \hline
 $(2~3~7)\equiv(0~1~5)$ & $1$ & $4$ & $9+2(\max{(\beta',\gamma')})$\\
 \hline
 $(2~4~6)\equiv(0~2~4)$ & $1$ & $4$ & $9+2(\max{(\beta',\gamma')})$\\
 \hline\hline
  $(0~8~13)$ & $4$ & $10$ & $15+2(\max{(\beta',\gamma')})$\\
 \hline
 $(0~4~17)$ & $5$ & $12$ & $17+2(\max{(\beta',\gamma')})$\\
 \hline
 $(0~13~17)$ & $5$ & $12$ & $17+2(\max{(\beta',\gamma')})$\\
 \hline
  $(0~7~14)$ &  $4$ & $6$ &$15+2(\max{(\beta',\gamma')})$\\
 \hline
  $(0~5~16)$ &  $5$ & $6$ &$17+2(\max{(\beta',\gamma')})$\\
 \hline
  $(0~14~16)$ &  $5$ & $6$ &$17+2(\max{(\beta',\gamma')})$\\
 \hline
 $(1~8~12) \equiv (0~7~11)$ &  $3$ & $8$ & $13+2(\max{(\beta',\gamma')})$\\
 \hline
 $(1~3~17) \equiv (0~2~16)$ &  $5$ & $12$ & $17+2(\max{(\beta',\gamma')})$\\
 \hline
 $(1~12~17) \equiv (0~11~16)$ & $5$ & $12$ & $17+2(\max{(\beta',\gamma')})$\\
 \hline
  $(1~6~14) \equiv(0~5~13)$ &  $4$ & $10$ & $15+2(\max{(\beta',\gamma')})$\\
 \hline
  $(1~5~15) \equiv(0~4~14)$ &  $4$ & $10$ & $15+2(\max{(\beta',\gamma')})$\\
 \hline
  $(1~14~15) \equiv(0~13~14)$ &  $4$ & $10$ & $15+2(\max{(\beta',\gamma')})$\\
 \hline
 $(2~7~12)\equiv(0~5~10)$ & $3$ & $8$ & $13+2(\max{(\beta',\gamma')})$\\
 \hline
 $(2~3~16)\equiv(0~1~14)$ & $4$ & $10$ & $15+2(\max{(\beta',\gamma')})$\\
 \hline
 $(2~12~16)\equiv(0~10~14)$ & $4$ & $10$ & $15+2(\max{(\beta',\gamma')})$\\
 \hline
  $(2~6~13)\equiv(0~4~11)$ & $3$ & $8$ & $13+2(\max{(\beta',\gamma')})$\\
 \hline
  $(2~4~15)\equiv(0~2~13)$ & $4$ & $10$ & $15+2(\max{(\beta',\gamma')})$\\
 \hline
  $(2~13~15)\equiv(0~11~13)$ & $4$ & $10$ & $15+2(\max{(\beta',\gamma')})$\\
 \hline
\end{tabular}
\caption{All unique permutation matrices of length $N_c =9$ for weight-$3$ RTZ inputs}
\label{tb3}
\end{table}

\paragraph{To Do}
\begin{enumerate}
\item add elements for inter-block Weight-3 RTZ inputs
\end{enumerate}
 \end{document}