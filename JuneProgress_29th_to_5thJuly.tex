\documentclass[11pt, oneside, dvipdfmx]{book}
\newcommand{\folder}{/usr/local/share/texmf}
%\newcommand{\folder}{/home/hanchenggao/Documents/texmf}
\input{\folder/hfiles/ebook}
\usepackage {graphicx}
\usepackage {graphics}
%\setCJKmainfont{SimSun}
\title{``
Progress So Far'' }
\author{Kwame Ackah Bohulu}
\date{\today}
\begin{document}

\maketitle
\section{Notation}
\begin{enumerate}
\item RTZ (Return-To-Zero) input :- A RTZ input is a binary input which causes a RSC encoder's final state to be return to zero after it has exited the zero state.

\item $\tau$ :- cycle length of the RSC encoder. For the $5/7$ RSC encoder $\tau = 3$

\item $N$ :- Interleaver length. 

\item $\cN$:- Integer set of $\{0,1,\cdots,N-1\}$

\item $\bbN$: Indexed set  of $\{0,1,\cdots,N-1\}$ in the natural order.

\item We assume that $N/\tau=C$

\item $\cC$ and $\bbC$ are definded in a similar manner.

\item $\cC^{t}:=\left\{c+t\right\}_{c \in \cC}$ and $\bbC^t$ is the indexed set with the elements of $\cC^t$ where  $t=(0,1,...,\tau-1)$. Where it becomes necessary to distinguish between the elements of $\cC^{t}$ and $\bbC^{t}$, we will write the elements of $\bbC^{t}$ as $c_{x'}^{t'}$ and the elements of $\cC^{t}$ as $c_x^{t}$

\item Permutation matrix 
\begin{equation*}
\bPi = \begin{bmatrix}
\bpi^0\cr
\bpi^1\cr
\vdots\cr
\bpi^{K-1}
\end{bmatrix}
= \begin{bmatrix}
\bpi_0 , \bpi_1,\cdots,\bpi_{\tau-1}
\end{bmatrix}
= \begin{bmatrix}
\pi_{t}^{(i)}
\end{bmatrix}_{i=0,~t=0}^{K-1,~\tau -1}
\end{equation*}
where $\pi_{t}^{(i)} \in \{0,1,\tau-1\}$. 

\item For the row vector $\bpi^{(i)}$, let $\mathscr{S}^e[\bpi^{(i)}]$ be the left-hand cycle shift of $\bpi^{(i)}$ and $\mathscr{S}^e[\bpi_t]$ be the up cycle shift of $\bpi_t$
\item We assume that the operation outputs the elements in $\bbC^t$ in order while $t$ is appeared in $\bpi^k$. For example, $\bpi^0 = (0,0,1)$ outputs $(c_0^0,c_1^0,c_0^1)$. From this example, we can see that the column index of $i$ in $\pi^{(i)}$ represents the coset it belongs to before interleaving and the value $\pi_{j}^{(i)}$ specifies the coset after interleaving
\item Our goal is to find a prefer $\bPi$ and $\bbC^t$, $t = 0,1,\cdots,\tau-1$.
\end{enumerate}


\section{Cosets and RTZ inputs}

\begin{enumerate}
\item a weight $2$ input sequence
\begin{itemize}
	\item polynomial: $P(x)=x^{h\tau+t}(1+x^{\alpha \tau}) = x^t(x^{h\tau}+x^{(h+\alpha)\tau})$
	\item coset: the $h$th and $(h+\alpha)$th elements in $\bbC^t$
\end{itemize}
\item a weight $3$ input sequence
\begin{itemize}
	\item polynomial: $Q(x) =x^{h\tau+t}(1+x^{\beta \tau +1}+x^{\gamma \tau +2})=x^{h\tau+t}+x^{(h+\beta) \tau +t+1}+x^{(h+\gamma) \tau +t+2}$. 
	Notice that $h \leq \beta$ is not a necessary condition.
	\item coset: the $h$th element in $\bbC^{t}$, $(h+\beta)$th element in $\bbC^{[t+1]_\tau}$, and $(h+\gamma)$th element in $\bbC^{[t+2]_\tau}$.
\end{itemize}
\end{enumerate}

\section{Representation of interleaver}
If the mapping relationship between elements in $\bx$ and $\by$ are read column wise as shown below

$$  
 \begin{bmatrix}
0 & 1 & 2 & 3 & 4 & 5 & 6 & 7 & 8 \\
0 & 5 & 1 & 6 & 2 & 7 & 3 & 8 & 4 \\
\end{bmatrix}
$$
the interleaver is represented by $\bbN=\{0,5,1,6,2,7,3,8,4\}$.

Let $\bbC^0=\{0,6,3\}$, $\bbC^1=\{1,7,4\}$, and $\bbC^2=\{5,2,8\}$. Then, the permutation matrix of $\bbN$ is
$\bPi = (0,2,1)$. Notice the row of $\bPi$ takes cyclicly.


\section{Coset Interleaver Design For Weight-$2$ RTZ inputs}
From the definition of Weight-$2$ RTZ inputs in the previous section, we know that the index of the ``1'' bits are in the same coset. Our aim is to make sure that the interleaver that we design is either able to break such weight-$2$ RTZ inputs or convert it into a large separation weight-$2$ RTZ. 
The condition to break weight-$2$ RTZs is given as

\begin{equation}
\pi_{j}^{(i)} \neq \pi_{j}^{(i')},~|i-i'| \leq N_c
\label{eq1}
\end{equation}

Since $\bPi$ consisting of $\tau$ elements, the maximum length of column elements consisting of values different each other is $\tau$. Thus, the cut-off interleaver length for which (\ref{eq1}) is satisfied is $N_c=\tau=3$.
For this interleaver length, we investigate 3 different compositions of permutation matrices that can be used to achieve this condition in in \ref{eq1}

\begin{enumerate}
	\item One cycle permutation: Each row is permutation of the sequence $(0,1,2)$. Setting the element at the first row and first column to $0$, there are exactly 4 permutation matrices that exist for cut-off length $N_c$.
	Let
	\begin{equation*}
	\bpsi=\begin{bmatrix} 0\cr 1\cr 2\cr \end{bmatrix},~
	\bpsi'=\begin{bmatrix} 0\cr 2\cr 1\cr \end{bmatrix}
	\end{equation*}
We then have 
	\begin{equation}
	\begin{split}
	[\bpsi,\mathscr{S}^1[\bpsi],\mathscr{S}^2[\bpsi]]=
	&
	\begin{bmatrix}
	0 & 1 & 2\cr
	1 & 2 & 0\cr
	2 & 0 & 1
	\end{bmatrix}:=\bpsi(\bpsi) \\
	[\bpsi',\mathscr{S}^1[\bpsi'],\mathscr{S}^2[\bpsi']]=
	&
	\begin{bmatrix}
	0 & 1 & 2\cr
	2 & 0 & 1\cr
	1 & 2 & 0
	\end{bmatrix}:=\bpsi(\bpsi')\\
	[\bpsi,\mathscr{S}^2[\bpsi],\mathscr{S}^1[\bpsi]]=
	&
	\begin{bmatrix}
	0 & 2 & 1\cr
	2 & 1 & 0\cr
	1 & 0 & 2
	\end{bmatrix}:=\bpsi'(\bpsi)\\
	[\bpsi',\mathscr{S}^2[\bpsi'],\mathscr{S}^1[\bpsi']]=
	&
	\begin{bmatrix}
	0 & 2 & 1\cr
	1 & 0 & 2\cr
	2 & 1 & 0
	\end{bmatrix}:=\bpsi'(\bpsi')\\
	\end{split}
	\end{equation}
	
	%{\bf find all such matrices.}
	
	\item Two cycle permutation: Two rows are permutation of the sequence $(0,0,1,1,2,2)$.
	
	There are no permutation matrices that satisfying cut-off length $N_c $. This is because the sequence length is not divisible by $N_c$, there will always be 2 elements of the same value in each row of $\bPi$
	
	
	\item Three cycle permutation: Three rows are permutation of the sequence$(0,0,0,1,1,1,2,2,2)$. 
	
	Example of the permutation matrices satisfying cut-off length $N_c = 9$ are shown in \ref{tb1}
	
	
\end{enumerate}





Table \ref{tb1} shows all unique coset interleaving arrays of length $N_c$ that convert weight-$2$ RTZ inputs to non-RTZ inputs. They are labeled from $A$ to $X$. A coset interleaving array is unique if a shift of the elements in the array does not produce another another coset interleaving array.

\begin{table}[h!]
\centering
\begin{tabular}{|c || c | c|| c|c || c | c|| c|} 
 \hline
 $A$ & $ \begin{bmatrix}0 & 0 & 0\cr1 & 1 & 1\cr2 & 2 & 2\end{bmatrix}$ 
 &
  $B$ & $\begin{bmatrix} 0 & 0 & 0 \cr 1 & 1 & 2 \cr 2 & 2 & 1\end{bmatrix}$ 
  &
  $C$ &$\begin{bmatrix} 0 & 0 & 0 \cr 1 & 2 & 1 \cr 2 & 1 & 2\end{bmatrix}$
  &
  $D$ & $\begin{bmatrix}0 & 0 & 0 \cr 1 & 2 & 2 \cr 2 & 1 & 1\end{bmatrix}$\\
 \hline
  $E$ & $\begin{bmatrix}0 & 0 & 0 \cr 2 & 1 & 1 \cr 1 & 2 & 2\end{bmatrix}$ 
 &
 $F$ & $\begin{bmatrix}0 & 0 & 0 \cr 2 & 1 & 2 \cr 1 & 2 & 1\end{bmatrix}$ 
 &
  $G$ & $\begin{bmatrix}0 & 0 & 0 \cr 2 & 2 & 1 \cr 1 & 1 & 2\end{bmatrix}$ 
 &
  $H$ & $\begin{bmatrix} 0 & 0 & 0\cr 2 & 2 & 2\cr 1 & 1 & 1\end{bmatrix}$\\ 
 \hline
   $I$ & $\begin{bmatrix} 0 & 0 & 1 \cr 1 & 1 & 0 \cr 2 & 2 & 2\end{bmatrix}$
 &
  $J$ & $\begin{bmatrix}0 & 0 & 1 \cr 1 & 2 & 0 \cr 2 & 1 & 2\end{bmatrix}$ 
 &
  $K$ & $\begin{bmatrix}0 & 0 & 1 \cr 2 & 1 & 0 \cr 1 & 2 & 2 \end{bmatrix}$
 &
 $L$ & $\begin{bmatrix}0 & 0 & 1 \cr 2 & 2 & 0 \cr 1 & 1 & 2\end{bmatrix}$\\ 
 \hline
 $M$ & $\begin{bmatrix}0 & 0 & 2 \cr 1 & 1 & 0 \cr 2 & 2 & 1 \end{bmatrix}$
 &
  $N$ & $\begin{bmatrix}0 & 0 & 2 \cr 1 & 2 & 0 \cr 2 & 1 & 1 \end{bmatrix}$ 
 &
 $O$ & $\begin{bmatrix}0 & 0 & 2 \cr 2 & 1 & 0 \cr 1 & 2 & 1\end{bmatrix}$ 
&
 $P$ & $\begin{bmatrix}0 & 0 & 2 \cr 2 & 2 & 0 \cr 1 & 1 & 1\end{bmatrix}$\\ 
 \hline
 $Q$ & $\begin{bmatrix}0 & 1 & 0 \cr 1 & 0 & 1 \cr 2 & 2 & 2 \end{bmatrix}$
&
  $R$ & $\begin{bmatrix}0 & 1 & 0 \cr 1 & 0 & 2 \cr 2 & 2 & 1\end{bmatrix}$ 
&
 $S$ & $\begin{bmatrix}0 & 1 & 0 \cr 1 & 2 & 1 \cr 2 & 0 & 2 \end{bmatrix}$
 &
 $T$ & $\begin{bmatrix}0 & 1 & 0 \cr 2 & 0 & 1 \cr 1 & 2 & 2\end{bmatrix}$\\ 
 \hline
 $U$ & $\begin{bmatrix}0 & 1 & 0 \cr 2 & 0 & 2 \cr 1 & 2 & 1\end{bmatrix}$ 
 &
 $V$ & $\begin{bmatrix}0 & 1 & 0 \cr 2 & 2 & 1 \cr 1 & 0 & 2\end{bmatrix}$ 
 &
 $W$ & $\begin{bmatrix}0 & 1 & 1 \cr 1 & 2 & 0 \cr 2 & 0 & 2 \end{bmatrix}$
 &
 $X$ & $\begin{bmatrix}0 & 2 & 0 \cr 2 & 0 & 2 \cr 1 & 1 & 1\end{bmatrix}$\\ 
   \hline

  \end{tabular}
\caption{All unique coset interleaving arrays of length $N_c =9$ for weight-$2$ RTZ inputs}
\label{tb1}
\end{table}

The interleaver length used in turbo coding are way greater than $N_c$ and it is not possible to transform weight-$2$ RTZ inputs into non-RTZ inputs for all values of $i$. All is not lost however, since not all weight-$2$ RTZ inputs produce low-weight codewords. 
The formula for calculating the Hamming weight ($w_H$) of the Turbo codeword produced by a weight-$2$ RTZ input occuring in both component codes is given by[SunTakeshita] 
\begin{equation}
\begin{split}
w_H=&2+(2 + \frac{\Delta_c}{\tau} )w_0+ (2 + \frac{\Delta_{c'}}{\tau})w_0\\
=&6+\Big(\frac{\Delta_c+\Delta_{c'}}{\tau}\Big)w_0,~w_0=2
\end{split}
\label{eq3}
\end{equation}

For all the $\Pi$ in Table \ref{tb1}, since $\Delta_c = 9=3\tau$ and $\Delta_{c'}:=(c_{(h'+\alpha')}^{t}-c_{(h')}^{t})$  we have
\begin{equation}
\begin{split}
w_H=&6+\Big(3+\frac{\Delta_{c'}}{3}\Big)w_0,~w_0=2
\end{split}
\label{eq4}
\end{equation}


%This means that by altering the value of $t$ and $s$, we can increase the weight of the codeword produced. With this knowledge,
%all we need to do is to make sure that the if the input to the interleaver is a weight-$2$ RTZ inputs with a small value of $t$, it is converted to a weight-$2$ RTZ with a large value of $s$

%In summary, an interleaver designed to deal with weight-$2$ RTZ inputs when $N>N_c$ should
%\begin{enumerate}
%\item Convert  weight-$2$ RTZ inputs to non-RTZ inputs

%\item Convert  weight-$2$ RTZ inputs to a weight-$2$ RTZ inputs with a large value of $s$ when condition 1 isnt possible.

%\end{enumerate}

%An interleaver design method which makes use of the above points is as follows.
%Given an interlever with length $N$, we break it up into $\frac{N}{N_c}$ blocks each of length $N_c$. At the beginning of the $n$th block, the coset interleaving pattern is repeated until the last block. By applying this method, we make sure that when the weight-$2$ RTZ occurs within a  block, condition 1 is met and condition 2 is met when the weight-$2$ RTZ occurs in 2 consecutive blocks i.e. when  $t=N_c$.

%This method only works when $N_c | N$ and we will delay the details of what to do when $N_c \not| N$.

\section{Coset Interleaver Design For Weight-$3$ RTZ inputs}
As mentioned earlier, a weight-$3$ RTZ input is formed when the indices of the ``1'' bits each occur in different cosets.  It goes without saying that the simplest way to convert a weight-$3$ RTZ input into a non-RTZ input is to make sure that at least two of indices of the ``1'' bits occur within the same coset after interleaving.
\begin{equation}
w_H=
7+2(l+l') 
\label{eq6}
\end{equation}



Unique permutation matrices which meet this criteria are shown in Table \ref{tb2} and they are labeled from $A$ to $L$

\begin{table}[h!]
\centering
\begin{tabular}{|c || c  |c  ||c  |} 
 \hline
 $A$ & $\begin{bmatrix} 0 & 0 & 0 \cr 1 & 1 & 1 \cr 2 & 2 & 2\end{bmatrix}$ 
  &
 $B$ & $\begin{bmatrix}0 & 0 & 0 \cr 1 & 1 & 2 \cr 1 & 2 & 2\end{bmatrix}$\\ 
 \hline
$C$ & $\begin{bmatrix}0 & 0 & 0 \cr 1 & 1 & 2 \cr 2 & 1 & 2\end{bmatrix}$ 
 &
$D$ & $\begin{bmatrix}0 & 0 & 0 \cr 1 & 1 & 2 \cr 2 & 2 & 1\end{bmatrix}$\\ 
 \hline
 $E$ & $\begin{bmatrix}0 & 0 & 0 \cr 2 & 2 & 1 \cr 1 & 1 & 2\end{bmatrix}$ 
 &
 $F$ & $\begin{bmatrix}0 & 0 & 0 \cr 2 & 2 & 1 \cr 1 & 2 & 1\end{bmatrix}$\\ 
 \hline
 $G$ & $\begin{bmatrix}0 & 0 & 0 \cr 2 & 2 & 1 \cr 2 & 1 & 1\end{bmatrix}$ 
 &
  $H$ & $\begin{bmatrix}0 & 0 & 1 \cr 0 & 1 & 1 \cr 2 & 2 & 2\end{bmatrix}$\\ 
 \hline
  $I$ & $\begin{bmatrix}0 & 0 & 1 \cr 1 & 1 & 2 \cr 2 & 0 & 2\end{bmatrix}$ 
 &
 $J$ & $\begin{bmatrix}0 & 0 & 2 \cr 0 & 2 & 2 \cr 1 & 1 & 1\end{bmatrix}$\\ 
 \hline
  $K$ & $\begin{bmatrix}0 & 0 & 2 \cr 2 & 2 & 1 \cr 1 & 0 & 1\end{bmatrix}$
 &
  $L$ & $\begin{bmatrix}0 & 1 & 0 \cr 1 & 1 & 2 \cr 2 & 0 & 2\end{bmatrix}$\\ 
 \hline
\end{tabular}
\caption{All unique permutation matrices of length $N_c =9$ for weight-$3$ RTZ inputs}
\label{tb2}
\end{table}

Depending on which permutation matrix is chosen from Table \ref{tb2}, Equation \ref{eq6} can be simplified. 

In general $w_H$ for turbo codewords as a result of weight-$3$ RTZ inputs can be written as $$w_H=3 + w_p+ w'_p$$, where $w_p,w'_p$ refer to the pre-interleaving parity weight and the post-interleaving parity weight respectively. The value of $w_p$ for the pre-interleaving weight-$3$ is dependent on the elements in $\cC^t$

Let $(c_{(h)}^{t},~c_{(h+\beta)}^{t+1},~c_{(h+\gamma)}^{t+2})$ be the vector representing a weight-$3$ RTZ input
%\begin{equation}
%\begin{split}
%c^{h}_{t}&=\alpha\tau+ t\\
%c^{h+\beta}_{t+1}&=(\alpha+\beta)\tau+ t+1\\
%c^{h+\gamma}_{t+2}&=(\alpha+\gamma)\tau+ t+2\\
%\end{split}
%\label{eq7}
%\end{equation}
Without loss of generality, we can assume that $h =t =0$. We then have 
\begin{equation}
l=\max{(\beta,\gamma)}
\label{eq8}
\end{equation}
And 
\begin{equation}
w_p=
2(\max{(\beta,\gamma)})+2
\label{eq9}
\end{equation}
By deciding on the $\Pi$ we can easily calculate all values of $l$ and $w_p$
$w'_p,\beta',\gamma' $ and $l'$ are similarly defined and are dependent on the elements in $\bbC^{t},~t=0,1,..,\tau-1$

As an example, Table \ref{tb3} shows all the weight-$3$ RTZ inputs and the corresponding equations for calculating $w_H$

\begin{table}
\centering
\begin{tabular}{||c |c  |c  |c |} 
 \hline
 RTZ index  & $ l$ & $w_p$& $w_H$\\
 \hline
 $(0~4~8)$ & $2$ & $6$ & $11+2(\max{(\beta',\gamma')})$\\
 \hline
 $(0~5~7)$ &  $2$ & $6$ &$11+2(\max{(\beta',\gamma')})$\\
 \hline
 $(1~3~8) $ &  $2$ & $6$ & $11+2(\max{(\beta',\gamma')})$\\
 \hline
 $(1~5~6) $ &  $1$ & $4$ & $9+2(\max{(\beta',\gamma')})$\\
 \hline
 $(2~3~7)$ & $1$ & $4$ & $9+2(\max{(\beta',\gamma')})$\\
 \hline
 $(2~4~6)$ & $1$ & $4$ & $9+2(\max{(\beta',\gamma')})$\\
 \hline\hline
  $(0~8~13)$ & $4$ & $10$ & $15+2(\max{(\beta',\gamma')})$\\
 \hline
 $(0~4~17)$ & $5$ & $12$ & $17+2(\max{(\beta',\gamma')})$\\
 \hline
 $(0~13~17)$ & $5$ & $12$ & $17+2(\max{(\beta',\gamma')})$\\
 \hline
  $(0~7~14)$ &  $4$ & $6$ &$15+2(\max{(\beta',\gamma')})$\\
 \hline
  $(0~5~16)$ &  $5$ & $6$ &$17+2(\max{(\beta',\gamma')})$\\
 \hline
  $(0~14~16)$ &  $5$ & $6$ &$17+2(\max{(\beta',\gamma')})$\\
 \hline
 $(1~8~12) $ &  $3$ & $8$ & $13+2(\max{(\beta',\gamma')})$\\
 \hline
 $(1~3~17) $ &  $5$ & $12$ & $17+2(\max{(\beta',\gamma')})$\\
 \hline
 $(1~12~17) $ & $5$ & $12$ & $17+2(\max{(\beta',\gamma')})$\\
 \hline
  $(1~6~14) $ &  $4$ & $10$ & $15+2(\max{(\beta',\gamma')})$\\
 \hline
  $(1~5~15)$ &  $4$ & $10$ & $15+2(\max{(\beta',\gamma')})$\\
 \hline
  $(1~14~15)$ &  $4$ & $10$ & $15+2(\max{(\beta',\gamma')})$\\
 \hline
 $(2~7~12)$ & $3$ & $8$ & $13+2(\max{(\beta',\gamma')})$\\
 \hline
 $(2~3~16)$ & $4$ & $10$ & $15+2(\max{(\beta',\gamma')})$\\
 \hline
 $(2~12~16)$ & $4$ & $10$ & $15+2(\max{(\beta',\gamma')})$\\
 \hline
  $(2~6~13)$ & $3$ & $8$ & $13+2(\max{(\beta',\gamma')})$\\
 \hline
  $(2~4~15)$ & $4$ & $10$ & $15+2(\max{(\beta',\gamma')})$\\
 \hline
  $(2~13~15)$ & $4$ & $10$ & $15+2(\max{(\beta',\gamma')})$\\
 \hline
\end{tabular}
\caption{All unique permutation matrices of length $N_c =9$ for weight-$3$ RTZ inputs}
\label{tb3}
\end{table}

\section{Coset Design}
Once the permutation matrix is decided upon, we have the necessary constraints which will help us design $\bbC^t$ with respect to weight $2$ and weight $3$ RTZ inputs.
We settle on $$\begin{bmatrix} 0 & 0 & 0 \cr 1 & 1 & 1 \cr 2 & 2 & 2\end{bmatrix}$$ which we will refer to as $\bPi^{(0)}$. It was decided upon because in the design process, we only need to focus on only one of the cosets, say $\bbC^0$ and replicate the results for the remaining cosets. We will make use of the Almost Linear Interleaver(ALI) is the design of $\bbC^0$. 
Since we are dealing with a single coset, the $(D,L)$ ALI interleaver equation is given by $$\pi(h)=D \cdot h + \Big\lfloor \frac{h}{A} \Big\rfloor \bmod L$$ where $A=L/C,~L=N/3$ and $C=\text{gcd}(D,L)$ Also, $D$ is the period of the interleaver.

In our design, $\pi(h)$ will represent the index of the elements in $\bbC^0$ whiles the value of the element at $\pi(h)$ is $3\pi(h)$

For weight-$2$ RTZ inputs, the weight can be calculated as
$
w_H=6+\Big(3+\frac{\Delta_{c'}}{3}\Big)w_0,~w_0=2
$ 
where
$\Delta_{c'}:=(c_{(h'+\alpha')}^{t}-c_{(h')}^{t}),~\alpha' =3$. Where $h$ is the index of the element in $\bbC$
 With respect to the ALI, $c_{(h'+3)}^{t}=3(\pi(h+3))$ and $c_{(h')}^{t}=3(\pi(h))$.
 We therefore have 
 \begin{equation}
 \begin{split}
 w_H=&6+2\Big(3+\frac{|3(\pi(h+3))- 3(\pi(h))|}{3}\Big)\\
  w_H=&6+2\Big(3+|(\pi(h+3))- (\pi(h))|\Big)\\
 \end{split}
 \label{eq8}
 \end{equation}
 
 For weight $3$-RTZ inputs, the various equations are given in Table \ref{tb3}. The positions where a weight-$3$ inputs occur are know, but they are given with respect to the complete interleaver and need to be scaled down to a single coset. 
 Let $h,~h+\beta,~h+\gamma$ be the inputs representing where the weight-$3$ RTZ inputs occur due to $\bPi^{(0)}$. Then the scaled down versions will be $h^s,~(h+\beta)^s,~(h+\gamma)^s$ and are calculated using the equation
 $$f(x)= x \bmod 3 + 3\Big(\Big\lfloor\frac{x}{9} \Big\rfloor\Big)$$
 
 We feed $h^s,~h^s+\beta^s,~h^s+\gamma^s$ into the ALI and we get $\bs=(\pi(h^s),~\pi(h^s+\beta^s),~\pi(h^s+\gamma^s))$ and 
 
 $$l'=\max(\beta,\gamma)= \bs_{\max}-\bs_{\min}$$
 
 For each possible value of $h,~h+\beta,~h+\gamma$, $l'$ is calculated using the above process

Table \ref{4}  shows corresponding hamming weight for $\bbC$ designed using ALI(87,23)
\begin{table}[h!]
\centering
\begin{tabular}{||c |c  | c |c |} 
 \hline
 RTZ index & $ \bs$ & $ l'$ & $w_H$\\
 \hline
 $(0~4~8)$ & $(0~1~2)$& $46$ & $11+2(46)=103$\\
 \hline
 $(0~5~7)$ & $(0~2~1)$&  $46$  &$11+2(46)=103$\\
 \hline
 $(1~3~8) $ & $(1~0~2)$&  $46$  & $11+2(46)=103$\\
 \hline
 $(1~5~6) $ & $(1~2~0)$ &  $46$  & $9+2(46)=101$\\
 \hline
 $(2~3~7)$ & $(2~0~1)$ & $46$  & $9+2(46)=101$\\
 \hline
 $(2~4~6)$ & $(2~1~0)$ & $46$  & $9+2(46)=101$\\
 \hline\hline
 $(0~8~13)$ & $(0~2~4)$& $46$ & $15+2(46)=107$\\
 \hline
 $(0~7~14)$ & $(0~1~5)$&  $28$  &$15+2(28)=71$\\
 \hline
 $(1~8~12) $ & $(1~3~2)$&  $46$  & $13+2(46)=105$\\
 \hline
 $(1~6~14) $ & $(1~0~5)$ &  $28$  & $15+2(28)=71$\\
 \hline
 $(2~7~12)$ & $(2~1~3)$ & $46$  & $13+2(46)=105$\\
 \hline
 $(2~6~13)$ & $(2~0~4)$ & $46$  & $13+2(46)=105$\\
 \hline\hline
 $(0~4~17)$ & $(0~1~5)$& $46$ & $17+2(28)=73$\\
 \hline
 $(0~5~16)$ & $(0~2~4)$&  $28$  &$17+2(46)=109$\\
 \hline
 $(1~3~17) $ & $(1~0~5)$&  $46$  & $17+2(28)=73$\\
 \hline
 $(1~5~15) $ & $(1~2~3)$ &  $28$  & $15+2(46)=107$\\
 \hline
 $(2~3~16)$ & $(2~0~4)$ & $46$  & $15+2(46)=107$\\
 \hline
 $(2~4~15)$ & $(2~1~3)$ & $46$  & $15+2(46)=107$\\
 \hline\hline
 $(0~13~17)$ & $(0~4~5)$& $46$ & $17+2(28)=73$\\
 \hline
 $(0~14~16)$ & $(0~5~4)$&  $28$  &$17+2(28)=73$\\
 \hline
 $(1~12~17) $ & $(1~3~5)$&  $46$  & $17+2(46)=109$\\
 \hline
 $(1~14~15) $ & $(1~5~3)$ &  $28$  & $15+2(46)=107$\\
 \hline
 $(2~12~16)$ & $(2~3~4)$ & $46$  & $15+2(64)=143$\\
 \hline
 $(2~13~15)$ & $(2~4~3)$ & $46$  & $15+2(64)=143$\\
 \hline\hline
\end{tabular}
\caption{Hamming Weight for weight-$3$ RTZ using ALI(87,23)} and
\label{tb4}
\end{table}


\begin{table}[h!]
\centering
\begin{tabular}{||c | c|c ||} 
 \hline
h &  $\Delta_{c'}$ &$w_H$\\
 \hline\hline
 $0$ &$69$& $150$\\
 \hline
 $1$ & $18$ & $48$\\
 \hline
 $2$ & $18$ & $48$\\
 \hline
 $3$ & $69$ & $150$\\
 \hline
$4$ & $18$ & $48$\\
 \hline\hline
\end{tabular}
\caption{Hamming Weight for weight-$2$ RTZ using ALI(87,23) for different values of $h$}
\label{tb5}
\end{table}

\section{Simulation Results and Discussion}
Simulations for the coset interleaver with $N=261$ are done using the ALI($L,D$) interleaver. The values for $D=\{28,29,...,35\}~,L=N/3 =87$ and $$\Pi =\begin{bmatrix} 0 & 0 & 0 \cr 1 & 1 & 1 \cr 2 & 2 & 2\end{bmatrix}$$

The minimum weight for each interleaver with respect to weight-2 and weight 3 RTZs are shown in Table(\ref{tb6}) and the simulation results are shown in Figure()
\begin{table}[h!]
\centering
\begin{tabular}{||c | c|c ||} 
 \hline
D &  $w_H^{(2)}$ &$w_H^{(3)}$\\
 \hline\hline
 $28$ &$18$& $115$\\
 \hline
 $29$ & $14$ & $123$\\
 \hline
 $30$ & $18$ & $121$\\
 \hline
 $31$ & $24$ & $103$\\
 \hline
$32$ & $30$ & $89$\\
 \hline
$33$ & $36$ & $75$\\
 \hline
$34$ & $42$ & $61$\\
 \hline
$35$ & $48$ & $49$\\
 \hline\hline
\end{tabular}
\caption{Hamming Weight for weight-$2$ RTZ using ALI(87,23) for different values of $h$}
\label{tb6}
\end{table}
As expected, the interleaver designed with ALI$(87,29)$ performs the worst, but the performance of the other interleavers does not perform as expected of the dta from Table(). Further examination of the simulation results reveals that even though the other interleavers have high Hamming weight related to weight-$2$ and weight-$3$ RTZ inputs, the minimum distance of the code seem to be bound at by RTZ inputs with higher weights, specifically those of weight-$4$. Specifically weight-$4$ input of the form $(1+x^{\tau}) +x^{(\tau^2)}(1+x^{\tau})$ is transformed into weight-$4$ input of the form $(1+x)+x^{\tau}(1+x)$ after interleaving. In the end, the Hamming weight as a result of these combinations is $$w_H=w_m+w_p+w_{p'}=4+8+2=14$$ where $w_m$ is the weight of the RTZ-input. 

Specifically weight-$4$ inputs of the form $(1+x^{\tau}) +x^{\tau^2}(1+x^{\tau})$ are transformed into weight-$4$ inputs of the form $(1+x)+x^3(1+x)$ after interleaving because the start index for picking the elements for all the cosets are the same and these causes somewhat well separated pre-interleaving inputs to be bunched together post-interleaving inputs. This is shown in the graph of the input output relation of any of the interleavers.

This can be remedied by making sure that the start position for the other cosets are different and this can be accomplished by shifting the elements of the cosets other than $\bbC^0$ a certain value to the left. Let $a,b$ be the factor by which the $\bbC^1$ and $\bbC^2$ are shifted. The resulting Coset interleaver will be represented as CI$(D,a,b)$. We proceed to re-design  CI$(23,0,0)$  by introducing $a=30$ and $b=7$.
The simulation results for CI$(23,0,0)$ and CI$(23,30,7)$ are shown in Figure() as can be seen adjusting the start position of the other cosets, the error-correcting performance is greatly improved. A further investigation of the simulation results reveals that for CI$(23,30,7)$, the Hamming distance for the turbo code is again bounded by weight-$4$ RTZ inputs. A specific example is that the weight-$4$ input of the form $(1+x^{\tau})+(x^{155})(1+x^{\tau})$ is transformed into another weight-$4$ RTZ input of the form$ (1+x^{\tau})+(x^{244})(1+x^{\tau})$. Then $$w_H=w_m+w_p+w_{p'}=4+8+8=20$$. The multiplicity of such weight-$4$ RTZ inputs is extremely high. 

Since we know that a weight-$4$RTZ input is just a combination of 2 weight-$2$ RTZ inputs 

 \end{document}