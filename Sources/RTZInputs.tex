\section{RTZ Inputs}
in this section we talk a little bit more about the types of  RTZ inputs and introduce their polynomial and coset definitions. Finally we talk about how certain RTZ inputs may be dealt with after interleaving.

\subsection{Types of RTZ inputs}
Regardless of the component code used in turbo coding, the RTZ inputs can be grouped into two basic forms. These are \textit{base RTZ inputs} and \textit{compound RTZ inputs}. Base RTZ inputs are dependent on the component code and cannot be broken down into 2 or more RTZ inputs. Compound RTZ inputs as the name implies are formed from 2 or more base RTZ inputs and therefore can be broken down into base RTZ input form.

For the $5/7$ component code, its base RTZ inputs are weight-$2$ RTZ inputs  (W2RTZs) and weight-$3$ RTZ inputs (W3RTZs). Every RTZ input with a weight higher than 3 is a compound RTZ input.

The permutation matrix that generates the set $\cN$ is given by 

$$\Pi'=\begin{bmatrix} 0 & 1 & 2 \end{bmatrix}$$ 

where $\Pi$ was used repeatedly untill all elements in $\cC^t$ are picked. From $\Pi$ we can derive the defintions for W2RTZs and W3RTZs as well as ways to break up such RTZs

\subsection{W2RTZs : Definitions and Breaking them}
Given below is the definition of W2RTZs in terms of polynomials and cosets.
\begin{itemize}
	\item polynomial: $P(x)=x^{h\tau+t}(1+x^{\alpha \tau}) = x^t(x^{h\tau}+x^{(h+\alpha)\tau})$
	\item coset: the $h$th and $(h+\alpha)$th elements in $\cC^t$
\end{itemize}
%where $\alpha=1,2,\cdot,N-\alpha-h$

From the coset definition, it is easy to see that W2RTZs can be broken if after interleaving,  the $h$th and $(h+\alpha)$th elements in $\cC^t$ are mapped to different cosets.
%, which translates to the columns of $Pi$ (which generates $\bbN$) having different elements.

\subsection{W3RTZs : Definitions and Breaking them}
Given below is the definition of W3RTZs in terms of polynomials and cosets.
\begin{itemize}
	\item polynomial: $Q(x) =x^{h\tau+t}(1+x^{\beta \tau +1}+x^{\gamma \tau +2})=x^{h\tau+t}+x^{(h+\beta) \tau +t+1}+x^{(h+\gamma) \tau +t+2}$. 
	Notice that $h \leq \beta$ is not a necessary condition.
	\item coset: the $h$th element in $\cC^{t}$, $(h+\beta)$th element in $\cC^{[t+1]_\tau}$, and $(h+\gamma)$th element in $\cC^{[t+2]_\tau}$.
\end{itemize}

Again, from the coset definition, we see that the easiest way to break up W3RTZs is to make sure that after interleaving, a minimum of 2 elements are mapped into the same coset.

\subsection{W4RTZs : Definitions and Breaking them}
Given below is the definition of W4RTZs in terms of polynomials and cosets.
\begin{itemize}
	\item polynomial: $P(x)=x^{h\tau+t}(1+x^{\alpha \tau}) + x^{h'\tau+(t+i)}(1+x^{\alpha' \tau})= x^t(x^{h\tau}+x^{(h+\alpha)\tau}) + x^{t+i}(x^{h'\tau}+x^{(h'+\alpha')\tau})$
	\item coset: the $h$th and $(h+\alpha)$th elements in $\cC^t$ and  the $h'$th and $(h'+\alpha')$th elements in $\cC^{[t+i]_\tau}$ 
\end{itemize}
where $i=0,1,2$

There is not much that can be done to break up W4RTZs using just $\Pi$. However careful selection of $\Pi$ conbined with coset design can be used to effectively break up W4RTZs