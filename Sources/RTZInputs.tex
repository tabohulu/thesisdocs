\section{RTZ Inputs}
in this section we talk a little bit more about the types of  RTZ inputs and introduce their polynomial and coset definitions. Finally we talk about how certain RTZ inputs may be dealt with after interleaving.

\subsection{Types of RTZ inputs}
Regardless of the component code used in turbo coding, the RTZ inputs can be grouped into two basic forms. These are \textit{base RTZ inputs} and \textit{compound RTZ inputs}. Base RTZ inputs are dependent on the component code and cannot be broken down into 2 or more RTZ inputs. Compound RTZ inputs as the name implies are formed from 2 or more base RTZ inputs and therefore can be broken down into base RTZ input form.

For the $5/7$ component code, its base RTZ inputs are weight-$2$ RTZ inputs  (W2RTZs) and weight-$3$ RTZ inputs (W3RTZs). Every RTZ input with a weight higher than 3 is a compound RTZ input. In general for RTZs with weight $w$ greater than 3, if $w \bmod 2=0$, then the RTZ is made up of $w/2$ W2RTZs. On the other hand, if $w \bmod 2=1$, then the RTZ is made up of $\lfloor w/2 \rfloor -1$ W2RTZs and 1 W3RTZ.

The impulse response of the RSC encoder is the output of the encoder when the input is $\brho=(1 0 0 0 0 ...)$. The impulse response can be used to calculate the weight of any input sequence of weight $w$ in general. This is done by noting that any input sequence of weight $w$ is just a summation of $w$ $\brho$'s, where the consequetive $w-1$ $\brho$'s have leading zeros. 
For the 5/7 RSC encoder, the impulse response is given by $$(1 1 1 0 1 1 0 1 1 0 ...)$$
The permutation matrix that generates the set $\cN$ is given by 

$$\Pi'=\begin{bmatrix} 0 & 1 & 2 \end{bmatrix}$$ 
where $\Pi$ was used repeatedly untill all elements in $\cC^t$ are picked. From $\Pi$ we can derive the defintions for W2RTZs and W3RTZs as well as ways to break up such RTZs

\subsection{W2RTZs : Definitions and Breaking them}
Given below is the definition of W2RTZs in terms of polynomials and cosets.
\begin{itemize}
	\item polynomial: $P(x)=x^{h\tau+t}(1+x^{\alpha \tau}) = x^t(x^{h\tau}+x^{(h+\alpha)\tau})$
	\item coset: the $h$th and $(h+\alpha)$th elements in $\cC^t$
\end{itemize}
%where $\alpha=1,2,\cdot,N-\alpha-h$


From the impulse response,let $\bphi_2:=(0 1 1)$. The above definition for W2RTZs holds because for any W2RTZ defined like the above, you have $\bphi_2+\bphi_2=(0 0 0)$ after the second $1$ bit which is repeated infinitely leading to a low-weight codeword.


From the coset definition, it is easy to see that W2RTZs can be broken if after interleaving,  the $h$th and $(h+\alpha)$th elements in $\cC^t$ are mapped to different cosets.
%, which translates to the columns of $Pi$ (which generates $\bbN$) having different elements.

\subsection{W3RTZs : Definitions and Breaking them}
Given below is the definition of W3RTZs in terms of polynomials and cosets.
\begin{itemize}
	\item polynomial: $Q(x) =x^{h\tau+t}(1+x^{\beta \tau +1}+x^{\gamma \tau +2})=x^{h\tau+t}+x^{(h+\beta) \tau +t+1}+x^{(h+\gamma) \tau +t+2}$. 
	Notice that $h \leq \beta$ is not a necessary condition.
	\item coset: the $h$th element in $\cC^{t}$, $(h+\beta)$th element in $\cC^{[t+1]_\tau}$, and $(h+\gamma)$th element in $\cC^{[t+2]_\tau}$.
\end{itemize}

From the impulse response,let $\bphi_2:=(0 1 1),~\bphi_2':=(1 1 0),\bphi_2'':=(1 0 1)$. The above definition for W2RTZs holds because for any W2RTZ defined like the above, you have $\bphi_2+\bphi_2'+\bphi_2''=(0 0 0)$ after the third $1$ bit which is repeated infinitely leading to a low-weight codeword.

Again, from the coset definition, we see that the easiest way to break up W3RTZs is to make sure that after interleaving, a minimum of 2 elements are mapped into the same coset.

\subsection{W4RTZs : Definitions and Breaking them}
Given below is the definition of W4RTZs in terms of polynomials and cosets.
\begin{itemize}
	\item polynomial: $P(x)=x^{h\tau+t}(1+x^{\alpha_1 \tau}) + x^{h'\tau+t'}(1+x^{\alpha_2 \tau})= x^t(x^{h\tau}+x^{(h+\alpha_1)\tau}) + x^{t'}(x^{h'\tau}+x^{(h'+\alpha_2)\tau})$
	\item coset: the $h$th and $(h+\alpha_1)$th elements in $\cC^t$ and  the $h'$th and $(h'+\alpha_2)$th elements in $\cC^{t'}$ 
\end{itemize}
where $t,~t' \in\{0,1,2\},h \neq h'$

The above defintion holds because a W4RTZ is a combination of 2 W2RTZs

\subsection{W5RTZs : Definitions and Breaking them}
Given below is the definition of W5RTZs in terms of polynomials and cosets.
\begin{itemize}
	\item polynomial: $P(x)=x^{h\tau+t}(1+x^{\alpha \tau}) 
	+
	x^{h'\tau+t'}(1+x^{\beta \tau +1}+x^{\gamma \tau +2})= 
	x^t(x^{h\tau}+x^{(h+\alpha)\tau}) 
	+x^{h'\tau+t'}+x^{(h'+\beta) \tau +t'+1}+x^{(h'+\gamma) \tau +t'+2}
	$
	\item coset: the $h$th and $(h+\alpha)$th elements in $\cC^t$ and  the 
	$h'$th element in $\cC^{t'}$, $(h'+\beta)$th element in $\cC^{[t'+1]_\tau}$, and $(h'+\gamma)$th element in $\cC^{[t'+2]_\tau}$
\end{itemize}
where $t,~t' \in\{0,1,2\},h \neq h'$

The above defintion holds because a W5RTZ is a combination of a W2RTZ and a W3RTZ

There is not much that can be done to break up W4RTZs or W5RTZs using just $\Pi$. However careful selection of $\Pi$ conbined with coset design can be used to effectively break up these higher weight RTZs. Permutation matrix design will be focused solely on W2RTZs and W3RTZs

\subsection{Calculating Codeword Weights For Turbo Codes}
Low weight turbo codewords will occur only if an RTZ input is fed into both component encoders. We will give (or derive) the equations that can be used to calculate the codeword weight for turbo codes generated as a result of the above defined RTZs being fed into both component codes.

\subsubsection{Hamming Weight for W2RTZ Turbo Codewords }
We modify the equation given in [Sun Takeshita] to fit our notation. The Hamming weight $w^{(2)}_H$ for a turbo codeword generated by a W2RTZ is calculated using the equation below 
\begin{equation}
\begin{split}
w^{(2)}_H=&6+2\Big(\frac{\alpha \tau}{\tau}+\frac{\alpha'\tau}{\tau}\Big)\\
=& 6 +2(\alpha + \alpha')
\end{split}
\label{RTZinputs-1}
\end{equation}
where $\alpha'$ is the value of $\alpha$ after interleaving

\subsubsection{Hamming Weight for W3RTZ Turbo Codewords }
The equation for the Hamming weight $w^{(3)}_H$ for a turbo codeword generated by a W3RTZ is given by
\begin{equation}
7+2(l+l')
\label{RTZInputs-2}
\end{equation}
%=============proof begins=====================
\begin{proof}
The polynomial representation of a weight-$3$ RTZ input is given by $$Q(x) =x^{h\tau+t}(1+x^{\beta \tau +1}+x^{\gamma \tau +2})$$
With reference to the impulse response of the 5/7  RSC encoder, 

Let \\$\bphi_1=(0~0~1),~\bphi'_1=(0~1~0),~\bphi''_1=(1~0~0)$, \\
$\bphi_2=(0~1~1),~\bphi'_2=(1~1~0),~\bphi''_2=(1~0~1)$, \\
$\bphi_3=(1~1~1)$. 

Now, we consider the weight of the vector derived by the sumation of the followings vectors.
\begin{eqnarray*}
(\bzero_{3(\gamma+h)}~\bphi_1~\bphi'_2~\cdots)\cr
(\bzero_{3(\beta+h)}~\bphi_2~\bphi''_2~\cdots)\cr
(\bzero_{3 h}~~~~~~\bphi_3~\bphi_2~\cdots)
\end{eqnarray*}

Without loss of generality, we can assume that all weight-$3$ RTZ inputs begin at the $0$th position, ie $h=t=0$. This is because the case where $h>0$ or $t>0$ is just a right-shifted version of the weight-$3$ RTZ. With this assumption, we we only need to consider cases where $h= 0,~\gamma \geq h$.
To simplify calculation, we have included an addition table for all the vectors which is shown in Table \ref{tb1}

\begin{table}[h!]
\centering
\begin{tabular}{c || c  | c  | c  | c  | c  | c  | c } 
 $$ & $\bphi_1$ & $\bphi'_1$ & $\bphi''_1$ & $\bphi_2$ & $\bphi'_2$ & $\bphi''_2$ & $\bphi_3$ \\
   \hline\hline
   %row1
$\bphi_1$ & $\bzero_3$ & $-$ & $-$ & $-$ & $-$ & $-$ & $-$ \\
   \hline
      %row2
$\bphi'_1$ & $\bphi_2$ & $\bzero_3$ & $-$ & $-$ & $-$ & $-$ & $-$ \\
   \hline
      %row3
$\bphi''_1$ & $\bphi''_2$ & $\bphi'_2$ & $\bzero_3$ & $-$ & $-$ & $-$ & $-$ \\
   \hline
      %row4
$\bphi_2$ & $\bphi'_1$ & $\bphi_1$ & $\bphi_3$ & $\bzero_3$ & $-$ & $-$ & $-$ \\
   \hline
      %row5
$\bphi'_2$ & $\bphi_3$ & $\bphi''_1$ & $\bphi'_1$ & $\bphi''_2$ & $\bzero_3$ & $-$ & $-$ \\
   \hline
      %row6
$\bphi''_2$ & $\bphi''_1$ & $\bphi_3$ & $\bphi_1$ & $\bphi'_2$ & $\bphi_2$ & $\bzero_3$ & $-$ \\
   \hline
      %row7
$\bphi_3$ & $\bphi'_2$ & $\bphi''_2$ & $\bphi_2$ & $\bphi''_1$ & $\bphi_1$ & $\bphi'_1$ & $\bzero_3$ \\
   \hline
  \end{tabular}
\caption{Truth Table}
\label{tb1}
\end{table}
Furthermore, we consider 4 general cases for all possible values of $i,j,k$ where $i \geq k$ These cases are $(=~=),~(=~<),~(<~=)$ and $(<~<)$
\paragraph{Case 0: $\gamma=\beta=h$ \newline}

 For this case, the vectors to sum will be 
 \begin{align*}
(\bphi_1~\bphi'_2~\cdots)\\
(\bphi_2~\bphi''_2~\cdots)\\
(\bphi_3~\bphi_2~\cdots)\\
\cline{1-2}
(\bphi''_2~\bzero_{3}~\cdots)
\end{align*}
 
and  the derived vector will be $(\bphi''_2~\bzero_{3}~\cdots)$ with a weight of $w_p=2$
 
 %========case =  < ===========
 
 %\paragraph{Case 1a: $i=j<k$\newline}
 %vector to sum:
 %\begin{align*}
 %(\bzero_{3}~\cdots~\bzero_{3}~\bphi_1~\bphi'_2~\cdots~\bphi_2'~\bphi_2'~\bphi_2'~\cdots)\\
% (\bzero_{3}~\cdots~\bzero_{3}~\bphi_2~\bphi''_2~\cdots~\bphi''_2~\bphi''_2\bphi''_2~\cdots)\\
%+(\bzero_{3}~~\cdots~\cdots~\cdots~\cdots~\bzero_{3}~\bphi_3~\bphi_2~\cdots)\\
%\cline{1-2}
%(\bzero_{3}~\cdots~\bzero_{3}~\bphi'_1~\bphi_2~\cdots~\bphi_2~\bphi''_1~\bzero_3~\cdots)
%\end{align*}
%derived vector : $(\bzero_{3j}~\bphi'_1~(\bphi_2)_{k-j-1}~\bphi''_1~\bzero_3~\cdots)$
%\newline
%Parity weight: \begin{equation}
%\begin{split}
%w_p=2(k-j)
%\end{split}
%\end{equation}

\paragraph{Case 1a: $\gamma=h<\beta$ \newline}
 vectors to sum:
 \begin{align*}
(\bphi_1~\bphi'_2~\bphi'_2~\bphi'_2~\bphi'_2~\cdots)\\
(\bzero_{3}~\cdots~\bzero_{3}~\bphi_2~\bphi''_2~\cdots)\\
+(\bphi_3~\bphi_2~\bphi_2~\bphi_2~\bphi_2~\cdots)\\
\cline{1-2}
(\bphi'_2~\bphi''_2~\bphi''_2~\bphi'_2~\bzero_3~\cdots)
\end{align*}
derived vector : $(\bphi'_2~(\bphi''_2)_{\beta-h-1}~\bphi'_2~\bzero_3~\cdots)$
\newline
Parity weight: \begin{equation}
\begin{split}
w_p=2(\beta-h)+2 =2\beta+2
\end{split}
\end{equation}

\paragraph{Case 1b: $\beta=h<\gamma$\newline}
 vectors to sum:
\begin{align*}
(\bzero_{3}~\cdots~\cdots~\bzero_{3}~\bphi_1~\bphi'_2~\cdots)\\
(\bphi_2~\bphi''_2~\cdots~\bphi''_2~\bphi''_2\bphi''_2~\cdots)\\
+(\bphi_3~\bphi_2~\cdots~\bphi_2~\bphi_2~\bphi_2~\cdots)\\
\cline{1-2}
(\bphi''_1~\bphi'_2~\cdots~\bphi'_2~\bphi_3~\bzero_3~\cdots)
\end{align*}
derived vector : $(\bphi''_1~(\bphi_2)_{\gamma-h-1}~\bphi_3~\bzero_3~\cdots)$\newline
Parity weight: \begin{equation}
\begin{split}
w_p=2(\gamma-h)+2=2\gamma+2
\end{split}
\end{equation}
%========case < = ==========
%\paragraph{Case 2a: $i<j=k$\newline}
% vectors to sum:
%\begin{align*}
%(\bzero_3 ~\cdots~\bphi_1~\bphi'_2~\cdots~\bphi'_2~\bphi'_2~\bphi'_2~\cdots)\\
%(\bzero_3~\cdots~\cdots~\cdots~\cdots~\bzero_3~\phi_2~\phi''_2~\cdots)\\
%+(\bzero_3~\cdots~\cdots~\cdots~\cdots~\bzero_3~\phi_3~\phi_2~\cdots)\\
%\cline{1-2}
%(\bzero_{3}~\cdots~\bzero_{3}\bphi_1~\bphi'_2~\cdots~\bphi'_2~\bphi'_1~\bzero_3~\cdots)
%\end{align*}
%derived vector : $(\bzero_{3i}~\bphi_1~(\bphi'_2)_{k-i-1}~\bphi'_1~\bzero_3~\cdots)$
%\newline
%Parity weight: \begin{equation}
%\begin{split}
%w_p=2(k-i)
%\end{split}
%\end{equation}

%\paragraph{Case 2a: $j<k=i$ \newline}
% vectors to sum:
%\begin{align*}
%(\bzero_3~\cdots~\cdots~\cdots~\cdots~\bzero_3~\phi_1~\phi'_2~\cdots)\\
%(\bzero_3~\cdots~\bzero_3~\phi_2~\phi''_2~\cdots~\phi''_2~\phi''_2~\phi''_2~\cdots)\\
%+(\bzero_3~\cdots~\cdots~\cdots~\cdots~\bzero_3~\phi_3~\phi_2~\cdots)\\
%\cline{1-2}
%(\bzero_{3}~\cdots\bzero_3~\bphi_2~\bphi''_2~\cdots~\bphi''_2~\bphi_2~\bzero_3~\cdots)
%\end{align*}
%derived vector : $(\bzero_{3j}~\bphi_2~\bphi''_2)_{i-j-1}~\bphi_2~\bzero_3~\cdots)$\newline
%Parity weight: \begin{equation}
%\begin{split}
%w_p=2(i-j)+2
%\end{split}
%\end{equation}
\newpage
\paragraph{Case 2a: $h<\gamma=\beta$ \newline}
 vectors to sum:
\begin{align*}
(\bzero_3\cdots~\cdots~\bzero_3~\phi_1~\phi'_2~\cdots)\\
(\bzero_3\cdots~\cdots~\bzero_3~\phi_2~\phi''_2~\cdots)\\
+(\phi_3~\phi_2~\cdots~\phi_2~\phi_2~\phi_2~\cdots)\\
\cline{1-2}
(\bphi_3~\bphi_2~\cdots~\bphi_2~\bphi_1~\bzero_3~\cdots)
\end{align*}


derived vector : $(\bphi_3~(\bphi_2)_{\gamma-h-1}~\bphi_1~\bzero_3~\cdots)$
\newline
Parity weight: \begin{equation}
\begin{split}
w_p=2(\gamma-h)+2 =2\gamma+2
\end{split}
\end{equation}
 %=====case < <==========
% \paragraph{Case 3a: $i<j<k$\newline}
 % vectors to sum:
%\begin{eqnarray*}
%(\bzero_3~\cdots~\bzero_3~\phi_1~\phi'_2~\cdots~\phi'_2~\phi'_2~\phi'_2~\cdots~\phi'_2~\phi'_2~\phi'_2~\cdots)\cr
%(\bzero_3~\cdots~\bzero_3~\bzero_3~\bzero_3~\cdots~\bzero_3~\phi_2~\phi''_2~\cdots~\phi''_2~\phi''_2~\phi''_2~\cdots)\cr
%+(\bzero_3~\cdots~\bzero_3~\bzero_3~\bzero_3~\cdots~\cdots~\cdots~\cdots~\bzero_3~\phi_3~\phi_2~\cdots)\cr
%\cline{1-2}
%(\bzero_{3}~\cdots~\bzero_3~\bphi_1~\bphi'_2\cdots\bphi'_2~\bphi''_2~\bphi_2~\cdots~\bphi_2~\bphi''_1~\bzero_3~\cdots)
%\end{eqnarray*}


%derived vector : $(\bzero_{3i}~\bphi_1~(\bphi'_2)_{j-i-1}~\bphi''_2~(\bphi_2)_{k-j-1}~\bphi''_1~\bzero_3~\cdots)$
%\newline
%Parity weight: \begin{equation}
%\begin{split}
%w_p&=2(j-i)+1+2(k-j-1)+1\\
%&=2(k-i)
%\end{split}
%\end{equation}

%\paragraph{Case 3b: $i<k<j$ \newline}
 % vectors to sum:
%\begin{eqnarray*}
%(\bzero_3~\cdots~\bzero_3~\phi_1~\phi'_2~\cdots~\phi'_2~\phi'_2~\phi'_2~\cdots~\phi'_2~\phi'_2~\phi'_2\cdots)\cr
%(\bzero_3~\cdots~\bzero_3~\bzero_3~\bzero_3~\cdots~\cdots~\cdots~\cdots~\bzero_3~\phi_2~\phi''_2\cdots)\cr
%+(\bzero_3~\cdots~\bzero_3~\bzero_3~\bzero_3~\cdots~\bzero_3~\phi_3~\phi_2~\cdots~\phi_2~\phi_2~\phi_2\cdots)\cr
%\cline{1-2}
%(\bzero_{3}~\cdots~\bzero_3~\bphi_1~\bphi'_2\cdots~\bphi'_2~\bphi_1~\bphi''_2~\cdots~\bphi''_2~\bphi'_2~\bzero_3\cdots)
%\end{eqnarray*}

%derived vector : $(\bzero_{3i}~\bphi_1~(\bphi'_2)_{k-i-1}~\bphi_1~(\bphi''_2)_{j-i-1}~\bphi'_2~\bzero_3~\cdots)$
%\newline
%Parity weight: \begin{equation}
%\begin{split}
%w_p&=2(k-i)+2(j-k)\\
%&=2(j-i)
%\end{split}
%\end{equation}


%\paragraph{Case 3a: $j<k<i$ \newline}
% vectors to sum:
%\begin{eqnarray*}
%(\bzero_3~\cdots~\bzero_3~\bzero_3~\bzero_3~\cdots~\cdots~\cdots~\cdots~\bzero_3~\phi_1~\phi'_2\cdots)\cr
%(\bzero_3~\cdots~\bzero_3~\phi_2~\phi''_2~\cdots~\phi''_2~\phi''_2~\phi''_2~\cdots~\phi''_2~\phi''_2~\phi''_2\cdots)\cr
%+(\bzero_3~\cdots\bzero_3~\bzero_3~\bzero_3~\cdots~\bzero_3~\phi_3~\phi_2~\cdots~\phi_2~\phi_2~\phi_2\cdots)\cr
%\cline{1-2}
%(\bzero_{3}~\cdots~\bzero_3~\bphi_2~\bphi''_2\cdots~\bphi''_2~\bphi'_1~\bphi'_2\cdots~\bphi'_2~\bphi_3~\bzero_3\cdots)
%\end{eqnarray*}
%derived vector : $(\bzero_{3k}~\bphi_2~(\bphi''_2)_{k-j-1}~\bphi'_1~(\bphi'_2)_{i-k-1}~\bphi_3~\bzero_3~\cdots)$\newline
%Parity weight: \begin{equation}
%\begin{split}
%w_p &=2(k-j)+1 +2(i-k)+1 \\
%&=2(i-j)+2
%\end{split}
%\end{equation}

%\paragraph{Case 3d: $j<i<k$\newline}
% vectors to sum:
%\begin{eqnarray*}
%(\bzero_3~\cdots\bzero_3~\bzero_3~\bzero_3~\cdots~\bzero_3~\phi_1~\phi'_2~\cdots~\phi'_2~\phi'_2~\phi'_2\cdots)\cr
%(\bzero_3~\cdots~\bzero_3~\phi_2~\phi''_2~\cdots~\phi''_2~\phi''_2~\phi''_2~\cdots~\phi''_2~\phi''_2~\phi''_2\cdots)\cr
%+(\bzero_3~\cdots~\bzero_3~\bzero_3~\bzero_3~~\cdots~\cdots~\cdots~\cdots~\bzero_3~\phi_3~\phi_2\cdots)\cr
%\cline{1-2}
%(\bzero_{3}~\cdots~\bzero_3~\bphi_2~\bphi''_2\cdots~\bphi''_2~\bphi''_1~\bphi_2\cdots~\bphi_2~\bphi''_1~\bzero_3\cdots)
%\end{eqnarray*}
%derived vector : $(\bzero_{3j}~\bphi_2~(\bphi''_2)_{i-j-1}~\bphi''_1~(\bphi_2)_{k-i-1}~\bphi''_1~\bzero_3~\cdots)$
%\newline
%Parity weight: \begin{equation}
%\begin{split}
%w_p&=2(i-j)+1+2(k-i-1)+1\\
%&=2(k-j)
%\end{split}
%\end{equation}


\paragraph{Case 3a: $h<\gamma<\beta$ \newline}
vectors to sum:
\begin{eqnarray*}
(\bzero_3\cdots\cdots~\bzero_3~\phi_1~\phi'_2~\cdots~\phi'_2~\phi'_2~\phi'_2\cdots)\cr
(\bzero_3\cdots~\cdots~\cdots~\cdots~\cdots~\bzero_3~\phi_2~\phi''_2\cdots)\cr
+(\phi_3~\phi_2~\cdots~\phi_2~\phi_2~\phi_2~\cdots~\phi_2~\phi_2~\phi_2\cdots)\cr
\cline{1-2}
(\bphi_3~\bphi_2\cdots~\bphi_2~\bphi'_1~\bphi''_2\cdots~\bphi''_2~\bphi'_2~\bzero_3\cdots)
\end{eqnarray*}


derived vector : $(\bphi_3~(\bphi_2)_{\gamma-h-1}~\bphi'_1~(\bphi''_2)_{\beta-\gamma-1}~\bphi'_2~\bzero_3~\cdots)$
\newline
Parity weight: \begin{equation}
\begin{split}
w_p&=2(\gamma-h)+2+2(\beta-i)\\
&=2(\beta-h)+2\\
& = 2\beta+2
\end{split}
\end{equation}

\paragraph{Case 3b: $h<\beta<\gamma$\newline}
\begin{eqnarray*}
(\bzero_3~\cdots~\cdots~\cdots~\cdots~\cdots~\bzero_3~\phi_1~\phi'_2\cdots)\cr
(\bzero_3~\cdots\cdots~\bzero_3~\phi_2~\phi''_2~\cdots~\phi''_2~\phi''_2~\phi''_2\cdots)\cr
+(\phi_3~\phi_2~\cdots~\phi_2~\phi_2~\phi_2~\cdots~\phi_2~\phi_2~\phi_2\cdots)\cr
\cline{1-2}
(\bphi_3~\bphi_2\cdots~\bphi_2~\bzero_3~\bphi'_2\cdots~\bphi'_2~\bphi_3~\bzero_3\cdots)
\end{eqnarray*}
derived vector : $(\bphi_3~(\bphi_2)_{j-k-1}~\bzero_3~(\bphi'_2)_{i-j-1}~\bphi_3~\bzero_3~\cdots)$\newline
Parity weight: \begin{equation}
\begin{split}
w_p &=2(\beta-h)+1 +2(\gamma-\beta)+1 \\
&=2(\gamma-h)+2\\
&=2\gamma+2
\end{split}
\end{equation}

From all the above cases we can conclude that the parity weight for a weight-$3$ RTZ sequence may be calculated as
\begin{equation}
w_p=
2l+2 
\end{equation}
where $l=\max \{ \gamma,\beta \} - k=\max \{ \gamma,\beta \}$ since $k=0$

Assuming that after interleaving, another weight-$3$ RTZ input is produced. Let $\gamma',\beta',h',l'$ and $w'_p$ be similarly defined. Then the Hamming weight $w_H$ of the turbo codeword produced can be calculated as
\begin{equation}
w_H=
7+2(l+l') 
\end{equation}

\end{proof}
%==============proof end==================

\subsubsection{Hamming weight for W4RTZ Turbo Codewords}
According to [SunTakeshita] the equation for the Hamming weight $w_H^{(4)}$ for a codeword generated by a W4RTZ is given by

\begin{equation}
w_H^{(4)} = 6m+2\Big(\sum_{i=1}^{m} \alpha_i+\sum_{i=1}^{m} \alpha'_i \Big)
\label{RTZInputs-3}
\end{equation}
where $m=w/2 = 4/2 = 2$

The above equation is only accurate when in both component codes, the given W4RTZ is such that $h<h+\alpha_1<h'<h'+\alpha_2$. When in both component codes,  the given W4RTZ is such that $h<h'<h+\alpha_1<h'+\alpha_2$ or $h<h'<h'+\alpha_2<h+\alpha_1$

\begin{equation}
w_H^{(4)}=4+w_p+w'_p
\end{equation}
where $w_p=\lfloor \frac{2f_1}{3} \rfloor + \lfloor \frac{2f_3}{3} \rfloor+\lfloor \frac{2(f_2-1)}{3} \rfloor + 2$ and \newline
$w'_p=\lfloor \frac{2f'_1}{3} \rfloor + \lfloor \frac{2f'_3}{3} \rfloor+\lfloor \frac{2(f'_2-1)}{3} \rfloor + 2$

\begin{proof}{case: $h<h'<h+\alpha_1<h'+\alpha_2$}

let $f_1 = h' - h, f_2= h+\alpha_1 - h', f_3 = h'+\alpha_2 - h+\alpha$
for $f_1$ and $f_3$ the bits do not overlap and the weight is maintained. The weights $(w_{f_1},~w_{f_3})$ can be calculated
using the equations below

$$w_{f_1}=\lfloor \frac{2f_1}{3} \rfloor+1$$
and
$$w_{f_3}=\lfloor \frac{2f_3}{3} \rfloor+1$$

For $f_2$, the bits do overlap and the overall weight is reduced
The weight for $f_2~ (w_{f_2})$ is calculated using the equation below
$w_{f_2}=\lfloor \frac{2f_2-1}{3} \rfloor$

the total parity weight 
\begin{equation}
\begin{split}
w_p &= w_{f_1} + w_{f_2} + w_{f_2}\\
&= \lfloor \frac{2f_1}{3} \rfloor + \lfloor \frac{2f_3}{3} \rfloor+\lfloor \frac{2(f_2-1)}{3} \rfloor + 2
\end{split}
\end{equation}
Assuming a similar W4RTZ occurs in the 2nd component code, $w'_p$ will be calculated similarly to $w_p$ and the total Hamming weight will be 
\begin{equation}
w_H^{(4)}=4+w_p+w'_p
\end{equation}

{case: $h<h'<h'+\alpha_2<h+\alpha_1$}
let $f_1 = h' - h, f_2= h+\alpha_2 - h', f_3 = h+\alpha_1 - (h+\alpha_2)$.
For $f_1$ and $f_3$ the bits do not overlap and the weight is maintained. The weights $(w_{f_1},~w_{f_3})$ can be calculated
using the equations below

$$w_{f_1}=\lfloor \frac{2f_1}{3} \rfloor+1$$
and 
$$w_{f_3}=\lfloor \frac{2f_3}{3} \rfloor+1$$

For $f_2$, the bits do overlap and the overall weight is reduced
The weight for $f_2~ (w_{f_2})$ is calculated using the equation below
$w_{f_2}=\lfloor \frac{2f_2-1}{3} \rfloor$

the total parity weight 
\begin{equation}
\begin{split}
w_p &= w_{f_1} + w_{f_2} + w_{f_2}\\
&= \lfloor \frac{2f_1}{3} \rfloor + \lfloor \frac{2f_3}{3} \rfloor+\lfloor \frac{2(f_2-1)}{3} \rfloor + 2
\end{split}
\end{equation}
Assuming a similar W4RTZ occurs in the 2nd component code, $w'_p$ will be calculated similarly to $w_p$ and the total Hamming weight will be 
\begin{equation}
w_H^{(4)}=4+w_p+w'_p
\end{equation}
\end{proof}

