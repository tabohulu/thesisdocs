

\section{Preliminaries}
This section outlines the definitions and notations that will be used. An example is also given at the end to better clarify the use of the notations.
\subsection{Definitions and Notations}
\begin{enumerate}
\item RTZ (Return-To-Zero) input :- A RTZ input is a binary input which causes a RSC encoder's final state to be return to zero after it has exited the zero state.

\item $\tau$ :- cycle length of the RSC encoder. For the $5/7$ RSC encoder $\tau = 3$

\item $N$ :- Interleaver length. 

\item $L$ :- Outer coset length where $L=N/\tau$

\item $M$ :- Inner coset length where $M=L/\tau$

\item $\cN$:- Integer set of $\{0,1,\cdots,N-1\}$

\item $\bbN$: Indexed set  of $\{0,1,\cdots,N-1\}$ in the natural order.

\item $\cC$:- Integer set of $\{0,1,\cdots,L-1\}$

\item $\bbC$: Indexed set  of $\{0,1,\cdots,L-1\}$ in the natural order.

\item $\cM$:- Integer set of $\{0,1,\cdots,M-1\}$

\item $\bbM$: Indexed set  of $\{0,1,\cdots,M-1\}$ in the natural order.

\item $\cC^{t}:=\left\{c+t\right\}_{c \in \cC}$ and $\bbC^t$ is the indexed set with the elements of $\cC^t$ where  $t=(0,1,...,\tau-1)$. 

\item $\cC^{tt'}:=\left\{m+t\right\}_{m \in \cM}$ and $\bbC^{tt'}$ is the indexed set with the elements of $\cC^{tt'}$ where  $tt'=(0,1,...,\tau-1)$. 
\item Permutation matrix 
\begin{equation*}
\bPi = \begin{bmatrix}
\bpi^0\cr
\bpi^1\cr
\vdots\cr
\bpi^{K-1}
\end{bmatrix}
= \begin{bmatrix}
\bpi_0 , \bpi_1,\cdots,\bpi_{\tau-1}
\end{bmatrix}
= \begin{bmatrix}
\pi_{t}^{(k)}
\end{bmatrix}_{k=0,~t=0}^{K-1,~\tau -1}
\end{equation*}
where $\pi_{t}^{(k)} \in \{0,1,\tau-1\}$. 

\item For the row vector $\bpi^{(k)}$, let $\mathscr{S}^e[\bpi^{(k)}]$ be the left-hand cycle shift of $\bpi^{(k)}$ and $\mathscr{S}^e[\bpi_t]$ be the up cycle shift of $\bpi_t$
\item We assume that the permutation matrix operation outputs the elements in $\bbC^t$ in the order which $t$ appers in $\bpi^k$. 
\item Our goal is to find the best $\bPi$ and $\bbC^t$, $t = 0,1,\cdots,\tau-1$. 
\end{enumerate}

\subsection{Example}
Lets assume we have a turbo code using the $5/7$ RSC encoder as its component code ($\tau=3$) and an interleaver length of $N=27$. We have the following values
\begin{enumerate}
\item $C=9$ and $M=3$. Also $\cN=\{0,1,\cdots,26\}$

\item $\cC^0=\{0,\tau,\cdots,(L-1)\tau\} = \{0,3,\cdots,24\}$, $\cC^1=\cC^0+1$ and $\cC^2=\cC^0+2$

\item $\cC^{00}=\{0,\tau,\cdots,(M-1)\tau\} = \{0,3,6\}$, $\cC^{01}=\cC^{00}+1$ and $\cC^{02}=\cC^{00}+2$

\item Let $\bbC^0=\{0, 3\tau, 6\tau, 1\tau, 4\tau, 7\tau, 2\tau, 5\tau, 8\tau \} = \{0, 9, 18, 3,12, 21, 6, 15, 24 \} $, $\bbC^1=\{4, 13, 22, 7, 16, 25, 1, 10, 19 \}$,$\bbC^2=\{ 23, 8, 17, 26, 2, 11, 20, 5, 14\} $ and $\bPi=\begin{bmatrix}0 & 0 & 0\cr2 & 2 & 2\cr1 & 1 & 1\end{bmatrix}$.

then 
 \begin{equation}
\begin{split}
\bbN=&\{c^0_0,c^0_1,c_2^0 ,c^2_0,c^2_1,c_2^2 ,c^1_0,c^1_1,c^1_2,\cdots, c^1_6,c^1_7,c^1_8\}\\
 =& \{0,9,18, 23,8,17,4,13,22 ,3,12,21,26,2,11,7,16,25,6,15,24,20,5,14,1,10,19\}
 \end{split}
 \end{equation}

\end{enumerate}
$\bbN$ represents the interleaved sequence. From this example, we can see that the column index of $i$ in $\pi^{(k)}$ represents the coset it belongs to before interleaving and the value $\pi_{t}^{(k)}$ specifies the coset after interleaving. Also notice that the rows of $\bPi$ are taken cyclicly untill all elements of $\bbC^t$ are placed in $\bbN$.

