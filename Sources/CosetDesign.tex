\section{Coset Design}
Once the permutation matrix is decided upon, we have the necessary constraints which will help us design $\bbC^t$ with respect to base RTZs (W2RTZs, W3RTZ)  .
We will make use of the Almost Linear Interleaver(ALI) is the design of $\bbC^0$. 
The interleaving equation for the ALI(L,D) Interleaver is given by 
$$
\pi(h)=D \cdot h + \Big\lfloor \frac{h}{A} \Big\rfloor \bmod L
$$ 
where $A=L/C$ and $C=\text{gcd}(D,L)$ Also, $D$ is the period of the interleaver and $h=0,1,...,L-1$. 
The value of a coset element at position $h$ will be $3\pi(h)+t$. The resulting coset interleaver will be reffered to by the notation CI(N,D)


\subsection{Coset Design for W2RTZs}
$w_H^{(2)}$ is calculated using (\ref{eq3})
 It is convinient to write $\Delta_{c'}$ in terms of $D$ 

$$c_{(h'+\alpha')}^{t}=3*(D(h+\alpha)+ \lfloor \frac{h+\alpha}{A} \rfloor \bmod L)$$ and $$c_{(h')}^{t})=3*(D(h)+ \lfloor \frac{h}{A} \rfloor \bmod L)$$
where $A=L/C,~C=\text{gcd}(L,D)$. 
For the chosen permutation matrix $\bPi^{(0)}$, we know that W2RTZs occur after 2 repititions and therefore we can set $\alpha =3$ and without loss of generality, we can also set $h=0$. 
$A$ can take on $3$ different values,$L,~L/3,~3$.

if $A=L,~L/3$ $\Delta_{c'}$ simplifies to 
\begin{equation}
\begin{split}
\Delta_{c'}=&3\Bigg(\min{\Big(D(\alpha) \bmod L,~ L-(D(\alpha) \bmod L)\Big)\Bigg)}\\
=&3\Bigg(\min{\Big(3D \bmod L,~ L-(3D \bmod L)\Big)\Bigg)}\\
\end{split}
\end{equation}

Feeding this into \ref{eq3} we get

\begin{equation*}
\begin{split}
w^{(2)}_H=&6+2\Bigg(3+\frac{3\Big(\min{\Big(3D \bmod L,~ L-(3D \bmod L)\Big)\Big)}}{3}\Bigg)\\
=&6+2\Big(3+\min{\Big(3D \bmod L,~ L-(3D \bmod L)\Big)}\Big)
\end{split}
\end{equation*}

For the case where $A=3$
 \begin{equation}
 \begin{split}
\Delta_{c'}=&3\Bigg(\min{\Big(D(\alpha)+1 \bmod L,~ L-(D(\alpha)+1 \bmod L)\Big)\Bigg)}\\
=&3\Bigg(\min{\Big(3D +1 \bmod L,~ L-(3D+1 \bmod L)\Big)\Bigg)}\\
\end{split}
 \end{equation}
 
Substituting into (\ref{eq3}) ,we have
\begin{equation*}
\begin{split}
w^{(2)}_H=&6+2\Big(3+\frac{3\Bigg(\min{\Big(3D+1 \bmod L,~ L-(3D+1 \bmod L)\Big)\Bigg)}}{3}\Big)\\
=&6+2\Big(3+\min{\Big(3D+1 \bmod L,~ L-(3D+1 \bmod L)\Big)}\Big)
\end{split}
\end{equation*}

Combining all the equations gives us 
\begin{equation}
w^{(2)}_H=
\begin{cases}
6+2\Big(3+\min{\Big(3D \bmod L,~ L-(3D \bmod L)\Big)\Bigg)}\Big),~A=L/3 ~\text{or}~A=L\\
6+2\Big(3+\min{\Big(3D+1 \bmod L,~ L-(3D+1 \bmod L)\Big)}\Big), ~A=3\\
\end{cases}
\label{eq9}
\end{equation}

 \subsection{Coset Design for W3RTZs}
 Unlike the W2RTZs, W3RTZs exist for the first and second repititions of $\bPi^{(0)}$. These RTZs can potentially reduce the $d_{\text{min}}$ value of the turbo code and coset design is carried out to prevent that.
 
 The W3RTZs as well as the various equations to calculate the resultant codeword weight are given in Table \ref{tb3}. 
 Since the design for $\bbC^0$ is applied to the other cosets we should be able to calculate the minimum value for $w_H^{(3)}$ with respect to Table \ref{tb3} whiles focusing on $\bbC^0$ only.
 The positions within 2 repititions of $\bPi^{(0)}$ where a weight-$3$ inputs occur are given with respect to the complete interleaver and need to be scaled down to $\bbC^0$
 Let $h,~h+\beta,~h+\gamma$ be the inputs representing where the weight-$3$ RTZ inputs occur due to $\bPi^{(0)}$. Then the scaled down versions will be $h^{(0)},~(h+\beta)^{(0)},~(h+\gamma)^{(0)}$ and are calculated using the equation
 \begin{equation}
 f(x)= x \bmod 3 + 3\Big(\Big\lfloor\frac{x}{9} \Big\rfloor\Big)
 \label{eq9}
 \end{equation}
 
 We feed $h^{(0)},~(h+\beta)^{(0)},~(h+\gamma)^{(0)}$ into the ALI and we get \begin{equation}
 \bs=(\pi(h^{(0)}),~\pi((h+\beta)^{(0)}),~\pi((h+\gamma)^{(0)}))-\min(\pi(h^{(0)}),~\pi((h+\beta)^{(0)}),~\pi((h+\gamma)^{(0)}))
 \label{eq10}
 \end{equation}
  and 
 
 \begin{equation}
 l'=\max(\beta,\gamma)= \bs_{\max}-\bs_{\min}=\bs_{\max}
 \label{eq11}
 \end{equation}
 
 When all $l'$ values for the $W3RTZs$ in Table \ref{tb3} are calculated, we take note of the ones that give the least value of $w_H^{(3)}$. Then for every shifted version of that $W3RTZ$ we calculate  $w_H^{(3)}$ and select the minimum value from all calculated $w_H^{(3)}$ to get the true minimum value.
 
 Below is a summary of the steps involved in determining the Hamming weight for all weight-$3$ RTZ associated with $\Pi^{(0)}$ 
 \begin{enumerate}
 \item convert  $h,~h+\beta,~h+\gamma$ into $h^{(0)},~(h+\beta)^{(0)},~(h+\gamma)^{(0)}$ using (\ref{eq9})
 
 \item input the indices into the ALI equation to obtain $\bs using (\ref{eq10})$
 
 \item find the value of $l'$ and  Hamming weight using (\ref{eq11}) and corresponding equation in Table \ref{tb3}
 
 \item repeat above steps for all shifted versions of the W3RTZ that produced the least value of $w_H^{(3)}$, where each shift is a multiple of $\tau^2$
 
\item Finally, the least Hamming weight value associated with the shifted version of the W3RTZ input(s) is selected.
 \end{enumerate}
 
 
 If we assume that the start position for all the cosets is the same, then the above calculations with respect to the weight-$3$ RTZ inputs are sufficient, however if we desire to apply a shift factor to all cosets but $\bbC^{(0)}$ , a slight change in notation will be needed. 
 
 First off we adjust the notation for a Coset interleaver from CI$(N,D)$ to CI$(N,D,s_1,s_2)$ where $s_1,~s_2$ indicate the start position for $\bbC^1,~\bbC^2$ respectively. It is worth noting that the value of $D$ used in the ALI(L,D) is the same for all the cosets, just that $\bbC^1,~\bbC^2$ are shifted by  $s_1,~s_2$ respective positions to the right.
 
 This means that values of $h^{(0)},~(h+\beta)^{(0)},~(h+\gamma)^{(0)}$ need to be adjusted by $0,~s_1,~s_2$ respectively. This statement is made assuming that $h^{(0)},~(h+\beta)^{(0)},~(h+\gamma)^{(0)}$ are in $\bbC^0,\bbC^1,\bbC^2$ respectively.
 
  Below is a summary of the steps involved in determining the Hamming weight for all weight-$3$ RTZ associated with $\Pi^{(0)}$ when $s_1,s_2>0$
 \begin{enumerate}
 \item convert  $h,~h+\beta,~h+\gamma$ into $h^{(0)},~(h+\beta)^{(0)},~(h+\gamma)^{(0)}$ using (\ref{eq9})
 
 \item input the indices $h^{(0)},~(h+\beta)^{(0)}+s_1,~(h+\gamma)^{(0)}+s_2$
 into the ALI equation to obtain $\bs$ using (\ref{eq10})
 
 \item find the value of $l'$ and  Hamming weight using (\ref{eq11}) and corresponding equation in Table \ref{tb3}
 
 \item repeat above steps for all shifted versions of the W3RTZ that produced the least value of $w_H^{(3)}$, where each shift is a multiple of $\tau^2$
 
\item Finally, the least Hamming weight value associated with the shifted version of the W3RTZ input(s) is selected.
 \end{enumerate}
 
