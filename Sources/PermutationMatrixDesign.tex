\section{Permutation Matrix Design}
In this section, we outline the procedure for selecting a good permutation matrix $\bPi$ with respect to W2RTZs and W3RTZs.

\subsection{Permutation Matrix selection for W2RTZs}
From the definition of Weight-$2$ RTZ inputs in the previous section, we know that the index of the ``1'' bits are in the same coset. Our aim is to make sure that the permutation matrix we select enbles the interleaver that we design to either break such weight-$2$ RTZ inputs or convert it into a large separation weight-$2$ RTZ. 
The condition to break weight-$2$ RTZs is given as

\begin{equation}
\pi_{j}^{(i)} \neq \pi_{j}^{(i')},~|i-i'| \leq N_c
\label{eq1}
\end{equation}

Since $\bPi$ consisting of $\tau$ elements, the maximum length of column elements consisting of values different from each other is $\tau$. Thus, the cut-off interleaver length for which (\ref{eq1}) is satisfied is $N_c^2=\tau^2=9$.
For this interleaver length, we investigate 3 different compositions of permutation matrices that can be used to achieve this condition in in \ref{eq1}

\begin{enumerate}
	\item One cycle permutation: Each row is permutation of the sequence $(0,1,2)$. Setting the element at the first row and first column to $0$, there are exactly 4 permutation matrices that exist for cut-off length $N_c^2$.
	Let
	\begin{equation*}
	\bpsi=\begin{bmatrix} 0\cr 1\cr 2\cr \end{bmatrix},~
	\bpsi'=\begin{bmatrix} 0\cr 2\cr 1\cr \end{bmatrix}
	\end{equation*}
We then have 
	\begin{equation}
	\begin{split}
	[\bpsi,\mathscr{S}^1[\bpsi],\mathscr{S}^2[\bpsi]]=
	&
	\begin{bmatrix}
	0 & 1 & 2\cr
	1 & 2 & 0\cr
	2 & 0 & 1
	\end{bmatrix}:=\bpsi(\bpsi) \\
	[\bpsi',\mathscr{S}^1[\bpsi'],\mathscr{S}^2[\bpsi']]=
	&
	\begin{bmatrix}
	0 & 1 & 2\cr
	2 & 0 & 1\cr
	1 & 2 & 0
	\end{bmatrix}:=\bpsi(\bpsi')\\
	[\bpsi,\mathscr{S}^2[\bpsi],\mathscr{S}^1[\bpsi]]=
	&
	\begin{bmatrix}
	0 & 2 & 1\cr
	2 & 1 & 0\cr
	1 & 0 & 2
	\end{bmatrix}:=\bpsi'(\bpsi)\\
	[\bpsi',\mathscr{S}^2[\bpsi'],\mathscr{S}^1[\bpsi']]=
	&
	\begin{bmatrix}
	0 & 2 & 1\cr
	1 & 0 & 2\cr
	2 & 1 & 0
	\end{bmatrix}:=\bpsi'(\bpsi')\\
	\end{split}
	\end{equation}
	
	%{\bf find all such matrices.}
	
	\item Two cycle permutation: Two rows are permutation of the sequence $(0,0,1,1,2,2)$.
	
	There are no permutation matrices that satisfying cut-off length $N_c^2$. This is because the sequence length is not divisible by $N_c^2$, there will always be 2 elements of the same value in each row of $\bPi$
	
	
	\item Three cycle permutation: Three rows are permutation of the sequence$(0,0,0,1,1,1,2,2,2)$. 
	
	Example of the permutation matrices satisfying cut-off length $N_c^2$ are shown in \ref{tb1}
	
	
\end{enumerate}





Table \ref{tb1} shows all unique coset interleaving arrays of length $N_c$ that convert weight-$2$ RTZ inputs to non-RTZ inputs. They are labeled from $A$ to $X$. A coset interleaving array is unique if a shift of the elements in the array does not produce another another coset interleaving array.

\begin{table}[h!]
\centering
\begin{tabular}{|c || c | c|| c|c || c | c|| c|} 
 \hline
 $A$ & $ \begin{bmatrix}0 & 0 & 0\cr1 & 1 & 1\cr2 & 2 & 2\end{bmatrix}$ 
 &
  $B$ & $\begin{bmatrix} 0 & 0 & 0 \cr 1 & 1 & 2 \cr 2 & 2 & 1\end{bmatrix}$ 
  &
  $C$ &$\begin{bmatrix} 0 & 0 & 0 \cr 1 & 2 & 1 \cr 2 & 1 & 2\end{bmatrix}$
  &
  $D$ & $\begin{bmatrix}0 & 0 & 0 \cr 1 & 2 & 2 \cr 2 & 1 & 1\end{bmatrix}$\\
 \hline
  $E$ & $\begin{bmatrix}0 & 0 & 0 \cr 2 & 1 & 1 \cr 1 & 2 & 2\end{bmatrix}$ 
 &
 $F$ & $\begin{bmatrix}0 & 0 & 0 \cr 2 & 1 & 2 \cr 1 & 2 & 1\end{bmatrix}$ 
 &
  $G$ & $\begin{bmatrix}0 & 0 & 0 \cr 2 & 2 & 1 \cr 1 & 1 & 2\end{bmatrix}$ 
 &
  $H$ & $\begin{bmatrix} 0 & 0 & 0\cr 2 & 2 & 2\cr 1 & 1 & 1\end{bmatrix}$\\ 
 \hline
   $I$ & $\begin{bmatrix} 0 & 0 & 1 \cr 1 & 1 & 0 \cr 2 & 2 & 2\end{bmatrix}$
 &
  $J$ & $\begin{bmatrix}0 & 0 & 1 \cr 1 & 2 & 0 \cr 2 & 1 & 2\end{bmatrix}$ 
 &
  $K$ & $\begin{bmatrix}0 & 0 & 1 \cr 2 & 1 & 0 \cr 1 & 2 & 2 \end{bmatrix}$
 &
 $L$ & $\begin{bmatrix}0 & 0 & 1 \cr 2 & 2 & 0 \cr 1 & 1 & 2\end{bmatrix}$\\ 
 \hline
 $M$ & $\begin{bmatrix}0 & 0 & 2 \cr 1 & 1 & 0 \cr 2 & 2 & 1 \end{bmatrix}$
 &
  $N$ & $\begin{bmatrix}0 & 0 & 2 \cr 1 & 2 & 0 \cr 2 & 1 & 1 \end{bmatrix}$ 
 &
 $O$ & $\begin{bmatrix}0 & 0 & 2 \cr 2 & 1 & 0 \cr 1 & 2 & 1\end{bmatrix}$ 
&
 $P$ & $\begin{bmatrix}0 & 0 & 2 \cr 2 & 2 & 0 \cr 1 & 1 & 1\end{bmatrix}$\\ 
 \hline
 $Q$ & $\begin{bmatrix}0 & 1 & 0 \cr 1 & 0 & 1 \cr 2 & 2 & 2 \end{bmatrix}$
&
  $R$ & $\begin{bmatrix}0 & 1 & 0 \cr 1 & 0 & 2 \cr 2 & 2 & 1\end{bmatrix}$ 
&
 $S$ & $\begin{bmatrix}0 & 1 & 0 \cr 1 & 2 & 1 \cr 2 & 0 & 2 \end{bmatrix}$
 &
 $T$ & $\begin{bmatrix}0 & 1 & 0 \cr 2 & 0 & 1 \cr 1 & 2 & 2\end{bmatrix}$\\ 
 \hline
 $U$ & $\begin{bmatrix}0 & 1 & 0 \cr 2 & 0 & 2 \cr 1 & 2 & 1\end{bmatrix}$ 
 &
 $V$ & $\begin{bmatrix}0 & 1 & 0 \cr 2 & 2 & 1 \cr 1 & 0 & 2\end{bmatrix}$ 
 &
 $W$ & $\begin{bmatrix}0 & 1 & 1 \cr 1 & 2 & 0 \cr 2 & 0 & 2 \end{bmatrix}$
 &
 $X$ & $\begin{bmatrix}0 & 2 & 0 \cr 2 & 0 & 2 \cr 1 & 1 & 1\end{bmatrix}$\\ 
   \hline

  \end{tabular}
\caption{All unique coset interleaving arrays of length $N_c =9$ for weight-$2$ RTZ inputs}
\label{tb1}
\end{table}

The interleaver length used in turbo coding are way greater than $N_c^2$ and it is not possible to transform weight-$2$ RTZ inputs into non-RTZ inputs for all values of $i$. All is not lost however, since not all weight-$2$ RTZ inputs produce low-weight codewords. 
The formula for calculating the Hamming weight of the Turbo codeword produced by a weight-$2$ RTZ input occuring in both component codes  ($w^{(2)}_H$)  is given by [SunTakeshita] 
\begin{equation}
\begin{split}
w^{(2)}_H=&2+(2 + \frac{\Delta_c}{\tau} )w_0+ (2 + \frac{\Delta_{c'}}{\tau})w_0\\
=&6+\Big(\frac{\Delta_c+\Delta_{c'}}{\tau}\Big)w_0,~w_0=2
\end{split}
\label{eq3}
\end{equation}

For all the $\Pi$ in Table \ref{tb1} $\Delta_c = 9=3\tau$ and $\Delta_{c'}:=(c_{(h+\alpha')}^{t}-c_{(h)}^{t})$ .
%This means that by altering the value of $t$ and $s$, we can increase the weight of the codeword produced. With this knowledge,
%all we need to do is to make sure that the if the input to the interleaver is a weight-$2$ RTZ inputs with a small value of $t$, it is converted to a weight-$2$ RTZ with a large value of $s$

%In summary, an interleaver designed to deal with weight-$2$ RTZ inputs when $N>N_c$ should
%\begin{enumerate}
%\item Convert  weight-$2$ RTZ inputs to non-RTZ inputs

%\item Convert  weight-$2$ RTZ inputs to a weight-$2$ RTZ inputs with a large value of $s$ when condition 1 isnt possible.

%\end{enumerate}

%An interleaver design method which makes use of the above points is as follows.
%Given an interlever with length $N$, we break it up into $\frac{N}{N_c}$ blocks each of length $N_c$. At the beginning of the $n$th block, the coset interleaving pattern is repeated until the last block. By applying this method, we make sure that when the weight-$2$ RTZ occurs within a  block, condition 1 is met and condition 2 is met when the weight-$2$ RTZ occurs in 2 consecutive blocks i.e. when  $t=N_c$.

%This method only works when $N_c | N$ and we will delay the details of what to do when $N_c \not| N$.

\subsection{Permutation Matrix selection for W3RTZs}
As mentioned earlier, a W3RTZ is formed when the indices of the ``1'' bits each occur in different cosets.  It goes without saying that the simplest way to convert a W3RTZ into a non-W3RTZ is to make sure that at least two of indices of the ``1'' bits occur within the same coset after interleaving. 

The formula for calculating the hamming weight for a turbo code created by a W3RTZ ($w^{(3)}_H$)
\begin{equation}
\begin{split}
w^{(3)}_H=& 3+(2l+2)+(2l'+2)\\
=&3+w_p+w'_p,~w_p=2l+2,~w'_p=2l'+2\\
=&7+2(l+l')
\end{split}
\label{eq6}
\end{equation}
where $w_p,w'_p$ refer to the pre-interleaving parity weight and the post-interleaving parity weight respectively.

In reality, W3RTZs are many and it is impossible to completely get rid of all of them, even within $N_c$. So in our selection of Permutation Matrices for W3RTZs, we make sure that the remaining W3RTZs have $w_p>2$. 
Unique permutation matrices which meet this criteria are shown in Table \ref{tb2} and they are labeled from $A$ to $L$

\begin{table}[h!]
\centering
\begin{tabular}{|c || c  |c  ||c  |} 
 \hline
 $A$ & $\begin{bmatrix} 0 & 0 & 0 \cr 1 & 1 & 1 \cr 2 & 2 & 2\end{bmatrix}$ 
  &
 $B$ & $\begin{bmatrix}0 & 0 & 0 \cr 1 & 1 & 2 \cr 1 & 2 & 2\end{bmatrix}$\\ 
 \hline
$C$ & $\begin{bmatrix}0 & 0 & 0 \cr 1 & 1 & 2 \cr 2 & 1 & 2\end{bmatrix}$ 
 &
$D$ & $\begin{bmatrix}0 & 0 & 0 \cr 1 & 1 & 2 \cr 2 & 2 & 1\end{bmatrix}$\\ 
 \hline
 $E$ & $\begin{bmatrix}0 & 0 & 0 \cr 2 & 2 & 1 \cr 1 & 1 & 2\end{bmatrix}$ 
 &
 $F$ & $\begin{bmatrix}0 & 0 & 0 \cr 2 & 2 & 1 \cr 1 & 2 & 1\end{bmatrix}$\\ 
 \hline
 $G$ & $\begin{bmatrix}0 & 0 & 0 \cr 2 & 2 & 1 \cr 2 & 1 & 1\end{bmatrix}$ 
 &
  $H$ & $\begin{bmatrix}0 & 0 & 1 \cr 0 & 1 & 1 \cr 2 & 2 & 2\end{bmatrix}$\\ 
 \hline
  $I$ & $\begin{bmatrix}0 & 0 & 1 \cr 1 & 1 & 2 \cr 2 & 0 & 2\end{bmatrix}$ 
 &
 $J$ & $\begin{bmatrix}0 & 0 & 2 \cr 0 & 2 & 2 \cr 1 & 1 & 1\end{bmatrix}$\\ 
 \hline
  $K$ & $\begin{bmatrix}0 & 0 & 2 \cr 2 & 2 & 1 \cr 1 & 0 & 1\end{bmatrix}$
 &
  $L$ & $\begin{bmatrix}0 & 1 & 0 \cr 1 & 1 & 2 \cr 2 & 0 & 2\end{bmatrix}$\\ 
 \hline
\end{tabular}
\caption{All unique permutation matrices of length $N_c =9$ for weight-$3$ RTZ inputs}
\label{tb2}
\end{table}

Depending on which permutation matrix is chosen from Table \ref{tb2}, Equation \ref{eq6} can be simplified. 
The value of $w_p$ for the pre-interleaving weight-$3$ is dependent on the elements in $\cC^t$

Let $(c_{(h)}^{t},~c_{(h+\beta)}^{t+1},~c_{(h+\gamma)}^{t+2})$ be the vector representing a weight-$3$ RTZ input

Without loss of generality, we can assume that $h =t =0$. We then have 
\begin{equation}
l=\max{(\beta,\gamma)}
\label{eq8}
\end{equation}
And 
\begin{equation}
w_p=
2(\max{(\beta,\gamma)})+2
\label{eq9}
\end{equation}
By deciding on the $\Pi$ we can easily calculate all values of $l$ and $w_p$. 
$w'_p,\beta',\gamma' $ and $l'$ are similarly defined and are dependent on the elements in $\bbC^{t},~t=0,1,..,\tau-1$

As an example, Table \ref{tb3} shows all the weight-$3$ RTZ inputs and the corresponding equations for calculating $w_H$

\begin{table}
\centering
\begin{tabular}{||c |c  |c  |c |} 
 \hline
 RTZ index  & $ l$ & $w_p$& $w_H$\\
 \hline
 $(0~4~8)$ & $2$ & $6$ & $11+2(\max{(\beta',\gamma')})$\\
 \hline
 $(0~5~7)$ &  $2$ & $6$ &$11+2(\max{(\beta',\gamma')})$\\
 \hline
 $(1~3~8) $ &  $2$ & $6$ & $11+2(\max{(\beta',\gamma')})$\\
 \hline
 $(1~5~6) $ &  $1$ & $4$ & $9+2(\max{(\beta',\gamma')})$\\
 \hline
 $(2~3~7)$ & $1$ & $4$ & $9+2(\max{(\beta',\gamma')})$\\
 \hline
 $(2~4~6)$ & $1$ & $4$ & $9+2(\max{(\beta',\gamma')})$\\
 \hline\hline
  $(0~8~13)$ & $4$ & $10$ & $15+2(\max{(\beta',\gamma')})$\\
 \hline
 $(0~4~17)$ & $5$ & $12$ & $17+2(\max{(\beta',\gamma')})$\\
 \hline
 $(0~13~17)$ & $5$ & $12$ & $17+2(\max{(\beta',\gamma')})$\\
 \hline
  $(0~7~14)$ &  $4$ & $6$ &$15+2(\max{(\beta',\gamma')})$\\
 \hline
  $(0~5~16)$ &  $5$ & $6$ &$17+2(\max{(\beta',\gamma')})$\\
 \hline
  $(0~14~16)$ &  $5$ & $6$ &$17+2(\max{(\beta',\gamma')})$\\
 \hline
 $(1~8~12) $ &  $3$ & $8$ & $13+2(\max{(\beta',\gamma')})$\\
 \hline
 $(1~3~17) $ &  $5$ & $12$ & $17+2(\max{(\beta',\gamma')})$\\
 \hline
 $(1~12~17) $ & $5$ & $12$ & $17+2(\max{(\beta',\gamma')})$\\
 \hline
  $(1~6~14) $ &  $4$ & $10$ & $15+2(\max{(\beta',\gamma')})$\\
 \hline
  $(1~5~15)$ &  $4$ & $10$ & $15+2(\max{(\beta',\gamma')})$\\
 \hline
  $(1~14~15)$ &  $4$ & $10$ & $15+2(\max{(\beta',\gamma')})$\\
 \hline
 $(2~7~12)$ & $3$ & $8$ & $13+2(\max{(\beta',\gamma')})$\\
 \hline
 $(2~3~16)$ & $4$ & $10$ & $15+2(\max{(\beta',\gamma')})$\\
 \hline
 $(2~12~16)$ & $4$ & $10$ & $15+2(\max{(\beta',\gamma')})$\\
 \hline
  $(2~6~13)$ & $3$ & $8$ & $13+2(\max{(\beta',\gamma')})$\\
 \hline
  $(2~4~15)$ & $4$ & $10$ & $15+2(\max{(\beta',\gamma')})$\\
 \hline
  $(2~13~15)$ & $4$ & $10$ & $15+2(\max{(\beta',\gamma')})$\\
 \hline
\end{tabular}
\caption{All unique permutation matrices of length $N_c =9$ for weight-$3$ RTZ inputs}
\label{tb3}
\end{table}

Finally, we need to choose a Permutation Matrix which is able to efectively deal with both 
W2RTZs and W3RTZs. This part is simple as the only thing that we need to do is to select the permutation matrices that appear in both Table \ref{tb1} and Table \ref{tb2}. This leaves us with $$\begin{bmatrix} 0 & 0 & 0 \cr 1 & 1 & 1 \cr 2 & 2 & 2\end{bmatrix}$$  and  $$\begin{bmatrix} 0 & 0 & 0 \cr 2 & 2 & 2 \cr 1 & 1 & 1\end{bmatrix}$$

Moving foward we will use the permutation matrix $$\bPi^{(0)}=\begin{bmatrix} 0 & 0 & 0 \cr 1 & 1 & 1 \cr 2 & 2 & 2\end{bmatrix}$$ in all our considerations. It was chosen because in the design process, we only need to focus on only one of the cosets, say $\bbC^0$ and replicate the results for the remaining cosets.

\subsection{Higher Weight RTZ inputs}
Higher weight RTZ inputs are made up of some combination of W2RTZs and/or W3RTZs.
Let the weight of the higher weight RTZ be represented by $w$. If $w \bmod 2 =0$, then the RTZ is made up of $w/2$ W2RTZs. If $w \bmod 2 =1$, then the RTZ is made up of $\lfloor w/2 \rfloor -1 $ W2RTZs and a single W3RTZ
As the weight of the RTZ inputs increases beyond the largest base RTZ, the Permutation Matrices cannot be designed to effectively get rid of these Higher Weight RTZ inputs. However, with the the structure of the Permutation Matrix, we know exactly where these Higher Weight RTZ inputs occur.
In addition to the hamming weight for W2RTZs and W3RTZs, we will consider the hamming weight for W4RTZs and W5RTZs in our design of good interleavers. From [SunTakeshita] we deduce that Hamming weight for a Turbo code produced as a result of an even weight RTZ is given by
\begin{equation}
w_H^{(\text{even})}=6m+2\Big(\frac{\sum_{i=1}^{m}\Delta_{c_i}+\sum_{i=1}^{m}\Delta_{c'_i}}{\tau}\Big)
\end{equation}
where $m=w/2$
whiles that for an odd weight RTZ is given by 
\begin{equation}
\begin{split}
w_H^{(\text{odd})}&=6m+2\Big(\frac{\sum_{i=1}^{m}\Delta_{c_i}+\sum_{i=1}^{m}\Delta_{c'_i}}{\tau}\Big)+7+2(l+l')\\
&=7+6m+2\Bigg(l+l'+\Big(\frac{\sum_{i=1}^{m}\Delta_{c_i}+\sum_{i=1}^{m}\Delta_{c'_i}}{\tau}\Big)\Bigg)
\end{split}
\end{equation}
where $m=\lfloor w/2 \rfloor-1$
For W4RTZs we have
\begin{equation}
w_H^{(4)}=12+2\Big(\frac{\Delta_{c1}+\Delta_{c2}+\Delta_{c'1}+\Delta_{c'2}}{\tau}\Big)
\end{equation}

and for W5RTZs we have 
\begin{equation}
w_H^{(5)}=13+2\Bigg(l+l'+\Big(\frac{\Delta_{c_1}+\Delta_{c'_1}}{\tau}\Big)\Bigg)
\end{equation}

These equations will be used as secondary checks to confirm the minimum distance of the turbo code with our designed interleaver