\documentclass[11pt, oneside, dvipdfmx]{book}
\newcommand{\folder}{/usr/local/share/texmf}
%\newcommand{\folder}{/home/hanchenggao/Documents/texmf}
\input{\folder/hfiles/ebook}
\usepackage {graphicx}
\usepackage {graphics}
%\setCJKmainfont{SimSun}
\title{``
Equation for calculating W4RTZ weights:Proof'' }
\author{Kwame Ackah Bohulu}
\date{\today}
\begin{document}

\maketitle
\section{Hamming weight for W4RTZ Turbo Codewords}
According to [SunTakeshita] the equation for the Hamming weight $w_H^{(4)}$ for a codeword generated by a W4RTZ is given by

\begin{equation}
w_H^{(4)} = 6m+2\Big(\alpha+\alpha' +\alpha^{(\pi)} +\alpha^{(\pi)'}\Big)
\label{RTZInputs-3}
\end{equation}
where $m=w/2 = 4/2 = 2$

The above equation is only accurate when in both component codes, the given W4RTZ is such that 
$h\tau+t<(h + \alpha)\tau+t<h'\tau+t'<(h' + \alpha')\tau+t'$.  For the case where $h\tau+t<h'\tau+t'<(h + \alpha)\tau+t<(h' + \alpha')\tau+t'$ and 
$h\tau+t<h'\tau+t'<(h' + \alpha')\tau+t'<(h + \alpha)\tau+t$, the above equation is not accurate.

First we will derive the Hamming weight equation for all the various classes of W4RTZs with respect to the convolutional code and then with respect to the turbo code.

Before we begin the proof, the following notations will be used

\begin{enumerate}
\item Let 
$$\bphi_1=(0~0~1),~\bphi'_1=(0~1~0),~\bphi''_1=(1~0~0)$$ \\
$$\bphi_2=(0~1~1),~\bphi'_2=(1~1~0),~\bphi''_2=(1~0~1)$$ \\
$$\bphi_3=(1~1~1)$$.

\item For a W2RTZ, the parity vector $\bv_p=(\bphi_1~ (\bphi'_2)_{\alpha-1}~ \bphi_3)$
\end{enumerate} 

\begin{theorem}[Type1 W4RTZ]

For a W4RTZ such that $h\tau+t<(h + \alpha)\tau+t<h'\tau+t'<(h' + \alpha')\tau+t'$, the parity weight of the convolutional code is $w_p=2(\alpha +\alpha')+4$
\end{theorem}
\begin{proof}
For the above stated case, we have the following parity vector summation.
\begin{eqnarray*}
(\bphi_1~ \bphi'_2~ \bphi'_2~\cdots~ \bphi'_2~ \bphi_3~\cdots ~\bzero_3~\bzero_3~\bzero_3~\cdots~\bzero_3~\bzero_3~
\cdots~\bzero_3)\cr
+(\bzero_3~\bzero_3~\bzero_3~\cdots~\bzero_3~\bzero_3~\cdots~\bphi_1~ \bphi'_2~ \bphi'_2~\cdots~\bphi'_2~ \bphi_3~\cdots~\bzero_3)\cr
\cline{1-2}
(\bphi_1~ \bphi'_2~ \bphi'_2~\cdots~ \bphi'_2~ \bphi_3~\cdots~\bphi_1~ \bphi'_2~ \bphi'_2~\cdots~\bphi'_2~ \bphi_3~\cdots~\bzero_3)
\end{eqnarray*}

The derived parity vector is then $(\bphi_1~ (\bphi'_2)_{\alpha-1}~ \bphi_3~\cdots~\bphi_1~ (\bphi'_2)_{\alpha'-1}~ \bphi_3)$ and the weight for the derived parity vector is calculated as 
\begin{equation}
\begin{split}
w_p=&2(\alpha)+2+2(\alpha')+2\\
=&2(\alpha + \alpha')+4
\end{split}
\end{equation}
\end{proof}

\begin{theorem}[Type2 W4RTZ]
For a W4RTZ such that $h\tau+t<h'\tau+t'<(h' + \alpha')\tau+t'<(h + \alpha)\tau+t,~
t'\neq t$, the parity weight of the convolutional code is $w_p=2\alpha$
\end{theorem}
\begin{proof}
For the above case, there are 3 possible vector summations as shown below
\begin{eqnarray}
(\bphi_1~ \bphi'_2~\cdots~\bphi'_2~ \bphi'_2~ \bphi'_2~\cdots~ \bphi'_2~ \bphi'_2~ \bphi'_2~\cdots~ \bphi'_2~\bphi_3)\cr
+(\bzero_3~~\bzero_3~\cdots~\bzero_3~\bphi_3~\bphi_2~\cdots~\bphi_2~\bphi''_1~\bzero_3
~\cdots~\bzero_3~\bzero_3)\cr
\cline{1-2}
(\bphi_1~ \bphi'_2~\cdots~\bphi'_2~\bphi_1~\bphi''_2~\cdots~\bphi''_2~\bphi'_1~\bphi'_2~
\cdots ~\bphi'_2~\bphi_3)
\label{2-1}
\end{eqnarray}
for \ref{2-1} the derived parity vector is $$(\bphi_1~ (\bphi'_2)_{(h'-h)}~\bphi_1~(\bphi''_2)_{(\alpha'-1)}~\bphi_1~(\bphi'_2)_{((h-h')+(\alpha-\alpha')-2)}~\bphi_3)$$
with a corresponding weight of 
\begin{equation*}
\begin{split}
w_p&=2(h'-h)+1+2(\alpha'-1)+1+2((h-h')+(\alpha-\alpha')-2)+4\\
&=2(h'-h)+1+2\alpha'-1+2(h-h')+2(\alpha-\alpha')\\
&=2(\alpha'-\alpha'+\alpha)\\
&=2\alpha
\end{split}
\end{equation*}
\begin{eqnarray}
(\bphi_1~ \bphi'_2~\cdots~ \bphi'_2~ \bphi'_2~ \bphi'_2~\cdots~ \bphi'_2~ \bphi'_2~ \bphi'_2~\cdots~\bphi'_2~ \bphi_3)\cr
+(\bzero_3~~\bzero_3~\cdots~\bzero_3~\bphi_2~\bphi''_2~\cdots~\bphi''_2~
\bphi'_2~\bzero_3~\cdots~\bzero_3
~\bzero_3)\cr
\cline{1-2}
(\bphi_1~ \bphi'_2~\cdots~ \bphi'_2~\bphi''_2~\bphi_2~\cdots~\bphi_2~\bzero_3~\bphi'_2~\cdots~\bphi'_2 \bphi_3)
\label{2-2}
\end{eqnarray}
For \ref{2-2} The derived parity vector is $$
(\bphi_1~(\bphi'_2)_{(h'-h)}~\bphi''_2~(\bphi_2)_{(\alpha-1)}~\bzero_3~(\bphi'_2)_{((h-h')+(\alpha-\alpha')-2)}~\bphi_3)
$$
And the parity weight is 
\begin{equation*}
\begin{split}
w_p&=2(h'-h)+1+2(\alpha'-1)+2+2((h-h')+(\alpha-\alpha')-2)+3\\
&=2(h'-h)+1+2\alpha'-2+2+2(h-h')+2(\alpha-\alpha')-1\\
&=2(\alpha'-\alpha'+\alpha)\\
&=2\alpha
\end{split}
\end{equation*}

\begin{eqnarray}
(\bphi_1~ \bphi'_2~ \bphi'_2~ \bphi'_2~\cdots~ \bphi'_2~ \bphi'_2~ \bphi'_2~ \bphi_3)\cr
+(\bzero_3~~\bzero_3~\bphi_1~ \bphi'_2~\cdots~\bphi'_2~ \bphi_3~\bzero_3
~\bzero_3)\cr
\cline{1-2}
(\bphi_1~ \bphi'_2~\bphi_3~\bzero_3~\cdots~\bzero_3~\bphi_1~\bphi'_2~ \bphi_3)
\label{2-3}
\end{eqnarray}

For \ref{2-3}, $t=t'$ and the derived vector as well as the parity weight is the same as that of the Type1 W4RTZ.
Therefore for Type2 W4RTZ the parity weight is given by
\begin{equation}
w_p=2\alpha 
\end{equation}
\end{proof}

\begin{theorem}[Type3 W4RTZ]
For a W4RTZ such that $h\tau+t<h'\tau+t'<(h + \alpha)\tau+t<(h' + \alpha')\tau+t',~
t'\neq t$, the parity weight of the convolutional code is $w_p=2(\alpha' +(h'-h) +(t'-t)-1)$
\end{theorem}
\begin{proof}
Again for the above case, there are 3 possible vector summations as shown below
\begin{eqnarray}
(\bphi_1~ \bphi'_2~\cdots~ \bphi'_2~ \bphi'_2~ \bphi'_2~\cdots~ \bphi'_2~
 \bphi_3~\bzero_3~\cdots~\bzero_3~\bzero_3)\cr
+(\bzero_3~\bzero_3~\cdots~\bzero_3~\bphi_3~\bphi_2~\cdots~\bphi_2~\bphi_2
~\bphi_2\cdots~\bphi_2~\bphi''_1)\cr
\cline{1-2}
(\bphi_1~\bphi'_2~\cdots~\bphi'_2~\bphi_1~\bphi''_2~\cdots~\bphi''_2~\bphi''_1
~\bphi_2\cdots~\bphi_2~\bphi''_1)
\label{3-1}
\end{eqnarray}

For \ref{3-1} the derived parity vector is $$
(\bphi_1~(\bphi'_2)_{(h'-h)}~\bphi_1~(\bphi''_2)_{(h-h'+\alpha)-2}~\bphi''_1~(\bphi_2)_{((h'-h)+(\alpha'-\alpha))}~\bphi''_1)
$$
with a weight of 
\begin{equation*}
\begin{split}
w_p&=2(h'-h)+1+2(h-h'+\alpha-2)+2+2((h'-h)+(\alpha'-\alpha))+1\\
&=2(h'-h)+2(\alpha-\alpha+\alpha')+1-1\\
&=2((h'-h)+\alpha')
\end{split}
\end{equation*}

\begin{eqnarray}
(\bphi_1~ \bphi'_2~\cdots~ \bphi'_2~ \bphi'_2~ \bphi'_2~\cdots~ \bphi'_2~
 \bphi_3~\bzero_3~\cdots~\bzero_3~\bzero_3)\cr
(\bzero_3~\bzero_3~\cdots~\bzero_3~\bphi_2~\bphi''_2~\cdots~\bphi''_2~\bphi''_2
~\bphi''_2\cdots~\bphi''_2~\bphi'_2)\cr
\cline{1-2}
(\bphi_1~\bphi'_2~\cdots~\bphi'_2~\bphi''_2~\bphi_2~\cdots~\bphi_2~\bphi'_1
~\bphi''_2\cdots~\bphi''_2~\bphi'_2)
\label{3-2}
\end{eqnarray}

For \ref{3-2} the derived parity vector is $$
(\bphi_1~(\bphi'_2)_{(h'-h)}~\bphi''_2~(\bphi_2)_{(h-h'+\alpha)-2}~\bphi'_1~(\bphi''_2)_{((h'-h)+(\alpha'-\alpha))}~\bphi'_2)
$$
with a weight of 
\begin{equation*}
\begin{split}
w_p&=2(h'-h)+1+2(h-h'+\alpha-2)+3+2((h'-h)+(\alpha'-\alpha))+2\\
&=2(h'-h)+2(\alpha-\alpha+\alpha')+2\\
&=2((h'-h)+\alpha'+1)
\end{split}
\end{equation*}


\begin{eqnarray}
(\bphi_1~ \bphi'_2~\cdots~ \bphi'_2~ \bphi'_2~ \bphi'_2~\cdots~ \bphi'_2~
 \bphi_3~\bzero_3~\cdots~\bzero_3~\bzero_3)\cr
(\bzero_3~\bzero_3~\cdots~\bzero_3~\bphi_1~\bphi'_2~\cdots~\bphi'_2~\bphi'_2
~\bphi'_2\cdots~\bphi'_2~\bphi_3)\cr
\cline{1-2}
\bphi_1~\bphi'_2~\cdots~\bphi'_2~\bphi_3~\bzero_3~\cdots~\bzero_3~\bphi_1
~\bphi'_2\cdots~\bphi'_2~\bphi_3)
\label{3-3}
\end{eqnarray}

For \ref{3-3}, $t=t'$ and the derived vector as well as the parity weight is the same as that of the Type1 W4RTZ.

The parity weight equations for \ref{3-1} and \ref{3-2} are different but without loss of generality we can assuming  that $t'>t,~t=0$ we see that \ref{3-1} corresponds to the case where $t'-t=1$ whiles \ref{3-2} corresponds to the case where$t'-t=2$
Therefore for Type3 W4RTZ the parity weight is given by
\begin{equation}
w_p=2(\alpha' +(h'-h) +(t'-t)-1)
\end{equation}
\end{proof}

The hamming weight with respect to all possible combinations of W4RTZ types is given below

\begin{enumerate}
\item 2 Type1 W4RTZ combinations
\begin{equation}
w_H^{(4)} = 6m+2\Big(\alpha+\alpha' +\alpha^{(\pi)} +\alpha^{(\pi)'}\Big)
\end{equation}

\item 2 Type2 W4RTZ combinations
\begin{equation}
w_H^{(4)} = 2(\alpha + \alpha^{(\pi)})
\end{equation}

\item 2 Type3 W4RTZ combinations
\begin{equation}
w_H^{(4)} =2(\alpha' +(h'-h) +(t'-t)) + 2(\alpha^{(\pi)'} +(h^{(\pi)'}-h^{(\pi)}) 
+(t^{(\pi)'}-t^{(\pi)}))
\end{equation}

\item A combination of a Type1 and Type2 W4RTZ
\begin{equation}
w_H^{(4)} =8+2(\alpha + \alpha^{(\pi)} + \alpha')
\end{equation}

\item A combination of a Type1 and Type3 W4RTZ
\begin{equation}
w_H^{(4)} =6+2(\alpha + \alpha')+2(\alpha^{(\pi)'} +(h^{(\pi)'}-h^{(\pi)}) 
+(t^{(\pi)'}-t^{(\pi)}))
\end{equation}

\item A combination of a Type2 and Type3 W4RTZ
\begin{equation}
w_H^{(4)} =2\Big((\alpha + \alpha^{(\pi)})+(h'-h)+(t'-t) - 1\Big)
\end{equation}

\end{enumerate}





%\begin{enumerate}
%\item Calculating weight for W4RTZs
%\item Calculating weight for W5RTZs

%\tem Search for good interleavers
%\item Simulation results comparison with QPP
%\end{enumerate}

\end{document}