\documentclass[11pt, oneside, dvipdfmx]{book}
\newcommand{\folder}{/usr/local/share/texmf}
%\newcommand{\folder}{/home/hanchenggao/Documents/texmf}
\input{\folder/hfiles/ebook}
\usepackage {graphicx}
\usepackage {graphics}
%\setCJKmainfont{SimSun}
\title{``
Equation for calculating W4RTZ weights:Proof'' }
\author{Kwame Ackah Bohulu}
\date{\today}
\begin{document}

\maketitle
\section{Hamming weight for W4RTZ Turbo Codewords}
According to [SunTakeshita] the equation for the Hamming weight $w_H^{(4)}$ for a codeword generated by a W4RTZ is given by

\begin{equation}
w_H^{(4)} = 6m+2\Big(\alpha+\alpha' +\alpha^{(\pi)} +\alpha^{(\pi)'}\Big)
\label{RTZInputs-3}
\end{equation}
where $m=w/2 = 4/2 = 2$

The above equation is only accurate when in both component codes, the given W4RTZ is such that 
$h\tau+t<(h + \alpha)\tau+t<h'\tau+t'<(h' + \alpha')\tau+t'$.  For the case where $h\tau+t<h'\tau+t'<(h + \alpha)\tau+t<(h' + \alpha')\tau+t'$ and 
$h\tau+t<h'\tau+t'<(h' + \alpha')\tau+t'<(h + \alpha)\tau+t$, the above equation is not accurate.

First we will derive the Hamming weight equation for all the various classes of W4RTZs with respect to the convolutional code and then with respect to the turbo code.

\begin{theorem}[Type1 W4RTZ]

For a W4RTZ such that $h\tau+t<(h + \alpha)\tau+t<h'\tau+t'<(h' + \alpha')\tau+t'$, the Hammming weight of the convolutional code $w_p=2(\alpha +\alpha')+4$
\end{theorem}
\begin{proof}
We split the W4RTZ into 3 parts. The separation between $(h + \alpha)\tau+t$ and $h\tau+t$ is
$$h\tau +\alpha\tau + t -h\tau - t=\alpha\tau$$ and the associated weight is $$2\alpha +2$$
The separation between $(h' + \alpha')\tau+t'$ and $h'\tau+t'$ is
$$h'\tau +\alpha'\tau + t' -h'\tau - t'=\alpha'\tau$$ and the associated weight is $$2\alpha' +2$$
Since the seperation between $(h + \alpha)\tau+t$ and $h\tau+t$ as well as $(h' + \alpha')\tau+t'$ and $h'\tau+t'$ are multiples of $\tau$, it means that the weight between $h'\tau+t'$ and $(h + \alpha)\tau+t$ is 0. The total parityweight is then
\begin{equation}
\begin{split}
w_p=&2(\alpha)+2+2(\alpha')+2\\
=&2(\alpha + \alpha')+4
\end{split}
\end{equation}
\end{proof}

\begin{theorem}[Type2 W4RTZ]
For a W4RTZ such that $h\tau+t<h'\tau+t'<(h' + \alpha')\tau+t'<(h + \alpha)\tau+t$, the Hammming weight of the convolutional code $w_p=2\alpha$
\end{theorem}
\begin{proof}
The separation between $h'\tau+t'$ and $h\tau+t$ is
$$(h'-h)\tau + (t'-t)$$ and the associated weight is $$2(h'-h) +(t'-t)$$

The separation between $(h + \alpha)\tau+t$ and $(h' + \alpha')\tau+t'$ is
$$(h-h')\tau+(\alpha-\alpha')\tau + (t -t')$$ and the associated weight is $$2(h-h')+2(\alpha - \alpha') +(t-t') +1$$

Since the seperation between 
$h'\tau+t'$ and $h\tau+t$ as well as 
$(h + \alpha)\tau+t$ and $(h' + \alpha')\tau+t'$ 
are not multiples of $\tau$, it means that the weight between 
$h'\tau+t'$ and $(h' + \alpha')\tau+t'$ is non-zero. 

The separation between $h'\tau+t'$ and $(h' + \alpha')\tau+t'$ is $$\alpha'\tau$$ and the associated weight is $$2\alpha' -1$$
The total parity weight is then
\begin{equation}
\begin{split}
w_p=&2(h'-h) +(t'-t) + 2(h-h')+2(\alpha - \alpha') +(t-t') +1 +2\alpha' -1\\
=&2(\alpha - \alpha' + \alpha') \\
=&2\alpha 
\end{split}
\end{equation}
\end{proof}

\begin{theorem}[Type3 W4RTZ]
For a W4RTZ such that $h\tau+t<h'\tau+t'<(h + \alpha)\tau+t<(h' + \alpha')\tau+t'$, the Hammming weight of the convolutional code $w_p=2(\alpha' -1+(h'-h)+(t'-t))$
\end{theorem}
\begin{proof}
The separation between $h'\tau+t'$ and $h\tau+t$ is
$$(h'-h)\tau + (t'-t)$$ and the associated weight is $$2(h'-h) +(t'-t)$$

The separation between $(h' + \alpha')\tau+t'$ and $(h + \alpha)\tau+t$ is
$$(h'-h)\tau+(\alpha'-\alpha)\tau + (t' -t)$$ and the associated weight is $$2(h'-h)+2(\alpha' - \alpha) +(t'-t)$$

Since the seperation between 
$h'\tau+t'$ and $h\tau+t$ as well as 
$(h' + \alpha')\tau+t'$ and $(h + \alpha)\tau+t$ 
are not multiples of $\tau$, it means that the weight between 
$(h' + \alpha')\tau+t'$ and $(h + \alpha)\tau+t$ is non-zero. 

The separation between $(h' + \alpha')\tau+t'$ and $(h + \alpha)\tau+t$ is 
$$(h-h')\tau +\alpha\tau +(t-t')$$ 
and the associated weight is $$2(h-h')+2\alpha -2$$
The total parity weight is then
\begin{equation}
\begin{split}
w_p=&2(h'-h) +(t'-t) + 2(h-h')+2\alpha -2 +2(h'-h) +2(\alpha' -\alpha)+(t'-t) \\
=&2(\alpha' +(h'-h) +(t'-t)-1)
\end{split}
\end{equation}
\end{proof}

The hamming weight with respect to all possible combinations of W4RTZ types is given below

\begin{enumerate}
\item 2 Type1 W4RTZ combinations
\begin{equation}
w_H^{(4)} = 6m+2\Big(\alpha+\alpha' +\alpha^{(\pi)} +\alpha^{(\pi)'}\Big)
\end{equation}

\item 2 Type2 W4RTZ combinations
\begin{equation}
w_H^{(4)} = 2(\alpha + \alpha^{(\pi)})
\end{equation}

\item 2 Type3 W4RTZ combinations
\begin{equation}
w_H^{(4)} =2(\alpha' +(h'-h) +(t'-t)) + 2(\alpha^{(\pi)'} +(h^{(\pi)'}-h^{(\pi)}) 
+(t^{(\pi)'}-t^{(\pi)}))
\end{equation}

\item A combination of a Type1 and Type2 W4RTZ
\begin{equation}
w_H^{(4)} =8+2(\alpha + \alpha^{(\pi)} + \alpha')
\end{equation}

\item A combination of a Type1 and Type3 W4RTZ
\begin{equation}
w_H^{(4)} =6+2(\alpha + \alpha')+2(\alpha^{(\pi)'} +(h^{(\pi)'}-h^{(\pi)}) 
+(t^{(\pi)'}-t^{(\pi)}))
\end{equation}

\item A combination of a Type2 and Type3 W4RTZ
\begin{equation}
w_H^{(4)} =2\Big((\alpha + \alpha^{(\pi)})+(h'-h)+(t'-t) - 1\Big)
\end{equation}

\end{enumerate}





%\begin{enumerate}
%\item Calculating weight for W4RTZs
%\item Calculating weight for W5RTZs

%\tem Search for good interleavers
%\item Simulation results comparison with QPP
%\end{enumerate}

\end{document}