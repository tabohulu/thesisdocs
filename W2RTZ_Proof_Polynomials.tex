\documentclass[11pt, oneside, dvipdfmx]{book}
\newcommand{\folder}{/usr/local/share/texmf}
%\newcommand{\folder}{/home/hanchenggao/Documents/texmf}
\input{\folder/hfiles/ebook}
%\usepackage[ruled,vlined]{algorithm2e}
\usepackage {graphicx}[dvips]
%\usepackage {graphics}
%\setCJKmainfont{SimSun}
\title{
Proving the Parity Weight Equation for RTZ Inputs Via Polynomials}
\author{Kwame Ackah Bohulu}
\date{\today}
\begin{document}

\maketitle

\newpage

\section{Parity Weight Equation for W2RTZs given the 5/7 RSC code}
\begin{theorem}
The parity weight equation for the 5/7 RSC code's W2RTZ is given by
\begin{equation}
w_p^{(2)} = 2\alpha+2
\end{equation}
\end{theorem}

\begin{proof}
In polynomial for, the parity-bit sequence $h(x)$ for any RSC code is given by 
\begin{equation}
h(x)=f(x)g^{-1}(x)b(x)
\label{eq1}
\end{equation}
where $f(x)=1+x^2$ and $b(x)$ is the message input and if it is an RTZ input and it can be written as
$$b(x)=a(x)g(x)$$

We then have
\begin{equation}
\begin{split}
h(x)=&f(x)g^{-1}(x)a(x)g(x)\\
=&f(x)a(x)
\end{split}
\label{eq2}
\end{equation}
Specifically, if it is a W2RTZ input for the 5/7 RSC code, $b(x)$ has the general for 
$$b(x)=1+x^{3\alpha}$$.
Given $g(x)$ and $b(x)$, it is possible to find $a(x)$ in its general form as it relates to W2RTZs simply by dividing $b(x)$ by $g(x)=1+x+x^2$
Then, we have 
\begin{equation}
\begin{split}
a(x)=&x^{3\alpha-2}+x^{3\alpha-3}+x^{3\alpha-5}+x^{3\alpha-6}+\cdots\\
=&\sum_{\alpha=1}^{i}x^{3(\alpha-1)+1}+x^{3(\alpha-1)}
\end{split}
\label{eq3}
\end{equation}

Fixing (\ref{eq3}) into (\ref{eq2}) we have
\begin{equation}
\begin{split}
h(x)=&f(x)\Big[\sum_{\alpha=1}^{i}x^{3(\alpha-1)+1}+x^{3(\alpha-1)} \Big]\\
=&1+x^2\Big[\sum_{\alpha=1}^{i}x^{3(\alpha-1)+1}+x^{3(\alpha-1)} \Big]\\
=&\sum_{\alpha=1}^{i}x^{3(\alpha-1)+1}+x^{3(\alpha-1)} + \sum_{\alpha=1}^{i}x^{3\alpha}+x^{3\alpha-1}
\end{split}
\end{equation}
\end{proof}
\end{document}