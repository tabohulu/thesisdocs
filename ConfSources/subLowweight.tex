\subsection{The Characteristics of Low-weight Codewords}
Since each RSC codeword is made up of two codeword components $b(x)$ and $h(x)$, it is obvious that the weight of the codeword $c(x)$ is given by 
\begin{equation}
w_H(c(x))=w_H(b(x)) + w_H(h(x))
\label{novelEq-1}
\end{equation} 
%(quarantine)%The distance spectrum derived via the transfer function method is an insufficient tool when it comes to to interleaver design. In this section, we present a novel method that generates what we refered to as the structured distance spectrum, which is the distance spectrum with the structure of the RTZ inputs as well as the corresponding parity-check sequence revealed, therefore making it a very useful tool for interleaver design.

%(quarantine)%For an RSC code, the Hamming weight of the codeword $w_H(\bc)$ is the sum of the weights of the parity bit sequence and message input. 
We first consider the parity check component, which can be expressed as 
\begin{equation}
h(x) =f(x)\cdot g^{-1}(x)\cdot b(x)
\label{novelEq0}
\end{equation}
If we consider large frame sizes, the presence of $g^{-1}(x)$ means that within $h(x)$ is a particular sequence of bits that is repeated a large number of times. This results in a large parity weight, and by extension, a relatively high-weight codeword. The only time this is not the case is when
\begin{equation}
b(x) \bmod g(x) \equiv 0
\label{novelEq1}
\end{equation}
This results in a relatively low-weight parity bit sequence, which might produce a low-weight codeword. Any $b(x)$ that meets the condition in (\ref{novelEq1}) can be written as 
\begin{equation}
b(x) =a(x)g(x)
\label{novelEq2}
\end{equation}
where $a(x)$ is a monic polynomial with $a_0=1$.
By fixing $b(x)$ from (\ref{novelEq2}) into (\ref{novelEq0}), we have 
\begin{equation}
\begin{split}
h(x)&=f(x)\cdot g^{-1}(x)\cdot a(x)g(x)\\
&=a(x)f(x)
\end{split}
\label{novelEq3}
\end{equation}
%(quarantine)%Using both (\ref{novelEq2})  and (\ref{novelEq3}), we wish to list all low-weight codewords for a given RSC code.A low-weight codeword is any codeword which satisfies the condition, $w_H(\bc) \leq d_{\text{max}}$. This list is known as the \textit{partial structured distance spectrum}. To generate the partial structured distance spectrum, we take note of a few things. 

%From (\ref{novelEq2})  and (\ref{novelEq3}), we observe that $a(x)$ is a common factor in both equations and if we are able to solve for $a(x)$ via either of the equations, the remaining equation can be solved. To solve for $a(x)$ requires that in either equation, it should be the only unknown variable. At first glance, it might seem that $g(x)$ and $f(x)$ are the only known variables because they are dependent on the RSC code in question. However, if we remember that the weight of $h(x)$ and $b(x)$ is directly proportional to the number of terms it has, then we are on our way to obtain our second known variable. What is left is to determine the valid power values for the polynomial  terms, depending on the weight of $h(x)$ or $b(x)$.

Thus, for a given $f(x)$ and $g(x)$, our goal is to find all $a(x)$s which generate low-weight codewords components in  (\ref{novelEq2}) and  (\ref{novelEq3}) simultaneously. 
However, since there is essentially no difference between the general structure of $h(x)$ and $b(x)$, we restrict our our attention to the low-weight parity check patterns in  (\ref{novelEq3}) and in the following, we present a method for determining valid values of $h(x)$ when $2 \leq w_H(h(x))\leq 3$. 
%In (\ref{novelEq3}), once $f(x)$ is given, our goal is to find $a(x)$ that results in a low-weight $h(x)$. To this end, we consider the roots of $f(x)$ 
%If $f(x)$ is a prime polynomial or can be factorized into prime polynomials, the the roots of $f(x)$ are its primitive elem
 %denoted by $\beta_i,~ 0 \leq i < 2^{m}-1)$. Then it is obvious that $h(\beta^i)=0$~ for all $\beta^i$ that are primitive elements 
%and we can reformulate our goal as to find weight-$w$ polynomials ($h(x)$) which take all the roots of $f(x)$ as its roots. The roots of $f(x)$ depend on its characteristic make-up and once that is known, we can easily determine the structure of $h(x)$ for a given value of $w_H(h(x))$. 
%The characteristic make-up of $f(x)$ can be grouped into the three cases below. 
%\begin{enumerate}
%\item Single primitive polynomial.
%\item Prime but not a primitive polynomial.
%\item Made up of repeated polynomial roots.
%\end{enumerate}


%It is worth noting that the method to be discussed can also be used to obtain valid values of $b(x)$, because there is no difference between the general structure of $h(x)$ and $b(x)$ once the Hamming weight is fixed. 