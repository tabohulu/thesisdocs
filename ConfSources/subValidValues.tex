\subsection{The patterns of the weight-2 PCs}
\label{sec:PC2}
Each weight-2 PC can be written as 
\begin{equation}
h(x)=1+x^a
\label{eq:wt2-gen-form}
\end{equation}
without loss of generality. Thus, we have from \eqref{Eq:rootcondition} that
\begin{equation}
(\beta_0^i)^a =1,~~~ 0 \leq i < \epsilon_0
\label{novelEq5b}
\end{equation}
On the other hand, the order $\epsilon_0$ is the least integer satisfying $\beta_0^{\epsilon_0} \equiv 1$, thus, $a$ should satisfy the condition
$$a \bmod \epsilon_0  \equiv 0$$

\begin{example}$f(x)=1+x+x^2$.\newline 
For this case, since $x^1=x$, $x^2 \equiv 1+x$, and $x^3 \equiv 1 \bmod f(x)$, the order of the root $\beta_0$ is $\epsilon_0=3$ and $a$ should be a multiple of $3$. The corresponding values for $a(x)$ and $h(x)$ are shown in Table \ref{novelTab2} for the first four valid values of $a$.
\begin{table}[htbp]
%\parbox{.5\linewidth}{
 \caption{$f(x)=1+x+x^2$}
\centering
 \begin{tabular}{c c c} 
%\hline
 $a(x)$ & $h(x)$ \\ [0.5ex] 
 \hline\hline
$1+x$
 & $1+x^{3}$ \\
\hline
$1+x+x^3+x^4$
 & $1+x^{6}$ 
 \\
\hline
$1+x+x^3+x^4+x^6+x^{7}$ 
&  $1+x^{9}$ 
\\
\hline
$1+x+x^3+x^4+x^6+x^{7}+x^9+x^{10}$
 &  $1+x^{12}$ \\
 \end{tabular}
 \label{novelTab2}
\end{table}
\label{ex-2}

We may write the weight-2 PCs in general form as $h(x)=1+x^{3\ell},~\ell>1$ and the corresponding $a(x)$ is given by 
\begin{equation*}
a(x)=\sum_{\ell=0}^{L-1} x^{3\ell}(1+x)
\end{equation*}
\label{ex-1}
\end{example}




\begin{example}
$f(x)=1+x+x^2+x^3+x^4$\newline
We can confirm that the order of $\beta_0$ is $\epsilon_0=5$. This means that $a$ should be a multiple of $5$. The corresponding values for $a(x)$ and $h(x)$ are shown in Table \ref{novelTab3} with general forms for $\ell>1$

%}
\begin{table}[htbp]
%\parbox{.5\linewidth}{
\caption{$f(x)=1+x+x^2+x^3+x^4$}
\centering
\begin{tabular}{c c} 
 \hline
 $a(x)=\sum_{\ell=0}^{L-1} x^{5\ell}(1+x)$ & $h(x)=1+x^{5\ell}$  \\ [0.5ex] 
 \hline\hline
$1+x$ &$1+x^5$\\ 
$1+x+x^5+x^6$ &$1+x^{10}$  \\
$1+x+x^5+x^6+x^{10}+x^{11}$ & $1+x^{15}$ \\
$1+x+x^5+x^6+x^{10}+x^{11}+x^{15}+x^{16}$ &$1+x^{20}$  
 \end{tabular}
 \label{novelTab3}
%}ll
\end{table}
\label{ex-2}
\end{example}

\begin{example}
	$f(x)=1+x^2$\newline
	Since 
	\[
	f(x)=(1+x)^2\]
	and the order of the root $\beta_0=1$ is $\epsilon_0=1$, we obtain from \eqref{Eq:rootcondition} and \eqref{Eq:differential}
	\begin{align}
		(\beta_0)^a = 1
		\label{Eq:example31}
	\end{align}
	\begin{align}
		a(\beta_0)^{(a-1)} = 0
		\label{Eq:example32}
	\end{align}	
	Although \eqref{Eq:example31} indicates $a$ can be any positive integer, we can see from \eqref{Eq:example32} that $a$ should be even number.
	The corresponding values for $a(x)$ and $h(x)$ are shown in Table \ref{novelTab1} with general forms for $\ell>1$.
	\begin{table}[htbp]
		\renewcommand{\arraystretch}{1.3}
		%\parbox{.3\linewidth}{
		\caption{$f(x)=1+x^2$}
		\centering
		\begin{tabular}{c c } 
			\hline
			$a(x)=\sum_{\ell=0}^{L-1} x^{2\ell}$ & $h(x)=1+x^{2\ell}$ \\ [0.5ex] 
			\hline\hline
			$1$ & $1+x^2$\\ 
			$1+x^2$ & $1+x^4$ \\
			$1+x^2+x^4$ & $1+x^6$\\
			$1+x^2+x^4+x^6$ & $1+x^8$ 
		\end{tabular}
		\label{novelTab1}
	\end{table}
\label{ex-3}
\end{example}



---------------------------
%=====================Deleted Examples  =======================%
%\begin{example}
%$f(x)=1+x^2+x^3+x^4$\newline
%$f(x)$ can be written as 
%$$f(x)=\prod_{k=0}^{1}f_k(x)$$
%where 
%$$f_0(x)=1+x,~f_1(x)=1+x+x^3$$ 
%For $f_0(x), x \equiv 1$, and $x^1 \equiv 1 \bmod f_0(x)$, which means the order of the root $\beta_0$ is $\epsilon_0=1$ and $a_0$ should be a multiple of $1$. Again, for  $f_1(x), x^3 \equiv 1+x$, and $x^7 \equiv 1 \bmod f_1(x)$, which means the order of the root $\beta_1$ is $\epsilon_1=7$ and $a_1$ should be a multiple of $7$.
%Finally, valid values of $a$ should be a multiple of the least common multiples of $a_0$ and $a_1$, which means $a$ should be a multiple of $7$.
%The corresponding values for $a(x)$ and $h(x)$ are shown in Table \ref{novelTab1-a} with general forms for $\ell>1$.
%\begin{table}[htbp]
%\renewcommand{\arraystretch}{1.3}
%\parbox{.3\linewidth}{
 %\caption{$f(x)=1+x^2+x^3+x^4$}
 %\centering
%\begin{tabular}{c c } 
%\hline
 %$a(x)=\sum_{\ell=0}^{L-1} x^{7\ell}(1+x^2+x^3)$ & $h(x)=1+x^{7\ell}$ \\ [0.5ex] 
%\hline\hline
%$1+x^2+x^3$ & $1+x^7$\\ 
%$1+x^2+x^3+x^7+x^9+x^{10}$ & $1+x^{14}$ \\
%$1+x^2+x^3+x^7+x^9+x^{10}+x^{14}+x^{16}+x^{17}$ & $1+x^{21}$
%\end{tabular}
% \label{novelTab1-a}
%\end{table}
%\end{example}

%\begin{example}
%$f(x)=1+x+x^2+x^3+x^4+x^5+x^6$\newline
%$f(x)$ can be written as 
%$$f(x)=\prod_{k=0}^{1}f_k(x)$$
%where 
%$$f_0(x)=1+x^2+x^3,~f_1(x)=1+x+x^3$$ 
%For $f_0(x), x^3 \equiv x^2+1$, and $x^7 \equiv 1 \bmod f_0(x)$, which means the order of the root $\beta_0$ is $\epsilon_0=7$ and $a_0$ should be a multiple of $7$. Again, for  $f_1(x), x^3 \equiv 1+x$, and $x^7 \equiv 1 \bmod f_1(x)$, which means the order of the root $\beta_1$ is $\epsilon_1=7$ and $a_1$ should be a multiple of $7$.
%Finally, valid values of $a$ should be a multiple of the least common multiples of $a_0$ and $a_1$, which means $a$ should be a multiple of $7$.
%The corresponding values for $a(x)$ and $h(x)$ are shown in Table \ref{novelTab1-b} with general forms for $\ell>1$.
%\begin{table}[htbp]
%\renewcommand{\arraystretch}{1.3}
%\parbox{.3\linewidth}{
% \caption{$f(x)=1+x+x^2+x^3+x^4+x^5+x^6$}
 %\centering
%\begin{tabular}{c c } 
%\hline
 %$a(x)=\sum_{\ell=0}^{L-1} x^{7\ell}(1+x)$ & $h(x)=1+x^{7\ell}$ \\ [0.5ex] 
%\hline\hline
%$1+x$ & $1+x^7$\\ 
%$1+x+x^7+x^8$ & $1+x^{14}$ \\
%$1+x+x^7+x^8+x^{14}+x^{15}$ & $1+x^{21}$
%\end{tabular}
% \label{novelTab1-b}
%\end{table}
%\end{example}
%=====================================End of deleted examples========================%
\begin{example}
$f(x)=1+x^2+x^3+x^4+x^6$\newline
$f(x)$ can be written as 
$$f(x)=\prod_{k=0}^{1}f_k(x)$$
where 
$$f_0(x)=1+x+x^2,~f_1(x)=1+x+x^2+x^3+x^4$$ 
From Example \ref{ex-1} and Example \ref{ex-2}, we know that $a_0=3$ and $a_1=5$.
Hence, valid values of $a$ should be a multiple of the least common multiples of $a_0$ and $a_1$, which means $a$ should be a multiple of $15$.
The corresponding values for $a(x)$ and $h(x)$ are shown in Table \ref{novelTab1-c} with general forms for $\ell>1$.
\begin{table}[htbp]
\renewcommand{\arraystretch}{1.3}
%\parbox{.3\linewidth}{
 \caption{$f(x)=1+x^2+x^3+x^4+x^6$}
 \centering
\begin{tabular}{c c } 
\hline
 $a(x)=\sum_{\ell=0}^{L-1} x^{15\ell}(1+x^2+x^3+x^6+x^7+x^9)$ & $h(x)=1+x^{15\ell}$ \\ [0.5ex] 
\hline\hline
$1+x^2+x^3+x^6+x^7+x^9$ & $1+x^{15}$\\ 
$1+x^2+x^3+x^6+x^7+x^9+x^{15}+x^{17}+x^{18}+x^{21}+x^{22}+x^{24}$ & $1+x^{30}$ \\
\end{tabular}
 \label{novelTab1-c}
\end{table}

%We may write the weight-2 PCs in general form as $h(x)=1+x^{7\ell},~\ell>1$ and the corresponding $a(x)$ is given by 
%\begin{equation*}
%a(x)=\sum_{\ell=0}^{L-1} x^{7\ell}(1+x)
%\end{equation*}
\end{example}



\subsection{The patterns of the weight-3 PCs}
Each weight-3 PC can be written as 
\begin{equation}
h(x)=1+x^a+x^b,~a\neq b
\label{novelEqwt3}
\end{equation}
without loss of generality. 
Fixing $\beta_0$ into $h(x)$ we have
\begin{equation}
\beta_0^a+\beta_0^b= 1
\label{novelEq5b}
\end{equation}

To solve \eqref{novelEq5b}, we refer to the table of the extended field for GF$(2^M)$, and we can find the valid $(\eta,~\zeta)$ pairs $\st X^{\eta}+X^{\zeta}=1$. If there are no valid $(\eta,~\zeta)$ pairs, then there is no parity check component of weight $3$ for the given $f(x)$. Depending on the Galois field, there might be multiple values of $(\eta,~\zeta)$ that satisfy \eqref{novelEq5b},
 so we represent the set of all $(\eta,~\zeta)$ pairs by $\cZ$. Then, any valid value $(a,~b)$ values should satisfy the condition
\begin{equation}
(a,b) \equiv (\eta,~\zeta) \bmod \epsilon_0,~(\eta,~\zeta)\in \cZ
\end{equation}
% since $\beta^{2^{m}-1}=1$.
\begin{example}
$f(x)=1+x+x^2$ \newline
The elements of GF$(2^2)$ are shown in Table \ref{novelTab7} and it is obvious that $\cZ=\{(1,2)\}$.
This means that $(a,b) = (3\ell+1,~3n+2),~l=n=\left\{0,1,...\right\}$.  The corresponding values for $a(x)$ and $h(x)$ are shown in the table below for the first four valid values of $(a,b)$.
\label{ex-5}
\end{example}

 \begin{table}[htbp]
 \caption{Non-zero Elements of GF$(2^2)$ generated by $f(x)=1+x+x^2$}
\centering
 \begin{tabular}{c c} 
 \hline
 power representation & actual value \\ [0.5ex] 
 \hline\hline
$X^0~=X^3=1$ & $1$\\
\hline
$X$ & $x$\\
\hline
$X^2$ &  $1+x$\\
\hline
 \end{tabular}
 \label{novelTab7}
\end{table}

\begin{table}[htbp]
 \caption{$f(x)=1+x+x^2$}
\centering
 \begin{tabular}{c c} 
 \hline
 $a(x)$ & $h(x)$\\ [0.5ex] 
 \hline\hline
$1$ & $1+x+x^2$\\ 
\hline
$1+x+x^2$ &  $1+x^2+x^4$\\
\hline
$1+x+x^3$ & $1+x^4+x^5$\\
\hline
$1+x^2+x^3$ & $1+x+x^5$ 
 \end{tabular}
 \label{novelTab8}
\end{table}

We may write the weight-3 PCs in general form as $h(x)=1+x^{3\ell+1}+x^{3n+2},~\ell,~n \geq 0$ 
%and the corresponding $a(x)$ is given by 
%\begin{equation*}
%a(x)=\sum_{\ell=0}^{L-1} x^{3\ell}(1+x)
%\end{equation*}
%\label{ex-1}



\paragraph{Case2}$f(x)$ can be factorised into multiple irreducible polynomials. \newline
For this case, we can write $f(x)$ as $$f(x)=\prod_{k=1}^{K}f_k(x)$$ where $f_k(x)$ is an irreducible polynomial of order $M_k$ with root $\beta \in $ GF$(2^{M_k})$, $\beta^{\epsilon_k}=1$. 
For each $f_k(x)$, we refer to the table of the extended field it generates and form the set $\cZ_k$, which contains all the valid $(\eta^{(k)},~\zeta^{(k)})$ pairs for that particular $f_k(x)$. If that set exists, then, for that $f_k(x)$ the following condition is met
\begin{equation}
\cA\cB_k=\{(a_k,~b_k)~|~(a_k,~b_k) \equiv (\eta^{(k)},~\zeta^{(k)}) \bmod \epsilon_k,~(\eta^{(k)},~\zeta^{(k)})\in \cZ_k\}
\end{equation}
and 
\begin{equation}
(a,~b) \in \bigcup_{k=1}^{K} \cA\cB_k
\end{equation}

For the special case where $f(x)$ can be factorised into equal irreducible polynomial, the above condition simplifies to 
 \begin{equation*}
 \begin{split}
 &(a,~b) \equiv (K\eta,~K\zeta) \bmod \epsilon,~(\eta,~\zeta) \in \cZ \\
 \end{split}
 \end{equation*}
 
 \begin{example}
 $f(x)=1+x^2$ \newline $f(x)$ can be written as $(1+x)^2$. $(1+x)$ is a primitive polynomial for GF$(2)$. The elements in GF$(2)$ are $1$ and $\beta$. In this field,  there are no valid $(e,f)$ pair values; therefore, $h(x)$ such that  $w_H(h(x))=3$ does not exist for $f(x)=1+x^2$.
\label{ex-6}
 \end{example}
 
