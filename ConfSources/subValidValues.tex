\subsection{The patterns of the weight-2 PCs}
Each weight-2 PC can be written as 
\begin{equation}
h(x)=1+x^a
\label{eq:wt2-gen-form}
\end{equation}
without loss of generality. Thus, we have 
\begin{equation}
\beta_k^a =1
\label{novelEq5b}
\end{equation}

Starting from the simplest case $K=1$, then, it is trivial that $a$ should satisfy the condition
$$a \bmod \epsilon_0  \equiv 0$$

\begin{example}$f(x)=1+x+x^2$.\newline 
For this case $x^2 \equiv 1+x$, and $x^3 \equiv 1 \bmod f(x)$, which means the order of the root $\beta_0$ is $\epsilon_0=3$ and $a$ should be a multiple of $3$. The corresponding values for $a(x)$ and $h(x)$ are shown in Table \ref{novelTab2} for the first four valid values of $a$.
\begin{table}[htbp]
%\parbox{.5\linewidth}{
 \caption{$f(x)=1+x+x^2$}
\centering
 \begin{tabular}{c c c} 
%\hline
 $a(x)$ & $h(x)$ \\ [0.5ex] 
 \hline\hline
$1+x$
 & $1+x^{3}$ \\
\hline
$1+x+x^3+x^4$
 & $1+x^{6}$ 
 \\
\hline
$1+x+x^3+x^4+x^6+x^{7}$ 
&  $1+x^{9}$ 
\\
\hline
$1+x+x^3+x^4+x^6+x^{7}+x^9+x^{10}$
 &  $1+x^{12}$ \\
 \end{tabular}
 \label{novelTab2}
\end{table}
\label{ex-2}
We may write the weight-2 PCs in general form as $h(x)=1+x^{3\ell},~\ell>1$ and the corresponding $a(x)$ is given by 
\begin{equation*}
a(x)=\sum_{\ell=0}^{L-1} x^{3\ell}(1+x)
\end{equation*}
\end{example}
%============================
%\begin{example}$g(x)=1+x+x^4$\newline
%$g(x)$ can be used to generate the extended field GF$(2^4)$. In this field, $\beta^{15}=1$. The valid values of $a$ are $a=\{15,30,45,\cdots \}$. The corresponding values for $a(x)$ and $b(x)$ are shown in the table below for the first two valid values of $a$.
% \begin{table*}[h]
 %\caption{$23/35$ RSC Code, $f(x)=1+x+x^4$}
%\centering
%\begin{tabular}{p{4cm} | c} 
% \hline
 %$a(x)$ & $b(x)$  \\ [0.5ex] 
 %\hline\hline
%$1+x^2+x^3+x^5+x^7+x^8+x^{11}$ 
%& $1+x^{15}$ \\ 
%\hline
%$1+x^2+x^3+x^5+x^7+x^8+x^{11}+x^{15}+x^{16}+x^{17}+x^{18}+x^{20}+x^{22}+x^{23}+x^{26}$ 
%&$1+x^{30}$\\
 %\end{tabular}
 %\label{novelTab5}
%\end{table*}
%\end{example}
%=====================
%\paragraph{Case2}$f(x)$ is prime polynomial but not primitive \newline
%Similar to the case for primitive polynomials, we need to find the order of $\beta$. For fields generated by prime polynomials, there is a value $j < 2^m-1$ such that 
%$$\beta^j=1,~j~|~2^{m}-1 $$ where $j$ is the order. Therefore, any valid value of $a$ should satisfy the condition below.
%$$ a \bmod j \equiv 0$$

\begin{example}
$f(x)=1+x+x^2+x^3+x^4$\newline
Since $x^4 \equiv 1+x+x^2+x^3,~x^5 \equiv 1 \bmod f(x)$ and the order of $\beta_0$ is $\epsilon_0=5$. This means that $a$ should be a multiple of $5$. The corresponding values for $a(x)$ and $h(x)$ are shown in Table \ref{novelTab3} for the first four valid values of $a$.

%}
\begin{table}[htbp]
%\parbox{.5\linewidth}{
\caption{$f(x)=1+x+x^2+x^3+x^4$}
\centering
\begin{tabular}{c c} 
 \hline
 $a(x)$ & $h(x)$  \\ [0.5ex] 
 \hline\hline
$1+x$ &$1+x^5$\\ 
$1+x+x^5+x^6$ &$1+x^{10}$  \\
$1+x+x^5+x^6+x^{10}+x^{11}$ & $1+x^{15}$ \\
$1+x+x^5+x^6+x^{10}+x^{11}+x^{15}+x^{16}$ &$1+x^{20}$  
 \end{tabular}
 \label{novelTab3}
%}ll
\end{table}
\label{ex-3}
We may write the weight-2 PCs in general form as $h(x)=1+x^{5\ell},~\ell>1$ and the corresponding $a(x)$ is given by 
\begin{equation*}
a(x)=\sum_{\ell=0}^{L-1} x^{5\ell}(1+x)
\end{equation*}
\end{example}

%\paragraph{ Case3: $g(x)$ is made up of repeated polynomial roots.\newline}
%Given the above condition we have, $r(x)=(r^{o_p}_p(x))^k$, where $r^{o_p}_p(x)$ represents the prime polynomial $r(x)$ can be factorised into and $k$ is the number of times it is repeated. $r^{o_p}_p(x)$ has $\beta$ as its roots and we need to find $d \st \beta^j=1$ in the (extended) field it generates. Because $\beta^{kj}=1$ in the same (extended) field , the valid values for $a$ should satisfy the condition below:
%$$kj ~| ~a$$

For the case of $K > 1$
\[
	f_{\beta_k}(x) = (x-\beta_k)\left(x-\beta_k^{2}\right)\left(x-\beta_k^{2^2}\right)\cdots\left(x-\beta_k^{2^{T_k}}\right)
\]
where $T_k$ is the smallest integer such that $2^{T_k} \equiv 1 \mod \epsilon_k$ 

For this case, we can write $f(x)$ as $$f(x)=\prod_{k=0}^{K-1}f_k(x)$$ where $f_k(x)$ is an irreducible polynomial of order $M_k$ with root $\beta \in $ GF$(2^{M_k})$, $\beta^{\epsilon_k}=1$. 
For each $f_k(x)$, the valid values of  $a_k$ are such that 
$$ \cA_k=\{a_k~ |~ a_k \bmod \epsilon_k \equiv 0\}$$ and the valid values of $a$ are such that
$$a \in  \bigcap_{k=0}^{K-1} \cA_k$$
This means that $a$ satisfies the condition
$$ a \bmod  \prod_{k=0}^{K-1} \epsilon_k \equiv 0$$
For the special case where $f(x)$ can be factorised into equal irreducible polynomials, the above condition simplifies to 
$$a \bmod \epsilon K \equiv 0$$

\begin{example}
$f(x)=1+x^2+x^3+x^4$\newline
$f(x)$ can be written as 
$$f(x)=\prod_{k=0}^{1}f_k(x)$$
where 
$$f_0(x)=1+x,~f_1(x)=1+x+x^3$$ 
For $f_0(x), x \equiv 1$, and $x^1 \equiv 1 \bmod f_0(x)$, which means the order of the root $\beta_0$ is $\epsilon_0=1$ and $a_0$ should be a multiple of $1$. Again, for  $f_1(x), x^3 \equiv 1+x$, and $x^7 \equiv 1 \bmod f_1(x)$, which means the order of the root $\beta_1$ is $\epsilon_1=7$ and $a_1$ should be a multiple of $7$.
Finally, valid values of $a$ should be a multiple of the least common multiples of $a_0$ and $a_1$, which means $a$ should be a multiple of $7$.
The corresponding values for $a(x)$ and $h(x)$ are shown in Table \ref{novelTab1-a} for the first 3 valid values of $a$.
\begin{table}[htbp]
\renewcommand{\arraystretch}{1.3}
%\parbox{.3\linewidth}{
 \caption{$f(x)=1+x^2+x^3+x^4$}
 \centering
\begin{tabular}{c c } 
\hline
 $a(x)$ & $h(x)$ \\ [0.5ex] 
\hline\hline
$1+x^2+x^3$ & $1+x^7$\\ 
$1+x^2+x^3+x^7+x^9+x^{10}$ & $1+x^{14}$ \\
$1+x^2+x^3+x^7+x^9+x^{10}+x^{14}+x^{16}+x^{17}$ & $1+x^{21}$
\end{tabular}
 \label{novelTab1-a}
\end{table}

We may write the weight-2 PCs in general form as $h(x)=1+x^{7\ell},~\ell>1$ and the corresponding $a(x)$ is given by 
\begin{equation*}
a(x)=\sum_{\ell=0}^{L-1} x^{7\ell}(1+x^2+x^3)
\end{equation*}
\end{example}

\begin{example}
$f(x)=1+x+x^2+x^3+x^4+x^5+x^6$\newline
$f(x)$ can be written as 
$$f(x)=\prod_{k=0}^{1}f_k(x)$$
where 
$$f_0(x)=1+x^2+x^3,~f_1(x)=1+x+x^3$$ 
For $f_0(x), x^3 \equiv x^2+1$, and $x^7 \equiv 1 \bmod f_0(x)$, which means the order of the root $\beta_0$ is $\epsilon_0=7$ and $a_0$ should be a multiple of $7$. Again, for  $f_1(x), x^3 \equiv 1+x$, and $x^7 \equiv 1 \bmod f_1(x)$, which means the order of the root $\beta_1$ is $\epsilon_1=7$ and $a_1$ should be a multiple of $7$.
Finally, valid values of $a$ should be a multiple of the least common multiples of $a_0$ and $a_1$, which means $a$ should be a multiple of $7$.
The corresponding values for $a(x)$ and $h(x)$ are shown in Table \ref{novelTab1-b} for the first three valid values of $a$.
\begin{table}[htbp]
\renewcommand{\arraystretch}{1.3}
%\parbox{.3\linewidth}{
 \caption{$f(x)=1+x+x^2+x^3+x^4+x^5+x^6$}
 \centering
\begin{tabular}{c c } 
\hline
 $a(x)$ & $h(x)$ \\ [0.5ex] 
\hline\hline
$1+x$ & $1+x^7$\\ 
$1+x+x^7+x^8$ & $1+x^{14}$ \\
$1+x+x^7+x^8+x^{14}+x^{15}$ & $1+x^{21}$
\end{tabular}
 \label{novelTab1-b}
\end{table}

We may write the weight-2 PCs in general form as $h(x)=1+x^{7\ell},~\ell>1$ and the corresponding $a(x)$ is given by 
\begin{equation*}
a(x)=\sum_{\ell=0}^{L-1} x^{7\ell}(1+x)
\end{equation*}
\end{example}

\begin{example}
$f(x)=1+x^2$\newline
$f(x)$ can be written as $$f(x)=(1+x)^2$$ $1+x$ is prime in $GF(2)$, where $\epsilon =1$. Since $f(x)$ is made up of equal repeated polynomial and $K=2$, the valid values of $a=\{2,4,6,\cdots \}$.
The corresponding values for $a(x)$ and $h(x)$ are shown in Table \ref{novelTab1} for the first four valid values of $a$.
\begin{table}[htbp]
\renewcommand{\arraystretch}{1.3}
%\parbox{.3\linewidth}{
 \caption{$f(x)=1+x^2$}
 \centering
\begin{tabular}{c c } 
\hline
 $a(x)$ & $h(x)$ \\ [0.5ex] 
\hline\hline
$1$ & $1+x^2$\\ 
$1+x^2$ & $1+x^4$ \\
$1+x^2+x^4$ & $1+x^6$\\
$1+x^2+x^4+x^6$ & $1+x^8$ 
\end{tabular}
 \label{novelTab1}
\end{table}

We may write the weight-2 PCs in general form as $h(x)=1+x^{2\ell},~\ell>1$ and the corresponding $a(x)$ is given by 
\begin{equation*}
a(x)=\sum_{\ell=0}^{L-1} x^{2\ell}(1)
\end{equation*}
\end{example}

%==========
%\begin{example}
%$g(x)=1+x^4$\newline $g(x)$ is made up of equal repeated polynomial roots and can be written as $$g(x)=(1+x)^4,~k=4$$. $1+x$ is prime in $GF(2)$ and $\beta^{1}=1$. Since $k=4$, we have $\beta^{k}=\beta^{4}=1$. $b(x)=1+x^b$ and the valid values of $a=\{4,8,12,\cdots \}$.
%The corresponding values for $a(x),~b(x)$ and $h(x)$ are shown in the table below for the first four valid values of $a$

%\begin{table*}[h]
%\caption{$g(x)=1+x^4$}
%\centering
 %\begin{tabular}{c c c} 
 %\hline
%$a(x)$ & $h(x)$ \\ [0.5ex] 
%\hline\hline
%$1$ &  $1+x^4$\\ 
%$1+x^4$ & $1+x^8$ \\
%$1+x^4+x^8$ & $1+x^{12}$ \\
%$1+x^4+x^8+x^{12}$ & $1+x^{16}$ 
%\end{tabular}
%\label{novelTab4}
%\end{table*}
%\end{example}

%========
%\begin{example}
%$g(x)=1+x^2+x^3+x^4$\newline
%$g(x)$ can be written as $$g(x)=(1+x)(1+x+x^3),~K=2$$
 %$1+x$ is prime in $GF(2^1)$ and $\beta^{1}=1$. $1+x+x^3$ is prime in $GF(2^3)$ and $\beta^{7}=1$. $b(x)=1+x^b$ and consequently, the valid values of $a$ that meet the condition $$ \bigcap_{k=1}^{K} \{j_k~| a\}$$ are $a=\{7,14,21,\cdots \}$.
%The corresponding values for $a(x)$ and $b(x)$ are shown in Table \ref{novelTab6} for the first four valid values of $a$
%\end{example}


%\hfill
%\parbox{\linewidth}{
%\begin{table*}[h]
%\caption{$g(x)=1+x^2+x^3+x^4$}
%\centering
% \begin{tabular}{p{4cm}| c} 
 %\hline
 %$a(x)$ & $b(x)$  \\ [0.5ex] 
 %\hline\hline
%$1+x^2+x^3$ & $1+x^7$ \\ 
%\hline
%$1+x^2+x^3+x^7+x^9+x^{10}$ &  $1+x^{14}$ \\
%\hline
%$1+x^2+x^3+x^7+x^{9}+x^{10}+x^{14}+x^{16}+x^{17}$ & $1+x^{21}$ 
%\\
%\hline
%$1+x^2+x^3+x^7+x^{9}+x^{10}+x^{14}+x^{16}+x^{17}+x^{21}+x^{23}+x^{24}$ & $1+x^{28}$
% \end{tabular}
 %\label{novelTab6}
%\end{table*}
%}
%\end{table}


\subsection{The patterns of the weight-3 PCs}
Each weight-3 PC can be written as 
\begin{equation}
h(x)=1+x^a+x^b,~a\neq b
\label{novelEqwt3}
\end{equation}
without loss of generality. Thus, we have 
\begin{equation}
\beta_k^a+\beta_k^b= 1
\label{novelEq5b}
\end{equation}

Starting from the simplest case $K=1$, we refer to the table of the extended field for GF$(2^M)$, and we can find the valid $(\eta,~\zeta)$ pairs $\st \beta^{\eta}+\beta^{\zeta}=1$. If there are no valid $(\eta,~\zeta)$ pairs, then there is no parity check component of weight $3$ for the given $f(x)$.
 We represent the set of $(\eta,~\zeta)$ pairs as 
$\cZ=\{ (\eta_1,~\zeta_1) ,( \eta_2,~\zeta_2),\cdots\}$. Then, any valid value $(a,~b)$ values should satisfy the condition
\begin{equation}
(a,b) \equiv (\eta,~\zeta) \bmod \epsilon,~(\eta,~\zeta)\in \cZ
\end{equation}
% since $\beta^{2^{m}-1}=1$.
\begin{example}
$f(x)=1+x+x^2$ \newline
The elements of GF$(2^2)$ are shown in Table \ref{novelTab7} and it is obvious that there is exactly 1 valid $(\eta,\zeta)$ pair $\st \beta^{\eta}+\beta^{\zeta} = 1$ and that is the pair $(1,2)$.
This means that valid values of the $(a,b)$ pairs are any values $\st (a,b) \equiv (1,2) \bmod 3$.  The corresponding values for $a(x)$ and $h(x)$ are shown in the table below for the first four valid values of $(a,b)$.
\label{ex-5}
\end{example}

 \begin{table}[htbp]
 \caption{Non-zero Elements of GF$(2^2)$ generated by $f(x)=1+x+x^2$}
\centering
 \begin{tabular}{c c} 
 \hline
 power representation & actual value \\ [0.5ex] 
 \hline\hline
$\beta^0~=\beta^3=1$ & $1$\\
\hline
$\beta$ & $\beta$\\
\hline
$\beta^2$ &  $1+\beta$\\
\hline
 \end{tabular}
 \label{novelTab7}
\end{table}

\begin{table}[htbp]
 \caption{$f(x)=1+x+x^2$}
\centering
 \begin{tabular}{c c} 
 \hline
 $a(x)$ & $h(x)$\\ [0.5ex] 
 \hline\hline
$1$ & $1+x+x^2$\\ 
\hline
$1+x+x^2$ &  $1+x^2+x^4$\\
\hline
$1+x+x^3$ & $1+x^4+x^5$\\
\hline
$1+x^2+x^3$ & $1+x+x^5$ 
 \end{tabular}
 \label{novelTab8}
\end{table}

%\paragraph{Case2}$r(x)$ is prime but not a primitive polynomial\newline
%Similar to the case where $f(x)$ is primitive, we confirm the existence of $(e,f)$ pairs $ \st \beta^e + \beta^f =1$. If there are no values for the pair $(e,f)$, then there is no $h(x)$ such that $w_H(h(x)) =3$ for the given $f(x)$.
%We represent the set of $(e,f)$ pairs as 
%$\bz=\{ (e_1,f_1) , e_2,f_2),\cdots\} $. Then any valid value for $a$ and $b$ should satisfy the condition
%$$(a,b) \equiv (e,f) \bmod j,~(e,f)\in \bz$$ where $j$ is the order of $\beta$.

%\begin{example}
%$f(x)=1+x+x^2+x^3+x^4$ \newline
%From the table of the extended field generated by $f(x)$ (Table \ref{novelTab9}), we see that there are no valid $(e,~f)$ and as such $h(x) \st w_H(h(x))=3$ is non-existent for $f(x)=1+x+x^2+x^3+x^4$.


%\begin{table*}[h!]
% \parbox{\linewidth}{
%\caption{Non-zero Elements of GF$(2^4)$ generated by $f(x)=1+x+x^2+x^3+x^4$}
%\centering
 %\begin{tabular}{c c} 
 %\hline
 %power & polynomial \\ [0.5ex] 
% \hline\hline
%$\beta^0~=\beta^5=\beta^{10}=\beta^{15}=1$ & $1$\\
%\hline
%$\beta=\beta^6=\beta^{11}$ & $\beta$\\
%\hline
%$\beta^2=\beta^7=\beta^{12}$ &  $\beta^2$\\
%\hline
%$\beta^3=\beta^8=\beta^{13}$ &  $\beta^3$\\
%\hline
%$\beta^4=\beta^9=\beta^{14}$ &  $\beta^3+\beta^2+\beta+1$\\
 %\end{tabular}
% \label{novelTab9}
%}
%\end{table*}

%\end{example}

%\paragraph{Case3: $r(x)$ is made up of equal repeated polynomial roots\newline}
%For this case, we can write $r(x)$ as 
%$r(x)=(r^{o_p}_p(x))^k$, where $r^{o_p}_p(x)$ represents the prime polynomial, which is the root of $r(x)$ and $k$ is the number of times it is repeated. If $r^{o_p}_p(x)$ is a primitive polynomial and $m>1$, then from Case1 there are exactly $q=2^{(m-1)}-1~(e,~f)$ pairs $\st \beta^e+\beta^f=1,~e \neq f$ in set $\bz$. 
%If $r^{o_p}_p(x)$ is prime but not a primitive polynomial, then we determine the number of elements in the set $\bz$ directly from the table representing the field it generates.
% If we focus on $f^{o_p}_p(x)$ only, $(u',v')$ should satisfy the condition$$ (u',v') \equiv (e,f)\bmod 2^{m}-1,~(e, f) \in \bz$$. However, since $f(x)$ is made up of $f^{o_p}_p(x)$ repeated $k$ times, and $\beta^{ke}+\beta^{kf}=1$, the valid values for $u$ and $v$
 %should satisfy the condition
% $$(u,v)=(ku',kv'),~(u,v) \equiv (e,f)\bmod 2^{m}-1,~(e, f) \in \bz $$.

\paragraph{Case2}$f(x)$ can be factorised into multiple irreducible polynomials. \newline
For this case, we can write $f(x)$ as $$f(x)=\prod_{k=1}^{K}f_k(x)$$ where $f_k(x)$ is an irreducible polynomial of order $M_k$ with root $\beta \in $ GF$(2^{M_k})$, $\beta^{\epsilon_k}=1$. 
For each $f_k(x)$, we refer to the table of the extended field it generates and form the set $\cZ_k$, which contains all the valid $(\eta^{(k)},~\zeta^{(k)})$ pairs for that particular $f_k(x)$. If that set exists, then, for that $f_k(x)$ the following condition is met
\begin{equation}
\cA\cB_k=\{(a_k,~b_k)~|~(a_k,~b_k) \equiv (\eta^{(k)},~\zeta^{(k)}) \bmod \epsilon_k,~(\eta^{(k)},~\zeta^{(k)})\in \cZ_k\}
\end{equation}
and 
\begin{equation}
(a,~b) \in \bigcup_{k=1}^{K} \cA\cB_k
\end{equation}

For the special case where $f(x)$ can be factorised into equal irreducible polynomial, the above condition simplifies to 
 \begin{equation*}
 \begin{split}
 &(a,~b) \equiv (K\eta,~K\zeta) \bmod \epsilon,~(\eta,~\zeta) \in \cZ \\
 \end{split}
 \end{equation*}
 
 \begin{example}
 $f(x)=1+x^2$ \newline $f(x)$ can be written as $(1+x)^2$. $(1+x)$ is a primitive polynomial for GF$(2)$. The elements in GF$(2)$ are $1$ and $\beta$. In this field,  there are no valid $(e,f)$ pair values; therefore, $h(x)$ such that  $w_H(h(x))=3$ does not exist for $f(x)=1+x^2$.
\label{ex-6}
 \end{example}
 
  %\begin{example}
 %$g(x) = 1+x^4$.\newline
 %After factorisation, we have $g(x)=(1+x)^4. $(1+x) is a primitive polynomial that generates the field GF$(2)$ and in this field, there are also no valid $(e,~f)$ and as such $b(x) \st w_H(\bb)=3$ is also non-existent.
 %\end{example}
 
  %\begin{example}
 %$g(x)=1+x^2+x^3+x^4$.\newline
 % Upon factorising, we have $g(x)=(1+x)(1+x+x^3)$. Table \ref{novelTab12} shows the elements of GF$(2^3)$ generated by $(1+x+x^3)$ and we can confirm that there are three valid $(e,~f)$ pairs. However, since there are no valid $(e,~f)$ pairs in GF$(2)$, it also means that there cannot be any valid $(u,~v)$ pairs and $b(x) \st w_H(\bb)=3$ does not exist.
 
 %\begin{table*}[h]
 %\caption{Non-zero Elements of GF$(2^3)$ generated by $1+x+x^3$}
%\centering
 %\begin{tabular}{c c} 
 %\hline
 %power & polynomial \\ [0.5ex] 
 %\hline\hline
%$\beta^0~=\beta^{7}=1$ & $1$\\
%\hline
%$\beta$ & $\beta$\\
%\hline
%$\beta^2$ &  $\beta^2$\\
%\hline
%$\beta^3$ & $\beta+1$\\
%\hline
%$\beta^4$ &  $\beta^2+\beta$\\
%\hline
%$\beta^5$ & $\beta^2+\beta+1$\\
%\hline
%$\beta^6$ &  $\beta^2+1$\\
 %\end{tabular}
 %\label{novelTab12}
%\end{table*}
 %\end{example}

%\begin{example}
%$5/7$ RSC Code, $f(x)=1+x^2,~g(x)=1+x+x^2$\newline
%$f(x)$ is a match for Case3 and can be written as $(1+x)^2$. $(1+x)$ is a primitive polynomial for GF$(2)$. The elements in GF$(2)$ are $1$ and $\beta$. In this field,  there are no valid $(e,f)$; therefore, $h(x) \st w_H(\bh)=3$ does not exist.

%$g(x)$ is a match for Case1, i.e. it is a primitive polynomial for GF$(2^2)$ with $\beta^{3}=1$. 
%The elements of GF$(2^2)$ are shown in Table \ref{novelTab5} and it is obvious that there is exactly 1 ($q=2^{(2-1)}-1~=1$) valid $(e,f)$ pair $\st \beta^e+\beta^f = 1$ and that is $(1,2)$.
%This means that valid values of the $(u,v)$ pairs are any values $\st (u,v) \equiv (1,2) \bmod 3$.  The corresponding values for $a(x),~b(x)$ and $h(x)$ are shown in the table below for the first four valid values of $(u,v)$

% \begin{table*}[h]
 %\caption{Non-zero Elements of GF$(2^2)$ generated by $g(x)=1+x+x^2$}
%\centering
% \begin{tabular}{c c} 
% \hline
 %power representation & actual value \\ [0.5ex] 
% \hline\hline
%$\beta^0~=\beta^3=1$ & $1$\\
%\hline
%$\beta$ & $\beta$\\
%\hline
%$\beta^2$ &  $1+\beta$\\
% \end{tabular}
 %\label{novelTab7}
%\end{table*}

%\begin{table*}[h]
% \caption{$5/7$ RSC, $g(x)=1+x+x^2$}
%\centering
% \begin{tabular}{c c c} 
% \hline
% $a(x)$ & $b(x)$ & $h(x)$\\ [0.5ex] 
% \hline\hline
%$1$ & $1+x+x^2$ & $1+x^2$\\ 
%\hline
%$1+x+x^2$ &  $1+x^2+x^4$ & $1+x+x^3+x^4$ \\
%\hline
%$1+x+x^3$ & $1+x^4+x^5$ & $1+x+x^2+x^5$\\
%\hline
%$1+x^2+x^3$ & $1+x+x^5$  &$1+x^3+x^4+x^5$
 %\end{tabular}
 %\label{novelTab8}
%\end{table*}
%\end{example}

%\newpage

 
 %\begin{example}
%$37/21$ RSC Code, $f(x)=1+x+x^2+x^3+x^4,~g(x)=1+x^4$\newline
%$f(x)$ is a match for Case2, i.e. it is a prime but not a primitive polynomial. From the table of the extended field generated by $f(x)$ (Table \ref{novelTab9}), we see that there are no valid $(e,~f)$ and as such $h(x) \st w_H(\bh)=3$ is non-existent.


 %\begin{table*}[h]
 %\caption{Non-zero Elements of GF$(2^2)$ generated by $f(x)=1+x+x^2+x^3+x^4$}
%\centering
 %\begin{tabular}{c c} 
 %\hline 
 %power & polynomial \\ [0.5ex] 
 %\hline\hline
%$\beta^0~=\beta^5=\beta^{10}=\beta^{15}=1$ & $1$\\
%\hline
%$\beta=\beta^6=\beta^{11}$ & $\beta$\\
%\hline
%$\beta^2=\beta^7=\beta^{12}$ &  $\beta^2$\\
%\hline
%$\beta^3=\beta^8=\beta^{13}$ &  $\beta^3$\\
%\hline
%$\beta^4=\beta^9=\beta^{14}$ &  $\beta^3+\beta^2+\beta+1$\\
 %\end{tabular}
 %\label{novelTab9}
%\end{table*}

%$g(x)$ is a match for Case3, i.e. it is made up of equal repeated roots. After factorisation, we have $g(x)=(1+x)^4. $(1+x) is a primitive polynomial that generates the field GF$(2)$ and in this field, there are also no valid $(e,~f)$ and as such $b(x) \st w_H(\bb)=3$ is also non-existent.
%\end{example}
 
%\begin{example}
%$23/35$ RSC Code, $f(x)=1+x+x^4,~g(x)=1+x^2+x^3+x^4$ \newline
%$f(x)$ is a match for Case1, i.e. it is a primitive polynomial that can be used to generate the extended field GF$(2^4)$ with $\beta^{15}=1$.. The elements of GF$(2^4)$ are shown in Table \ref{novelTab10} and we can see that there are seven valid $(e,f)$ pairs.
 
 %This means that the valid values of the $(u,v)$ pairs are any values $\st (u,v) \equiv (e,f) \bmod 15, (e,f) \in \bz,~\bz=\{(1,4),~(2,8) ,~(3,14) ,~(5,10) ,~(6,13) ,~(7,9) ,~(11,12))$.
 %The corresponding values for $a(x),~b(x)$ and $h(x)$ are shown in Table \ref{novelTab11} below for the first four valid values of $(u,v)$
 
  %\begin{table*}[h]
 %\caption{Non-zero Elements of GF$(2^4)$ generated by $f(x)=1+x+x^4$}
%\centering
 %\begin{tabular}{c c} 
 %\hline
 %power & polynomial \\ [0.5ex] 
 %\hline\hline
%$\beta^0~=\beta^{15}=1$ & $1$\\
%\hline
%$\beta$ & $\beta$\\
%\hline
%$\beta^2$ &  $\beta^2$\\
%\hline
%$\beta^3$ & $\beta^3$\\
%\hline
%$\beta^4$ &  $\beta+1$\\
%\hline
%$\beta^5$ & $\beta^2+\beta$\\
%\hline
%$\beta^6$ &  $\beta^3+\beta^2$\\
%\hline
%$\beta^7$ & $\beta^3+\beta+1$\\
%\hline
%$\beta^8$ &  $\beta^2+1$\\
%\hline
%$\beta^9$ & $\beta^3+\beta$\\
%\hline
%$\beta^{10}$ &  $\beta^2+\beta+1$\\
%\hline
%$\beta^{11}$ & $\beta^3+\beta^2+\beta$\\
%\hline
%$\beta^{12}$ &  $\beta^3+\beta^2+\beta+1$\\
%\hline
%$\beta^{13}$ & $\beta^3+\beta^2+1$\\
%\hline
%$\beta^{14}$ &  $\beta^3+1$\\
% \end{tabular}
% \label{novelTab10}
%\end{table*}
 
% \begin{table*}[h]
% \caption{$23/35$ RSC, $f(x)=1+x+x^4$}
%\centering
% \begin{tabular}{c c c} 
% \hline
% $a(x)$ & $b(x)$ & $h(x)$\\ [0.5ex] 
% \hline\hline
%$1$ & $1+x^2+x^3+x^4$ & $1+x+x^4$\\ 
%\hline
%$1+x+x^2+x^3+x^5$ &  $1+x^2+x^3+x^4+x^8+x^9$ & $1+x^7+x^9$ \\
%\hline
%$1+x+x^2+x^3+x^5+x^7+x^8$ & $1+x^2+x^3+x^4+x^7+x^{12}$ & $1+x^{11}+x^{12}$\\
%\hline
%$1+x+x^4$ & $1+x+x^2+x^4+x^5+x^6+x^7+x^8$  &$1+x^2+x^8$
 %\end{tabular}
% \label{novelTab11}
%\end{table*}
 
% $g(x)$ is a match for Case4, i.e. it is made up of unique repeated roots. Upon factorising, we have $g(x)=(1+x)(1+x+x^3)$. Table \ref{novelTab12} shows the elements of GF$(2^3)$ generated by $(1+x+x^3)$ and we can confirm that there are three valid $(e,~f)$ pairs. However, since there are no valid $(e,~f)$ pairs in GF$(2)$, it also means that there cannot be any valid $(u,~v)$ pairs and $b(x) \st w_H(\bb)=3$ does not exist.
 
% \begin{table*}[h]
% \caption{Non-zero Elements of GF$(2^3)$ generated by $1+x+x^3$}
%\centering
 %\begin{tabular}{c c} 
% \hline
 %power & polynomial \\ [0.5ex] 
% \hline\hline
%$\beta^0~=\beta^{7}=1$ & $1$\\
%\hline
%$\beta$ & $\beta$\\
%\hline
%$\beta^2$ &  $\beta^2$\\
%\hline
%$\beta^3$ & $\beta+1$\\
%\hline
%$\beta^4$ &  $\beta^2+\beta$\\
%\hline
%$\beta^5$ & $\beta^2+\beta+1$\\
%\hline
%$\beta^6$ &  $\beta^2+1$\\
%\end{tabular}
% \label{novelTab12}
%\end{table*}
%\end{example}



%\subsubsection{Higher order weight:$w_H=4$}%: Equal Repeated Polynomial Roots}
%If $w_H(\bq)=4$, then $q(x) = 1+x^{u}+x^{v}+x^{w}$.
%Finding the structure of $q(x)$ when $w_H=4$ for general cases has proven to be a difficult problem, for which we have not found a feasible solution to. However, there is a special case where have been able to come up with a method for determining the structure of $q(x)$. 

%\paragraph{$r(x)=1+x^{2i},~i=1,2,...$}
%Since $q(x)$ is divisible by $r(x)$, the roots of $r(x)$ are also the roots of $q(x)$. Inserting $\beta$ into $q(x)$, we get

%$$1+\beta^{u}+\beta^{v}+\beta^{w} =0$$
%Taking the derivative of the above equation, we get 
%\begin{equation*}
%\begin{split}
% &u\beta^{u-1}+v\beta^{v-1}+w\beta^{w-1} =0 \\
% & u\beta^{u-1}+v\beta^{v-1}+w\beta^{w-1} \equiv 0 \bmod 2
% \end{split}
% \end{equation*}
%We can see that to satisfy the equation above, there are 2 possible cases
%\begin{enumerate}
%\item $(u,~v,~w)$ is made up of two odd numbers, one even number.
%\item $(u,~v,~w)$ are all even numbers.
%\end{enumerate}

%For a valid $(u,~v,~w)$, we fix it into $q(x)$ and if it is divisible by $q(x)$, then $(u,~v,~w)$ is valid.
%In summary, valid $(u,~v,~w)$ should satisfy the condition
%$$u+v+w \equiv 0\bmod 2, ~1+\beta^{u}+\beta^{v}+\beta^{w} \bmod r(x) \equiv 0$$

%\begin{example}
%$5/7$ RSC code $f(x)=1+x^2,~g(x)=1+x+x^2$\newline
%$f(x)$ satisfies the special condition case and we can use the method above to obtain valid values of $(u,~v,~,w)$. The corresponding values for $a(x),~b(x)$ and $h(x)$ are shown in Table \ref{novelTab16} for the first four valid values of $(u,~v,~,w)$.

% \begin{table*}[h]
% \caption{$5/7$ RSC, $f(x)=1+x^2,~w_H(\bh)=4$}
%\centering
 %\begin{tabular}{c c c} 
% \hline
% $a(x)$ & $b(x)$ & $h(x)$\\ [0.5ex] 
% \hline\hline
%$1+x$ & $1+x^3$ & $1+x+x^2+x^3$\\ 
%\hline
%$1+x+x^2$ &  $1+x^2+x^4$ & $1+x+x^3+x^4$ \\
%\hline
%$1+x+x^3$ & $1+x^4+x^5$ & $1+x+x^2+x^5$\\
%\hline
%$1+x^2+x^3$ & $1+x+x^5$  &$1+x^3+x^4+x^5$
% \end{tabular}
% \label{novelTab16}
%\end{table*}

%\end{example}