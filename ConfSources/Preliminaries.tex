\section{Preliminaries}
\label{secPrelim}
\begin{definition}{Polynomials over GF($p$) \newline}
A polynomial $v(x)$ over GF($p$) is defined as 
\begin{equation*}
v(x) = \sum_{i=0}^{m} v_ix^i
\end{equation*}
where $v_i \in $ GF($p$), $v_m \neq 0$. $p$ is a prime number and $m$ is the degree of  $v(x)$. Because we are working in the binary domain, we set $p=2$.
\end{definition}

%\begin{definition}{Order of a polynomial \newline}
%The order of the polynomial $v(x)$ is equal to the highest degree non-zero coefficient  term.
%\end{definition}

\begin{definition}{Prime Polynomial \newline}
If $v_m=1$, then $v(x)$ is a \textit{monic} polynomial. If $v(x)$ cannot be factorised into lower degree poynomials over GF($2$), it is an \textit{irreducible} polynomial. If $v(x)$ is both monic an irreducible, it is a \textit{prime} polynomial.
\end{definition}

\begin{definition}{Order of an Element in GF($2^m$) \newline}
Let $\beta^i$ represent a non-zero element in GF($2^m$, $1 \leq i \leq 2^m-1$). The smallest value of $j$ such that $(\beta^i)^j=1$ is known as the \textit{order} of $\beta^i$
\end{definition}

\begin{definition}{Primitive Element \newline}
Let $\beta^i$ represent a non-zero element in GF($2^m$). If its order $j$ is such that $j=2^m-1$, then it is a primitive element.
\end{definition}

\begin{definition}{Primitive Polynomial \newline}
Let $v(x)$ be a prime polynomial with degree $m$. If GF($2^m$) is constructed based on $v(x)$ and $\beta^1 = \beta$ is a primitive element, then $v(x)$ is a primitive polynomial. Alternately, if $v(x)$ does not divide $1+x^j$ for any $j<2^m-1$, then it is a primitive polynomial. 
\end{definition}

\begin{definition}{ $(e,~f) \bmod 2^m-1$ \newline}
Let $(e~f)$ represent a pair of non-zero positive integers. Then $(e,~f) \bmod 2^m-1$ is shorthand for the operation $(e \bmod 2^m-1,~f \bmod 2^m-1)$.
\end{definition}



%$m$ is used to represent the order of $v(x)$ and GF($2^{m}$) represents the extended Galois field generated by a prime polynomial which has order $m$.
%$\beta^i$ represents a non-zero element in  GF($2^{m}$), $1 \leq i \leq 2^m-1$.
%Any $\beta^i \st (\beta^i)^j=1,~j\leq 2^m-1$ is know as a \textit{primitive element}


% an infinite repetition of the vector $\bv$ whiles $(\bv)_j$ represents the repetition of vector $\bv~j$ times   

%$\bphi$ represents a primitive element in the extended Galois field GF($2^{\tau}$) and the vector representation for all elements in GF($2^{\tau}$) are written as $\bphi_i,~0 \leq i \leq 2^{\tau}$. 

%The subscript $i$ represents the decimal value of the binary vector, which means $\bphi_0$ represents the all-zero vector. All addition and multiplication operations are done in GF($2^{\tau}$).

%The operation $(e \bmod M,~f \bmod M)$ is represented by $(e,~f) \bmod M$, where $(e,~f)$ are integer pairs.
