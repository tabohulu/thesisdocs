\section{Preliminaries}
\label{secPrelim}

A polynomial in $x$, with degree $M$ is an expression of the form
%and coefficients in $\cV$ is denoted by $v(x)$, and defined as 
\begin{equation*}
v(x) = \sum_{m=0}^{M} v_mx^m
\end{equation*}
where $M$ is the \textit{degree} of the polynomial which is denoted by deg($v(x)$) = $M$ and $v_m$ is called the \textit{coefficient}. In this paper, we assume that $v_m$ is an integer, $0 \leq v_m \leq p$, where $p$ is a prime number. 
 Let $w_H(v(x))$ be the Hamming weight of $v(x)$, \textit{i.e.} the total number of non-zero coefficients.

Given a prime number $p$, we denote the Galois field as GF($p$) and if the coefficients $v_m, 0 \leq m \leq M$ are elements of GF($p$), $v(x)$ is called a polynomial over GF($p$). $v(x)$ is called \textit{monic} if $v_m=1$, \textit{irreducible} if it is defined over GF($p$) and cannot be factorised into lower degree polynomials over GF($p$) and a \textit{prime polynomial} if it is both monic and irreducible. 
For two polynomials $v(x),~\text{deg}(v(x))=M$ and $w(x),~\text{deg}(w(x))=N$, the sum and product of $v(x)$ and $w(x)$ are defined as 

\begin{equation*}
v(x)+w(x)=\sum_{m=0}^{map\{ M,N\}} (v_m +w_n)x^m
\end{equation*}

\begin{equation*}
v(x)w(x)=\sum_{m=0}^{ M+N} \sum_{i=0}^{m} (v_i w_{m-i})x^m
\end{equation*}
 respectively.
If both $v(x)$ and $w(x)$ are polynomials over GF($p$), then addition and multiplication of $v_m$ and $w_n$ are done modulo-p.
To multiply 2 polynomials modulo-$v(x)$, where $v(x)$ is a prime polynomial means to divide the product of the 2 polynomials by $v(x)$ and return the remainder.
GF($p^M$) with polynomial elements is constructed by a prime polynomial $v(x)$  with degree $M$ by considering the set of all polynomials with degree less than $M$, with ordinary addition and with polynomial multiplication modulo-$v(x)$, where polynomial multiplication modulo-$v(x)$ refers to dividing the product by $v(x)$ and returning the remainder.

The polynomial elements of GF($p^M$) are in $\beta$, in a form such as $1+\beta$. Alternatively, we can represent polynomial elements in power notation as $\beta^m,~0 \leq m \leq p^M-1$, where $\beta^2$ may be used in place of $1+\beta$, for example. Through out this paper, the power notation will be used more often for the sake of convenience, with the appropriate conversion between the power and polynomial notation made known where necessary. From this point onwards, GF($p^M$) represents the extension field of GF($p$) with polynomial elements, unless otherwise stated.

Let $\beta$ be a non-zero element of GF($2^M$). Then, $\epsilon$ denotes the \textit{order} of $\beta$, and is defined as the least positive integer value such that $\beta^{\epsilon}=1$. $\beta$ is a \textit{primitive element} of GF($2^M$) if $\epsilon=2^M-1$. A prime polynomial $v(x)$ with degree $M$ is called a \textit{primitive polynomial} if $\beta$ is a primitive element in GF($2^M$) generated by $v(x)$. 

Finally, let $(e,~f)$ represent a pair of non-zero positive integers. Then $(e,~f) \bmod 2^M-1$ is shorthand for the operation $(e \bmod 2^M-1,~f \bmod 2^M-1)$.

 GF($p^M$) is called the \textit{extension field} of GF($p$) and GF($p$) is called the \textit{ground field} of GF($p^M$).


%$m$ is used to represent the order of $v(x)$ and GF($2^{m}$) represents the extended Galois field generated by a prime polynomial which has order $m$.
%$\beta^i$ represents a non-zero element in  GF($2^{m}$), $1 \leq i \leq 2^m-1$.
%Any $\beta^i \st (\beta^i)^j=1,~j\leq 2^m-1$ is know as a \textit{primitive element}


% an infinite repetition of the vector $\bv$ whiles $(\bv)_j$ represents the repetition of vector $\bv~j$ times   

%$\bphi$ represents a primitive element in the extended Galois field GF($2^{\tau}$) and the vector representation for all elements in GF($2^{\tau}$) are written as $\bphi_i,~0 \leq i \leq 2^{\tau}$. 

%The subscript $i$ represents the decimal value of the binary vector, which means $\bphi_0$ represents the all-zero vector. All addition and multiplication operations are done in GF($2^{\tau}$).

%The operation $(e \bmod M,~f \bmod M)$ is represented by $(e,~f) \bmod M$, where $(e,~f)$ are integer pairs.
