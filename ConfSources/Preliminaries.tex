\section{Preliminaries}
\label{secPrelim}
$v(x)$ denotes the polynomial representation of any binary sequence. 
Specifically, the symbols $b(x)$ and $h(x)$ represent the input message and parity-check sequence respectively in polynomial form. 
The Hamming weight of $b(x)$ and $h(x)$, denoted by $w_H(b(x))$ and $w_H(h(x))$ respectively, is equal to the length of the polynomial $b(x)$ or $h(x)$.

$m$ is used to represent the order of $v(x)$ and GF($2^{m}$) represents the extended Galois field generated by a prime polynomial which has order $m$.
$\beta^i$ represents a non-zero element in  GF($2^{m}$), $1 \leq i \leq 2^m-1$.
Any $\beta^i \st (\beta^i)^j=1,~j\leq 2^m-1$ is know as a \textit{primitive element}

Given a pair of integers $(e,~f)$, $(e,~f) \bmod 2^m-1$ is shorthand for the operation $(e \bmod 2^m-1,~f \bmod 2^m-1)$.


% an infinite repetition of the vector $\bv$ whiles $(\bv)_j$ represents the repetition of vector $\bv~j$ times   

%$\bphi$ represents a primitive element in the extended Galois field GF($2^{\tau}$) and the vector representation for all elements in GF($2^{\tau}$) are written as $\bphi_i,~0 \leq i \leq 2^{\tau}$. 

%The subscript $i$ represents the decimal value of the binary vector, which means $\bphi_0$ represents the all-zero vector. All addition and multiplication operations are done in GF($2^{\tau}$).

%The operation $(e \bmod M,~f \bmod M)$ is represented by $(e,~f) \bmod M$, where $(e,~f)$ are integer pairs.
