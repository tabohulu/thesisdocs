\section{Preliminaries}
\label{secPrelim}

A polynomial in $x$, with degree $M$ is an expression of the form
%and coefficients in $\cV$ is denoted by $v(x)$, and defined as 
\begin{equation*}
v(x) = \sum_{m=0}^{M} v_mx^m
\end{equation*}
where $v_m,~0 \leq m \leq M$, are called the \textit{coefficients}  and $v_m \neq 0$. If $v_M=1,~v(x)$ is called a \textit{monic} polynomial.
 %In this paper, we assume that $v_m$ is an integer, $0 \leq v_m \leq p$, where $p$ is a prime number. 
Moreover, the \textit{Hamming weight} of $v(x)$, which is denoted by $w_H(v(x))$, is defined as the total number of non-zero coefficients.
For two polynomials $v(x),~\text{deg}(v(x))=M$ and $w(x),~\text{deg}(w(x))=N$, the sum and product of $v(x)$ and $w(x)$ are defined as 

\begin{equation*}
v(x)+w(x)=\sum_{m=0}^{map\{ M,N\}} (v_m +w_n)x^m
\end{equation*}

\begin{equation*}
v(x)w(x)=\sum_{m=0}^{ M+N} \sum_{i=0}^{m} (v_i w_{m-i})x^m
\end{equation*}
 respectively.

For a prime number $p$, the Galois field with $p$ elements, denoted as GF($p$) and if the the set of integers $\{ 0,1,p-1\}$ integers where addition and multiplication of 2 elements are carried out modulo-$p$. If the coefficients $v_m, 0 \leq m \leq M$ are elements of GF($p$), $v(x)$ is called a polynomial over GF($p$). 
A monic polynomial which cannot be factorised into lower degree polynomials over GF($p$) is called a \textit{prime polynomial}.

Let $v(x)$ be a prime polynomial with degree $M,~M>1$. Then a Galois field with $p^M$ polynomial elements, represented by GF($p^M$) can be constructed by considering the set of all polynomials with degree less than $M$ in GF($p$).
GF($p^M$) is called the \textit{extension field} of GF($p$) and GF($p$) is called the \textit{ground field} of GF($p^M$).The addition operation in GF($p^M$) is the same as regular addition, whiles the multiplication operation is carried out modulo-$v(x)$, where multiplication modulo-$v(x)$ means to divide the product of the 2 polynomials by $v(x)$ and return the remainder.

Elements in GF($p^M$) can be represented by a power notation, \textit{i.e.} $X^m,~0 \leq m \leq p^M-1$, where $X^2$ may be used in place of $1+x$, for example. Through out this paper, the power notation will be used more often for the sake of convenience, with the appropriate conversion between the power and polynomial notation made known where necessary.  

Let $X$ be a non-zero element of GF($p^M$).Then, $\epsilon$ denotes the \textit{order} of $X$, and is defined as the least positive integer value such that $X^{\epsilon}=1$, and $X$ is called a \textit{primitive element} if $\epsilon=p^M-1$ . Let $v(x)$ be a prime polynomial with degree $M$. If $v(x)$ generates GF($p^M$) such that $X$ is a primitive element in GF($p^M$), then  $v(x)$ is called a \textit{primitive polynomial}. Finally, the root of $v(x)$, denoted by $\beta$, is any element in GF($p^M$) for which $v(x)=0$.

%Finally, let $(e,~f)$ represent a pair of non-zero positive integers. Then $(e,~f) \bmod 2^M-1$ is shorthand for the operation $(e \bmod 2^M-1,~f \bmod 2^M-1)$.

 


%$m$ is used to represent the order of $v(x)$ and GF($2^{m}$) represents the extended Galois field generated by a prime polynomial which has order $m$.
%$\beta^i$ represents a non-zero element in  GF($2^{m}$), $1 \leq i \leq 2^m-1$.
%Any $\beta^i \st (\beta^i)^j=1,~j\leq 2^m-1$ is know as a \textit{primitive element}


% an infinite repetition of the vector $\bv$ whiles $(\bv)_j$ represents the repetition of vector $\bv~j$ times   

%$\bphi$ represents a primitive element in the extended Galois field GF($2^{\tau}$) and the vector representation for all elements in GF($2^{\tau}$) are written as $\bphi_i,~0 \leq i \leq 2^{\tau}$. 

%The subscript $i$ represents the decimal value of the binary vector, which means $\bphi_0$ represents the all-zero vector. All addition and multiplication operations are done in GF($2^{\tau}$).

%The operation $(e \bmod M,~f \bmod M)$ is represented by $(e,~f) \bmod M$, where $(e,~f)$ are integer pairs.
