\section{Union Bound and the Codeword Pattern Distance Spectrum}
\label{sec4}
Having determined how to find valid values of $b(x)$ and $h(x)$ for Hamming weights $\leq 3$, we are now in a position to generate the codeword pattern distance spectrum for a given RSC code. We take a union bound like approach towards the generation of the codeword pattern distance spectrum. The approach is outlined below.
\begin{enumerate}
 \item Beginning with (\ref{novelEq2}), we find all values of $b(x),~w_H(b(x))=2$ that have the same roots as $g(x)$ and divide $g(x)$ by each valid polynomial to obtain the corresponding $a(x)$.\label{ubStep1}
 \item Then using (\ref{novelEq3}), we multiply each $a(x)$ by $f(x)$ to obtain the corresponding value of $h(x)$. It is worth noting that $w_H(h(x))$ may be $\geq w_H(b(x))$. \label{ubStep2}
 \item Since we are interested in only the low weight codewords, we ignore any $b(x) \st w_H(b(x))+w_H(h(x)) \geq d_{\text{max}}$. \label{ubStep3}
 \item Next we set the weight value of $b(x)$ to $w_H(b(x))=3$, and repeat steps \ref{ubStep1} and \ref{ubStep2} while ignoring $b(x)$ that meet the condition in step \ref{ubStep3}.\label{ubStep4}
 \item To obtain a complete codeword pattern distance spectrum, we do a reverse operation, \textit{i.e.} we focus on (\ref{novelEq2}) and find all values of $g(x),~w_H(\bh)=2$ that have the same roots as $f(x)$ and divide $f(x)$ by each valid polynomial to obtain the corresponding $a(x)$.
 \item Then using (\ref{novelEq2}), we repeat steps \ref{ubStep2} through \ref{ubStep4}, being careful to avoid repitition.
 \item Finally we arrange all valid values of $b(x)$ and $h(x)$ in ascending value of codeword weight,$w_H(b(x)) + w_H(h(x))$.
 \end{enumerate}
The codeword pattern distance spectrum for the $5/7,~37/21$ and $23/35$ RSC codes are shown in Tables \ref{novelTab13},  \ref{novelTab14} and \ref{novelTab15} respectively.
%(quarantine)%For each RSC codeThese were obtained by dividing the general form of $h(x)$ for $w_H(\bh)=2$ and ($w_H(\bh)=2$ if it exists) by $f(x)$ and multiplying it by $g(x)$ to obtain $b(x)$ for a given RSC code. This process is repeated doing the same for the general form of $b(x)$. In both cases $a(x),~b(x)$ and $h(x)$ are only added to the list if $w_H(\bc) \leq d_{\text{max}}$.

\begin{table*}[h!]
 \caption{Partial Structured Distance Spectrum for the $5/7$ RSC code, $d_{\text{max}}=8$}
\centering
 \begin{tabular}{c c c} 
 \hline
 $a(x)$ & $b(x)$ & $h(x)$ \\ [0.5ex] 
 \hline\hline
$1$ & $1+x+x^{2}$ & $1+x^2$\\
\hline
$1+x^2$ & $1+x+x^3+x^4$ & $1+x^{4}$\\
\hline
$1+x$ & $1+x^3$ & $1+x+x^2+x^3$\\
\hline
$1+x^2+x^4$ & $1+x+x^3+x^5+x^6$ & $1+x^{6}$\\
\hline
$1+x^2+x^3$ & $1+x+x^5$ & $1+x^3+x^4+x^5$\\
\hline
$1+x+x^2$ & $1+x^2+x^4$ & $1+x+x^3+x^4$\\
\hline
$1+x+x^3$ & $1+x^4+x^5$ & $1+x+x^2+x^5$\\
\hline
$1+x^2+x^4+x^6$ & $1+x+x^3+x^5+x^7+x^8$ & $1+x^8$\\
\hline
$1+x+x^3+x^4$ & $1+x^6$ & $1+x+x^2+x^4+x^5+x^6$\\
%======extra
%\hline
%$1+x+x^3+x^5$ & $1+x^4+x^6+x^7$ & $1+x+x^2+x^7$\\
%\hline
%$1+x+x^2+x^4$ & $1+x^2+x^5+x^6$ & $1+x+x^3+x^6$\\
%\hline
%$1+x+x^2+x^3$ & $1+x^2+x^3+x^5$ & $1+x+x^4+x^5$\\
%\hline
%$1+x^2+x^3+x^5$ & $1+x+x^6+x^7$ & $1+x^3+x^4+x^7$\\
%\hline
%$1+x^2+x^3+x^4$ & $1+x+x^4+x^6$ & $1+x^3+x^5+x^6$\\
%\hline
%$1+x^2+x^4+x^5$ & $1+x+x^3+x^7$ & $1+x^5+x^6+x^7$\\
 \end{tabular}
 
 \label{novelTab13}
\end{table*}

\begin{table*}[h!]
 \caption{Partial Structured Distance Spectrum for the $37/21$ RSC code,$d_{\text{max}}=9$}
\centering
 \begin{tabular}{c c c} 
 \hline
 $a(x)$ & $b(x)$ & $h(x)$ \\ [0.5ex] 
 \hline\hline
$1+x$ & $1+x+x^{4}+x^5$ & $1+x^5$\\
\hline
$1$ & $1+x^4$ & $1+x+x^2+x^3+x^4$\\
\hline
$1+x+x^5+x^6$ & $1+x+x^4+x^6+x^9+x^{10}$ & $1+x^{10}$\\
%\hline
%$1+x+x^4+x^5$ & $1+x+x^8+x^9$ & $1+x^4+x^5+x^9$\\
%\hline
%$1+x^2$ & $1+x^2+x^4+x^6$ & $1+x+x^5+x^6$\\
%\hline
%$1+x+x^5$ & $1+x+x^4+x^9$ & $1+x^6+x^7+x^8+x^9$\\
%\hline
%$1+x+x^4$ & $1+x+x^5+x^8$ & $1+x^4+x^6+x^7+x^8$\\
%\hline
%$1+x^2+x^4$ & $1+x^2+x^6+x^8$ & $1+x+x^4+x^7+x^8$\\
%\hline
%$1+x^3+x^4$& $1+x^3+x^7+x^8$ & $1+x+x^2+x^4+x^8$\\
%\hline
%$1+x^4+x^5$ & $1+x^5+x^8+x^9$ & $1+x+x^2+x^3+x^9$\\
 \end{tabular}
 
 \label{novelTab14}
\end{table*}

\begin{table*}[h!]
 \caption{Partial Structured Distance Spectrum for the $23/35$ RSC code,$d_{\text{max}}=10$}
\centering
 \begin{tabular}{c c c} 
 \hline
 $a(x)$ & $b(x)$ & $h(x)$ \\ [0.5ex] 
 \hline\hline
$1+x^2+x^3$ & $1+x^7$ & $1+x+x^2+x^6+x^7$\\
\hline
$1$ & $1+x^2+x^3+x^4$ & $1+x+x^{4}$\\
\hline
$1+x+x^2+x^3+x^5$ & $1+x^3+x^4+x^8+x^9$ & $1+x^7+x^9$\\
\hline
$1+x+x^2+x^3+x^5+x^7+x^8$ & $1+x+x^3+x^4+x^7+x^{12}$ & $1+x^{11}+x^{12}$\\
\hline
$1+x^2+x^3+x^7+x^9+x^{10}$ & $1+x^{14}$ & $1+x+x^2+x^6+x^8+x^9+x^{13}+x^{14}$\\
 \end{tabular}
 
 \label{novelTab15}
\end{table*}
We use the codeword pattern distance spectrum to calculate the bit-error bounds for each RSC and compare them to the bit-error bounds obtained via the distance spectrum as well as simulation results. We use the probability of bit-error in doing this and a more general formula for calculating $P_b$ is shown below [4]:

\begin{equation}
P_b \leq \frac{1}{k} \sum_{d=d_{\text{free}}}^{\infty} w(d) Q\Bigg( \sqrt{\frac{2dE_c}{N_0}}\Bigg)
\label{novelEq6}
\end{equation}
where $w(d)=\sum_{i=1}^{\infty} i~ a(d,i)$ and $ a(d,i)$ is the number of codewords of weight $d$ generated by an input message of weight $i$. If we set a limit on the maximum value of the codeword weight $d_{\text{max}}$
 we can rewrite (\ref{novelEq6}) as 
\begin{equation}
P_b \leq \frac{1}{k} \sum_{d=d_{\text{free}}}^{d_{\text{max}}} w(d) Q\Bigg( \sqrt{\frac{2dE_c}{N_0}}\Bigg)
\label{novelEq7}
\end{equation}
 From the simulation results, we observed $d_{\text{max}}=d_{\text{min}}+3$ is a sufficient value for obtaining the BER bounds.
%In order to confirm the validity of our method, we use the values obtained from Tables \ref{novelTab8}, \ref{novelTab9} and \ref{novelTab10} to find the bounds for the BER of the RSC code, $P_b$.Finally, we compare the results obtained to $P_b$ found using the Transfer Function method as well as the simulation results.