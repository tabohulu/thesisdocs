\subsection{ Distance Spectrum of RSC Codes and the Union Bound}
\label{sec4}
%For a given RSC code, the distance spectrum provides information concerning the multiplicity of a codeword for a fixed weight and it is an effective tool to evaluate its error-correcting capability. In practice however, since higher-weight codewords have very little effect on its overall error-correcting capability, we usually use a partial distance spectrum, where the largest codeword weight value is set to $d_{\text{max}}$. 

The distance spectrum of the RSC code can be obtained from its transfer function, denoted by $$T(Y,X)=\sum_{d=0}^{\infty}\sum_{w=0}^{\infty} a(d,w)Y^dX^w$$ where $a(d,w)$ is the number of codewords of weight $d$ generated by an input bit sequence of weight $w$. 
%The transfer function enumerates all the paths that diverge from and then return to the initial state \cite{ref3}, \textit{i.e.} the RTZ input paths. 
Once the transfer function of an RSC code is known, it can be used to obtain bounds on the error-correcting capability using the union bound.
Unfortunately, the complexity involved in deriving the transfer function increases as the number of states of the RSC code increases and other methods such as Mason's Rule \cite{ref3} have to be used. 

%For a given RSC code, we have shown in \ref{subsec:low-weight} that each codeword $c(x)$ is made up of $b(x)$ and $h(x)$ which have $a(x)$ as their common factor as shown in (\ref{novelEq2}) and (\ref{novelEq3}).
 Let $\cA_h(d)$ be the set of all $a(x)$ which yields weight-$d$ parity-check component \ie, $w_H(h(x))=w_H(a(x)f(x))=d$ for $a(x) \in \cA_h(d)$. 
Similarly $\cA_b(d)$ is the set of all $a(x)$ which yields weight-$d$ systematic component \ie, $w_H(b(x))=w_H(a(x)g(x))=d$ for $a(x) \in \cA_b(d)$
 and $\cA_c(d)$ is the set of all $a(x)$ which yields weight-$d$ codeword \ie, $w_H(c(x))=w_H(a(x)f(x))+ w_H(a(x)g(x))=d$ for $a(x) \in \cA_c(d)$.  

Then, the union bound of the bit-error rate \cite{ref4} can be calculated as 
%\begin{align}
%P_b \leq \frac{1}{k} \sum_{d=d_{\text{free}}}^{\infty} \sum_{a(x) \in \cA_c(d)}w_H(a(x)g(x)) Q\Bigg( \sqrt{\frac{2dE_c}{N_0}}\Bigg)
%\label{novelEq6-1}
%\end{align}
%However, since the high-weight codewords have minor contribution on the unioin bound, \eqref{novelEq6-1} can be further approximated by setting a limit on the maximum value of the codeword weight $d_{\text{max}}$, resulting in
\begin{align}
P_b \leq \frac{1}{k} \sum_{d=d_{\text{free}}}^{d_{\text{max}}} \sum_{a(x) \in \cA_c(d)}w_H(a(x)g(x)) Q\Bigg( \sqrt{\frac{2dE_c}{N_0}}\Bigg)
\label{novelEq7}
\end{align}

From \eqref{eq:cw-weight}, when $w_H(b(x)), w_H(h(x)) \geq 2$, we have
\begin{align}
\cA_c(d) = \bigcup_{\ell = 2}^{d-2} \left\{\cA_b(\ell) \cap \cA_h(d-\ell)\right\}
\label{Eq:exactset}
\end{align}
However, to determine $\cA_b(\ell)$ or $\cA_h(\ell)$ for a large $\ell$ is a complex task in general. Thus, in this paper, we replace the set $\cA_c(d)$ by the approximated set $\cA_c'(d)$ as defined in  (\ref{setApprox})
\begin{equation}
\begin{split}
\cA_c(d) \approx \cA_c'(d) &= \left\{\bigcup_{\ell = 2}^{\ell+\alpha} \left\{\cA_b(\ell) \cap \cA_h(d-\ell)\right\}\right\}\bigcup \\
&\left\{\bigcup_{\ell = 2}^{\ell+\alpha} \left\{\cA_b(d-\ell) \cap \cA_h(\ell)\right\}\right\}
\end{split}
\label{setApprox}
\end{equation}
where some codewords in $\cA_c(d)$ with $\ell \approx d-\ell$ may be ignored in $\cA_c'(d)$.
%We refer to the distance spectrum obtained using this method as the \textit{codeword component pattern distance spectrum}
\begin{example}
If we set $d=8$ and  $\alpha=1$, $\cA_c'(8)$ becomes
\begin{equation*}
\begin{split}
\cA_c'(8) &=\left\{\left\{\cA_b(2) \cap \cA_h(6)\right\} \bigcup  \left\{\cA_b(3) \cap \cA_h(5)\right\} \right\} \bigcup \\
& \left\{\left\{\cA_b(6) \cap \cA_h(2)\right\} \bigcup  \left\{\cA_b(5) \cap \cA_h(3)\right\} \right\} \\
\end{split}
\end{equation*}

We can see that $\left\{\cA_b(4) \cap \cA_h(4)\right\}$ is not used in $\cA_c'(8)$, event though it is used in $\cA_c(8)$.
\end{example}





%Having determined how to find valid values of $b(x)$ and $h(x)$ for Hamming weights $\leq 3$, we are now in a position to generate the codeword pattern distance spectrum for a given RSC code. We take a union bound like approach towards the generation of the codeword pattern distance spectrum. The approach is outlined below.
%\begin{enumerate}
 %\item Beginning with (\ref{novelEq2}), we find all values of $b(x),~w_H(b(x))=2$ that have the same roots as $g(x)$ and divide $g(x)$ by each valid polynomial to obtain the corresponding $a(x)$.\label{ubStep1}
 %\item Then using (\ref{novelEq3}), we multiply each $a(x)$ by $f(x)$ to obtain the corresponding value of $h(x)$. It is worth noting that $w_H(h(x))$ may be $\geq w_H(b(x))$. \label{ubStep2}
 %\item Since we are interested in only the low weight codewords, we ignore any $b(x) \st w_H(b(x))+w_H(h(x)) \geq d_{\text{max}}$. \label{ubStep3}
 %\item Next we set the weight value of $b(x)$ to $w_H(b(x))=3$, and repeat steps \ref{ubStep1} and \ref{ubStep2} while ignoring $b(x)$ that meet the condition in step \ref{ubStep3}.\label{ubStep4}
 %\item To obtain a complete codeword pattern distance spectrum, we do a reverse operation, \textit{i.e.} we focus on (\ref{novelEq2}) and find all values of $g(x),~w_H(\bh)=2$ that have the same roots as $f(x)$ and divide $f(x)$ by each valid polynomial to obtain the corresponding $a(x)$.
 %\item Then using (\ref{novelEq2}), we repeat steps \ref{ubStep2} through \ref{ubStep4}, being careful to avoid repitition.
 %\item Finally we arrange all valid values of $b(x)$ and $h(x)$ in ascending value of codeword weight,$w_H(b(x)) + w_H(h(x))$.
 %\end{enumerate}

%We use the codeword pattern distance spectrum to calculate the bit-error bounds for each RSC and compare them to the bit-error bounds obtained via the distance spectrum as well as simulation results. We use the probability of bit-error in doing this and a more general formula for calculating $P_b$ is shown below [4]:

%\begin{equation}
%P_b \leq \frac{1}{k} \sum_{d=d_{\text{free}}}^{\infty} w(d) Q\Bigg( \sqrt{\frac{2dE_c}{N_0}}\Bigg)
%\label{novelEq6}
%\end{equation}
%where $w(d)=\sum_{i=1}^{\infty} i~ a(d,i)$ and $ a(d,i)$ is the number of codewords of weight $d$ generated by an input message of weight $i$. If we set a limit on the maximum value of the codeword weight $d_{\text{max}}$
% we can rewrite (\ref{novelEq6}) as 
%\begin{equation}
%P_b \leq \frac{1}{k} \sum_{d=d_{\text{free}}}^{d_{\text{max}}} w(d) Q\Bigg( \sqrt{\frac{2dE_c}{N_0}}\Bigg)
%\label{novelEq7}
%\end{equation}
 %From the simulation results, we observed $d_{\text{max}}=d_{\text{min}}+3$ is a sufficient value for obtaining the BER bounds.
%In order to confirm the validity of our method, we use the values obtained from Tables \ref{novelTab8}, \ref{novelTab9} and \ref{novelTab10} to find the bounds for the BER of the RSC code, $P_b$.Finally, we compare the results obtained to $P_b$ found using the Transfer Function method as well as the simulation results.