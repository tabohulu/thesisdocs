\section{The characteristics of the low-weights codewords of RSC code}
\label{sec2}
The outputs of an RSC code are determined by the input bit sequence $b(x)$, the states of the shift registers and the feedforward and feedback connections of the shift registers that can be represented by a generator function. 

For instance,  the generator function of a rate $1/2$ RSC code may be written as  $$\left[1 ~\frac{f(x)}{g(x)}\right]$$ where $1$ yields the \textit{systematic  component} (SC) $b(x)$ while the \textit{parity-check component} (PC) $h(x)$ is associated with the feedforward and feedback connections of the shift registers, specified by $f(x)$ and $g(x)$, respectively. Each output $c(x)$ is a mixture of the SC and PC as
\begin{equation}
c(x) = b(x^2)+xh(x^2)
\label{codeword-comp}
\end{equation}
where 
\begin{equation}
h(x) =f(x)g^{-1}(x)b(x)
\label{eq:parity-def}
\end{equation}
From \eqref{codeword-comp}, it is trivial that
\begin{equation}
w_H(c(x))=w_H(b(x)) + w_H(h(x))
\label{eq:cw-weight}
\end{equation}
and hence, each low-weight codeword is a combination of a low-weight SC and PC.

Under the assumption of large frame sizes, the presence of $g^{-1}(x)$  in \eqref{eq:parity-def} may involve a particular sequence of bits that is repeated a large number of times, hence generating a high-weight PC. A low-weight PC occurs if and only if
\begin{equation}
b(x) \bmod g(x) \equiv 0
\label{eq:rtz-input}
\end{equation}
Any input $b(x)$ which meet the condition in \eqref{eq:rtz-input} is called a \textit{return-to-zero} (RTZ) input. Thus, every RTZ input can be factorized by  
\begin{equation}
b(x) =a(x)g(x)
\label{eq:low-weight-msg}
\end{equation}
Substituting (\ref{eq:low-weight-msg}) into \eqref{eq:parity-def}, we can characterize the low-weight PC as
\begin{equation}
\begin{split}
h(x)&=f(x)\cdot g^{-1}(x)\cdot a(x)g(x)\\
&=a(x)f(x)
\end{split}
\label{eq:low-weight-parity}
\end{equation}

Finally, for a given RSC code, we can formulate our goal as, to find all $a(x)$s which satisfy  \eqref{eq:low-weight-msg} and  \eqref{eq:low-weight-parity} simultaneously. However, since there is no essential mathematical difference between the two equations, in the next section, we present a method for determining the low-weight PC patterns for $2 \leq w_H(h(x)) \leq 3$


