\section{Preliminaries}
\label{secPrelim}

A polynomial in $x$ with degree $M$ is an expression of the form
%and coefficients in $\cV$ is denoted by $v(x)$, and defined as 
\begin{equation}
v(x) = \sum_{m=0}^{M} v_mx^m
\label{Eq:polynomial}
\end{equation}
where $v_m,~0 \leq m \leq M$, are called the \textit{coefficients} and $v_M \neq 0$. If $v_M=1,~v(x)$ is called the \textit{monic} polynomial. We call the total number of non-zero coefficients the \textit{Hamming weight} of $v(x)$, and write $w_H(v(x))$.


For a prime number $p$, if the addition and multiplication of two elements in the integer set$\{ 0,1,p-1\}$ are performed on the terms $\bmod p$, we call the set the Galois field and denoted as $\GF(p)$. If the coefficients in \eqref{Eq:polynomial} are elements of $\GF(p)$, $v(x)$ is called {\it polynomial over} $\GF(p)$.


For two polynomials $v(x)$ and $w(x)$ with degrees $M$ and $N$, respectively, the addition and multiplication over $\GF(p)$ are defined as 
\begin{align}
v(x)+w(x)=\sum_{m=0}^{\max\{ M,N\}} [(v_m +w_n)\mod p] x^m
\label{Eq:addition}
\end{align}
and
\begin{align}
v(x)w(x)=\sum_{m=0}^{ M+N} \sum_{i=0}^{m} [v_i w_{m-i}\mod p]x^m
\label{Eq:multiplication}
\end{align}
respectively. 
A monic polynomial which cannot be represented by multiplication of some lower degree polynomials is called a \textit{prime polynomial}.
For two polynomials $v(x)$ and $w(x)$ over$\GF(p)$, we assume $w(x) \neq 0$. Then there exists polynomials $q(x)$ and $r(x)$ over $\GF(p)$ such that 
\begin{align}
v(x) = w(x)q(x)+r(x)
\label{Eq:decomposition}
\end{align}
with $\deg(r(x)) < \deg(w(x))$. $r(x)$ in the expression \eqref{Eq:decomposition}, is denoted by
\begin{align}
r(x) =v(x)\mod w(x)
\end{align}
is called the \textit{remainder polynomial} while $q(x)$ is called the \textit{quotient polynomial} of the division of $v(x)$ by $w(x)$.

Let $v(x)$ be a prime polynomial over $\GF(p)$ with $\deg(v(x)):=M>1$ and $\cV$ be the set of size $p^M$ consisting of all polynomials over $\GF(p)$ with degree less than $M$. Then, the \textit{extension field of $\GF(p)$}, denoted by $\GF\left(p^M\right)$, is the set $\cV$ with addition and multiplication over $\GF(p)$ where the multiplication is carried out modulo-$v(x)$ over $\GF(p)$.
Each non-zero elements in $\GF \left(p^M\right)$ can be represented by a power of $X$ uniquely as $X^m,~0 \leq m \leq p^M-1$. %, where $X^2$ may be used in place of $1+x$, for example.
%Through out this paper, the polynomilal nad power notations will be used more often for the sake of convenience, with the appropriate conversion between the power and polynomial notation made known where necessary.  

For each non-zero element of $\GF \left(p^M\right)$, there exist integers $\epsilon$ such that $X^{\epsilon}=1$ and the least positive integer among them is called the \textit{order} of $X$. The element with order $\epsilon=p^M-1$ is called \textit{primitive element}. For $\GF \left(p^M\right)$ generated by a prime polynomial $v(x)$ with $\deg(v(x))=M$, if $X$ is a primitive element in $\GF \left(p^M\right)$, then  $v(x)$ is called \textit{primitive polynomial}. 
%
Finally, the root of $v(x)$, is the non-zero element in $\varphi \in \GF \left(p^M\right)$ such that $v(\varphi)=0$. If $v(x)$ is a primitive polynomial, the order of $\varphi$ is $\epsilon=p^M-1$ while $\epsilon | p^M-1$ otherwise. 
Moreover, the elements $\varphi^i,~0 \leq i \leq \epsilon -1$, are all distinct each other.
 
%Finally, let $(e,~f)$ represent a pair of non-zero positive integers. Then $(e,~f) \bmod 2^M-1$ is shorthand for the operation $(e \bmod 2^M-1,~f \bmod 2^M-1)$.

