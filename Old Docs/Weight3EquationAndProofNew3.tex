\documentclass[11pt, oneside, dvipdfmx]{book}
\newcommand{\folder}{/usr/local/share/texmf}
%\newcommand{\folder}{/home/hanchenggao/Documents/texmf}
\input{\folder/hfiles/ebook}
\usepackage {graphicx}
\usepackage {graphics}
\usepackage {graphics}
\title{Formula to Calculate Weight for Low-Weight Weight 3 Inputs and Proof} 
\author{Kwame Ackah Bohulu}
\date{\today}
\begin{document}
\maketitle

\newpage


\section{Equation and Proof}
%\begin{theorem}
%Let $Q(x) =x^{a\tau+t}(1+x^{\beta \tau +1}+x^{\gamma \tau +2})$ be the polynomial representation of a weight $3$ RTZ input.
%The Hamming weight, $w_H$ of a turbo codeword generated by a weight-$3$ RTZ input is given by 
%\begin{equation}
%7+2(\max\{l_1,l_2\}+\max\{l^{\prime}_1,l^{\prime}_2\})
%\end{equation}
%\end{theorem}
\begin{proof}
The polynomial representation of a weight-$3$ RTZ input is given by $$Q(x) =x^{h\tau+t}(1+x^{\beta \tau +1}+x^{\gamma \tau +2})$$
The impulse response of the RSC encoder is
\[
(1~1~1~0~1~1~0~1~1~0~\cdots)
\]
and using the inpulse response, we can calculate the parity weight as well as Hamming weight of the turbo codeword. 

Let \\$\bphi_1=(0~0~1),~\bphi'_1=(0~1~0),~\bphi''_1=(1~0~0)$, \\
$\bphi_2=(0~1~1),~\bphi'_2=(1~1~0),~\bphi''_2=(1~0~1)$, \\
$\bphi_3=(1~1~1)$. 

Then, the weight-3 RTZ occurs since $\bphi_2+\bphi'_2+\bphi''_2=\bzero_{\tau}=\bzero_3$. 
Now, we consider the weight of the vector derived by the sumation of the followings vectors.
\begin{eqnarray*}
(\bzero_{3(\gamma+h)}~\bphi_1~\bphi'_2~\cdots)\cr
(\bzero_{3(\beta+h)}~\bphi_2~\bphi''_2~\cdots)\cr
(\bzero_{3 h}~~~~~~\bphi_3~\bphi_2~\cdots)
\end{eqnarray*}

Without loss of generality, we can assume that all weight-$3$ RTZ inputs begin at the $0$th position, ie $h=t=0$. This is because the case where $h>0$ or $t>0$ is just a right-shifted version of the weight-$3$ RTZ. With this assumption, we we only need to consider cases where $h= 0,~\gamma \geq h$.
To simplify calculation, we have included an addition table for all the vectors which is shown in Table \ref{tb1}

\begin{table}[h!]
\centering
\begin{tabular}{c || c  | c  | c  | c  | c  | c  | c } 
 $$ & $\bphi_1$ & $\bphi'_1$ & $\bphi''_1$ & $\bphi_2$ & $\bphi'_2$ & $\bphi''_2$ & $\bphi_3$ \\
   \hline\hline
   %row1
$\bphi_1$ & $\bzero_3$ & $-$ & $-$ & $-$ & $-$ & $-$ & $-$ \\
   \hline
      %row2
$\bphi'_1$ & $\bphi_2$ & $\bzero_3$ & $-$ & $-$ & $-$ & $-$ & $-$ \\
   \hline
      %row3
$\bphi''_1$ & $\bphi''_2$ & $\bphi'_2$ & $\bzero_3$ & $-$ & $-$ & $-$ & $-$ \\
   \hline
      %row4
$\bphi_2$ & $\bphi'_1$ & $\bphi_1$ & $\bphi_3$ & $\bzero_3$ & $-$ & $-$ & $-$ \\
   \hline
      %row5
$\bphi'_2$ & $\bphi_3$ & $\bphi''_1$ & $\bphi'_1$ & $\bphi''_2$ & $\bzero_3$ & $-$ & $-$ \\
   \hline
      %row6
$\bphi''_2$ & $\bphi''_1$ & $\bphi_3$ & $\bphi_1$ & $\bphi'_2$ & $\bphi_2$ & $\bzero_3$ & $-$ \\
   \hline
      %row7
$\bphi_3$ & $\bphi'_2$ & $\bphi''_2$ & $\bphi_2$ & $\bphi''_1$ & $\bphi_1$ & $\bphi'_1$ & $\bzero_3$ \\
   \hline
  \end{tabular}
\caption{Truth Table}
\label{tb1}
\end{table}
Furthermore, we consider 4 general cases for all possible values of $i,j,k$ where $i \geq k$ These cases are $(=~=),~(=~<),~(<~=)$ and $(<~<)$
\paragraph{Case 0: $\gamma=\beta=h$ \newline}

 For this case, the vectors to sum will be 
 \begin{align*}
(\bphi_1~\bphi'_2~\cdots)\\
(\bphi_2~\bphi''_2~\cdots)\\
(\bphi_3~\bphi_2~\cdots)\\
\cline{1-2}
(\bphi''_2~\bzero_{3}~\cdots)
\end{align*}
 
and  the derived vector will be $(\bphi''_2~\bzero_{3}~\cdots)$ with a weight of $w_p=2$
 
 %========case =  < ===========
 
 %\paragraph{Case 1a: $i=j<k$\newline}
 %vector to sum:
 %\begin{align*}
 %(\bzero_{3}~\cdots~\bzero_{3}~\bphi_1~\bphi'_2~\cdots~\bphi_2'~\bphi_2'~\bphi_2'~\cdots)\\
% (\bzero_{3}~\cdots~\bzero_{3}~\bphi_2~\bphi''_2~\cdots~\bphi''_2~\bphi''_2\bphi''_2~\cdots)\\
%+(\bzero_{3}~~\cdots~\cdots~\cdots~\cdots~\bzero_{3}~\bphi_3~\bphi_2~\cdots)\\
%\cline{1-2}
%(\bzero_{3}~\cdots~\bzero_{3}~\bphi'_1~\bphi_2~\cdots~\bphi_2~\bphi''_1~\bzero_3~\cdots)
%\end{align*}
%derived vector : $(\bzero_{3j}~\bphi'_1~(\bphi_2)_{k-j-1}~\bphi''_1~\bzero_3~\cdots)$
%\newline
%Parity weight: \begin{equation}
%\begin{split}
%w_p=2(k-j)
%\end{split}
%\end{equation}

\paragraph{Case 1a: $\gamma=h<\beta$ \newline}
 vectors to sum:
 \begin{align*}
(\bphi_1~\bphi'_2~\bphi'_2~\bphi'_2~\bphi'_2~\cdots)\\
(\bzero_{3}~\cdots~\bzero_{3}~\bphi_2~\bphi''_2~\cdots)\\
+(\bphi_3~\bphi_2~\bphi_2~\bphi_2~\bphi_2~\cdots)\\
\cline{1-2}
(\bphi'_2~\bphi''_2~\bphi''_2~\bphi'_2~\bzero_3~\cdots)
\end{align*}
derived vector : $(\bphi'_2~(\bphi''_2)_{\beta-h-1}~\bphi'_2~\bzero_3~\cdots)$
\newline
Parity weight: \begin{equation}
\begin{split}
w_p=2(\beta-h)+2 =2\beta+2
\end{split}
\end{equation}

\paragraph{Case 1b: $\beta=h<\gamma$\newline}
 vectors to sum:
\begin{align*}
(\bzero_{3}~\cdots~\cdots~\bzero_{3}~\bphi_1~\bphi'_2~\cdots)\\
(\bphi_2~\bphi''_2~\cdots~\bphi''_2~\bphi''_2\bphi''_2~\cdots)\\
+(\bphi_3~\bphi_2~\cdots~\bphi_2~\bphi_2~\bphi_2~\cdots)\\
\cline{1-2}
(\bphi''_1~\bphi'_2~\cdots~\bphi'_2~\bphi_3~\bzero_3~\cdots)
\end{align*}
derived vector : $(\bphi''_1~(\bphi_2)_{\gamma-h-1}~\bphi_3~\bzero_3~\cdots)$\newline
Parity weight: \begin{equation}
\begin{split}
w_p=2(\gamma-h)+2=2\gamma+2
\end{split}
\end{equation}
%========case < = ==========
%\paragraph{Case 2a: $i<j=k$\newline}
% vectors to sum:
%\begin{align*}
%(\bzero_3 ~\cdots~\bphi_1~\bphi'_2~\cdots~\bphi'_2~\bphi'_2~\bphi'_2~\cdots)\\
%(\bzero_3~\cdots~\cdots~\cdots~\cdots~\bzero_3~\phi_2~\phi''_2~\cdots)\\
%+(\bzero_3~\cdots~\cdots~\cdots~\cdots~\bzero_3~\phi_3~\phi_2~\cdots)\\
%\cline{1-2}
%(\bzero_{3}~\cdots~\bzero_{3}\bphi_1~\bphi'_2~\cdots~\bphi'_2~\bphi'_1~\bzero_3~\cdots)
%\end{align*}
%derived vector : $(\bzero_{3i}~\bphi_1~(\bphi'_2)_{k-i-1}~\bphi'_1~\bzero_3~\cdots)$
%\newline
%Parity weight: \begin{equation}
%\begin{split}
%w_p=2(k-i)
%\end{split}
%\end{equation}

%\paragraph{Case 2a: $j<k=i$ \newline}
% vectors to sum:
%\begin{align*}
%(\bzero_3~\cdots~\cdots~\cdots~\cdots~\bzero_3~\phi_1~\phi'_2~\cdots)\\
%(\bzero_3~\cdots~\bzero_3~\phi_2~\phi''_2~\cdots~\phi''_2~\phi''_2~\phi''_2~\cdots)\\
%+(\bzero_3~\cdots~\cdots~\cdots~\cdots~\bzero_3~\phi_3~\phi_2~\cdots)\\
%\cline{1-2}
%(\bzero_{3}~\cdots\bzero_3~\bphi_2~\bphi''_2~\cdots~\bphi''_2~\bphi_2~\bzero_3~\cdots)
%\end{align*}
%derived vector : $(\bzero_{3j}~\bphi_2~\bphi''_2)_{i-j-1}~\bphi_2~\bzero_3~\cdots)$\newline
%Parity weight: \begin{equation}
%\begin{split}
%w_p=2(i-j)+2
%\end{split}
%\end{equation}
\newpage
\paragraph{Case 2a: $h<\gamma=\beta$ \newline}
 vectors to sum:
\begin{align*}
(\bzero_3\cdots~\cdots~\bzero_3~\phi_1~\phi'_2~\cdots)\\
(\bzero_3\cdots~\cdots~\bzero_3~\phi_2~\phi''_2~\cdots)\\
+(\phi_3~\phi_2~\cdots~\phi_2~\phi_2~\phi_2~\cdots)\\
\cline{1-2}
(\bphi_3~\bphi_2~\cdots~\bphi_2~\bphi_1~\bzero_3~\cdots)
\end{align*}


derived vector : $(\bphi_3~(\bphi_2)_{\gamma-h-1}~\bphi_1~\bzero_3~\cdots)$
\newline
Parity weight: \begin{equation}
\begin{split}
w_p=2(\gamma-h)+2 =2\gamma+2
\end{split}
\end{equation}
 %=====case < <==========
% \paragraph{Case 3a: $i<j<k$\newline}
 % vectors to sum:
%\begin{eqnarray*}
%(\bzero_3~\cdots~\bzero_3~\phi_1~\phi'_2~\cdots~\phi'_2~\phi'_2~\phi'_2~\cdots~\phi'_2~\phi'_2~\phi'_2~\cdots)\cr
%(\bzero_3~\cdots~\bzero_3~\bzero_3~\bzero_3~\cdots~\bzero_3~\phi_2~\phi''_2~\cdots~\phi''_2~\phi''_2~\phi''_2~\cdots)\cr
%+(\bzero_3~\cdots~\bzero_3~\bzero_3~\bzero_3~\cdots~\cdots~\cdots~\cdots~\bzero_3~\phi_3~\phi_2~\cdots)\cr
%\cline{1-2}
%(\bzero_{3}~\cdots~\bzero_3~\bphi_1~\bphi'_2\cdots\bphi'_2~\bphi''_2~\bphi_2~\cdots~\bphi_2~\bphi''_1~\bzero_3~\cdots)
%\end{eqnarray*}


%derived vector : $(\bzero_{3i}~\bphi_1~(\bphi'_2)_{j-i-1}~\bphi''_2~(\bphi_2)_{k-j-1}~\bphi''_1~\bzero_3~\cdots)$
%\newline
%Parity weight: \begin{equation}
%\begin{split}
%w_p&=2(j-i)+1+2(k-j-1)+1\\
%&=2(k-i)
%\end{split}
%\end{equation}

%\paragraph{Case 3b: $i<k<j$ \newline}
 % vectors to sum:
%\begin{eqnarray*}
%(\bzero_3~\cdots~\bzero_3~\phi_1~\phi'_2~\cdots~\phi'_2~\phi'_2~\phi'_2~\cdots~\phi'_2~\phi'_2~\phi'_2\cdots)\cr
%(\bzero_3~\cdots~\bzero_3~\bzero_3~\bzero_3~\cdots~\cdots~\cdots~\cdots~\bzero_3~\phi_2~\phi''_2\cdots)\cr
%+(\bzero_3~\cdots~\bzero_3~\bzero_3~\bzero_3~\cdots~\bzero_3~\phi_3~\phi_2~\cdots~\phi_2~\phi_2~\phi_2\cdots)\cr
%\cline{1-2}
%(\bzero_{3}~\cdots~\bzero_3~\bphi_1~\bphi'_2\cdots~\bphi'_2~\bphi_1~\bphi''_2~\cdots~\bphi''_2~\bphi'_2~\bzero_3\cdots)
%\end{eqnarray*}

%derived vector : $(\bzero_{3i}~\bphi_1~(\bphi'_2)_{k-i-1}~\bphi_1~(\bphi''_2)_{j-i-1}~\bphi'_2~\bzero_3~\cdots)$
%\newline
%Parity weight: \begin{equation}
%\begin{split}
%w_p&=2(k-i)+2(j-k)\\
%&=2(j-i)
%\end{split}
%\end{equation}


%\paragraph{Case 3a: $j<k<i$ \newline}
% vectors to sum:
%\begin{eqnarray*}
%(\bzero_3~\cdots~\bzero_3~\bzero_3~\bzero_3~\cdots~\cdots~\cdots~\cdots~\bzero_3~\phi_1~\phi'_2\cdots)\cr
%(\bzero_3~\cdots~\bzero_3~\phi_2~\phi''_2~\cdots~\phi''_2~\phi''_2~\phi''_2~\cdots~\phi''_2~\phi''_2~\phi''_2\cdots)\cr
%+(\bzero_3~\cdots\bzero_3~\bzero_3~\bzero_3~\cdots~\bzero_3~\phi_3~\phi_2~\cdots~\phi_2~\phi_2~\phi_2\cdots)\cr
%\cline{1-2}
%(\bzero_{3}~\cdots~\bzero_3~\bphi_2~\bphi''_2\cdots~\bphi''_2~\bphi'_1~\bphi'_2\cdots~\bphi'_2~\bphi_3~\bzero_3\cdots)
%\end{eqnarray*}
%derived vector : $(\bzero_{3k}~\bphi_2~(\bphi''_2)_{k-j-1}~\bphi'_1~(\bphi'_2)_{i-k-1}~\bphi_3~\bzero_3~\cdots)$\newline
%Parity weight: \begin{equation}
%\begin{split}
%w_p &=2(k-j)+1 +2(i-k)+1 \\
%&=2(i-j)+2
%\end{split}
%\end{equation}

%\paragraph{Case 3d: $j<i<k$\newline}
% vectors to sum:
%\begin{eqnarray*}
%(\bzero_3~\cdots\bzero_3~\bzero_3~\bzero_3~\cdots~\bzero_3~\phi_1~\phi'_2~\cdots~\phi'_2~\phi'_2~\phi'_2\cdots)\cr
%(\bzero_3~\cdots~\bzero_3~\phi_2~\phi''_2~\cdots~\phi''_2~\phi''_2~\phi''_2~\cdots~\phi''_2~\phi''_2~\phi''_2\cdots)\cr
%+(\bzero_3~\cdots~\bzero_3~\bzero_3~\bzero_3~~\cdots~\cdots~\cdots~\cdots~\bzero_3~\phi_3~\phi_2\cdots)\cr
%\cline{1-2}
%(\bzero_{3}~\cdots~\bzero_3~\bphi_2~\bphi''_2\cdots~\bphi''_2~\bphi''_1~\bphi_2\cdots~\bphi_2~\bphi''_1~\bzero_3\cdots)
%\end{eqnarray*}
%derived vector : $(\bzero_{3j}~\bphi_2~(\bphi''_2)_{i-j-1}~\bphi''_1~(\bphi_2)_{k-i-1}~\bphi''_1~\bzero_3~\cdots)$
%\newline
%Parity weight: \begin{equation}
%\begin{split}
%w_p&=2(i-j)+1+2(k-i-1)+1\\
%&=2(k-j)
%\end{split}
%\end{equation}


\paragraph{Case 3a: $h<\gamma<\beta$ \newline}
vectors to sum:
\begin{eqnarray*}
(\bzero_3\cdots\cdots~\bzero_3~\phi_1~\phi'_2~\cdots~\phi'_2~\phi'_2~\phi'_2\cdots)\cr
(\bzero_3\cdots~\cdots~\cdots~\cdots~\cdots~\bzero_3~\phi_2~\phi''_2\cdots)\cr
+(\phi_3~\phi_2~\cdots~\phi_2~\phi_2~\phi_2~\cdots~\phi_2~\phi_2~\phi_2\cdots)\cr
\cline{1-2}
(\bphi_3~\bphi_2\cdots~\bphi_2~\bphi'_1~\bphi''_2\cdots~\bphi''_2~\bphi'_2~\bzero_3\cdots)
\end{eqnarray*}


derived vector : $(\bphi_3~(\bphi_2)_{\gamma-h-1}~\bphi'_1~(\bphi''_2)_{\beta-\gamma-1}~\bphi'_2~\bzero_3~\cdots)$
\newline
Parity weight: \begin{equation}
\begin{split}
w_p&=2(\gamma-h)+2+2(\beta-i)\\
&=2(\beta-h)+2\\
& = 2\beta+2
\end{split}
\end{equation}

\paragraph{Case 3b: $h<\beta<\gamma$\newline}
\begin{eqnarray*}
(\bzero_3~\cdots~\cdots~\cdots~\cdots~\cdots~\bzero_3~\phi_1~\phi'_2\cdots)\cr
(\bzero_3~\cdots\cdots~\bzero_3~\phi_2~\phi''_2~\cdots~\phi''_2~\phi''_2~\phi''_2\cdots)\cr
+(\phi_3~\phi_2~\cdots~\phi_2~\phi_2~\phi_2~\cdots~\phi_2~\phi_2~\phi_2\cdots)\cr
\cline{1-2}
(\bphi_3~\bphi_2\cdots~\bphi_2~\bzero_3~\bphi'_2\cdots~\bphi'_2~\bphi_3~\bzero_3\cdots)
\end{eqnarray*}
derived vector : $(\bphi_3~(\bphi_2)_{j-k-1}~\bzero_3~(\bphi'_2)_{i-j-1}~\bphi_3~\bzero_3~\cdots)$\newline
Parity weight: \begin{equation}
\begin{split}
w_p &=2(\beta-h)+1 +2(\gamma-\beta)+1 \\
&=2(\gamma-h)+2\\
&=2\gamma+2
\end{split}
\end{equation}

From all the above cases we can conclude that the parity weight for a weight-$3$ RTZ sequence may be calculated as
\begin{equation}
w_p=
2l+2 
\end{equation}
where $l=\max \{ \gamma,\beta \} - k=\max \{ \gamma,\beta \}$ since $k=0$

Assuming that after interleaving, another weight-$3$ RTZ input is produced. Let $\gamma',\beta',h',l'$ and $w'_p$ be similarly defined. Then the Hamming weight $w_H$ of the turbo codeword produced can be calculated as
\begin{equation}
w_H=
7+2(l+l') 
\end{equation}

\end{proof}
\newpage





\end{document}
