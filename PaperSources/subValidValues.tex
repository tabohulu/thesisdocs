\subsection{Finding Valid Values of $h(x)$ and $b(x)$ for a fixed Hamming weight and RSC code\newline}
We have established that the relationship between the roots of $f(x)$ and $h(x)$, as well as the relationship between the roots of $g(x)$ and $b(x)$. Depending on the characteristic make-up of $f(x)$ , we can easily determine the structure of $h(x)$ for its different weight values. The same logic applies to determining the structure of $b(x)$ for its different weight values from $g(x)$. 
We therefore, present a method for determining valid values of $h(x)$ and $b(x)$ for a given RSC code when the weight values $w_H(\bh),~w_H(\bb) \leq 3$. We state here that there is no RTZ input of weight 1, because it goes against the definition of RTZ inputs.Consequently, there also cannot be a corresponding parity-check sequence of weight 1 and we can ignore those cases. 
The characteristic make-up of both $f(x)$ and $g(x)$ can be grouped into the four cases below. 
\begin{enumerate}
\item Single primitive polynomial.
\item Prime but not a primitive polynomial.
\item Made up of repeated polynomial roots.
%\item Made up of unique repeated polynomial roots.
\end{enumerate}
Because there is no difference between the general structure of $h(x)$ and $b(x)$ once the Hamming weight is fixed, we will focus on only $g(x)$ and $b(x)$ (\ref{novelEq3}) in the discussion that follows.
\newpage
\subsubsection{Solution for $w_H(\bb) =2$}
For this weight case, we can write $b(x)$ as $b(x)=1+x^a$ without any loss of generality. Thus we wish to find the possible values of $a$ satisfying $$\beta_i^a+1=0$$
for all $\beta_i,~0 \leq i < 2^{order(g(x))}$ in GF$(2^{order(g(x))})$
\paragraph{ Case1: $g(x)$ is a single primitive polynomial}
Since $g(x)$ is a prime polynomial, it has a primitive element $\beta$ as its root, which means, $\beta$ is also a root of $b(x)=1+x^a$. Substituting $\beta$ into the equation, we have 
\begin{equation}
\begin{split}
1+\beta^a=&0\\
\beta^a=&1
\end{split}
\end{equation}
For any primitive polynomial that generates the extended field GF$(2^m),~\beta^{2^{m}-1}=1$, 
which means any valid of $a$ should satisfy the condition below.
$$2^{m}-1~|~a$$

\begin{example}
$g(x)=1+x+x^2$.\newline $g(x)$ generates the field GF$(2^2)$ where $\beta^{3}=1$. The valid values of $a$ are $a=\{3,6,9,\cdots \}$. The corresponding values for $a(x)$ and $b(x)$ are shown in the table below for the first four valid values of $a$.

 \begin{table*}[h]
 \caption{$g(x)=1+x+x^2$}
\centering
 \begin{tabular}{c c c} 
 \hline
 $a(x)$ & $b(x)$ \\ [0.5ex] 
 \hline\hline
$1+x$
 & $1+x^{3}$ \\
\hline
$1+x+x^3+x^4$
 & $1+x^{6}$ 
 \\
\hline
$1+x+x^3+x^4+x^6+x^{7}$ 
&  $1+x^{9}$ 
\\
\hline
$1+x+x^3+x^4+x^6+x^{7}+x^9+x^{10}$
 &  $1+x^{12}$ \\
 \end{tabular}
 \label{novelTab2}
\end{table*}
\end{example}

\begin{example}$g(x)=1+x+x^4$\newline
$g(x)$ can be used to generate the extended field GF$(2^4)$. In this field, $\beta^{15}=1$. The valid values of $a$ are $a=\{15,30,45,\cdots \}$. The corresponding values for $a(x)$ and $b(x)$ are shown in the table below for the first two valid values of $a$.

 \begin{table*}[h!]
 \caption{$23/35$ RSC Code, $f(x)=1+x+x^4$}
\centering
\begin{tabular}{p{7cm} | p{7cm}} 
 \hline
 $a(x)$ & $b(x)$  \\ [0.5ex] 
 \hline\hline
$1+x^2+x^3+x^5+x^7+x^8+x^{11}$ 
& $1+x^{15}$ \\ 
\hline
$1+x^2+x^3+x^5+x^7+x^8+x^{11}+x^{15}+x^{16}+x^{17}+x^{18}+x^{20}+x^{22}+x^{23}+x^{26}$ 
&$1+x^{30}$\\
 \end{tabular}
 \label{novelTab5}
\end{table*}

\end{example}
\paragraph{ Case2: $g(x)$ is prime polynomial but not primitive \newline}
Similar to the case for primitive polynomials, there is a value $j < 2^m-1$ such that 
$$\beta^j=1,~j~|~2^{m}-1 $$. Therefore, any valid value of $a$ should satisfy the condition below.
$$j~|~a$$

\begin{example}
$g(x)=1+x+x^2+x^3+x^4$\newline
$g(x)$ can be used used to generate GF$(2^4)$ and in the field it generates, $\beta^{5}=1$. Therefore, the valid values of $a$ are $a=\{5,10,15,\cdots\}$. The corresponding values for $b(x)$ and $a(x)$ are shown in the table below for the first four valid values of $a$.

\begin{table*}[h]
 \caption{$37/21$ RSC Code, $f(x)=1+x+x^2+x^3+x^4$}
\centering
 \begin{tabular}{c c} 
 \hline
 $a(x)$ & $b(x)$  \\ [0.5ex] 
 \hline\hline
$1+x$ &$1+x^5$\\ 
$1+x+x^5+x^6$ &$1+x^{10}$  \\
$1+x+x^5+x^6+x^{10}+x^{11}$ & $1+x^{15}$ \\
$1+x+x^5+x^6+x^{10}+x^{11}+x^{15}+x^{16}$ &$1+x^{20}$  
 \end{tabular}
 \label{novelTab3}
\end{table*}
\end{example}

%\paragraph{ Case3: $g(x)$ is made up of repeated polynomial roots.\newline}
%Given the above condition we have, $r(x)=(r^{o_p}_p(x))^k$, where $r^{o_p}_p(x)$ represents the prime polynomial $r(x)$ can be factorised into and $k$ is the number of times it is repeated. $r^{o_p}_p(x)$ has $\beta$ as its roots and we need to find $d \st \beta^j=1$ in the (extended) field it generates. Because $\beta^{kj}=1$ in the same (extended) field , the valid values for $a$ should satisfy the condition below:
%$$kj ~| ~a$$

\paragraph{Case3: $r(x)$ is made up of repeated polynomial roots. \newline}
We may write $r(x)$ as $$r(x)=\prod_{k=1}^{K}r_k(x)$$ The $kth$ polynomial is prime and has a value $d_k \st \beta^{j_k}=1$. Because each value of $j_k$ is unique, valid values of $a$ should satisfy the condition below. 

$$ \bigcap_{k=1}^{K} \{j_k~| a\}$$
For the special case where $g(x)$ has equal repeated roots, the above condition simplifies to 
$$kj ~| ~a$$

\begin{example}
$g(x)=1+x^2$\newline
$g(x)$ can be written as $$f(x)=(1+x)^2,~k=2$$ $1+x$ is prime in $GF(2)$ and $\beta^{1}=1$. Since $k=2$, we have $\beta^{k}=\beta^{2}=1$. $h(x)=1+x^a$ and valid values of $a=\{2,4,6,\cdots \}$.
The corresponding values for $a(x)$ and $b(x)$ are shown in the table below for the first four valid values of $a$

 \begin{table*}[h]
 \caption{$5/7$ RSC Code, $f(x)=1+x^2$}
\centering
\begin{tabular}{c c } 
\hline
 $a(x)$ & $b(x)$ \\ [0.5ex] 
 \hline\hline
$1$ & $1+x^2$\\ 
$1+x^2$ & $1+x^4$ \\
$1+x^2+x^4$ & $1+x^6$\\
$1+x^2+x^4+x^6$ & $1+x^8$ 
\end{tabular}
 \label{novelTab1}
\end{table*}
\end{example}

%==========
\begin{example}
$g(x)=1+x^4$\newline $g(x)$ is made up of equal repeated polynomial roots and can be written as $$g(x)=(1+x)^4,~k=4$$. $1+x$ is prime in $GF(2)$ and $\beta^{1}=1$. Since $k=4$, we have $\beta^{k}=\beta^{4}=1$. $b(x)=1+x^b$ and the valid values of $a=\{4,8,12,\cdots \}$.
The corresponding values for $a(x),~b(x)$ and $h(x)$ are shown in the table below for the first four valid values of $a$

\begin{table*}[h]
\caption{$37/21$ RSC Code, $g(x)=1+x^4$}
\centering
 \begin{tabular}{c c c} 
 \hline
$a(x)$ & $h(x)$ \\ [0.5ex] 
\hline\hline
$1$ &  $1+x^4$\\ 
$1+x^4$ & $1+x^8$ \\
$1+x^4+x^8$ & $1+x^{12}$ \\
$1+x^4+x^8+x^{12}$ & $1+x^{16}$ 
\end{tabular}
\label{novelTab4}
\end{table*}
\end{example}

%========
\begin{example}
$g(x)=1+x^2+x^3+x^4$\newline
$g(x)$ can be written as $$g(x)=(1+x)(1+x+x^3),~K=2$$
 $1+x$ is prime in $GF(2^1)$ and $\beta^{1}=1$. $1+x+x^3$ is prime in $GF(2^3)$ and $\beta^{7}=1$. $b(x)=1+x^b$ and consequently, the valid values of $a$ that meet the condition $$ \bigcap_{k=1}^{K} \{j_k~| a\}$$ are $a=\{7,14,21,\cdots \}$.
The corresponding values for $a(x)$ and $b(x)$ are shown in the table below for the first four valid values of $a$

\begin{table*}[h]
\caption{$23/35$ RSC Code, $g(x)=1+x^2+x^3+x^4$}
\centering
 \begin{tabular}{p{6cm}| p{8cm}} 
 \hline
 $a(x)$ & $b(x)$  \\ [0.5ex] 
 \hline\hline
$1+x^2+x^3$ & $1+x^7$ \\ 
\hline
$1+x^2+x^3+x^7+x^9+x^{10}$ &  $1+x^{14}$ \\
\hline
$1+x^2+x^3+x^7+x^{9}+x^{10}+x^{14}+x^{16}+x^{17}$ & $1+x^{21}$ 
\\
\hline
$1+x^2+x^3+x^7+x^{9}+x^{10}+x^{14}+x^{16}+x^{17}+x^{21}+x^{23}+x^{24}$ & $1+x^{28}$
 \end{tabular}
 \label{novelTab6}
\end{table*}
\end{example}
%=========Examples===============%

\newpage

%\begin{example}
%$5/7$ RSC Code $f(x)=1+x^2,~g(x) = 1+x+x^2$\newline
%$f(x)$ is a match for Case3, i.e. it is made up of equal repeated polynomial roots. It can be written as $$f(x)=(1+x)^2,~k=2$$ $1+x$ is prime in $GF(2)$ and $\beta^{1}=1$. Since $k=2$, we have $\beta^{k}=\beta^{2}=1$. $h(x)=1+x^a$ and valid values of $a=\{2,4,6,\cdots \}$.
%The corresponding values for $a(x),~b(x)$ and $h(x)$ are shown in the table below for the first four valid values of $a$

% \begin{table*}[h]
% \caption{$5/7$ RSC Code, $f(x)=1+x^2$}
%\centering
 %\begin{tabular}{c c c} 
 %\hline
% $a(x)$ & $b(x)$ & $h(x)$ \\ [0.5ex] 
% \hline\hline
%$1$ & $1+x+x^2$ & $1+x^2$ \\ 
%$1+x^2$ & $1+x+x^3+x^4$ & $1+x^{4}$  \\
%$1+x^2+x^4$ & $1+x+x^3+x^5+x^6$ & $1+x^{6}$  \\
%$1+x^2+x^4+x^6$ & $1+x+x^3+x^5+x^7+x^8$ & $1+x^{8}$  
 %\end{tabular}
% \label{novelTab1}
%\end{table*}

%$g(x)$ is a match for Case1, i.e. it is a primitive polynomial for GF$(2^2)$. In this extended field, $\beta^{3}=1$. $b(x)=1+x^b$ and the valid values of $b$ are $b=\{3,6,9,\cdots \}$. The corresponding values for $a(x),~b(x)$ and $h(x)$ are shown in the table below for the first four valid values of $b$.

 %\begin{table*}[h]
 %\caption{$5/7$ RSC Code, $g(x)=1+x+x^2$}
%\centering
 %\begin{tabular}{c c c} 
 %\hline
 %$a(x)$ & $b(x)$ & $h(x)$ \\ [0.5ex] 
 %\hline\hline
%$1+x$ & $1+x^{3}$ &$1+x+x^2+x^3$\\
%\hline
%$1+x+x^3+x^4$ & $1+x^{6}$ &$1+x+x^2+x^4+x^5+x^6$ \\
%\hline
%$1+x+x^3+x^4+x^6+x^{7}$ &  $1+x^{9}$ %&$1+x+x^2+x^4+x^5+x^7+x^8+x^9$\\
%\hline
%$1+x+x^3+x^4+x^6+x^{7}+x^9+x^{10}$ &  $1+x^{12}$ %&$1+x+x^2+x^4+x^5+x^7+x^8+x^{10}+x^{11}+x^{12}$\\
% \end{tabular}
 %\label{novelTab2}
%\end{table*}
%\end{example}


%=======37/21 RSC=========%
%\begin{example}
%$37/21$ RSC Code, $f(x)=1+x+x^2+x^3+x^4,~g(x)=1+x^4$\newline
%$f(x)$ is a match for Case2, i.e. it is a prime polynomial but not primitive. It can be used to generate GF$(2^4)$ and in the field it generates, $\beta^{5}=1$. Therefore, the valid values of $a$ are $a=\{5,10,15,\cdots\}$. The corresponding values for $h(x)$ and $a(x)$ are shown in the table below for the first four valid values of $a$.

% \begin{table*}[h]
% \caption{$37/21$ RSC Code, $f(x)=1+x+x^2+x^3+x^4$}
%\centering
% \begin{tabular}{c c c} 
% \hline
% $a(x)$ & $b(x)$ & $h(x)$ \\ [0.5ex] 
% \hline\hline
%$1+x$ &$1+x+x^4+x^5$ &  $1+x^5$\\ 
%$1+x+x^5+x^6$ &$1+x+x^4+x^6+x^9+x^{10}$ & $1+x^{10}$ \\
%$1+x+x^5+x^6+x^{10}+x^{11}$ & $1+x+x^4+x^6+x^9+x^{11}+x^{14}+x^{15}$ & $1+x^{15}$ \\
%$1+x+x^5+x^6+x^{10}+x^{11}+x^{15}+x^{16}$ &$1+x+x^4+x^6+x^9+x^{11}+x^{14}+x^{16}+x^{19}+x^{20}$ & $1+x^{20}$ 
% \end{tabular}
% \label{novelTab3}
%\end{table*}

%$g(x)$ is a match for Case3, i.e. it is made up of equal repeated polynomial roots. It can be written as $$g(x)=(1+x)^4,~k=4$$ $1+x$ is prime in $GF(2)$ and $\beta^{1}=1$. Since $k=4$, we have $\beta^{k}=\beta^{4}=1$. $b(x)=1+x^b$ and valid values of $b=\{4,8,12,\cdots \}$.
%The corresponding values for $a(x),~b(x)$ and $h(x)$ are shown in the table below for the first four valid values of $a$

 %\begin{table*}[h]
 %\caption{$37/21$ RSC Code, $g(x)=1+x^4$}
%\centering
% \begin{tabular}{c c c} 
% \hline
 %$a(x)$ & $h(x)$ \\ [0.5ex] 
 %\hline\hline
%$1$ &  $1+x^4$ & $1+x+x^2+x^3+x^4$\\ 
%$1+x^4$ & $1+x^{8}$ & $1+x+x^2+x^3+x^5+x^6+x^7+x^8$ \\
%$1+x^4+x^8$ & $1+x^{12}$ & $1+x+x^2+x^3+x^5+x^6+x^7+x^9+x^{10}+x^{11}+x^{12}$\\
%$1+x^4+x^8+x^{12}$ & $1+x^{16}$ & $1+x+x^2+x^3+x^5+x^6+x^7+x^9+x^{10}+x^{11}+x^{13}+x^{14}+x^{15}+x^{16}$
 %\end{tabular}
 %\label{novelTab4}
%\end{table*}

%\end{example}
%===========23/35 RSC=========%
%\begin{example}
%$23/35$ RSC Code, $f(x)=1+x+x^4,~g(x)=1+x^2+x^3+x^4$\newline
%$f(x)$ is a match for Case1, i.e. it is a primitive polynomial that can be used to generate the extended field GF$(2^4)$. In this field, $\beta^{15}=1$. $h(x)=1+x^a$ and the valid values of $a$ are $a=\{15,30,45,\cdots \}$. The corresponding values for $a(x),~b(x)$ and $h(x)$ are shown in the table below for the first two valid values of $a$.

% \begin{table*}[h!]
% \caption{$23/35$ RSC Code, $f(x)=1+x+x^4$}
%\centering
 %\begin{tabular}{p{7cm} | p{7cm} | c} 
% \hline
% $a(x)$ & $b(x)$ & $h(x)$ \\ [0.5ex] 
% \hline\hline
%$1+x^2+x^3+x^5+x^7+x^8+x^{11}$ 
%& $1+x+x^3+x^4+x^7+x^{11}+x^{12}+x^{13}+x^{14}+x^{15}$ 
%&$1+x^{15}$\\ 
%\hline
%$1+x^2+x^3+x^5+x^7+x^8+x^{11}+x^{15}+x^{16}+x^{17}+x^{18}+x^{20}+x^{22}+x^{23}+x^{26}$ 
%&$1+x+x^3+x^4+x^7+x^{11}+x^{12}+x^{13}+x^{14}+x^{16}+x^{18}+x^{19}+x^{22}+x^{26}+x^{27}+x^{28}+x^{29}+x^{30}$ 
%&$1+x^{30}$\\
% \end{tabular}
% \label{novelTab5}
%\end{table*}

%$g(x)$ is a match for Case4, i.e. it is made up of unique repeated polynomial roots. It be written as $$g(x)=(1+x)(1+x+x^3),~K=2$$
 %$1+x$ is prime in $GF(2^1)$ and $\beta^{1}=1$. $1+x+x^3$ is prime in $GF(2^3)$ and $\beta^{7}=1$. $b(x)=1+x^b$ and consequently, the valid values of $b$ that meet the condition $$ \bigcap_{k=1}^{K} \{j_k~| b\}$$ are $b=\{7,14,21,\cdots \}$.
%The corresponding values for $a(x),~b(x)$ and $h(x)$ are shown in the table below for the first four valid values of $b$

%\begin{table*}[h]
% \caption{$23/35$ RSC Code, $g(x)=1+x^2+x^3+x^4$}
%\centering
% \begin{tabular}{p{6cm}| c | p{8cm}} 
% \hline
% $a(x)$ & $b(x)$ & $h(x)$ \\ [0.5ex] 
% \hline\hline
%$1+x^2+x^3$ & $1+x^7$ &$1+x+x^2+x^6+x^7$\\ 
%\hline
%$1+x^2+x^3+x^7+x^9+x^{10}$ &  $1+x^{14}$ &$1+x+x^2+x^6+x^8+x^9+x^{13}+x^{14}$\\
%\hline
%$1+x^2+x^3+x^7+x^{9}+x^{10}+x^{14}+x^{16}+x^{17}$ & $1+x^{21}$ 
%&$1+x+x^2+x^6+x^8+x^9+x^{13}+x^{15}+x^{16}+x^{20}+x^{21}$\\
%\hline
%$1+x^2+x^3+x^7+x^{9}+x^{10}+x^{14}+x^{16}+x^{17}+x^{21}+x^{23}+x^{24}$ & $1+x^{28}$ &$1+x+x^2+x^6+x^8+x^9+x^{13}+x^{15}+x^{16}+x^{20}+x^{22}+x^{23}+x^{27}+x^{28}$
% \end{tabular}
% \label{novelTab6}
%\end{table*}

%\end{example}


%\begin{example}
%$f(x)=1+x^2+x^3+x^4$\newline
%$f(x)$ is reducible and can be written as $$f(x)=(1+x)(1+x+x^3),~K=2$$
 %$1+x$ is prime in $GF(2^1)$ and $\beta^{1}=1$. $1+x+x^3$ is prime in $GF(2^3)$ and $\beta^{7}=1$ Consequently, the valid values of $a$ that meet the condition $$ \bigcap_{k=1}^{K} \{j_k~| a\}$$ are $a=\{7,14,21,\cdots \}$.
%The corresponding values for $h(x)$ and $a(x)$ are shown in the table below for the first 4 valid values of $a$

%\begin{table*}[h]
 %\caption{$f(x)=1+x^2+x^3+x^4$}
%\centering
 %\begin{tabular}{c c} 
% \hline
 %$a(x)$ & $h(x)$ \\ [0.5ex] 
% \hline\hline
%$1+x^2+x^3$ & $1+x^7$\\ 
%\hline
%$1+x^2+x^3+x^7+x^9+x^{10}$ &  $1+x^{14}$ \\
%\hline
%$1+x^2+x^3+x^7+x^{9}+x^{10}+x^{14}+x^{16}+x^{17}$ & $1+x^{21}$ \\
%\hline
%$1+x^2+x^3+x^7+x^{9}+x^{10}+x^{14}+x^{16}+x^{17}+x^{21}+x^{23}+x^{24}$ & $1+x^{28}$ 
% \end{tabular}
% \label{novelTab4}
%\end{table*}

%\end{example}


\subsubsection{Solution for $w_H(\bb) =3$}
For this weight case, $b(x)=1+x^u+x^v,~v\neq u$ without loss of generality. Given $g(x)$, our task is to find valid $(u,~v)$ pair values satisfying the condition 
$$ \beta_i^u+\beta_j^v=1$$
for all $\beta_i,\beta_j,~0 \leq i <2^{order(g(x))},~0 \leq j <2^{order(g(x))},~i\neq j$.

 If there are no values for the pair $(u,v)$, then $q(x) \st w_H(\bq) =3$ does not exist for the given $g(x)$.
% It is worth noting that such $h(x)$ only exist iff $f(x)$ has exactly $w_H(\bh) =3$ terms. Moving forward, we assume that for all the cases, this condition holds true.
\paragraph{ Case1} $r(x)$ is a single primitive polynomial \newline
Because $r(x)$ is a prime polynomial, it has a primitive element $\beta$ as its root, which means\newline that $\beta$ is also a root of $q(x)=1+x^u+x^v$. Substituting $\beta$ into the equation, we have 
\begin{equation}
\begin{split}
1+\beta^u+\beta^v=&0\\
\beta^u+\beta^v=&1
\end{split}
\end{equation}
where $u\neq v$
By refering to the table of the extended field for GF$(2^m)$, we can find the valid $(e,f)$ pairs $\st \beta^e+\beta^f=1,~e\neq f$.  If we represent the set of $(e,f)$ pairs as 
$\bz=\{ (e_1,f_1) ,( e_2,f_2),\cdots\} $, then any valid value for $u$ and $v$ should satisfy the condition
$$(u,v) \equiv (e,f) \bmod 2^m-1,~(e,f)\in \bz$$ since $\beta^{2^{m}-1}=1$.

\begin{example}
$g(x)=1+x+x^2$ \newline
The elements of GF$(2^2)$ are shown in Table \ref{novelTab5} and it is obvious that there is exactly 1 valid $(e,f)$ pair $\st \beta^e+\beta^f = 1$ and that is the pair $(1,2)$.
This means that valid values of the $(u,v)$ pairs are any values $\st (u,v) \equiv (1,2) \bmod 3$.  The corresponding values for $a(x)$ and $b(x)$ are shown in the table below for the first four valid values of $(u,v)$

 \begin{table*}[h]
 \caption{Non-zero Elements of GF$(2^2)$ generated by $g(x)=1+x+x^2$}
\centering
 \begin{tabular}{c c} 
 \hline
 power representation & actual value \\ [0.5ex] 
 \hline\hline
$\beta^0~=\beta^3=1$ & $1$\\
\hline
$\beta$ & $\beta$\\
\hline
$\beta^2$ &  $1+\beta$\\
 \end{tabular}
 \label{novelTab7}
\end{table*}

\begin{table*}[h]
 \caption{$g(x)=1+x+x^2$}
\centering
 \begin{tabular}{c c} 
 \hline
 $a(x)$ & $b(x)$\\ [0.5ex] 
 \hline\hline
$1$ & $1+x+x^2$\\ 
\hline
$1+x+x^2$ &  $1+x^2+x^4$\\
\hline
$1+x+x^3$ & $1+x^4+x^5$\\
\hline
$1+x^2+x^3$ & $1+x+x^5$ 
 \end{tabular}
 \label{novelTab8}
\end{table*}
\end{example}

\begin{example}
$g(x) = 1+x+x^4$\newline  The elements of GF$(2^4)$ are shown in Table \ref{novelTab10} and we can see that there are seven valid $(e,f)$ pairs.
 
 This means that the valid values of the $(u,v)$ pairs are any values $\st (u,v) \equiv (e,f) \bmod 15, (e,f) \in \bz,~\bz=\{(1,4),~(2,8) ,~(3,14) ,~(5,10) ,~(6,13) ,~(7,9) ,~(11,12))$.
 The corresponding values for $a(x)$ and $b(x)$ are shown in Table \ref{novelTab11} below for the first four valid values of $(u,v)$
 
  \begin{table*}[h]
 \caption{Non-zero Elements of GF$(2^4)$ generated by $g(x)=1+x+x^4$}
\centering
 \begin{tabular}{c c} 
 \hline
 power & polynomial \\ [0.5ex] 
 \hline\hline
$\beta^0~=\beta^{15}=1$ & $1$\\
\hline
$\beta$ & $\beta$\\
\hline
$\beta^2$ &  $\beta^2$\\
\hline
$\beta^3$ & $\beta^3$\\
\hline
$\beta^4$ &  $\beta+1$\\
\hline
$\beta^5$ & $\beta^2+\beta$\\
\hline
$\beta^6$ &  $\beta^3+\beta^2$\\
\hline
$\beta^7$ & $\beta^3+\beta+1$\\
\hline
$\beta^8$ &  $\beta^2+1$\\
\hline
$\beta^9$ & $\beta^3+\beta$\\
\hline
$\beta^{10}$ &  $\beta^2+\beta+1$\\
\hline
$\beta^{11}$ & $\beta^3+\beta^2+\beta$\\
\hline
$\beta^{12}$ &  $\beta^3+\beta^2+\beta+1$\\
\hline
$\beta^{13}$ & $\beta^3+\beta^2+1$\\
\hline
$\beta^{14}$ &  $\beta^3+1$\\
 \end{tabular}
 \label{novelTab10}
\end{table*}
 
 \begin{table*}[h]
 \caption{ $f(x)=1+x+x^4$}
\centering
 \begin{tabular}{c c} 
 \hline
 $a(x)$ & $b(x)$\\ [0.5ex] 
 \hline\hline
$1$ & $1+x+x^4$\\ 
\hline
$1+x+x^2+x^3+x^5$ & $1+x^7+x^9$ \\
\hline
$1+x+x^2+x^3+x^5+x^7+x^8$ & $1+x^{11}+x^{12}$\\
\hline
$1+x+x^4$ &$1+x^2+x^8$
 \end{tabular}
 \label{novelTab11}
\end{table*}
\end{example}

\paragraph{Case2}$r(x)$ is prime but not a primitive polynomial\newline
A prime but not primitive polynomial generates an (extended) field with $j$ elements, where $j<2^m-1$. Similar to the case where $r(x)$ is primitive, we confirm the existence of $(e,f)$ pairs $ \st \beta^e + \beta^f =1$. 
If we represent the set of $(e,f)$ pairs as 
$\bz=\{ (e_1,f_1) , e_2,f_2),\cdots,(e_1,f_q)\} $, then any valid value for $u$ and $v$ should satisfy the condition
$$(u,v) \equiv (e,f) \bmod j,~(e,f)\in \bz$$ since $\beta^{j}=1$

\begin{example}
$g(x)=1+x+x^2+x^3+x^4$ \newline
From the table of the extended field generated by $g(x)$ (Table \ref{novelTab9}), we see that there are no valid $(e,~f)$ and as such $b(x) \st w_H(\bb)=3$ is non-existent.


 \begin{table*}[h]
 \caption{Non-zero Elements of GF$(2^4)$ generated by $f(x)=1+x+x^2+x^3+x^4$}
\centering
 \begin{tabular}{c c} 
 \hline
 power & polynomial \\ [0.5ex] 
 \hline\hline
$\beta^0~=\beta^5=\beta^{10}=\beta^{15}=1$ & $1$\\
\hline
$\beta=\beta^6=\beta^{11}$ & $\beta$\\
\hline
$\beta^2=\beta^7=\beta^{12}$ &  $\beta^2$\\
\hline
$\beta^3=\beta^8=\beta^{13}$ &  $\beta^3$\\
\hline
$\beta^4=\beta^9=\beta^{14}$ &  $\beta^3+\beta^2+\beta+1$\\
 \end{tabular}
 \label{novelTab9}
\end{table*}

\end{example}

%\paragraph{Case3: $r(x)$ is made up of equal repeated polynomial roots\newline}
%For this case, we can write $r(x)$ as 
%$r(x)=(r^{o_p}_p(x))^k$, where $r^{o_p}_p(x)$ represents the prime polynomial, which is the root of $r(x)$ and $k$ is the number of times it is repeated. If $r^{o_p}_p(x)$ is a primitive polynomial and $m>1$, then from Case1 there are exactly $q=2^{(m-1)}-1~(e,~f)$ pairs $\st \beta^e+\beta^f=1,~e \neq f$ in set $\bz$. 
%If $r^{o_p}_p(x)$ is prime but not a primitive polynomial, then we determine the number of elements in the set $\bz$ directly from the table representing the field it generates.
% If we focus on $f^{o_p}_p(x)$ only, $(u',v')$ should satisfy the condition$$ (u',v') \equiv (e,f)\bmod 2^{m}-1,~(e, f) \in \bz$$. However, since $f(x)$ is made up of $f^{o_p}_p(x)$ repeated $k$ times, and $\beta^{ke}+\beta^{kf}=1$, the valid values for $u$ and $v$
 %should satisfy the condition
% $$(u,v)=(ku',kv'),~(u,v) \equiv (e,f)\bmod 2^{m}-1,~(e, f) \in \bz $$.

\paragraph{Case3}$g(x)$ is made up of repeated polynomial roots\newline
We may write $g(x)$ as $$r(x)=\prod_{k=1}^{K}r_k(x)$$ 
For the $kth$ polynomial, we refer to the (extended) field table and determine if there exists any $(e^{(k)},~f^{(k)})$ pairs $\st \beta^{e^{(k)}}+\beta^{f^{(k)}}=1$. If it exists for all $K$ polynomials, then the valid values for $u$ and $v$
 should satisfy the condition

 $$\textcolor{red}{\bigcap_{k=1}^{K} (u,v)=(ku',kv'),~(u,v) \equiv (e^{(k)},f^{(k)})\bmod 2^{m}-1,~(e^{(k)}, f^{(k)}) \in \bz^{(k)}} $$
 
A special case is when $g(x)$ has equal and repeated roots. The above condition simplifies to 
 $$(u,v)=(ku',kv'),~(u,v) \equiv (e,f)\bmod 2^{m}-1,~(e, f) \in \bz $$.
 
 \begin{example}
 $g(x)=1+x^2$ \newline $g(x)$ can be written as $(1+x)^2$. $(1+x)$ is a primitive polynomial for GF$(2)$. The elements in GF$(2)$ are $1$ and $\beta$. In this field,  there are no valid $(e,f)$ pair values; therefore, $h(x) \st w_H(\bh)=3$ does not exist.
 \end{example}
 
  \begin{example}
 $g(x) = 1+x^4$.\newline
 After factorisation, we have $g(x)=(1+x)^4. $(1+x) is a primitive polynomial that generates the field GF$(2)$ and in this field, there are also no valid $(e,~f)$ and as such $b(x) \st w_H(\bb)=3$ is also non-existent.
 \end{example}
 
  \begin{example}
 $g(x)=1+x^2+x^3+x^4$.\newline
  Upon factorising, we have $g(x)=(1+x)(1+x+x^3)$. Table \ref{novelTab12} shows the elements of GF$(2^3)$ generated by $(1+x+x^3)$ and we can confirm that there are three valid $(e,~f)$ pairs. However, since there are no valid $(e,~f)$ pairs in GF$(2)$, it also means that there cannot be any valid $(u,~v)$ pairs and $b(x) \st w_H(\bb)=3$ does not exist.
 
 \begin{table*}[h]
 \caption{Non-zero Elements of GF$(2^3)$ generated by $1+x+x^3$}
\centering
 \begin{tabular}{c c} 
 \hline
 power & polynomial \\ [0.5ex] 
 \hline\hline
$\beta^0~=\beta^{7}=1$ & $1$\\
\hline
$\beta$ & $\beta$\\
\hline
$\beta^2$ &  $\beta^2$\\
\hline
$\beta^3$ & $\beta+1$\\
\hline
$\beta^4$ &  $\beta^2+\beta$\\
\hline
$\beta^5$ & $\beta^2+\beta+1$\\
\hline
$\beta^6$ &  $\beta^2+1$\\
 \end{tabular}
 \label{novelTab12}
\end{table*}
 \end{example}

%\begin{example}
%$5/7$ RSC Code, $f(x)=1+x^2,~g(x)=1+x+x^2$\newline
%$f(x)$ is a match for Case3 and can be written as $(1+x)^2$. $(1+x)$ is a primitive polynomial for GF$(2)$. The elements in GF$(2)$ are $1$ and $\beta$. In this field,  there are no valid $(e,f)$; therefore, $h(x) \st w_H(\bh)=3$ does not exist.

%$g(x)$ is a match for Case1, i.e. it is a primitive polynomial for GF$(2^2)$ with $\beta^{3}=1$. 
%The elements of GF$(2^2)$ are shown in Table \ref{novelTab5} and it is obvious that there is exactly 1 ($q=2^{(2-1)}-1~=1$) valid $(e,f)$ pair $\st \beta^e+\beta^f = 1$ and that is $(1,2)$.
%This means that valid values of the $(u,v)$ pairs are any values $\st (u,v) \equiv (1,2) \bmod 3$.  The corresponding values for $a(x),~b(x)$ and $h(x)$ are shown in the table below for the first four valid values of $(u,v)$

% \begin{table*}[h]
 %\caption{Non-zero Elements of GF$(2^2)$ generated by $g(x)=1+x+x^2$}
%\centering
% \begin{tabular}{c c} 
% \hline
 %power representation & actual value \\ [0.5ex] 
% \hline\hline
%$\beta^0~=\beta^3=1$ & $1$\\
%\hline
%$\beta$ & $\beta$\\
%\hline
%$\beta^2$ &  $1+\beta$\\
% \end{tabular}
 %\label{novelTab7}
%\end{table*}

%\begin{table*}[h]
% \caption{$5/7$ RSC, $g(x)=1+x+x^2$}
%\centering
% \begin{tabular}{c c c} 
% \hline
% $a(x)$ & $b(x)$ & $h(x)$\\ [0.5ex] 
% \hline\hline
%$1$ & $1+x+x^2$ & $1+x^2$\\ 
%\hline
%$1+x+x^2$ &  $1+x^2+x^4$ & $1+x+x^3+x^4$ \\
%\hline
%$1+x+x^3$ & $1+x^4+x^5$ & $1+x+x^2+x^5$\\
%\hline
%$1+x^2+x^3$ & $1+x+x^5$  &$1+x^3+x^4+x^5$
 %\end{tabular}
 %\label{novelTab8}
%\end{table*}
%\end{example}

%\newpage

 
 %\begin{example}
%$37/21$ RSC Code, $f(x)=1+x+x^2+x^3+x^4,~g(x)=1+x^4$\newline
%$f(x)$ is a match for Case2, i.e. it is a prime but not a primitive polynomial. From the table of the extended field generated by $f(x)$ (Table \ref{novelTab9}), we see that there are no valid $(e,~f)$ and as such $h(x) \st w_H(\bh)=3$ is non-existent.


 %\begin{table*}[h]
 %\caption{Non-zero Elements of GF$(2^2)$ generated by $f(x)=1+x+x^2+x^3+x^4$}
%\centering
 %\begin{tabular}{c c} 
 %\hline
 %power & polynomial \\ [0.5ex] 
 %\hline\hline
%$\beta^0~=\beta^5=\beta^{10}=\beta^{15}=1$ & $1$\\
%\hline
%$\beta=\beta^6=\beta^{11}$ & $\beta$\\
%\hline
%$\beta^2=\beta^7=\beta^{12}$ &  $\beta^2$\\
%\hline
%$\beta^3=\beta^8=\beta^{13}$ &  $\beta^3$\\
%\hline
%$\beta^4=\beta^9=\beta^{14}$ &  $\beta^3+\beta^2+\beta+1$\\
 %\end{tabular}
 %\label{novelTab9}
%\end{table*}

%$g(x)$ is a match for Case3, i.e. it is made up of equal repeated roots. After factorisation, we have $g(x)=(1+x)^4. $(1+x) is a primitive polynomial that generates the field GF$(2)$ and in this field, there are also no valid $(e,~f)$ and as such $b(x) \st w_H(\bb)=3$ is also non-existent.
%\end{example}
 
%\begin{example}
%$23/35$ RSC Code, $f(x)=1+x+x^4,~g(x)=1+x^2+x^3+x^4$ \newline
%$f(x)$ is a match for Case1, i.e. it is a primitive polynomial that can be used to generate the extended field GF$(2^4)$ with $\beta^{15}=1$.. The elements of GF$(2^4)$ are shown in Table \ref{novelTab10} and we can see that there are seven valid $(e,f)$ pairs.
 
 %This means that the valid values of the $(u,v)$ pairs are any values $\st (u,v) \equiv (e,f) \bmod 15, (e,f) \in \bz,~\bz=\{(1,4),~(2,8) ,~(3,14) ,~(5,10) ,~(6,13) ,~(7,9) ,~(11,12))$.
 %The corresponding values for $a(x),~b(x)$ and $h(x)$ are shown in Table \ref{novelTab11} below for the first four valid values of $(u,v)$
 
  %\begin{table*}[h]
 %\caption{Non-zero Elements of GF$(2^4)$ generated by $f(x)=1+x+x^4$}
%\centering
 %\begin{tabular}{c c} 
 %\hline
 %power & polynomial \\ [0.5ex] 
 %\hline\hline
%$\beta^0~=\beta^{15}=1$ & $1$\\
%\hline
%$\beta$ & $\beta$\\
%\hline
%$\beta^2$ &  $\beta^2$\\
%\hline
%$\beta^3$ & $\beta^3$\\
%\hline
%$\beta^4$ &  $\beta+1$\\
%\hline
%$\beta^5$ & $\beta^2+\beta$\\
%\hline
%$\beta^6$ &  $\beta^3+\beta^2$\\
%\hline
%$\beta^7$ & $\beta^3+\beta+1$\\
%\hline
%$\beta^8$ &  $\beta^2+1$\\
%\hline
%$\beta^9$ & $\beta^3+\beta$\\
%\hline
%$\beta^{10}$ &  $\beta^2+\beta+1$\\
%\hline
%$\beta^{11}$ & $\beta^3+\beta^2+\beta$\\
%\hline
%$\beta^{12}$ &  $\beta^3+\beta^2+\beta+1$\\
%\hline
%$\beta^{13}$ & $\beta^3+\beta^2+1$\\
%\hline
%$\beta^{14}$ &  $\beta^3+1$\\
% \end{tabular}
% \label{novelTab10}
%\end{table*}
 
% \begin{table*}[h]
% \caption{$23/35$ RSC, $f(x)=1+x+x^4$}
%\centering
% \begin{tabular}{c c c} 
% \hline
% $a(x)$ & $b(x)$ & $h(x)$\\ [0.5ex] 
% \hline\hline
%$1$ & $1+x^2+x^3+x^4$ & $1+x+x^4$\\ 
%\hline
%$1+x+x^2+x^3+x^5$ &  $1+x^2+x^3+x^4+x^8+x^9$ & $1+x^7+x^9$ \\
%\hline
%$1+x+x^2+x^3+x^5+x^7+x^8$ & $1+x^2+x^3+x^4+x^7+x^{12}$ & $1+x^{11}+x^{12}$\\
%\hline
%$1+x+x^4$ & $1+x+x^2+x^4+x^5+x^6+x^7+x^8$  &$1+x^2+x^8$
 %\end{tabular}
% \label{novelTab11}
%\end{table*}
 
% $g(x)$ is a match for Case4, i.e. it is made up of unique repeated roots. Upon factorising, we have $g(x)=(1+x)(1+x+x^3)$. Table \ref{novelTab12} shows the elements of GF$(2^3)$ generated by $(1+x+x^3)$ and we can confirm that there are three valid $(e,~f)$ pairs. However, since there are no valid $(e,~f)$ pairs in GF$(2)$, it also means that there cannot be any valid $(u,~v)$ pairs and $b(x) \st w_H(\bb)=3$ does not exist.
 
% \begin{table*}[h]
% \caption{Non-zero Elements of GF$(2^3)$ generated by $1+x+x^3$}
%\centering
 %\begin{tabular}{c c} 
% \hline
 %power & polynomial \\ [0.5ex] 
% \hline\hline
%$\beta^0~=\beta^{7}=1$ & $1$\\
%\hline
%$\beta$ & $\beta$\\
%\hline
%$\beta^2$ &  $\beta^2$\\
%\hline
%$\beta^3$ & $\beta+1$\\
%\hline
%$\beta^4$ &  $\beta^2+\beta$\\
%\hline
%$\beta^5$ & $\beta^2+\beta+1$\\
%\hline
%$\beta^6$ &  $\beta^2+1$\\
%\end{tabular}
% \label{novelTab12}
%\end{table*}
%\end{example}



%\subsubsection{Higher order weight:$w_H=4$}%: Equal Repeated Polynomial Roots}
%If $w_H(\bq)=4$, then $q(x) = 1+x^{u}+x^{v}+x^{w}$.
%Finding the structure of $q(x)$ when $w_H=4$ for general cases has proven to be a difficult problem, for which we have not found a feasible solution to. However, there is a special case where have been able to come up with a method for determining the structure of $q(x)$. 

%\paragraph{$r(x)=1+x^{2i},~i=1,2,...$}
%Since $q(x)$ is divisible by $r(x)$, the roots of $r(x)$ are also the roots of $q(x)$. Inserting $\beta$ into $q(x)$, we get

%$$1+\beta^{u}+\beta^{v}+\beta^{w} =0$$
%Taking the derivative of the above equation, we get 
%\begin{equation*}
%\begin{split}
% &u\beta^{u-1}+v\beta^{v-1}+w\beta^{w-1} =0 \\
% & u\beta^{u-1}+v\beta^{v-1}+w\beta^{w-1} \equiv 0 \bmod 2
% \end{split}
% \end{equation*}
%We can see that to satisfy the equation above, there are 2 possible cases
%\begin{enumerate}
%\item $(u,~v,~w)$ is made up of two odd numbers, one even number.
%\item $(u,~v,~w)$ are all even numbers.
%\end{enumerate}

%For a valid $(u,~v,~w)$, we fix it into $q(x)$ and if it is divisible by $q(x)$, then $(u,~v,~w)$ is valid.
%In summary, valid $(u,~v,~w)$ should satisfy the condition
%$$u+v+w \equiv 0\bmod 2, ~1+\beta^{u}+\beta^{v}+\beta^{w} \bmod r(x) \equiv 0$$

%\begin{example}
%$5/7$ RSC code $f(x)=1+x^2,~g(x)=1+x+x^2$\newline
%$f(x)$ satisfies the special condition case and we can use the method above to obtain valid values of $(u,~v,~,w)$. The corresponding values for $a(x),~b(x)$ and $h(x)$ are shown in Table \ref{novelTab16} for the first four valid values of $(u,~v,~,w)$.

% \begin{table*}[h]
% \caption{$5/7$ RSC, $f(x)=1+x^2,~w_H(\bh)=4$}
%\centering
 %\begin{tabular}{c c c} 
% \hline
% $a(x)$ & $b(x)$ & $h(x)$\\ [0.5ex] 
% \hline\hline
%$1+x$ & $1+x^3$ & $1+x+x^2+x^3$\\ 
%\hline
%$1+x+x^2$ &  $1+x^2+x^4$ & $1+x+x^3+x^4$ \\
%\hline
%$1+x+x^3$ & $1+x^4+x^5$ & $1+x+x^2+x^5$\\
%\hline
%$1+x^2+x^3$ & $1+x+x^5$  &$1+x^3+x^4+x^5$
% \end{tabular}
% \label{novelTab16}
%\end{table*}

%\end{example}