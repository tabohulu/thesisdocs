\section{Introduction}
A Convolutional Code (CC) is generated by passing an input message through a linear finite-state shift register. The structure of this code is such that it is best described using a trellis. This structure makes it possible to employ soft decision decoding algorithms, the most popular of these algorithms being the Viterbi algorithm. CC are used extensively in mobile communication and space communication application as a major component in concatenated code.  Depending on the configuration of the shift register being used to generate the code, a CC can either be \textit{recursive} or \textit{nonrecursive}. In the case of the recursive CC, a feedback shift register is used to generate the code. Furthermore if the input message appears in the CC, it is known as \textit{systematic}. Recursive Sytematic Convolutional (RSC) codes are used as component codes for turbo codes, which are one of the few error correcting codes with performance very close to the Shannon limit [1].

Low weight codeword are produced when the parity bit sequence has a very low weight. Amongst all such codewords, the one with the lowest weight determines the free distance $d_{\text{free}}$ of the code. 
$d_{\text{free}}$  of a RSC code is a very important factor and determines its error-correction performance [4].  This can be obtained from the distance spectrum of the RSC code which requires the calculation of the transfer function. The distance spectrum provides information about the number of codewords of weight $d$ generated by a message input of weight $w$. 
The message inputs are such that they diverge from and then return to the initial state, assuming edge effect is ignored. These message inputs are referred to as Return-To-Zero (RTZ) inputs and with respect to interleaver design for turbo codes, the structure of these inputs makes it possible to design good interleavers.
For this reason, the distance spectrum obtained as a result of the transfer function is not very useful, since it provides no information about the structure of the RTZ inputs. As an added downside, the complexity of calculating the transfer function for a given RSC code increases with the number of states.

In this paper,we present a novel alternate method to the Transfer Function whose complexity is independent of the number of states of the component code, and has the added benefit of making known the structure of the RTZ inputs that make up the distance spectrum.
With the knowledge of the structure of the message inputs, we derive a general polynomial representation for them based on the weight of the message input after which we go a step further and derive corresponding parity-weight equations for the codewords they generate.
 %%%%%%%%%%%%%%%%%%%%%%
Finally, we compare the upper bound for both methods to simulation results and it is revealed that the upper bound obtained by the novel method is much tighter.

The rest of the research paper is organised as follows. In Section \ref{sec2}, we give a brief review of RSC codes. In section \ref{sec3}, we describe how the distance spectrum of a RSC Code is obtained using the transfer function, followed by the presentation of our novel method in Section \ref{sec4}. Simulation results are presented in Section \ref{sec5} and we draw conclusions in Section \ref{sec6}