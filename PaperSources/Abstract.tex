\section{Abstract}
The knowledege of the distance spectrum, as well as the structure of the message inputs that make up the distance spectrum for a specific Recursive Systematic Convolutional (RSC) code is vital to the design of Turbo Code interleavers. Whiles the distance spectrum of a RSC code can be obtained by calculationg its transfer function, it provides no information about the structure of the message inputs and the complexity involved in calculating the transfer function increases with the number of states of the RSC code.
%%%%%%%%%%%%%%%%%%%%%%%%%%%

In this paper,we present a novel low-complexity method for determining the distance spectrum of any RSC code which has the added benefit of revealing the structure of the message inputs that make up the distance spectrum.
%%%%%%%%%%%%%%%%%%%%%%%%%%%
 With the knowledge of the structure of the message inputs, we derive a general polynomial representation for them based on the weight of the message input after which we go a step further and derive corresponding parity-weight equations for the codewords they generate.
 %%%%%%%%%%%%%%%%%%%%%%
Finally, we compare the upper bound for both methods to simulation results and it is revealed that the upper bound obtained by the novel method is much tighter.