\section{Input-Structure Distance Spectrum }
\label{sec4}
The regular distance spectrum is an insufficient tool where it relates to interleaver design. In this section, we present a novel method that generates what we refered to as the input-structure distance spectrum. It is the distance spectrum with the structure of the RTZ inputs revealed making it a very useful tool for interleaver design.

For convinience sake we will switch to polynomial representation, where $g(x)$ and $f(x)$ represent the feedback and feedforward connections of the RSC code respectively. Also $b(x)$ and $h(x)$ are the polynomial representations of the input message and its corresponding parity-bit sequence respectively.

This method is simple and relies of the fact that a message input $b(x)$ is an RTZ input if 
\begin{equation}
b(x) \bmod g(x) =0
\label{eq1}
\end{equation}

Which means that we may rewrite $b(x)$ as $b(x)=a(x)g(x)$, where $a(x)$ is the polynomial representation of any binary vector or arbitrary length.

$h(x)$ is simply calculated as 
\begin{equation}
h(x) =a(x)f(x)
\label{eq2}
\end{equation}

With the above equations, finding the input-structure distance spectrum for an input message of length $N$ is as simple as using (\ref{eq1}) or  (\ref{eq2}) (or both ) for all values of $a(x),~x^0\in a(x), a(x) \in GF(2^N)$. The condition $x^0\in a(x)$ ensures that there is no repitition of $b(x)$ or $h(x)$

 For $N=16$ we find all $b(x) ~\text{and}~ h(x)$ for which $w_H(\textbf{h})=2 ~\text{and} ~ w_H(\textbf{h})=4$. The results are shown in Table \ref{tab1} and \ref{tab2} respectively . For $w_H(\textbf{h})=4$, only the first 10 results are shown.
 Also \ref{tab3} shows all $b(x)$ and $h(x)$ that produce a codeword weight $w_H(\bc) \leq 8$ for $N=32$.


 
    \begin{table*}[h]
 
 \caption{codewords with parity bit sequence weight $w_H(\textbf{h})=2$}
\centering
 \begin{tabular}{c c c} 
 \hline
 $w_H(\textbf{b})$ & $b(x)$ & $h(x)$ \\ [0.5ex] 
 \hline\hline
 3 &  $1+x+x^2$ & $1+x^2$\\ 
 4 & $1+x+x^3+x^4$ & $1+x^4$ \\
 5 & $1+x+x^3+x^5+x^6$ & $1+x^6$ \\
 6 & $1+x+x^3+x^5+x^7+x^8$& $1+x^8$ \\
 7 & $1+x+x^3+x^5+x^7+x^9+x^{10}$ & $1+x^{10}$ \\
 8 & $1+x+x^3+x^5+x^7+x^9+x^{11}+x^{12}$ & $1+x^{12}$\\ 
 9 & $1+x+x^3+x^5+x^7+x^9+x^{11}+x^{13}+x^{14}$ & $1+x^{14}$ \\ [1ex] 
 \hline
 \end{tabular}
 \label{tab1}
\end{table*}
 
 \begin{table*}[h!]
 
 \caption{codewords with parity bit sequence weight $w_H(\textbf{h})=4$}
\centering
 \begin{tabular}{c c c} 
 \hline
 $w_H(\textbf{b})$ & $b(x)$ & $h(x)$ \\ [0.5ex] 
 \hline\hline
 2 &  $1+x^3$ & $1+x+x^2+x^3$\\ 
 \hline 
  & $1+x^2+x^4$ &  $1+x+x^3+x^4$ \\
   3 &  $1+x^4+x^5$ & $1+x+x^2+x^5$ \\
  &  $1+x+x^5$ & $1+x^3+x^4+x^5$ \\
  \hline 
  & $1+x^2+x^3+x^5$ & $1+x+x^4+x^5$ \\
  &  $1+x^2+x^5+x^6$ & $1+x+x^3+x^6$ \\
 4 & $1+x^4+x^6+x^7$ & $1+x+x^2+x^7$\\ 
  &  $1+x+x^4+x^6$ & $1+x^{3}+x^5+x^6$ \\ 
  &  $1+x+x^6+x^7$ & $1+x^{3}+x^4+x^7$ \\  
  &  $1+x+x^3+x^7$ & $1+x^5+x^6+x^7$ \\ 
 [1ex]
 \hline
 \end{tabular}
 \label{tab2}
\end{table*}

\begin{table*}[h!]
 \caption{All $b(x)$ which produce codewords with weight $w_H(\bc) \leq 8$ for $M=32$}
\centering
 \begin{tabular}{c c c} 
 \hline
 $w_H(\textbf{c})$ & $b(x)$ & $h(x)$ \\ [0.5ex] 
 \hline\hline
 5 &  $1+x+x^2$ & $1+x^2$\\ 
 \hline 
  & $1+x^3$ &  $1+x+x^2+x^3$ \\
   6 &  $1+x+x^3+x^4$ & $1+x^4$ \\
  \hline 
  & $1+x^2+x^4$ & $1+x+x^3+x^4$ \\
  &  $1+x+x^5$ & $1+x^3+x^4+x^5$ \\
 7 & $1+x^4+x^5$ & $1+x+x^2+x^5$\\ 
  &  $1+x+x^3+x-5+x^6$ & $1+x^6$ \\ 
  \hline
  &$1+x^2+x^3+x^5$ & $1+x+x^4+x^5$ \\
  & $1+x^6$ & $1+x+x^2+x^4+x^5+x^6$\\
  & $1 +x +x^4+x^6$ & $1+x^3+x^5+x^6$ \\
  & $1 +x^2 +x^5 +x^6$ & $1+x+x^3+x^6$\\
  8 & $1+x+x^3+x^7$ & $1+x^5+x^6+x^7$ \\
  & $1+ x+x^6+x^7$ & $1+x^3+x^4+x^7$\\
  & $1+x^4+x^6+x^7$ & $1+x+x^2+x^7$\\
  & $1 +x +x^3 +x^5 +x^7 +x^8$ & $1+x^8$\\
 [1ex]
 \hline
 \end{tabular}
 \label{tab3}
\end{table*}