\section{Distance Spectrum via Transfer Function of RSC Code}
\label{sec4}
The distance spectrum of an RSC code gives information about codeword weights and the number of codewords present in the code for a given weight generated as a result of message inputs that begin from, exit and then return to the zero state of the trellis of that code. Such message inputs are known as Return-to-Zero (RTZ) inputs. The distance spectrum of the RSC code can be obtained from its transfer function, denoted by $$T(Y,X)=\sum_{d=0}^{\infty}\sum_{w=0}^{\infty} a(d,w)Y^dX^w$$ where $a(d,w)$ is the number of codewords of weight $d$ generated by an input message of weight $w$.
Based on the method described in \cite{ref3}, we outline the process involved in deriving the transfer function of the $5/7$ RSC code. 

\begin{figure}[h]
\centering
		\includegraphics[width=0.45\textwidth]{tf.png}
		\caption{State Diagram of the $5/7$ RSC code }
		\label{Txfig4}
		\end{figure}
First, the state diagram of the $5/7$ RSC code is redrawn, as shown in Figure \ref{Txfig4}. The zero state is split into two and the transition from the zero state to itself is omitted. On each edge, the variables $Y^d$ and $X^w$ are used to represent the output weight $d$ and input weight $w$ of the path, respectively. We treat each edge label as a transfer block and employ the following rules for graph simplification:
Assume two edges labelled $H$ and $B$. Then,

\begin{enumerate}
\item If $H$ is connected in series to $B$, the labels are merged as $HB$.

\item If $H$ is connected in parallel to $B$, the labels are merged as $H+B$.

\item If the edges are in a feedback configuration, where $H$ and $B$ are the feedfoward and feedback portions respectively, the labels are merged as $\frac{H}{1-HB}$.
\end{enumerate}

In the following, we demonstrate the process for deriving the transfer function for the RSCC shown in Figure {\ref{Txfig4}}. The respective state diagram transformations are also shown in Figure {\ref{Txfig5}}.

\begin{figure}[h!]
\centering
		\includegraphics[width=0.5\textwidth]{tfexample.png}
		\caption{State Diagram Transformations involved in Transfer Function Calculation }
		\label{Txfig5}
		\end{figure}

\begin{example} Define all edge labels\\
\label{ex3}
 edge $ a\triangleq 0 0 \rightarrow 1 0$,  edge $b \triangleq 1 0 \rightarrow 0 1$\\
 edge $ c \triangleq 0 1 \rightarrow 0 0$, edge $d \triangleq 0 1 \rightarrow 1 0$\\
 edge $ e \triangleq 1 0 \rightarrow 1 1$, edge $f \triangleq 1 1 \rightarrow 1 1$\\
 edge $ g \triangleq 1 1 \rightarrow 1 0$

\begin{enumerate}
\item Simplify edge $g$ with feedback configuration

$$ \frac{1}{1-YX}$$

\item Merge edges $e,f,g$ with series configuration and rename it edge $h$

$$\text{edge}~h = Y\times \frac{1}{1-YX}\times Y=\frac{Y^2}{1-YX}$$

\item Merge edges $h,b$ with parallel configuration and rename it edge $j$

$$\text{edge}~j = YX+ \frac{Y^2}{1-YX}= \frac{YX+Y^2X^2+Y^2}{1-YX}$$

\item Merge  edges $j,d$ with feedback configuration and rename it edge $k$ 
\begin{equation*}
\begin{split}
 \text{edge}~k&= \frac{\frac{YX+Y^2X^2+Y^2}{1-YX}}{1-(\frac{YX+Y^2X^2+Y^2}{1-YX})}\\
 &=\frac{Y(X-YX^2+Y)}{1-2YX+Y^2X^2-Y^2}
\end{split}
\end{equation*}

\item Calculate transfer function by merging edges $k,a,c$ with series configuration

\begin{equation*}
\begin{split}
T(Y,X)&=Y^2X \times \frac{Y(X-YX^2+Y)}{1-2YX+Y^2X^2-Y^2}\times Y^2X \\
&=\frac{Y^5X^2(Y-YX^2+Y)}{1-2YX+Y^2X^2-Y^2}\\
&=Y^5X^3+Y^6(X^4+X^2)+Y^7(X^5+3X^3)+\\
&Y^8(X^6+6X^4+X^2)+Y^9(X^7+10X^5+5X^3)+...
\end{split}
\end{equation*}
\end{enumerate}
\end{example}

From the example, it is clear that the complexity involved in deriving the transfer function increases as the number of states of the RSCC increases and other methods such as Mason's Rule \cite{ref3} have to be used. Also to obtain the distance spectrum requires an extra division operation. In the case of interleaver design for Turbo codes, this method for generating the distance spectrum is not particularly useful. This is because it reveals no extra details with respect to the structure of the RTZ inputs. In the next section, we present a novel method whose complexity is independent of the number of states in the RSC code. As an added bonus, information regarding the structure of the RTZ inputs can be obtained using this method.