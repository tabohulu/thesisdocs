\subsection{Distance Spectrum via the Transfer Function of an RSC code}
The distance spectrum of an RSC code gives information about codeword weights and the number of codewords present in the code for a given weight generated as a result of message inputs that begin from, exit and then return to the zero state of the trellis of that code. Such message inputs are known as Return-to-Zero (RTZ) inputs. The distance spectrum of the RSC code can be obtained from its transfer function, denoted by $$T(Y,X)=\sum_{d=0}^{\infty}\sum_{w=0}^{\infty} a(d,w)Y^dX^w$$ where $a(d,w)$ is the number of codewords of weight $d$ generated by an input message of weight $w$.
Interested readers are referred to \cite{ref3} where the transfer function for the $5/7$ RSC code is derived.
The complexity involved in deriving the transfer function increases as the number of states of the RSC code increases and other methods such as Mason's Rule \cite{ref3} have to be used. Also to obtain the distance spectrum requires an extra division operation. In the case of interleaver design for turbo codes, this method for generating the distance spectrum is not particularly useful. This is because it reveals no extra details with respect to the structure of the RTZ inputs. 
Next, we present our novel method, whose complexity is independent of the number of states in the RSC code. As an added bonus, information regarding the structure of the RTZ inputs can be obtained using this method.