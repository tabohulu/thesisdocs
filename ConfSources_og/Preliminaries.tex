\section{Preliminaries}
\label{secPrelim}

A polynomial in $x$, with degree $M$ is an expression of the form
%and coefficients in $\cV$ is denoted by $v(x)$, and defined as 
\begin{equation*}
v(x) = \sum_{m=0}^{M} v_mx^m
\end{equation*}
where $v_m,~0 \leq m \leq M$, are called the \textit{coefficients}  and $v_m \neq 0$. If $v_M=1,~v(x)$ is called a \textit{monic} polynomial.
 %In this paper, we assume that $v_m$ is an integer, $0 \leq v_m \leq p$, where $p$ is a prime number. 
Moreover, the \textit{Hamming weight} of $v(x)$, which is denoted by $w_H(v(x))$, is defined as the total number of non-zero coefficients.
For two polynomials $v(x),~\text{deg}(v(x))=M$ and $w(x),~\text{deg}(w(x))=N$, the sum and product of $v(x)$ and $w(x)$ are defined as 

\begin{equation*}
v(x)+w(x)=\sum_{m=0}^{map\{ M,N\}} (v_m +w_n)x^m
\end{equation*}

\begin{equation*}
v(x)w(x)=\sum_{m=0}^{ M+N} \sum_{i=0}^{m} (v_i w_{m-i})x^m
\end{equation*}
 respectively.

For a prime number $p$, the Galois field with $p$ elements, denoted as GF($p$) is the set of integers $\{ 0,1,p-1\}$ integers where addition and multiplication of 2 elements are carried out modulo-$p$. If the coefficients $v_m, 0 \leq m \leq M$ are elements of GF($p$), $v(x)$ is called a polynomial over GF($p$). Let $v(x)$ and $w(x)$ are both polynomials over GF($p$), $w(x) \neq 0$. Then there exists polynomials $q(x)$ and $r(x)$ over GF($p$) such that $v(x) = w(x)q(x)+r(x)$ and $r(x)=0 ~\text{or deg}~r(x) < ~\text{deg}~w(x)$. $q(x)$ and $r(x)$ are called the \textit{quotient polynomial} and \textit{remainder polynomials}, respectively of the division of $v(x)$ by $w(x)$.

A monic polynomial which cannot be factorised into lower degree polynomials over GF($p$) is called a \textit{prime polynomial}.
Let $v(x)$ be a prime polynomial with degree $M>1$. Then, we can define a set of $p^M$ polynomials with degree less than M over GF($p$). If within this set, addition and multiplication operations are carried out modulo-$v(x)$ over GF($p$), we obtain what is called an \textit{extension field} of GF($p$), which is denoted by GF($p^M$). Addition and multiplication modulo-$v(x)$ over GF($p$) means all addition and multiplication operations on the polynomials and their coefficients are carried out modulo-$v(x)$ and modulo-$p$ respectively, and to perform multiplication modulo-$v(x)$, means to divide the polynomial product by $v(x)$ and find the remainder polynomial. 


Elements in GF($p^M$) can be represented by a power notation, \textit{i.e.} $X^m,~0 \leq m \leq p^M-1$, where $X^2$ may be used in place of $1+x$, for example. Through out this paper, the power notation will be used more often for the sake of convenience, with the appropriate conversion between the power and polynomial notation made known where necessary.  

Let $X$ be a non-zero element of GF($p^M$).Then, $\epsilon$ denotes the \textit{order} of $X$, and is defined as the least positive integer value such that $X^{\epsilon}=1$, and $X$ is called a \textit{primitive element} iff $\epsilon=p^M-1$ . Let $v(x)$ be a prime polynomial with degree $M$. If $v(x)$ generates GF($p^M$) such that $X$ is a primitive element in GF($p^M$), then  $v(x)$ is called a \textit{primitive polynomial}. 

Finally, the root of $v(x)$, denoted by $\beta$, is any non-zero element in GF($p^M$) such that $v(\beta)=0$. The order of $\beta$ is defined similarly to that of $X$ and is also denoted by $\epsilon$, and if $\beta^{\epsilon}=1$, then all $\beta^i,~0 \leq i \leq \epsilon -1$ are distinct.
 If $v(x)$ is a primitive polynomial, then $\beta$ is a primitive element, \textit{i.e.} $\epsilon=p^M-1$, otherwise $\epsilon < p^M-1,~ \epsilon | p^M-1$ . 
%Finally, let $(e,~f)$ represent a pair of non-zero positive integers. Then $(e,~f) \bmod 2^M-1$ is shorthand for the operation $(e \bmod 2^M-1,~f \bmod 2^M-1)$.

