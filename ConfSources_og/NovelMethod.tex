\section{The patterns of the low-weight PCs}
\label{sec3}
To determine the details of the patterns of the low-weight PCs, we assume $f(x)$ can be factorized into $K$ prime polynomials as 
\begin{align}
f(x)&=\prod_{k=0}^{K-1}f_k^{\gamma_k}(x)
\end{align}
where $\gamma_0,\gamma_1,\cdots,\gamma_{K-1}$ are positive integers and we assume $\beta_k$ is a root of $f_{k}(x)$ of order $\epsilon_k$.

Refering \eqref{eq:low-weight-parity}, we consider the solution of
\begin{align}
	h(x) \mod f(x) \equiv 0
	\label{Eq:condition}
\end{align}
%
We start from the simplest case $K=1$, \ie, $f(x) = f_0^{\gamma_0}(x)$. Then, we can see from \eqref{eq:low-weight-parity} that each root is also a root of $h(x)$. For the case $\gamma_0 = 1$, since all $\beta_0^i$, $0 \leq i < \epsilon_0$, are distinct from each other, the condition
\begin{align}
	h(\beta_0^i)=0,~~~ 0 \leq i < \epsilon_0
	\label{Eq:rootcondition}
\end{align}
is necessary and sufficient to fulfil \eqref{Eq:condition}. For $\gamma_0 > 1$, on the other hand, \eqref{Eq:rootcondition} is necessary but not sufficient for \eqref{Eq:condition}. For this case, although we may derive some solutions by differential equations
\begin{align}
\left.\frac{d^{(j)}h(x)}{d x^j}\right|_{x=\beta_0^i}=0,~~~0 \leq i < \epsilon_0,~1 \leq j < \gamma_0
\label{Eq:differential}
\end{align}
we can not determine the patterns completely, since the operations on coefficients of the polynomial are done modulo-$p$. However, we can remove the ghost solutions 
%{\bf (perhaps has accurate name)} 
by careful confirmation.

By repeating the above discussion for the roots $\beta_k$, $0 < k < K$, and taking the intersection of the results, we can easily extend to the case $K>1$.

\subsection{The patterns of the weight-2 PCs}
\label{sec:PC2}
Each weight-2 PC can be written as 
\begin{equation}
h(x)=1+x^a
\label{eq:wt2-gen-form}
\end{equation}
without loss of generality. Thus, we have from \eqref{Eq:rootcondition} that
\begin{equation}
(\beta_0^i)^a =1,~~~ 0 \leq i < \epsilon_0
\label{novelEq5b}
\end{equation}
On the other hand, the order $\epsilon_0$ is the least integer satisfying $\beta_0^{\epsilon_0} \equiv 1$, thus, $a$ should satisfy the condition
$$a \bmod \epsilon_0  \equiv 0$$

\subsection{The patterns of the weight-3 PCs}

Each weight-3 PC can be written as 
\begin{equation}
h(x)=1+x^a+x^b,~a\neq b
\label{novelEqwt3}
\end{equation}
without loss of generality. 
Thus, $(a,b)$ should satisfying the condition
\begin{equation}
\beta_0^a+\beta_0^b= 1
\label{Eq:novelEq5b}
\end{equation}
Such pair can be find by referring to the table of the extended field for GF$(2^M)$. 
While $(a,b)$ is a pair satisfying \eqref{Eq:novelEq5b}, it is obvious that all pairs $(\eta,~\zeta)$ satisfying
\begin{equation}
(a,b) \equiv (\eta,~\zeta) \bmod \epsilon_0,~(\eta,~\zeta)\in \cZ
\end{equation}
also satisfies \eqref{Eq:novelEq5b}, where $(\eta,~\zeta) \bmod \epsilon_0$ is shorthand for the operation $(\eta \bmod \epsilon_0,~\zeta \bmod \epsilon_0)$. On the other hand, for a fixed $a$, since $\epsilon_0+i$, $0 \leq i < \epsilon_0$, are distinct from each other, any integer $b'$ that satisfies \eqref{Eq:novelEq5b} must be such that $b'\equiv b \mod \epsilon_0$.

\subsection{Examples}

\begin{example}$f(x)=1+x+x^2$.\newline 
\textbf{Weight-2 PCs}: 
For this case, since $x^1=x$, $x^2 \equiv 1+x$, and $x^3 \equiv 1 \bmod f(x)$, the order of the root $\beta_0$ is $\epsilon_0=3$ and $a$ should be a multiple of $3$. The corresponding values for $a(x)$ and $h(x)$ are shown in Table \ref{novelTab2} for the first four valid values of $a$.
\begin{table}[htbp]
%\parbox{.5\linewidth}{
 \caption{$f(x)=1+x+x^2$}
\centering
 \begin{tabular}{c c c} 
%\hline
 $a(x)$ & $h(x)$ \\ [0.5ex] 
 \hline\hline
$1+x$
 & $1+x^{3}$ \\
\hline
$1+x+x^3+x^4$
 & $1+x^{6}$ 
 \\
\hline
$1+x+x^3+x^4+x^6+x^{7}$ 
&  $1+x^{9}$ 
\\
\hline
$1+x+x^3+x^4+x^6+x^{7}+x^9+x^{10}$
 &  $1+x^{12}$ \\
 \end{tabular}
 \label{novelTab2}
\end{table}
We may write the weight-2 PCs in general form as $h(x)=1+x^{3\ell},~\ell>1$ and the corresponding $a(x)$ is given by 
\begin{equation*}
a(x)=\sum_{\ell=0}^{L-1} x^{3\ell}(1+x)
\end{equation*}

\textbf{Weight-3 PCs}: The elements of GF$(2^2)$ are shown in Table \ref{novelTab7} and it is obvious that $\cZ=\{(1,2)\}$.
This means that $(a,b) \in \cB_0=\{3\ell+1,~3n+2\},~\ell,n \geq 0$.  The corresponding values for $a(x)$ and $h(x)$ are shown in Table \ref{novelTab8} below for the first four valid values of $(a,b)$.
 \begin{table}[htbp]
 \caption{Non-zero Elements of GF$(2^2)$ generated by $f(x)=1+x+x^2$}
\centering
 \begin{tabular}{c c} 
 \hline
 power representation & actual value \\ [0.5ex] 
 \hline\hline
$X^0~=X^3=1$ & $1$\\
\hline
$X$ & $x$\\
\hline
$X^2$ &  $1+x$\\
\hline
 \end{tabular}
 \label{novelTab7}
\end{table}

\begin{table}[htbp]
 \caption{$f(x)=1+x+x^2$}
\centering
 \begin{tabular}{c c} 
 \hline
 $a(x)$ & $h(x)$\\ [0.5ex] 
 \hline\hline
$1$ & $1+x+x^2$\\ 
\hline
$1+x+x^2$ &  $1+x^2+x^4$\\
\hline
$1+x+x^3$ & $1+x^4+x^5$\\
\hline
$1+x^2+x^3$ & $1+x+x^5$ 
 \end{tabular}
 \label{novelTab8}
\end{table}

We may write the weight-3 PCs in general form as $h(x)=1+x^{3\ell+1}+x^{3n+2},~\ell,~n \geq 0$ 

\label{ex-1}
\end{example}




\begin{example}
$f(x)=1+x+x^2+x^3+x^4$\newline
\textbf{Weight-2 PCs}: 
We can confirm that the order of $\beta_0$ is $\epsilon_0=5$. This means that $a$ should be a multiple of $5$. The corresponding values for $a(x)$ and $h(x)$ are shown in Table \ref{novelTab3} with general forms for $\ell>1$

%}
\begin{table}[htbp]
%\parbox{.5\linewidth}{
\caption{$f(x)=1+x+x^2+x^3+x^4$}
\centering
\begin{tabular}{c c} 
 \hline
 $a(x)=\sum_{\ell=0}^{L-1} x^{5\ell}(1+x)$ & $h(x)=1+x^{5\ell}$  \\ [0.5ex] 
 \hline\hline
$1+x$ &$1+x^5$\\ 
$1+x+x^5+x^6$ &$1+x^{10}$  \\
$1+x+x^5+x^6+x^{10}+x^{11}$ & $1+x^{15}$ \\
$1+x+x^5+x^6+x^{10}+x^{11}+x^{15}+x^{16}$ &$1+x^{20}$  
 \end{tabular}
 \label{novelTab3}
%}ll
\end{table}
\newpage
\textbf{Weight-3 PCs}:
We refer to Table \ref{novelTabWt3-2} and it is obvious that $\cZ=\varnothing$ and therefore, there are no weight-3 PCs for $f(x)$
 \begin{table}[htbp]
 \caption{Non-zero Elements of GF$(2^4)$ generated by $f(x)=1+x+x^2+x^3+x^4$}
\centering
 \begin{tabular}{c c} 
 \hline
 power representation & actual value \\ [0.5ex] 
 \hline\hline
$X^0~=X^5~=X^{10}~=X^{15}$ & $1$\\
\hline
$X~=X^6~=X^{11}$ & $x$\\
\hline
$X^2~=X^7~=X^{12}$ &  $x^2$\\
\hline
$X^3~=X^8~=X^{13}$ &  $x^3$\\
\hline
$X^4~=X^9~=X^{14}$ &  $1+x+x^2+x^3$\\
\hline
 \end{tabular}
 \label{novelTabWt3-2}
\end{table}
\label{ex-2}
\end{example}

\begin{example}
	$f(x)=1+x^2$\newline
	
	\textbf{Weight-2 PCs}: Since 
	\[
	f(x)=(1+x)^2\]
	and the order of the root $\beta_0=1$ is $\epsilon_0=1$, we obtain from \eqref{Eq:rootcondition} and \eqref{Eq:differential}
	\begin{align}
		(\beta_0)^a = 1
		\label{Eq:example31}
	\end{align}
	\begin{align}
		a(\beta_0)^{(a-1)} = 0
		\label{Eq:example32}
	\end{align}	
	Although \eqref{Eq:example31} indicates $a$ can be any positive integer, we can see from \eqref{Eq:example32} that $a$ should be an even number.
	The corresponding values for $a(x)$ and $h(x)$ are shown in Table \ref{novelTab1} with general forms for $\ell>1$.
	\begin{table}[htbp]
		\renewcommand{\arraystretch}{1.3}
		%\parbox{.3\linewidth}{
		\caption{$f(x)=1+x^2$}
		\centering
		\begin{tabular}{c c } 
			\hline
			$a(x)=\sum_{\ell=0}^{L-1} x^{2\ell}$ & $h(x)=1+x^{2\ell}$ \\ [0.5ex] 
			\hline\hline
			$1$ & $1+x^2$\\ 
			$1+x^2$ & $1+x^4$ \\
			$1+x^2+x^4$ & $1+x^6$\\
			$1+x^2+x^4+x^6$ & $1+x^8$ 
		\end{tabular}
		\label{novelTab1}
	\end{table}
	
	\textbf{Weight-3 PCs}:
	Given that there is a single non-zero element in GF(2) which is generated by $1+x$ we can conclude that there are no weight-3 PCs associated with $f(x)$.
\label{ex-3}
\end{example}



---------------------------
%=====================Deleted Examples  =======================%
%\begin{example}
%$f(x)=1+x^2+x^3+x^4$\newline
%$f(x)$ can be written as 
%$$f(x)=\prod_{k=0}^{1}f_k(x)$$
%where 
%$$f_0(x)=1+x,~f_1(x)=1+x+x^3$$ 
%For $f_0(x), x \equiv 1$, and $x^1 \equiv 1 \bmod f_0(x)$, which means the order of the root $\beta_0$ is $\epsilon_0=1$ and $a_0$ should be a multiple of $1$. Again, for  $f_1(x), x^3 \equiv 1+x$, and $x^7 \equiv 1 \bmod f_1(x)$, which means the order of the root $\beta_1$ is $\epsilon_1=7$ and $a_1$ should be a multiple of $7$.
%Finally, valid values of $a$ should be a multiple of the least common multiples of $a_0$ and $a_1$, which means $a$ should be a multiple of $7$.
%The corresponding values for $a(x)$ and $h(x)$ are shown in Table \ref{novelTab1-a} with general forms for $\ell>1$.
%\begin{table}[htbp]
%\renewcommand{\arraystretch}{1.3}
%\parbox{.3\linewidth}{
 %\caption{$f(x)=1+x^2+x^3+x^4$}
 %\centering
%\begin{tabular}{c c } 
%\hline
 %$a(x)=\sum_{\ell=0}^{L-1} x^{7\ell}(1+x^2+x^3)$ & $h(x)=1+x^{7\ell}$ \\ [0.5ex] 
%\hline\hline
%$1+x^2+x^3$ & $1+x^7$\\ 
%$1+x^2+x^3+x^7+x^9+x^{10}$ & $1+x^{14}$ \\
%$1+x^2+x^3+x^7+x^9+x^{10}+x^{14}+x^{16}+x^{17}$ & $1+x^{21}$
%\end{tabular}
% \label{novelTab1-a}
%\end{table}
%\end{example}

%\begin{example}
%$f(x)=1+x+x^2+x^3+x^4+x^5+x^6$\newline
%$f(x)$ can be written as 
%$$f(x)=\prod_{k=0}^{1}f_k(x)$$
%where 
%$$f_0(x)=1+x^2+x^3,~f_1(x)=1+x+x^3$$ 
%For $f_0(x), x^3 \equiv x^2+1$, and $x^7 \equiv 1 \bmod f_0(x)$, which means the order of the root $\beta_0$ is $\epsilon_0=7$ and $a_0$ should be a multiple of $7$. Again, for  $f_1(x), x^3 \equiv 1+x$, and $x^7 \equiv 1 \bmod f_1(x)$, which means the order of the root $\beta_1$ is $\epsilon_1=7$ and $a_1$ should be a multiple of $7$.
%Finally, valid values of $a$ should be a multiple of the least common multiples of $a_0$ and $a_1$, which means $a$ should be a multiple of $7$.
%The corresponding values for $a(x)$ and $h(x)$ are shown in Table \ref{novelTab1-b} with general forms for $\ell>1$.
%\begin{table}[htbp]
%\renewcommand{\arraystretch}{1.3}
%\parbox{.3\linewidth}{
% \caption{$f(x)=1+x+x^2+x^3+x^4+x^5+x^6$}
 %\centering
%\begin{tabular}{c c } 
%\hline
 %$a(x)=\sum_{\ell=0}^{L-1} x^{7\ell}(1+x)$ & $h(x)=1+x^{7\ell}$ \\ [0.5ex] 
%\hline\hline
%$1+x$ & $1+x^7$\\ 
%$1+x+x^7+x^8$ & $1+x^{14}$ \\
%$1+x+x^7+x^8+x^{14}+x^{15}$ & $1+x^{21}$
%\end{tabular}
% \label{novelTab1-b}
%\end{table}
%\end{example}
%=====================================End of deleted examples========================%
\begin{example}
$f(x)=1+x^2+x^3+x^4+x^6$\newline

\textbf{Weight-2 PCs}: $f(x)$ can be written as 
$$f(x)=\prod_{k=0}^{1}f_k(x)$$
where 
$$f_0(x)=1+x+x^2,~f_1(x)=1+x+x^2+x^3+x^4$$ 
From Example \ref{ex-1} and Example \ref{ex-2}, we know that $a_0=3$ and $a_1=5$.
Hence, valid values of $a$ should be a multiple of the least common multiples of $a_0$ and $a_1$, which means $a$ should be a multiple of $15$.
The corresponding values for $a(x)$ and $h(x)$ are shown in Table \ref{novelTab1-c} with general forms for $\ell>1$.

\begin{table}[htbp]
\renewcommand{\arraystretch}{1.3}
%\parbox{.3\linewidth}{
 \caption{$f(x)=1+x^2+x^3+x^4+x^6$}
 \centering
\begin{tabular}{c c } 
\hline
 $a(x)=\sum_{\ell=0}^{L-1} x^{15\ell}(1+x^2+x^3+x^6+x^7+x^9)$ & $h(x)=1+x^{15\ell}$ \\ [0.5ex] 
\hline\hline
$1+x^2+x^3+x^6+x^7+x^9$ & $1+x^{15}$\\ 
$1+x^2+x^3+x^6+x^7+x^9+x^{15}+x^{17}+x^{18}+x^{21}+x^{22}+x^{24}$ & $1+x^{30}$ \\
\end{tabular}
 \label{novelTab1-c}
\end{table}

\textbf{Weight-3 PCs}:
From Example \ref{ex-1} and Example \ref{ex-2}, we have $\cZ_0=\{ (1,2)\}$ and $\cZ_1=\varnothing$. Therefore $\cZ_0 \bigcap \cZ_1= \varnothing$ and therefore there are no weight-3 PCs associated with $f(x)$.
%We may write the weight-2 PCs in general form as $h(x)=1+x^{7\ell},~\ell>1$ and the corresponding $a(x)$ is given by 
%\begin{equation*}
%a(x)=\sum_{\ell=0}^{L-1} x^{7\ell}(1+x)
%\end{equation*}
\end{example}

\begin{example}
$f(x)=1+x+x^5$\newline
\textbf{Weight-2 PCs}: $f(x)$ can be written as 
$$f(x)=\prod_{k=0}^{1}f_k(x)$$
where 
$$f_0(x)=1+x+x^2,~f_1(x)=1+x^2+x^3$$ 
From Example \ref{ex-1}, we know that $a_0=3$ and it can be confirmed that $a_1=7$.
Hence, valid values of $a$ should be a multiple of the least common multiples of $a_0$ and $a_1$, which means $a$ should be a multiple of $21$.
The corresponding values for $a(x)$ and $h(x)$ are shown in Table \ref{novelTab1-c} with general forms for $\ell>1$.

\begin{table}[htbp]
\renewcommand{\arraystretch}{1.3}
%\parbox{.3\linewidth}{
 \caption{$f(x)=1+x+x^5$}
 \centering
\begin{tabular}{c c } 
\hline
 $a(x)=\sum_{\ell=0}^{L-1} x^{21\ell}(1+x^2+x^3+x^4+x^6+x^8+x^{4}+x^{6}+x^{8}+x^{11}+x^{12}+x^{16})$ & $h(x)=1+x^{21\ell}$ \\ [0.5ex] 
\hline\hline
$1+x^2+x^3+x^4+x^6+x^8+x^{4}+x^{6}+x^{8}+x^{11}+x^{12}+x^{16}$ & $1+x^{21}$\\ 
\end{tabular}
 \label{novelTab1-c}
\end{table}

\textbf{Weight-3 PCs}:
From Example \ref{ex-1}, we have $\cZ_0=\{ (1,2)\}$ with $\cB_0=\{( 3\ell+1,~3n+2)\},~\ell,n \geq 0$ and from Table \ref{novelTabWt3-5}, $ \cZ_1=\{(1,5),~(2,3),~(4,6)\}$, with $ \cB_1=\{(7\ell+1,~7n+5),~(7\ell+2,~7n+3),~(7\ell+4,~7n+6) \},~\ell,n \geq 0$. Therefore $(a,b) \in \cB_0 \bigcap \cB_1$.

\begin{table}[htbp]
 \caption{Non-zero Elements of GF$(2^3)$ generated by $1+x^2+x^3$}
\centering
 \begin{tabular}{c c} 
 \hline
 power representation & actual value \\ [0.5ex] 
 \hline\hline
$X^0~=X^7$ & $1$\\
\hline
$X$ & $x$\\
\hline
$X^2$ &  $x^2$\\
\hline
$X^3$ &  $1+x^2$\\
\hline
$X^4$ &  $1+x+x^2$\\
\hline
$X^5$ &  $1+x$\\
\hline
$X^6$ &  $x+x^2$\\
\hline
 \end{tabular}
 \label{novelTabWt3-5}
\end{table}

The corresponding values for $a(x)$ and $h(x)$ are shown in Table \ref{novelTab8-b} below for the first three valid values of $(a,b)$.
\begin{table}[htbp]
 \caption{$f(x)=1+x+x^5$}
\centering
 \begin{tabular}{c c} 
 \hline
 $a(x)$ & $h(x)$\\ [0.5ex] 
 \hline\hline
$1$ & $1+x+x^{5}$\\ 
\hline
$1+x+x^5$ &  $1+x^2+x^{10}$\\
\hline
$1+x+x^2+x^3+x^4+x^{6}+x^{8}$ & $1+x^{11}+x^{13}$\\
 \end{tabular}
 \label{novelTab8-b}
\end{table}
\end{example}
 







