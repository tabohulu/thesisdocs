\section{Novel Method to Determine the Low-Weight Parity-Check Pattern}
\label{sec3}
In this section, we present our novel method for obtaining what we have named the \textit{codeword component pattern distance spectrum}. %Compared to the transfer function method, its complexity is independent of the number of states of the RSC. It is also able to reveal the pattern of low-weight codewords. 
Our novel method can be seen as the combination of two different but related methods. The first method is quite simple and makes use of the fact that in the polynomial domain, systematic components that are  RTZ inputs and their corresponding parity-check components share a common factor. 
The second method shows how to obtain this common factor when the weight of the parity check component is fixed for a given RSC code.  Through out this section, $c(x),~b(x)$ and $h(x)$ represesent the RSC codeword, the systematic component of the codeword and the parity check component of the codeword, respectively in polynomial notation.
%After explaining the inner working of our novel method, we used it to obtain the partial codeword pattern distance spectrum for the $5/7,~ 37/21$ and $23/35$  RSC codes.



\subsection{The Characteristics of Low-weight Codewords}
Since each RSC codeword is made up of two codeword components $b(x)$ and $h(x)$, it is obvious that the weight of the codeword $c(x)$ is given by 
\begin{equation}
w_H(c(x))=w_H(b(x)) + w_H(h(x))
\label{novelEq-1}
\end{equation} 
%(quarantine)%The distance spectrum derived via the transfer function method is an insufficient tool when it comes to to interleaver design. In this section, we present a novel method that generates what we refered to as the structured distance spectrum, which is the distance spectrum with the structure of the RTZ inputs as well as the corresponding parity-check sequence revealed, therefore making it a very useful tool for interleaver design.

%(quarantine)%For an RSC code, the Hamming weight of the codeword $w_H(\bc)$ is the sum of the weights of the parity bit sequence and message input. 
We first consider the parity check component, which can be expressed as 
\begin{equation}
h(x) =f(x)\cdot g^{-1}(x)\cdot b(x)
\label{novelEq0}
\end{equation}
If we consider large frame sizes, the presence of $g^{-1}(x)$ means that within $h(x)$ is a particular sequence of bits that is repeated a large number of times. This results in a large parity weight, and by extension, a relatively high-weight codeword. The only time this is not the case is when
\begin{equation}
b(x) \bmod g(x) \equiv 0
\label{novelEq1}
\end{equation}
This results in a relatively low-weight parity bit sequence, which might produce a low-weight codeword. Any $b(x)$ that meets the condition in (\ref{novelEq1}) can be written as 
\begin{equation}
b(x) =a(x)g(x)
\label{novelEq2}
\end{equation}
where $a(x)$ is a monic polynomial with $a_0=1$.
By fixing $b(x)$ from (\ref{novelEq2}) into (\ref{novelEq0}), we have 
\begin{equation}
\begin{split}
h(x)&=f(x)\cdot g^{-1}(x)\cdot a(x)g(x)\\
&=a(x)f(x)
\end{split}
\label{novelEq3}
\end{equation}
%(quarantine)%Using both (\ref{novelEq2})  and (\ref{novelEq3}), we wish to list all low-weight codewords for a given RSC code.A low-weight codeword is any codeword which satisfies the condition, $w_H(\bc) \leq d_{\text{max}}$. This list is known as the \textit{partial structured distance spectrum}. To generate the partial structured distance spectrum, we take note of a few things. 

%From (\ref{novelEq2})  and (\ref{novelEq3}), we observe that $a(x)$ is a common factor in both equations and if we are able to solve for $a(x)$ via either of the equations, the remaining equation can be solved. To solve for $a(x)$ requires that in either equation, it should be the only unknown variable. At first glance, it might seem that $g(x)$ and $f(x)$ are the only known variables because they are dependent on the RSC code in question. However, if we remember that the weight of $h(x)$ and $b(x)$ is directly proportional to the number of terms it has, then we are on our way to obtain our second known variable. What is left is to determine the valid power values for the polynomial  terms, depending on the weight of $h(x)$ or $b(x)$.

Thus, for a given $f(x)$ and $g(x)$, our goal is to find all $a(x)$s which generate low-weight codewords components in  (\ref{novelEq2}) and  (\ref{novelEq3}) simultaneously. 
However, since there is essentially no difference between the general structure of $h(x)$ and $b(x)$, we restrict our our attention to the low-weight parity check patterns in  (\ref{novelEq3}) and in the following, we present a method for determining valid values of $h(x)$ when $2 \leq w_H(h(x))\leq 3$. 
%In (\ref{novelEq3}), once $f(x)$ is given, our goal is to find $a(x)$ that results in a low-weight $h(x)$. To this end, we consider the roots of $f(x)$ 
%If $f(x)$ is a prime polynomial or can be factorized into prime polynomials, the the roots of $f(x)$ are its primitive elem
 %denoted by $\beta_i,~ 0 \leq i < 2^{m}-1)$. Then it is obvious that $h(\beta^i)=0$~ for all $\beta^i$ that are primitive elements 
%and we can reformulate our goal as to find weight-$w$ polynomials ($h(x)$) which take all the roots of $f(x)$ as its roots. The roots of $f(x)$ depend on its characteristic make-up and once that is known, we can easily determine the structure of $h(x)$ for a given value of $w_H(h(x))$. 
%The characteristic make-up of $f(x)$ can be grouped into the three cases below. 
%\begin{enumerate}
%\item Single primitive polynomial.
%\item Prime but not a primitive polynomial.
%\item Made up of repeated polynomial roots.
%\end{enumerate}


%It is worth noting that the method to be discussed can also be used to obtain valid values of $b(x)$, because there is no difference between the general structure of $h(x)$ and $b(x)$ once the Hamming weight is fixed. 
\subsection{The Characteristic of Weight $2$ Parity Check Pattern}
For this weight case, we can write $h(x)$ as $h(x)=1+x^a$ without any loss of generality. 
Let $M$ be the degree of $f(x)$ and $\alpha_m, ~0 \leq m \leq M-1$, be the roots of $f(x)$ satisfying $f(\alpha_m) = 0$ for all $0 \leq m \leq M-1$. Then from (\ref{novelEq3}), we have
\begin{equation}
\begin{split}
&h(\alpha_m)=0\\
&1+(\alpha_m)^a =0
\end{split}
\label{novelEq5b}
\end{equation}
for all $0 \leq m \leq M-1$ and therefore, we can reformulate our goal as to find low-weight polynomials $h(x)$ which satisfy (\ref{novelEq5b}) for all roots of $f(x)$. 

Now we consider the characteristics of the roots of $f(x)$ for the folowing 2 cases.
%Thus we wish to find the possible values of $a$ satisfying $$\beta^{ia}+1=0$$ for all $\beta^i,~1 \leq i \leq 2^{m}-1$ in GF$(2^{m}),~m=\text{order}(g(x))$
\paragraph{ Case1} $f(x)$ is a single irreducible polynomial\newline
For this case, $f(x)$ has $M$ distinct roots and each root $\alpha_m$ can be expressed as
\begin{equation}
\alpha_m=\beta^{2^m},~ 0\leq m \leq M-1
\end{equation}
where $\beta^{2^m}$ is primitive element in GF($2^M$) with order $\epsilon,~\epsilon | 2^M-1$. 
Now, if we consider the possible values of $a>0$ such that 
$$(\beta^{2^m})^a=1$$
then,
$$a \bmod \epsilon  \equiv 0$$
since $(\beta^{2^m})^{\epsilon}=1$

\begin{example}
$f(x)=1+x+x^2$.\newline $f(x)$ generates the field GF$(2^2)$. There are 2 distinct roots of $f(x)$, $\beta$ and $\beta^2$. The order of $\beta$ and $\beta^2$ is $\epsilon=3$, and the  valid values of $a$ are $a=\{3,6,9,\cdots \}$. The corresponding values for $a(x)$ and $h(x)$ are shown in Table \ref{novelTab2} for the first four valid values of $a$.
\begin{table}[htbp]
%\parbox{.5\linewidth}{
 \caption{$f(x)=1+x+x^2$}
\centering
 \begin{tabular}{c c c} 
%\hline
 $a(x)$ & $h(x)$ \\ [0.5ex] 
 \hline\hline
$1+x$
 & $1+x^{3}$ \\
\hline
$1+x+x^3+x^4$
 & $1+x^{6}$ 
 \\
\hline
$1+x+x^3+x^4+x^6+x^{7}$ 
&  $1+x^{9}$ 
\\
\hline
$1+x+x^3+x^4+x^6+x^{7}+x^9+x^{10}$
 &  $1+x^{12}$ \\
 \end{tabular}
 \label{novelTab2}
\end{table}
\end{example}
%============================
%\begin{example}$g(x)=1+x+x^4$\newline
%$g(x)$ can be used to generate the extended field GF$(2^4)$. In this field, $\beta^{15}=1$. The valid values of $a$ are $a=\{15,30,45,\cdots \}$. The corresponding values for $a(x)$ and $b(x)$ are shown in the table below for the first two valid values of $a$.
% \begin{table*}[h]
 %\caption{$23/35$ RSC Code, $f(x)=1+x+x^4$}
%\centering
%\begin{tabular}{p{4cm} | c} 
% \hline
 %$a(x)$ & $b(x)$  \\ [0.5ex] 
 %\hline\hline
%$1+x^2+x^3+x^5+x^7+x^8+x^{11}$ 
%& $1+x^{15}$ \\ 
%\hline
%$1+x^2+x^3+x^5+x^7+x^8+x^{11}+x^{15}+x^{16}+x^{17}+x^{18}+x^{20}+x^{22}+x^{23}+x^{26}$ 
%&$1+x^{30}$\\
 %\end{tabular}
 %\label{novelTab5}
%\end{table*}
%\end{example}
%=====================
%\paragraph{Case2}$f(x)$ is prime polynomial but not primitive \newline
%Similar to the case for primitive polynomials, we need to find the order of $\beta$. For fields generated by prime polynomials, there is a value $j < 2^m-1$ such that 
%$$\beta^j=1,~j~|~2^{m}-1 $$ where $j$ is the order. Therefore, any valid value of $a$ should satisfy the condition below.
%$$ a \bmod j \equiv 0$$

\begin{example}
$f(x)=1+x+x^2+x^3+x^4$\newline
$f(x)$ can be used used to generate GF$(2^4)$. There are 4 distinct roots and $\beta,~\beta^2,~\beta^3~\text{and}~\beta^4$. The order of each root is $\epsilon=5$ and therefore, the valid values of $a$ are $a=\{5,10,15,\cdots\}$. The corresponding values for $a(x)$ and $h(x)$ are shown in Table \ref{novelTab3} for the first four valid values of $a$.

%}
\begin{table}[htbp]
%\parbox{.5\linewidth}{
\caption{$f(x)=1+x+x^2+x^3+x^4$}
\centering
\begin{tabular}{c c} 
 \hline
 $a(x)$ & $h(x)$  \\ [0.5ex] 
 \hline\hline
$1+x$ &$1+x^5$\\ 
$1+x+x^5+x^6$ &$1+x^{10}$  \\
$1+x+x^5+x^6+x^{10}+x^{11}$ & $1+x^{15}$ \\
$1+x+x^5+x^6+x^{10}+x^{11}+x^{15}+x^{16}$ &$1+x^{20}$  
 \end{tabular}
 \label{novelTab3}
%}ll
\end{table}
\end{example}

%\paragraph{ Case3: $g(x)$ is made up of repeated polynomial roots.\newline}
%Given the above condition we have, $r(x)=(r^{o_p}_p(x))^k$, where $r^{o_p}_p(x)$ represents the prime polynomial $r(x)$ can be factorised into and $k$ is the number of times it is repeated. $r^{o_p}_p(x)$ has $\beta$ as its roots and we need to find $d \st \beta^j=1$ in the (extended) field it generates. Because $\beta^{kj}=1$ in the same (extended) field , the valid values for $a$ should satisfy the condition below:
%$$kj ~| ~a$$

\paragraph{Case2}$f(x)$ can be factorised into multiple irreducible polynomials. \newline
For this case, we can write $f(x)$ as $$f(x)=\prod_{k=1}^{K}f_k(x)$$ where $f_k(x)$ is an irreducible polynomial with order $\epsilon_k$. 
For each $f_k(x)$, the valid values of  $a_k$ are such that 
$$a_k \bmod \epsilon_k \equiv 0$$ and the valid values of $a$ are such that
$$a \in  \bigcap_{k=1}^{K} a_k$$
This means that $a$ satisfies the condition
$$ a \bmod  \prod_{k=1}^{K} \epsilon_k \equiv 0$$
For the special case where $f(x)$ can be factorised into equal irreducible polynomials, the above condition simplifies to 
$$a \bmod \epsilon K \equiv 0$$

\begin{example}
$f(x)=1+x^2$\newline
$f(x)$ can be written as $$f(x)=(1+x)^2,~K=2$$ $1+x$ is prime in $GF(2)$ and has $\beta$ as its root. The order of $\beta$ is $1$. Since $f(x)$ is made up of equal repeated polynomial and $K=2$, the valid values of $a=\{2,4,6,\cdots \}$.
The corresponding values for $a(x)$ and $h(x)$ are shown in Table \ref{novelTab1} for the first four valid values of $a$.
\begin{table}[htbp]
\renewcommand{\arraystretch}{1.3}
%\parbox{.3\linewidth}{
 \caption{$f(x)=1+x^2$}
 \centering
\begin{tabular}{c c } 
\hline
 $a(x)$ & $h(x)$ \\ [0.5ex] 
\hline\hline
$1$ & $1+x^2$\\ 
$1+x^2$ & $1+x^4$ \\
$1+x^2+x^4$ & $1+x^6$\\
$1+x^2+x^4+x^6$ & $1+x^8$ 
\end{tabular}
 \label{novelTab1}
\end{table}
\end{example}

%==========
%\begin{example}
%$g(x)=1+x^4$\newline $g(x)$ is made up of equal repeated polynomial roots and can be written as $$g(x)=(1+x)^4,~k=4$$. $1+x$ is prime in $GF(2)$ and $\beta^{1}=1$. Since $k=4$, we have $\beta^{k}=\beta^{4}=1$. $b(x)=1+x^b$ and the valid values of $a=\{4,8,12,\cdots \}$.
%The corresponding values for $a(x),~b(x)$ and $h(x)$ are shown in the table below for the first four valid values of $a$

%\begin{table*}[h]
%\caption{$g(x)=1+x^4$}
%\centering
 %\begin{tabular}{c c c} 
 %\hline
%$a(x)$ & $h(x)$ \\ [0.5ex] 
%\hline\hline
%$1$ &  $1+x^4$\\ 
%$1+x^4$ & $1+x^8$ \\
%$1+x^4+x^8$ & $1+x^{12}$ \\
%$1+x^4+x^8+x^{12}$ & $1+x^{16}$ 
%\end{tabular}
%\label{novelTab4}
%\end{table*}
%\end{example}

%========
%\begin{example}
%$g(x)=1+x^2+x^3+x^4$\newline
%$g(x)$ can be written as $$g(x)=(1+x)(1+x+x^3),~K=2$$
 %$1+x$ is prime in $GF(2^1)$ and $\beta^{1}=1$. $1+x+x^3$ is prime in $GF(2^3)$ and $\beta^{7}=1$. $b(x)=1+x^b$ and consequently, the valid values of $a$ that meet the condition $$ \bigcap_{k=1}^{K} \{j_k~| a\}$$ are $a=\{7,14,21,\cdots \}$.
%The corresponding values for $a(x)$ and $b(x)$ are shown in Table \ref{novelTab6} for the first four valid values of $a$
%\end{example}


%\hfill
%\parbox{\linewidth}{
%\begin{table*}[h]
%\caption{$g(x)=1+x^2+x^3+x^4$}
%\centering
% \begin{tabular}{p{4cm}| c} 
 %\hline
 %$a(x)$ & $b(x)$  \\ [0.5ex] 
 %\hline\hline
%$1+x^2+x^3$ & $1+x^7$ \\ 
%\hline
%$1+x^2+x^3+x^7+x^9+x^{10}$ &  $1+x^{14}$ \\
%\hline
%$1+x^2+x^3+x^7+x^{9}+x^{10}+x^{14}+x^{16}+x^{17}$ & $1+x^{21}$ 
%\\
%\hline
%$1+x^2+x^3+x^7+x^{9}+x^{10}+x^{14}+x^{16}+x^{17}+x^{21}+x^{23}+x^{24}$ & $1+x^{28}$
% \end{tabular}
 %\label{novelTab6}
%\end{table*}
%}
%\end{table}


\subsection{The Characteristic of Weight $3$ Parity Check Pattern}
For this weight case, 
\begin{equation}
h(x)=1+x^a+x^b,~a\neq b
\label{novelEqwt3}
\end{equation}
without loss of generality. We consider the characteristics of the roots of $f(x)$ for the folowing 2 cases.
%Given $f(x)$, our task is to find valid $(a,~b)$ pair values satisfying the condition 
%$$1+\beta^u+\beta^v=0$$
% If there are no values for the pair $(u,v)$, then $q(x) \st w_H(\bq) =3$ does not exist for the given $g(x)$.
% It is worth noting that such $h(x)$ only exist iff $f(x)$ has exactly $w_H(\bh) =3$ terms. Moving forward, we assume that for all the cases, this condition holds true.
\paragraph{ Case1} $f(x)$ is a single irreducible polynomial \newline
If $f(x)$ is an irreducible polynomial, then it can be used to generate the extended field GF($2^M$). The non-zero elements of the extended field are represented by $\beta^m,~1 \leq m \leq 2^M-1$. Also, from the discussion in the previous section, we know that $f(x)$ has $M$ distinct roots, one of them being $\beta$. Substituting $\beta$ into (\ref{novelEqwt3}), we get 
\begin{equation}
\begin{split}
&h(\beta)=0\\
&1+\beta^a+\beta^b=0
\end{split}
\label{novelEqwt3-1}
\end{equation}
We can then reformulate our task as to find all $\beta^m$ such that 
\begin{equation}
\beta^{\eta}+\beta^{\zeta}=1,~\eta \neq \zeta
\end{equation}
By refering to the table of the extended field for GF$(2^m)$, we can find the valid $(\eta,~\zeta)$ pairs $\st \beta^{\eta}+\beta^{\zeta}=1$. If there are no valid $(\eta,~\zeta)$ pairs, then there is no parity check component of weight $3$ for the given $f(x)$.
 We represent the set of $(\eta,~\zeta)$ pairs as 
$\bz=\{ (\eta_1,~\zeta_1) ,( \eta_2,~\zeta_2),\cdots\}$. Then, any valid value $(a,~b)$ values should satisfy the condition
\begin{equation}
(a,b) \equiv (\eta,~\zeta) \bmod \epsilon,~(\eta,~\zeta)\in \bz
\end{equation}
% since $\beta^{2^{m}-1}=1$.
\begin{example}
$f(x)=1+x+x^2$ \newline
The elements of GF$(2^2)$ are shown in Table \ref{novelTab7} and it is obvious that there is exactly 1 valid $(\eta,\zeta)$ pair $\st \beta^{\eta}+\beta^{\zeta} = 1$ and that is the pair $(1,2)$.
This means that valid values of the $(a,b)$ pairs are any values $\st (a,b) \equiv (1,2) \bmod 3$.  The corresponding values for $a(x)$ and $h(x)$ are shown in the table below for the first four valid values of $(a,b)$.
\end{example}

 \begin{table}[htbp]
 \caption{Non-zero Elements of GF$(2^2)$ generated by $f(x)=1+x+x^2$}
\centering
 \begin{tabular}{c c} 
 \hline
 power representation & actual value \\ [0.5ex] 
 \hline\hline
$\beta^0~=\beta^3=1$ & $1$\\
\hline
$\beta$ & $\beta$\\
\hline
$\beta^2$ &  $1+\beta$\\
\hline
 \end{tabular}
 \label{novelTab7}
\end{table}

\begin{table}[htbp]
 \caption{$f(x)=1+x+x^2$}
\centering
 \begin{tabular}{c c} 
 \hline
 $a(x)$ & $h(x)$\\ [0.5ex] 
 \hline\hline
$1$ & $1+x+x^2$\\ 
\hline
$1+x+x^2$ &  $1+x^2+x^4$\\
\hline
$1+x+x^3$ & $1+x^4+x^5$\\
\hline
$1+x^2+x^3$ & $1+x+x^5$ 
 \end{tabular}
 \label{novelTab8}
\end{table}

%\paragraph{Case2}$r(x)$ is prime but not a primitive polynomial\newline
%Similar to the case where $f(x)$ is primitive, we confirm the existence of $(e,f)$ pairs $ \st \beta^e + \beta^f =1$. If there are no values for the pair $(e,f)$, then there is no $h(x)$ such that $w_H(h(x)) =3$ for the given $f(x)$.
%We represent the set of $(e,f)$ pairs as 
%$\bz=\{ (e_1,f_1) , e_2,f_2),\cdots\} $. Then any valid value for $a$ and $b$ should satisfy the condition
%$$(a,b) \equiv (e,f) \bmod j,~(e,f)\in \bz$$ where $j$ is the order of $\beta$.

%\begin{example}
%$f(x)=1+x+x^2+x^3+x^4$ \newline
%From the table of the extended field generated by $f(x)$ (Table \ref{novelTab9}), we see that there are no valid $(e,~f)$ and as such $h(x) \st w_H(h(x))=3$ is non-existent for $f(x)=1+x+x^2+x^3+x^4$.


%\begin{table*}[h!]
% \parbox{\linewidth}{
%\caption{Non-zero Elements of GF$(2^4)$ generated by $f(x)=1+x+x^2+x^3+x^4$}
%\centering
 %\begin{tabular}{c c} 
 %\hline
 %power & polynomial \\ [0.5ex] 
% \hline\hline
%$\beta^0~=\beta^5=\beta^{10}=\beta^{15}=1$ & $1$\\
%\hline
%$\beta=\beta^6=\beta^{11}$ & $\beta$\\
%\hline
%$\beta^2=\beta^7=\beta^{12}$ &  $\beta^2$\\
%\hline
%$\beta^3=\beta^8=\beta^{13}$ &  $\beta^3$\\
%\hline
%$\beta^4=\beta^9=\beta^{14}$ &  $\beta^3+\beta^2+\beta+1$\\
 %\end{tabular}
% \label{novelTab9}
%}
%\end{table*}

%\end{example}

%\paragraph{Case3: $r(x)$ is made up of equal repeated polynomial roots\newline}
%For this case, we can write $r(x)$ as 
%$r(x)=(r^{o_p}_p(x))^k$, where $r^{o_p}_p(x)$ represents the prime polynomial, which is the root of $r(x)$ and $k$ is the number of times it is repeated. If $r^{o_p}_p(x)$ is a primitive polynomial and $m>1$, then from Case1 there are exactly $q=2^{(m-1)}-1~(e,~f)$ pairs $\st \beta^e+\beta^f=1,~e \neq f$ in set $\bz$. 
%If $r^{o_p}_p(x)$ is prime but not a primitive polynomial, then we determine the number of elements in the set $\bz$ directly from the table representing the field it generates.
% If we focus on $f^{o_p}_p(x)$ only, $(u',v')$ should satisfy the condition$$ (u',v') \equiv (e,f)\bmod 2^{m}-1,~(e, f) \in \bz$$. However, since $f(x)$ is made up of $f^{o_p}_p(x)$ repeated $k$ times, and $\beta^{ke}+\beta^{kf}=1$, the valid values for $u$ and $v$
 %should satisfy the condition
% $$(u,v)=(ku',kv'),~(u,v) \equiv (e,f)\bmod 2^{m}-1,~(e, f) \in \bz $$.

\paragraph{Case2}$f(x)$ can be factorised into multiple irreducible polynomials. \newline
We may write $f(x)$ as $$f(x)=\prod_{k=1}^{K}f_k(x)$$ where $f_k(x)$ is an irreducible polynomial with order $\epsilon_k$
For each $f_k(x)$, we refer to the table of the extended field it generates and form the set $\bz_k$, which contains all the valid $(\eta^{(k)},~\zeta^{(k)})$ pairs for that particular $f_k(x)$. If that set exists, then, for that $f_k(x)$ the following condition is met
\begin{equation}
(a_k,~b_k) \equiv (\eta^{(k)},~\zeta^{(k)}) \bmod \epsilon_k,~(\eta^{(k)},~\zeta^{(k)})\in \bz_k
\end{equation}
and 
\begin{equation}
(a,~b) \equiv \bigcup_{k=1}^{K} (\eta^{(k)},~\zeta^{(k)}) \bmod \epsilon_k,~(\eta^{(k)},~\zeta^{(k)})\in \bz_k
\end{equation}

For the special case where $f(x)$ can be factorised into equal irreducible polynomial, the above condition simplifies to 
 \begin{equation*}
 \begin{split}
 &(a,~b) \equiv (K\eta,~K\zeta) \bmod \epsilon,~(\eta,~\zeta) \in \bz \\
 \end{split}
 \end{equation*}
 
 \begin{example}
 $f(x)=1+x^2$ \newline $f(x)$ can be written as $(1+x)^2$. $(1+x)$ is a primitive polynomial for GF$(2)$. The elements in GF$(2)$ are $1$ and $\beta$. In this field,  there are no valid $(e,f)$ pair values; therefore, $h(x)$ such that  $w_H(h(x))=3$ does not exist for $f(x)=1+x^2$.
 \end{example}
 
  %\begin{example}
 %$g(x) = 1+x^4$.\newline
 %After factorisation, we have $g(x)=(1+x)^4. $(1+x) is a primitive polynomial that generates the field GF$(2)$ and in this field, there are also no valid $(e,~f)$ and as such $b(x) \st w_H(\bb)=3$ is also non-existent.
 %\end{example}
 
  %\begin{example}
 %$g(x)=1+x^2+x^3+x^4$.\newline
 % Upon factorising, we have $g(x)=(1+x)(1+x+x^3)$. Table \ref{novelTab12} shows the elements of GF$(2^3)$ generated by $(1+x+x^3)$ and we can confirm that there are three valid $(e,~f)$ pairs. However, since there are no valid $(e,~f)$ pairs in GF$(2)$, it also means that there cannot be any valid $(u,~v)$ pairs and $b(x) \st w_H(\bb)=3$ does not exist.
 
 %\begin{table*}[h]
 %\caption{Non-zero Elements of GF$(2^3)$ generated by $1+x+x^3$}
%\centering
 %\begin{tabular}{c c} 
 %\hline
 %power & polynomial \\ [0.5ex] 
 %\hline\hline
%$\beta^0~=\beta^{7}=1$ & $1$\\
%\hline
%$\beta$ & $\beta$\\
%\hline
%$\beta^2$ &  $\beta^2$\\
%\hline
%$\beta^3$ & $\beta+1$\\
%\hline
%$\beta^4$ &  $\beta^2+\beta$\\
%\hline
%$\beta^5$ & $\beta^2+\beta+1$\\
%\hline
%$\beta^6$ &  $\beta^2+1$\\
 %\end{tabular}
 %\label{novelTab12}
%\end{table*}
 %\end{example}

%\begin{example}
%$5/7$ RSC Code, $f(x)=1+x^2,~g(x)=1+x+x^2$\newline
%$f(x)$ is a match for Case3 and can be written as $(1+x)^2$. $(1+x)$ is a primitive polynomial for GF$(2)$. The elements in GF$(2)$ are $1$ and $\beta$. In this field,  there are no valid $(e,f)$; therefore, $h(x) \st w_H(\bh)=3$ does not exist.

%$g(x)$ is a match for Case1, i.e. it is a primitive polynomial for GF$(2^2)$ with $\beta^{3}=1$. 
%The elements of GF$(2^2)$ are shown in Table \ref{novelTab5} and it is obvious that there is exactly 1 ($q=2^{(2-1)}-1~=1$) valid $(e,f)$ pair $\st \beta^e+\beta^f = 1$ and that is $(1,2)$.
%This means that valid values of the $(u,v)$ pairs are any values $\st (u,v) \equiv (1,2) \bmod 3$.  The corresponding values for $a(x),~b(x)$ and $h(x)$ are shown in the table below for the first four valid values of $(u,v)$

% \begin{table*}[h]
 %\caption{Non-zero Elements of GF$(2^2)$ generated by $g(x)=1+x+x^2$}
%\centering
% \begin{tabular}{c c} 
% \hline
 %power representation & actual value \\ [0.5ex] 
% \hline\hline
%$\beta^0~=\beta^3=1$ & $1$\\
%\hline
%$\beta$ & $\beta$\\
%\hline
%$\beta^2$ &  $1+\beta$\\
% \end{tabular}
 %\label{novelTab7}
%\end{table*}

%\begin{table*}[h]
% \caption{$5/7$ RSC, $g(x)=1+x+x^2$}
%\centering
% \begin{tabular}{c c c} 
% \hline
% $a(x)$ & $b(x)$ & $h(x)$\\ [0.5ex] 
% \hline\hline
%$1$ & $1+x+x^2$ & $1+x^2$\\ 
%\hline
%$1+x+x^2$ &  $1+x^2+x^4$ & $1+x+x^3+x^4$ \\
%\hline
%$1+x+x^3$ & $1+x^4+x^5$ & $1+x+x^2+x^5$\\
%\hline
%$1+x^2+x^3$ & $1+x+x^5$  &$1+x^3+x^4+x^5$
 %\end{tabular}
 %\label{novelTab8}
%\end{table*}
%\end{example}

%\newpage

 
 %\begin{example}
%$37/21$ RSC Code, $f(x)=1+x+x^2+x^3+x^4,~g(x)=1+x^4$\newline
%$f(x)$ is a match for Case2, i.e. it is a prime but not a primitive polynomial. From the table of the extended field generated by $f(x)$ (Table \ref{novelTab9}), we see that there are no valid $(e,~f)$ and as such $h(x) \st w_H(\bh)=3$ is non-existent.


 %\begin{table*}[h]
 %\caption{Non-zero Elements of GF$(2^2)$ generated by $f(x)=1+x+x^2+x^3+x^4$}
%\centering
 %\begin{tabular}{c c} 
 %\hline 
 %power & polynomial \\ [0.5ex] 
 %\hline\hline
%$\beta^0~=\beta^5=\beta^{10}=\beta^{15}=1$ & $1$\\
%\hline
%$\beta=\beta^6=\beta^{11}$ & $\beta$\\
%\hline
%$\beta^2=\beta^7=\beta^{12}$ &  $\beta^2$\\
%\hline
%$\beta^3=\beta^8=\beta^{13}$ &  $\beta^3$\\
%\hline
%$\beta^4=\beta^9=\beta^{14}$ &  $\beta^3+\beta^2+\beta+1$\\
 %\end{tabular}
 %\label{novelTab9}
%\end{table*}

%$g(x)$ is a match for Case3, i.e. it is made up of equal repeated roots. After factorisation, we have $g(x)=(1+x)^4. $(1+x) is a primitive polynomial that generates the field GF$(2)$ and in this field, there are also no valid $(e,~f)$ and as such $b(x) \st w_H(\bb)=3$ is also non-existent.
%\end{example}
 
%\begin{example}
%$23/35$ RSC Code, $f(x)=1+x+x^4,~g(x)=1+x^2+x^3+x^4$ \newline
%$f(x)$ is a match for Case1, i.e. it is a primitive polynomial that can be used to generate the extended field GF$(2^4)$ with $\beta^{15}=1$.. The elements of GF$(2^4)$ are shown in Table \ref{novelTab10} and we can see that there are seven valid $(e,f)$ pairs.
 
 %This means that the valid values of the $(u,v)$ pairs are any values $\st (u,v) \equiv (e,f) \bmod 15, (e,f) \in \bz,~\bz=\{(1,4),~(2,8) ,~(3,14) ,~(5,10) ,~(6,13) ,~(7,9) ,~(11,12))$.
 %The corresponding values for $a(x),~b(x)$ and $h(x)$ are shown in Table \ref{novelTab11} below for the first four valid values of $(u,v)$
 
  %\begin{table*}[h]
 %\caption{Non-zero Elements of GF$(2^4)$ generated by $f(x)=1+x+x^4$}
%\centering
 %\begin{tabular}{c c} 
 %\hline
 %power & polynomial \\ [0.5ex] 
 %\hline\hline
%$\beta^0~=\beta^{15}=1$ & $1$\\
%\hline
%$\beta$ & $\beta$\\
%\hline
%$\beta^2$ &  $\beta^2$\\
%\hline
%$\beta^3$ & $\beta^3$\\
%\hline
%$\beta^4$ &  $\beta+1$\\
%\hline
%$\beta^5$ & $\beta^2+\beta$\\
%\hline
%$\beta^6$ &  $\beta^3+\beta^2$\\
%\hline
%$\beta^7$ & $\beta^3+\beta+1$\\
%\hline
%$\beta^8$ &  $\beta^2+1$\\
%\hline
%$\beta^9$ & $\beta^3+\beta$\\
%\hline
%$\beta^{10}$ &  $\beta^2+\beta+1$\\
%\hline
%$\beta^{11}$ & $\beta^3+\beta^2+\beta$\\
%\hline
%$\beta^{12}$ &  $\beta^3+\beta^2+\beta+1$\\
%\hline
%$\beta^{13}$ & $\beta^3+\beta^2+1$\\
%\hline
%$\beta^{14}$ &  $\beta^3+1$\\
% \end{tabular}
% \label{novelTab10}
%\end{table*}
 
% \begin{table*}[h]
% \caption{$23/35$ RSC, $f(x)=1+x+x^4$}
%\centering
% \begin{tabular}{c c c} 
% \hline
% $a(x)$ & $b(x)$ & $h(x)$\\ [0.5ex] 
% \hline\hline
%$1$ & $1+x^2+x^3+x^4$ & $1+x+x^4$\\ 
%\hline
%$1+x+x^2+x^3+x^5$ &  $1+x^2+x^3+x^4+x^8+x^9$ & $1+x^7+x^9$ \\
%\hline
%$1+x+x^2+x^3+x^5+x^7+x^8$ & $1+x^2+x^3+x^4+x^7+x^{12}$ & $1+x^{11}+x^{12}$\\
%\hline
%$1+x+x^4$ & $1+x+x^2+x^4+x^5+x^6+x^7+x^8$  &$1+x^2+x^8$
 %\end{tabular}
% \label{novelTab11}
%\end{table*}
 
% $g(x)$ is a match for Case4, i.e. it is made up of unique repeated roots. Upon factorising, we have $g(x)=(1+x)(1+x+x^3)$. Table \ref{novelTab12} shows the elements of GF$(2^3)$ generated by $(1+x+x^3)$ and we can confirm that there are three valid $(e,~f)$ pairs. However, since there are no valid $(e,~f)$ pairs in GF$(2)$, it also means that there cannot be any valid $(u,~v)$ pairs and $b(x) \st w_H(\bb)=3$ does not exist.
 
% \begin{table*}[h]
% \caption{Non-zero Elements of GF$(2^3)$ generated by $1+x+x^3$}
%\centering
 %\begin{tabular}{c c} 
% \hline
 %power & polynomial \\ [0.5ex] 
% \hline\hline
%$\beta^0~=\beta^{7}=1$ & $1$\\
%\hline
%$\beta$ & $\beta$\\
%\hline
%$\beta^2$ &  $\beta^2$\\
%\hline
%$\beta^3$ & $\beta+1$\\
%\hline
%$\beta^4$ &  $\beta^2+\beta$\\
%\hline
%$\beta^5$ & $\beta^2+\beta+1$\\
%\hline
%$\beta^6$ &  $\beta^2+1$\\
%\end{tabular}
% \label{novelTab12}
%\end{table*}
%\end{example}



%\subsubsection{Higher order weight:$w_H=4$}%: Equal Repeated Polynomial Roots}
%If $w_H(\bq)=4$, then $q(x) = 1+x^{u}+x^{v}+x^{w}$.
%Finding the structure of $q(x)$ when $w_H=4$ for general cases has proven to be a difficult problem, for which we have not found a feasible solution to. However, there is a special case where have been able to come up with a method for determining the structure of $q(x)$. 

%\paragraph{$r(x)=1+x^{2i},~i=1,2,...$}
%Since $q(x)$ is divisible by $r(x)$, the roots of $r(x)$ are also the roots of $q(x)$. Inserting $\beta$ into $q(x)$, we get

%$$1+\beta^{u}+\beta^{v}+\beta^{w} =0$$
%Taking the derivative of the above equation, we get 
%\begin{equation*}
%\begin{split}
% &u\beta^{u-1}+v\beta^{v-1}+w\beta^{w-1} =0 \\
% & u\beta^{u-1}+v\beta^{v-1}+w\beta^{w-1} \equiv 0 \bmod 2
% \end{split}
% \end{equation*}
%We can see that to satisfy the equation above, there are 2 possible cases
%\begin{enumerate}
%\item $(u,~v,~w)$ is made up of two odd numbers, one even number.
%\item $(u,~v,~w)$ are all even numbers.
%\end{enumerate}

%For a valid $(u,~v,~w)$, we fix it into $q(x)$ and if it is divisible by $q(x)$, then $(u,~v,~w)$ is valid.
%In summary, valid $(u,~v,~w)$ should satisfy the condition
%$$u+v+w \equiv 0\bmod 2, ~1+\beta^{u}+\beta^{v}+\beta^{w} \bmod r(x) \equiv 0$$

%\begin{example}
%$5/7$ RSC code $f(x)=1+x^2,~g(x)=1+x+x^2$\newline
%$f(x)$ satisfies the special condition case and we can use the method above to obtain valid values of $(u,~v,~,w)$. The corresponding values for $a(x),~b(x)$ and $h(x)$ are shown in Table \ref{novelTab16} for the first four valid values of $(u,~v,~,w)$.

% \begin{table*}[h]
% \caption{$5/7$ RSC, $f(x)=1+x^2,~w_H(\bh)=4$}
%\centering
 %\begin{tabular}{c c c} 
% \hline
% $a(x)$ & $b(x)$ & $h(x)$\\ [0.5ex] 
% \hline\hline
%$1+x$ & $1+x^3$ & $1+x+x^2+x^3$\\ 
%\hline
%$1+x+x^2$ &  $1+x^2+x^4$ & $1+x+x^3+x^4$ \\
%\hline
%$1+x+x^3$ & $1+x^4+x^5$ & $1+x+x^2+x^5$\\
%\hline
%$1+x^2+x^3$ & $1+x+x^5$  &$1+x^3+x^4+x^5$
% \end{tabular}
% \label{novelTab16}
%\end{table*}

%\end{example}
%\newpage
\section{Union Bound and the Codeword Component Pattern Distance Spectrum}
\label{sec4}
For a given RSC code, we have shown in the previous section that each codeword $c(x)$ is made up of $b(x)$ and $h(x)$ which have $a(x)$ as their common factor as shown in (\ref{novelEq2}) and (\ref{novelEq3}).
 Now, let $\cA_h(d)$ be the set of all $a(x)$ which yields weight-$d$ parity-check component \ie, $w_H(h(x))=w_H(a(x)f(x))=d$ for $a(x) \in \cA_h(d)$. 
Similarly $\cA_b(d)$ is the set of all $a(x)$ which yields weight-$d$ systematic component \ie, $w_H(b(x))=w_H(a(x)g(x))=d$ for $a(x) \in \cA_b(d)$
 and $\cA_c(d)$ is the set of all $a(x)$ which yields weight-$d$ codeword \ie, $w_H(c(x))=w_H(a(x)f(x))+ w_H(a(x)g(x))=d$ for $a(x) \in \cA_c(d)$.  

Then, the union bound of the bit-error rate can be calculated as \cite{ref4}
\begin{align}
P_b \leq \frac{1}{k} \sum_{d=d_{\text{free}}}^{\infty} \sum_{a(x) \in \cA_c(d)}w_H(a(x)g(x)) Q\Bigg( \sqrt{\frac{2dE_c}{N_0}}\Bigg)
\label{novelEq6-1}
\end{align}
However, since the high-weight codewords have minor contribution on the unioin bound, \eqref{novelEq6-1} can be further approximated by setting a limit on the maximum value of the codeword weight $d_{\text{max}}$, resulting in
\begin{align}
P_b \leq \frac{1}{k} \sum_{d=d_{\text{free}}}^{d_{\text{max}}} \sum_{a(x) \in \cA_c(d)}w_H(a(x)g(x)) Q\Bigg( \sqrt{\frac{2dE_c}{N_0}}\Bigg)
\label{novelEq7}
\end{align}
%In order to confirm the validity of our method, we use the values obtained from Tables \ref{novelTab8}, \ref{novelTab9} and \ref{novelTab10} to find the bounds for the BER of the RSC code, $P_b$.Finally, we compare the results obtained to $P_b$ found using the Transfer Function method as well as the simulation results.

On the other hand, since the weight of codeword is the summation of information and parity check parts as shown in (\ref{novelEq-1}), when $w_H(b(x)), w_H(h(x)) \geq 2$, we have
\begin{align}
\cA_c(d) = \bigcup_{\ell = 2}^{d-2} \left\{\cA_b(\ell) \cap \cA_h(d-\ell)\right\}
\label{Eq:exactset}
\end{align}
However, to determine $\cA_b(\ell)$ or $\cA_h(\ell)$ for a large $\ell$ is a complex task in general. Thus, in this paper, we replace the set $\cA_c(d)$ by the approximated set $\cA_c'(d)$ as defined in  (\ref{setApprox})
\begin{equation}
\begin{split}
\cA_c(d) \approx \cA_c'(d) &= \left\{\bigcup_{\ell = 2}^{\ell+\alpha} \left\{\cA_b(\ell) \cap \cA_h(d-\ell)\right\}\right\}\bigcup \\
&\left\{\bigcup_{\ell = 2}^{\ell+\alpha} \left\{\cA_b(d-\ell) \cap \cA_h(\ell)\right\}\right\}
\end{split}
\label{setApprox}
\end{equation}
where some codewords in $\cA_c(d)$ with $\ell \approx d-\ell$ may be ignored in $\cA_c'(d)$.

\begin{example}
 $\alpha=1$\newline
Let $d=7$. The $\cA_c'(7)$ becomes
\begin{equation*}
\begin{split}
\cA_c'(7) &= \left\{\left\{\cA_b(2) \cap \cA_h(7-2)\right\} \bigcup  \left\{\cA_b(3) \cap \cA_h(7-3)\right\} \right\} \bigcup \\
& \left\{\left\{\cA_b(7-2) \cap \cA_h(2)\right\} \bigcup  \left\{\cA_b(7-3) \cap \cA_h(3)\right\} \right\} \\
&=\left\{\left\{\cA_b(2) \cap \cA_h(5)\right\} \bigcup  \left\{\cA_b(3) \cap \cA_h(4)\right\} \right\} \bigcup \\
& \left\{\left\{\cA_b(5) \cap \cA_h(2)\right\} \bigcup  \left\{\cA_b(4) \cap \cA_h(3)\right\} \right\} \\
\end{split}
\end{equation*}

If we set $d=8$, $\cA_c'(8)$ becomes
\begin{equation*}
\begin{split}
\cA_c'(8) &= \left\{\left\{\cA_b(2) \cap \cA_h(8-2)\right\} \bigcup  \left\{\cA_b(3) \cap \cA_h(8-3)\right\} \right\} \bigcup \\
& \left\{\left\{\cA_b(8-2) \cap \cA_h(2)\right\} \bigcup  \left\{\cA_b(8-3) \cap \cA_h(3)\right\} \right\} \\
&=\left\{\left\{\cA_b(2) \cap \cA_h(6)\right\} \bigcup  \left\{\cA_b(3) \cap \cA_h(5)\right\} \right\} \bigcup \\
& \left\{\left\{\cA_b(6) \cap \cA_h(2)\right\} \bigcup  \left\{\cA_b(5) \cap \cA_h(3)\right\} \right\} \\
\end{split}
\end{equation*}

We can see that $\left\{\cA_b(4) \cap \cA_h(4)\right\}$ is not used in $\cA_c'(8)$, event though it is used in $\cA_c(8)$.
\end{example}





%Having determined how to find valid values of $b(x)$ and $h(x)$ for Hamming weights $\leq 3$, we are now in a position to generate the codeword pattern distance spectrum for a given RSC code. We take a union bound like approach towards the generation of the codeword pattern distance spectrum. The approach is outlined below.
%\begin{enumerate}
 %\item Beginning with (\ref{novelEq2}), we find all values of $b(x),~w_H(b(x))=2$ that have the same roots as $g(x)$ and divide $g(x)$ by each valid polynomial to obtain the corresponding $a(x)$.\label{ubStep1}
 %\item Then using (\ref{novelEq3}), we multiply each $a(x)$ by $f(x)$ to obtain the corresponding value of $h(x)$. It is worth noting that $w_H(h(x))$ may be $\geq w_H(b(x))$. \label{ubStep2}
 %\item Since we are interested in only the low weight codewords, we ignore any $b(x) \st w_H(b(x))+w_H(h(x)) \geq d_{\text{max}}$. \label{ubStep3}
 %\item Next we set the weight value of $b(x)$ to $w_H(b(x))=3$, and repeat steps \ref{ubStep1} and \ref{ubStep2} while ignoring $b(x)$ that meet the condition in step \ref{ubStep3}.\label{ubStep4}
 %\item To obtain a complete codeword pattern distance spectrum, we do a reverse operation, \textit{i.e.} we focus on (\ref{novelEq2}) and find all values of $g(x),~w_H(\bh)=2$ that have the same roots as $f(x)$ and divide $f(x)$ by each valid polynomial to obtain the corresponding $a(x)$.
 %\item Then using (\ref{novelEq2}), we repeat steps \ref{ubStep2} through \ref{ubStep4}, being careful to avoid repitition.
 %\item Finally we arrange all valid values of $b(x)$ and $h(x)$ in ascending value of codeword weight,$w_H(b(x)) + w_H(h(x))$.
 %\end{enumerate}

%We use the codeword pattern distance spectrum to calculate the bit-error bounds for each RSC and compare them to the bit-error bounds obtained via the distance spectrum as well as simulation results. We use the probability of bit-error in doing this and a more general formula for calculating $P_b$ is shown below [4]:

%\begin{equation}
%P_b \leq \frac{1}{k} \sum_{d=d_{\text{free}}}^{\infty} w(d) Q\Bigg( \sqrt{\frac{2dE_c}{N_0}}\Bigg)
%\label{novelEq6}
%\end{equation}
%where $w(d)=\sum_{i=1}^{\infty} i~ a(d,i)$ and $ a(d,i)$ is the number of codewords of weight $d$ generated by an input message of weight $i$. If we set a limit on the maximum value of the codeword weight $d_{\text{max}}$
% we can rewrite (\ref{novelEq6}) as 
%\begin{equation}
%P_b \leq \frac{1}{k} \sum_{d=d_{\text{free}}}^{d_{\text{max}}} w(d) Q\Bigg( \sqrt{\frac{2dE_c}{N_0}}\Bigg)
%\label{novelEq7}
%\end{equation}
 %From the simulation results, we observed $d_{\text{max}}=d_{\text{min}}+3$ is a sufficient value for obtaining the BER bounds.
%In order to confirm the validity of our method, we use the values obtained from Tables \ref{novelTab8}, \ref{novelTab9} and \ref{novelTab10} to find the bounds for the BER of the RSC code, $P_b$.Finally, we compare the results obtained to $P_b$ found using the Transfer Function method as well as the simulation results.






